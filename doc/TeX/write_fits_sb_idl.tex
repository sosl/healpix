% -*- LaTeX -*-

% PLEASE USE THIS FILE AS A TEMPLATE FOR THE DOCUMENTATION OF YOUR OWN
% FACILITIES: IN PARTICULAR, IT IS IMPORTANT TO NOTE COMMENTS MADE IN
% THE TEXT AND TO FOLLOW THIS ORDERING. THE FORMAT FOLLOWS ONE USED BY
% THE COBE-DMR PROJECT.	
% A.J. Banday, April 1999.




\renewcommand{\facname}{{write\_fits\_sb}}
\renewcommand{\FACNAME}{{WRITE\_FITS\_SB}}
\sloppy



\title{\healpix IDL Facility User Guidelines}
\docid{write\_fits\_sb} \section[write\_fits\_sb]{ }
\label{idl:write_fits_sb}
\docrv{Version 1.2}
\author{Eric Hivon}
\abstract{This document describes the \healpix facility \thedocid.}




\begin{facility}
{This IDL facility writes out a \healpix map into a FITS file according to
the \healpix convention. It can also write an arbitray data set into a FITS
binary table}
{src/idl/fits/write\_fits\_sb.pro}
\end{facility}

\begin{IDLformat}
{\FACNAME, \mylink{idl:write_fits_sb:File}{File}%
, \mylink{idl:write_fits_sb:Prim_Stc}{Prim\_Stc}%
 [, \mylink{idl:write_fits_sb:Xten_stc}{Xten\_stc}%
, \mylink{idl:write_fits_sb:Coordsys}{Coordsys=}%
, \mylink{idl:write_fits_sb:Nested}{/Nested}%
, \mylink{idl:write_fits_sb:Ring}{/Ring}%
,
\mylink{idl:write_fits_sb:Ordering}{Ordering=}%
, \mylink{idl:write_fits_sb:Partial}{/Partial}%
, \mylink{idl:write_fits_sb:Nside}{Nside=}%
, \mylink{idl:write_fits_sb:Extension}{Extension=}%
, \mylink{idl:write_fits_sb:Nothealpix}{/Nothealpix}%
]
}
\end{IDLformat}

\begin{qualifiers}
  \begin{qulist}{} %%%% NOTE the ``extra'' brace here %%%%
 	\item[{File}] \mytarget{idl:write_fits_sb:File}
          name of a FITS file in which the map is to be written

 	\item[{Prim\_stc}]  \mytarget{idl:write_fits_sb:Prim_Stc}
	IDL structure containing the following fields: \\
		- primary header \\
		- primary image \\
	Set it to 0 to get an empty primary unit

       \item[{Xten\_stc}]  \mytarget{idl:write_fits_sb:Xten_stc}
		  (optional), \\
	IDL structure containing the following fields: \\
		- extension header  \\
		- data column 1  \\
		- data column 2   \\
		... \\
	NB: because of some astron routines limitation, avoid using the single letters
		  'T' or 'F' as tagnames in the structures Prim\_stc and Xten\_stc.

  \end{qulist}
\end{qualifiers}

\begin{keywords}
  \begin{kwlist}{} %%% extra brace
       \item[{Coordsys=}]  \mytarget{idl:write_fits_sb:Coordsys}
		  (optional), \\
		if set to either 'C', 'E' or 'G',  specifies that the
		Healpix coordinate system is respectively Celestial=equatorial,
		  Ecliptic or Galactic.
		(The relevant keyword is then added/updated in the extension
		  header, but the map is NOT rotated)

	\item[{Ordering=}]  \mytarget{idl:write_fits_sb:Ordering}
		  (optional), \\
		if set to either 'ring' or 'nested' (case un-sensitive),
		  specifies that the map is respectively in RING or NESTED
		  ordering scheme\\
		\seealso Nested and Ring

	\item[{Nside=}]   \mytarget{idl:write_fits_sb:Nside}
		(optional), \\
		scalar integer, \healpix resolution parameter of the
		data set. Must be used when the data set does not
		cover the whole sky

	\item[{Extension=}]  \mytarget{idl:write_fits_sb:Extension}
		(optional), \\
		scalar integer, extension in which to write the data
		(0 based).\\
		\default 0

	\item[{/Nested}]   \mytarget{idl:write_fits_sb:Nested}
	  (optional), \\ 
	  if set, specifies that the map is in the NESTED ordering
	scheme\\
	\seealso Ordering and Ring 

	\item[{/Ring}]   \mytarget{idl:write_fits_sb:Ring}
	  (optional), \\ 
	  if set, specifies that the map is in the RING ordering
	scheme\\
	\seealso Ordering and Nested

	\item[{/Partial}]   \mytarget{idl:write_fits_sb:Partial}
	  (optional), \\ 
	  if set, the data set does not cover the whole sky. In
	that case the information on the actual map resolution should be given by the
	qualifier Nside (see above), or included in the FITS header enclosed in
	the Xten\_stc.

	\item[{/Nothealpix}]   \mytarget{idl:write_fits_sb:Nothealpix}
	  (optional), \\ 
	  if set, the data set can be arbitrary, and the
	restriction on the number of pixels do not apply. The keywords {\tt
	Ordering}, {\tt Nside}, {\tt Nested}, {\tt Ring} and {\tt Partial} are ignored.

   \end{kwlist}
\end{keywords}

\begin{codedescription}
%%%{\parbox[t]{\hsize}
{\parbox[t]{\hsize}{\facname{} writes out the information contained in {\tt{Prim\_stc}} and {\tt{Exten\_stc}} in the primary unit and extension of the FITS file
{\tt File} respectively . Coordinate systems can also be specified by {\tt Coordsys}. Specifying the
ordering scheme is compulsary for \healpix data sets and can be done either in {\tt Header} or by setting {\tt
Ordering} or {\tt Nested} or {\tt Ring} to the correct value. If {\tt
Ordering} or {\tt Nested} or {\tt Ring} is set, its value overrides what is
given in {\tt Header}. \\

The data is assumed to represent a full sky data set with 
the number of data points npix = 12*Nside*Nside
unless   \hfill\newline
Partial is set {\em or} the input FITS header contains OBJECT =
               'PARTIAL' \hfill\newline
       AND \hfill\newline
         the Nside qualifier is given a valid value {\em or} the FITS header contains
                 a NSIDE.\\

In the \healpix scheme, invalid or missing pixels should be given the value {\tt
!healpix.bad\_value}$= -1.63750\, 10^{30}$.\\

If {\tt Nothealpix} is set, the restrictions on Nside are void.}}
\end{codedescription}



\begin{related}
  \begin{sulist}{} %%%% NOTE the ``extra'' brace here %%%%
  \item[idl] version \idlversion or more is necessary to run \thedocid
  \item[\htmlref{read\_fits\_map}{idl:read_fits_map}] This \healpix IDL facility can be used to read in maps
  written by \thedocid.
  \item[\htmlref{read\_fits\_s}{idl:read_fits_s}] This \healpix IDL facility can be used to read
  into an IDL structure maps written by \thedocid.
  \item[sxaddpar] This IDL routine (included in \healpix package) can be used to update
  or add FITS keywords to the header in {\tt Prim\_stc} and {\tt Exten\_stc}
  \item[%
\htmlref{write\_fits\_cut4}{idl:write_fits_cut4},
\htmlref{write\_fits\_partial}{idl:write_fits_partial},
\htmlref{write\_fits\_map}{idl:write_fits_map}]
  \item[%
\htmlref{write\_tqu}{idl:write_tqu},
\htmlref{write\_fits\_sb}{idl:write_fits_sb}]
\healpix IDL routines to write cut-sky and partial maps, full-sky maps, polarized full-sky maps and
arbitrary data sets into FITS files

  \item[\htmlref{write\_tqu}{idl:write_tqu}] This \healpix IDL facility based on \thedocid{} is designed to write
  temperature+polarization (T,Q,U) maps
  \end{sulist}
\end{related}


% \begin{example}
% {
% \begin{tabular}{ll} %%%% use this tabular format %%%%
% npix =& \htmlref{nside2npix}{idl:nside2npix}(64) \\
% t =& randomn(seed,npix) \\
% q =& randomn(seed,npix) \\
% u =& randomn(seed,npix) \\
% map\_TQU =& create\_struct('HDR','[]','TEMPERATURE',t,'Q\_POL',q,'U\_POL',u) \\
% write\_fits\_sb, & 'map\_polarization.fits', 0, map\_TQU, coord='G', /ring\\
% \end{tabular}
% }
% {
% The structure map\_TQU is defined to contain a fictitious polarisation map, with
% the 3 Stokes parameters T, Q and U.
% \thedocid{} writes out the contents of map\_TQU into the extension 
% of the FITS file 'map\_polarization.fits'.
% }
% \end{example}

\begin{example}
{
\begin{tabular}{l} %%%% use this tabular format %%%%
npix =  \htmlref{nside2npix}{idl:nside2npix}(128) \\
f=  randomn(seed,npix) \\
n=  lindgen(npix)+3 \\
map\_FN =  create\_struct('HDR',[' '],'FLUX',f,'NUMBER',n) \\
\thedocid,  'map\_fluxnumber.fits', 0, map\_FN, coord='G', /ring\\
\end{tabular}
}
{
The structure map\_FN is defined to contain a fictitious Flux$+$number map, where
one field is a float and the other an integer.
\thedocid{} writes out the contents of map\_FN into the extension 
of the FITS file 'map\_fluxnumber.fits'.
}
\end{example}

