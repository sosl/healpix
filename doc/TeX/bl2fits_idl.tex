% -*- LaTeX -*-




\sloppy



\title{\healpix IDL Facility User Guidelines}
\docid{bl2fits} \section[bl2fits]{ }
\label{idl:bl2fits}
\docrv{Version 1.0}
\author{Eric Hivon}
\abstract{This document describes the \healpix IDL facility \thedocid.}




\begin{facility}
{This IDL facility provides a means to
write into a FITS file as an ascii table extension a (beam) window function
$W(\ell)$ or $W(\ell)$. Adds additional
headers if required. The facility is primarily intended to allow the
user to write an arbitrary window function into a FITS file in 
the correct format to be ingested by the \healpix simulation facility 
\textbf{synfast}.
}
{src/idl/fits/bl2fits.pro}
\end{facility}

\begin{IDLformat}
{{BL2FITS}%
, \mylink{idl:bl2fits:bl_array}{bl\_array}%
, \mylink{idl:bl2fits:fitsfile}{fitsfile}%
, [HDR = , \mylink{idl:bl2fits:HELP}{/HELP}%
, XHDR =]}
\end{IDLformat}

\begin{qualifiers}
  \begin{qulist}{} %%%% NOTE the ``extra'' brace here %%%%
    \item[bl\_array\mytarget{idl:bl2fits:bl_array}%
] real or double array of Bl coefficients to be written to
      file. This has dimension (lmax+1,$n$) with $1\le n \le 3$, given in the sequence T E B.
    \item[fitsfile\mytarget{idl:bl2fits:fitsfile}%
] String containing the name of the file to be written.
  \end{qulist}
\end{qualifiers}

\begin{keywords}
  \begin{kwlist}{} %%% extra brace
    \item[HDR\mytarget{idl:bl2fits:HDR}%
 =] String array containing the (non-trivial) primary header
      for the FITS file. 
    \item[/HELP\mytarget{idl:bl2fits:HELP}%
] If set, a help message is printed out, no file is written

    \item[XHDR\mytarget{idl:bl2fits:XHDR}%
 =] String array containing the (non-trivial) extension header
      for the FITS file. 

  \end{kwlist}
\end{keywords}  

\begin{codedescription}
{\thedocid\ writes the input $B(\ell)$ or $W(\ell)$ coefficients into a FITS
file containing an ascii table extension. Optional headers conforming
to the FITS convention can also be written to the output file. All
required FITS header keywords (like SIMPLE, BITPIX, ...) are automatically generated by the
routine and should NOT be duplicated in the optional header inputs
(they would be ignored anyway).
The one/two/three column(s) are automatically named 
{\tt TEMPERATURE}, {\tt GRAD}, {\tt CURL}
respectively.
If the window function is provided in a double precision array, the output format
will automatically feature more decimal places.}
\end{codedescription}



\begin{related}
  \begin{sulist}{} %%%% NOTE the ``extra'' brace here %%%%
    \item[idl] version \idlversion or more is necessary to run \thedocid.
    \item[\htmlref{fits2cl}{idl:fits2cl}] provides the complimentary routine to read in a
      window function or power spectrum from a FITS file.
    \item[synfast] utilises the output file generated by \thedocid (option {\tt beam\_file}).
  \end{sulist}
\end{related}

\begin{example}
{
\begin{tabular}{l} %%%% use this tabular format %%%%
beam1 =  gaussbeam(10., 2000, 1) \\
beam2 =  gaussbeam(15., 2000, 1) \\
beam  =  (beam1 + beam2) / 2. \\
\thedocid,  beam, 'beam.fits'
\end{tabular}
}
{
\thedocid\ writes the beam window function stored in the variable {\tt beam}
(=Legendre transform of a circular beam)
into the output FITS file {\tt beam.fits}.
}
\end{example}


