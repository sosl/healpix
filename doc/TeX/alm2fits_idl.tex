% -*- LaTeX -*-




\sloppy



\title{\healpix IDL Facility User Guidelines}
\docid{alm2fits} \section[alm2fits]{ }
\label{idl:alm2fits}
\docrv{Version 1.2}
\author{Anthony J.~Banday, E. Hivon}
\abstract{This document describes the \healpix facility alm2fits.}




\begin{facility}
{This IDL routine provides a means to write 
spherical harmonic coefficients (and optional errors) and their index label
to a FITS file. Each signal is written to a separate binary table
extension. The routine also writes header information if required.
The facility is primarily designed to allow the user to write 
a FITS files containing constraints for a constrained realisation
performed by the \healpix facility \textbf{synfast}.
}{src/idl/fits/alm2fits.pro}
\end{facility}

\begin{IDLformat}
{ALM2FITS, 
\mylink{idl:alm2fits:index}{index}, 
\mylink{idl:alm2fits:alm_array}{alm\_array}, 
\mylink{idl:alm2fits:fitsfile}{fitsfile}, [%
\mylink{idl:alm2fits:hdr}{HDR=}, 
\mylink{idl:alm2fits:help}{/HELP}, 
\mylink{idl:alm2fits:xhdr}{XHDR=}]}
\end{IDLformat}

\begin{qualifiers}
  \begin{qulist}{} %%%% NOTE the ``extra'' brace here %%%%
    \item[index] \mytarget{idl:alm2fits:index}Long array containing the index for the corresponding
                 array of alm coefficients (and erralm if required). The
                 index {$i$} is related to {$l,m$} by the relation \hfill\newline
                 $i$ = $\ell^2$ + $\ell$ + $m$ + 1
    \item[alm\_array] \mytarget{idl:alm2fits:alm_array}Real array of alm coefficients written to the
      file. This has dimension (nl,nalm,nsig) -- corresponding to\hfill\newline
      nl   = number of {l,m} indices \hfill\newline
      nalm = 2 for real and imaginary parts of alm coefficients or
             4 for above plus corresponding error values \hfill\newline
      nsig = number of signals to be written (1 for any of T E B
             or 3 if ALL to be written). Each signal is stored
             in a separate extension.
    \item[fitsfile] \mytarget{idl:alm2fits:fitsfile}String containing the name of the file to be
      written.
  \end{qulist}
\end{qualifiers}

\begin{keywords}
  \begin{kwlist}{} %%% extra brace
    \item[HDR =] \mytarget{idl:alm2fits:hdr}String array containing the primary header to be written in the FITS
      file. 
    \item[/HELP] \mytarget{idl:alm2fits:help}If set, the routine documentation header is shown and the routine exits	
    \item[XHDR =] \mytarget{idl:alm2fits:xhdr}String array containing the extension header. If
                  ALL signals are required, then each extension table
                  is given this header.
    \item[ ] NOTE: optional header strings should NOT include the
             header keywords explicitly written by this routine.
  \end{kwlist}
\end{keywords}  

\begin{codedescription}
{\thedocid\ writes the input alm coefficients (and associated errors if 
required) into a FITS file. Each signal type is written as a separate
binary table extension. Optional headers conforming
to the FITS convention can also be written to the output file. All
required FITS header keywords are automatically generated by the
routine and should NOT be duplicated in the optional header inputs.
The keywords EXTNAME and TTYPE* are now also automatically generated.
}
\end{codedescription}



\begin{related}
  \begin{sulist}{} %%%% NOTE the ``extra'' brace here %%%%
    \item[idl] version \idlversion or more is necessary to run \thedocid.
    \item[\htmlref{fits2alm}{idl:fits2alm}] provides the complimentary routine to read in 
      alm coefficients from a FITS file.
    \item[\htmlref{alm\_i2t}{idl:alm_i2t}, \htmlref{alm\_t2i}{idl:alm_t2i}]
these facilities turn indexed lists of $a_{\ell m}$ into 2D a(l,m) tables and back
    \item[\htmlref{lm2index}{idl:lm2index}] converts the $a_{\ell m}$ order and degree
    $(\ell, m)$ into the index $i$ = $\ell^2$ + $\ell$ + $m$ + 1 required by
\thedocid.
    \item[\htmlref{cl2fits}{idl:cl2fits}] routine to write a power spectrum into
a FITS file.
    \item[\htmlref{fits2cl}{idl:fits2cl}] routine to read/compute $C(l)$ power spectra from a file containing $C(l)$ or $a_{lm}$ coefficients
    \item[alteralm] utilises the output file generated by \thedocid.
    \item[synfast] utilises the output file generated by \thedocid.
  \end{sulist}
\end{related}

\begin{example}
{
\begin{tabular}{l} %%%% use this tabular format %%%%
\thedocid, index, alm, 'alm.fits', HDR = hdr, XHDR = xhdr \\
\end{tabular}
}
{
\thedocid\ writes the coefficients stored in the variable {\tt alm}
to the output FITS file {\tt alm.fits} with optional headers
passed by the string variables {\tt hdr} and {\tt xhdr}.
}
\end{example}

