
\sloppy


\title{\healpix Fortran Subroutines Overview}
\docid{nside2npix} \section[nside2npix]{ }
\label{sub:nside2npix}
\docrv{Version 1.1}
\author{E. Hivon}
\abstract{This document describes the \healpix Fortran90 subroutine NSIDE2NPIX.}

\begin{facility}
{Function to provide the number of pixels $\npix$ over the full sky corresponding
to resolution parameter $\nside$. 
}
{\modPixTools}
\end{facility}

\begin{f90function}
{\mylink{sub:nside2npix:nside}{nside}%
}
\end{f90function}

\begin{arguments}
{
\begin{tabular}{p{0.3\hsize} p{0.05\hsize} p{0.1\hsize} p{0.45\hsize}} \hline  
\textbf{name~\&~dimensionality} & \textbf{kind} & \textbf{in/out} & \textbf{description} \\ \hline
                   &   &   &                           \\ %%% for presentation
nside\mytarget{sub:nside2npix:nside} & I4B & IN & the $\nside$ parameter of the map. \\
var & I8B & OUT & the number of pixels $\npix$ of the map. If $\nside$ is valid (a power of 2 in
$\{1,\ldots,2^{28}=268435456\}$), $\npix=12\nside^2$ is returned; if not, an error message is
issued and -1 is returned.\\
\end{tabular}
}
\end{arguments}

\begin{example}
{
use \htmlref{healpix\_modules}{sub:healpix_modules} \\
integer(\htmlref{I8B}{sub:healpix_types}) :: npix \\
npix= nside2npix(256)  \\
}
{
Returns the number of \healpix pixels (786432) for the resolution
parameter 256.
}
\end{example}
\begin{related}
  \begin{sulist}{} %%%% NOTE the ``extra'' brace here %%%%
  \item[\htmlref{npix2nside}{sub:npix2nside}] returns resolution parameter corresponding to the number of pixels.
%   \item[pix2xxx] conversion between pixel index and position on the sky.
  \end{sulist}
\end{related}

\rule{\hsize}{2mm}

