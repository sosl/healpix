% -*- LaTeX -*-

\sloppy

\title{\healpix IDL Facility User Guidelines}
\docid{hpx2dm} \section[hpx2dm]{ }
\label{idl:\thedocid}
\docrv{Version 1.0}
\author{Eric Hivon}
\abstract{This document describes the \healpix facility \thedocid.}

% q customized for hpx2dm IDL routines adapted from mollview qualifiers
  \newenvironment{qualifiers_hpx2dm}
    {\rule{\hsize}{0.7mm}
     \textsc{\Large{\textbf{QUALIFIERS}}}\hfill\newline%
	\renewcommand{\arraystretch}{1.5}%
	}

% \newcommand{\sizeoneg}{0.19\hsize}
%\newcommand{\sizetwog}{0.08\hsize}
%\newcommand{\sizethrg}{0.70\hsize}
\newcommand{\sizeoneg}{0.12\hsize}
\newcommand{\sizethrg}{0.85\hsize}

\begin{facility}
{This IDL facility provides a means to turn a \healpix\ data set
into a DomeMaster
compliant image (azimuthal equidistant projection of the half-sphere
in a PNG or lossless JPEG file) that can be projected on a planetarium.
See eg 
\htmladdnormallink{http://fulldome.ryanwyatt.net/fulldome\_domemasterSpec\_v05.pdf}%
{http://fulldome.ryanwyatt.net/fulldome_domemasterSpec_v05.pdf}
}
{src/idl/visu/hpx2dm.pro}
\end{facility}

\begin{IDLformat}
{\thedocid, 
\mylink{idl:hpx2dm:file}{File}, 
[ \mylink{idl:hpx2dm:select}{Select}, ]
[ \mylink{idl:hpx2dm:help}{/HELP}, 
\mylink{idl:hpx2dm:jpeg}{JPEG=}, 
\mylink{idl:hpx2dm:png}{PNG=}, 
\mylink{idl:hpx2dm:preview}{PREVIEW=}, 
\mylink{idl:hpx2dm:pxsize}{PXSIZE=}, 
%
  + %
\mylink{idl:hpx2dm:other_keywords}{most of azeqview keywords}%
$\ldots$
]
}
\end{IDLformat}

\newpage
\begin{qualifiers_hpx2dm}
\begin{tabular}{p{\sizeoneg}%
% p{\sizetwog} 
p{\sizethrg}}
%% \begin{tabular}{p{0.14\hsize} p{0.08\hsize} p{0.75\hsize}}
\hline  
%\textbf{name}  & \textbf{description} \\ \hline
%                    &   &                            \\ %%% for presentation
{File}\mytarget{idl:hpx2dm:file}    & \parbox[t]{0.95\hsize}{Required\\
                    name of a FITS file containing 
               the \healpix\ map in an extension or in the image field, \\
          {\em or}\  \  name of an {\em online} variable (either array or
structure) containing the \healpix\ map (See note below);\\
          if Save is set   :    name of an IDL saveset file containing
               the \healpix\ map stored under the variable  {\tt data} \\
	\nodefault}\\ 

{Select}\mytarget{idl:hpx2dm:select}    & \parbox[t]{0.95\hsize}{Optional\\
		  column of the BIN FITS table to be plotted, can be either  \\
                -- a name : value given in TTYPEi of the FITS file \\
                        NOT case sensitive and can be truncated, \\
                        (only letters, digits and underscore are valid) \\
               -- an integer        : number i of the column
                            containing the data, starting with 1 (also valid if
		  {\tt File} is an online array) \\
                   \default {1 for full sky maps, 'SIGNAL' column for FITS files
		  containing cut sky maps}}\\
\end{tabular}
\end{qualifiers_hpx2dm}

\begin{keywords}
  \begin{kwlist}{} %%%% NOTE the ``extra'' brace here %%%%

\item [{JPEG=}\mytarget{idl:hpx2dm:jpeg}] name of the output {\em lossless}
JPEG file
\item [{PNG=}\mytarget{idl:hpx2dm:png}]  name of the output PNG file

\item [{/PREVIEW}\mytarget{idl:hpx2dm:preview}] if set, the output JPEG or PNG
file will be previewed

\item [{/HELP}\mytarget{idl:hpx2dm:help}]  Prints out the documentation header 

\item [{PXSIZE=}\mytarget{idl:hpx2dm:pxsize}]  number of pixels in each
dimension of the square output image

% \item [{KML=} \mytarget{idl:hpx2dm:kml}] Name of the KML file to be created (if the {\tt .kml} suffix is missing,
%      it will be added automatically)
%      \default {{\tt 'hpx2googlesky.kml'}}

% \item [{PNG=}\mytarget{idl:hpx2dm:png} ] Name of the PNG overlay file to be created. Only to be used if you want the
%      filename to be different from the default 
% 	(\default{ same as KML file, with a {\tt .png} suffix instead
%      of {\tt .kml}})

% \item [{RESO\_ARCMIN=} \mytarget{idl:hpx2dm:reso_arcmin}] Pixel angular size in arcmin (at the equator) of the cartesian
%      map generated \default{30}

% \item [{SUBTITLE=} \mytarget{idl:hpx2dm:subtitle}] information on the data, 
% will appear in KML file {\tt GroundOverlay
%      description} field

% \item [{TITLEPLOT=} \mytarget{idl:hpx2dm:titleplot}] information on the data, 
% will appear in KML file {\tt GroundOverlay
%      name} field

\item [\mytarget{idl:hpx2dm:other_keywords}
\mylink{idl:mollview:asinh}{/ASINH}, ]
%
\item [\mylink{idl:mollview:colt}{COLT}=,
\mylink{idl:mollview:coord}{COORD}=,
\mylink{idl:mollview:factor}{FACTOR}=,
\mylink{idl:mollview:flip}{/FLIP},
% \mylink{idl:mollview:glsize}{GLSIZE}=,
% \mylink{idl:mollview:graticule}{GRATICULE}=,
\mylink{idl:mollview:hbound}{HBOUND}=,]
%
\item [\mylink{idl:mollview:hist_equal}{/HIST\_EQUAL},
% \item [\mylink{idl:mollview:jpeg}{JPEG=},
% \mylink{idl:mollview:iglsize}{IGLSIZE}=,
% \mylink{idl:mollview:igraticule}{IGRATICULE}=,
\mylink{idl:mollview:log}{/LOG},
\mylink{idl:mollview:max}{MAX}=,
\mylink{idl:mollview:min}{MIN}=, %]
\mylink{idl:mollview:nested}{/NESTED},
% \mylink{idl:mollview:no_dipole}{/NO\_DIPOLE},
% \mylink{idl:mollview:no_monopole}{/NO\_MONOPLE},
\mylink{idl:mollview:offset}{OFFSET}=, ]
% \item [\mylink{idl:mollview:outline}{OUTLINE}=,
% \mylink{idl:mollview:polarization}{POLARIZATION}=,
% \item [\mylink{idl:mollview:png}{PNG=},
% \mylink{idl:mollview:preview}{/PREVIEW},]
%
\item [\mylink{idl:mollview:quadcube}{/QUADCUBE},
\mylink{idl:mollview:rot}{ROT}=,
\mylink{idl:mollview:save}{SAVE}=,
\mylink{idl:mollview:silent}{/SILENT}, ]
%
\item [\mylink{idl:mollview:truecolors}{TRUECOLORS}=]
 those keywords have the same meaning as in
\htmlref{azeqview}{idl:azeqview} and 
\htmlref{mollview}{idl:mollview}


  \end{kwlist}
\end{keywords}
%***************************************************************


\begin{codedescription}
{\thedocid\ reads in a \healpix sky map in FITS format or from a memory array
and generates a PNG or JPEG file containing a DomeMaster compliant map
(azimuthal equidistant projection of the half-sky).%
}
\end{codedescription}

%
% defines the field 'RELATED' for mollview, gnomview, orthview,
% cartview

\begin{related}
  \begin{sulist}{} %%%% NOTE the ``extra'' brace here %%%%
  \item[idl] version \idlversion or more is necessary to run \thedocid
  \item[ghostview] ghostview or a similar facility is required to view
	  the Postscript image generated by \thedocid.
  \item[xv] xv or a similar facility is required to view the
            GIF/PNG image generated by \thedocid  (a browser can also 
            be used).
  \item[synfast] This \healpix facility will generate the FITS format 
            sky map to be input to \thedocid.
  \item[{\htmlref{cartview}{idl:cartview}}] 
	IDL facility to generate a Cartesian projection of
  	a \healpix map.
  \item[{\htmlref{cartcursor}{idl:cartcursor}}] 
	interactive cursor to be used with cartview
  \item[{\htmlref{gnomview}{idl:gnomview}}] 
	IDL facility to generate a gnomonic projection of
  	a \healpix map.
  \item[{\htmlref{gnomcursor}{idl:gnomcursor}}] 
	interactive cursor to be used with gnomview
  \item[{\htmlref{mollview}{idl:mollview}}] 
	IDL facility to generate a Mollweide projection of
  	a \healpix map.
  \item[{\htmlref{mollcursor}{idl:mollcursor}}] interactive cursor to be used with mollview
  \item[{\htmlref{orthview}{idl:orthview}}] 
	IDL facility to generate an orthographic projection of
  	a \healpix map.
  \item[{\htmlref{orthcursor}{idl:orthcursor}}] 
	interactive cursor to be used with orthview
  \end{sulist}
\end{related}

\begin{related}
  \begin{sulist}{} %%%% NOTE the ``extra'' brace here %%%%
  \item[\htmlref{azeqview}{idl:azeqview}] performs Azimuthal Equidistant
projection required by \thedocid.
  \item[\htmlref{hpx2gs}{idl:hpx2gs}] turns Healpix maps into GoogleEarth or
GoogleSky images
  \end{sulist}
\end{related}


% \begin{example}
% {
% \begin{tabular}{l} %%%% use this tabular format %%%%

% map  = findgen(48) \\
% \thedocid, map, kml='my\_map.kml',title='my map in Google'\\
% \end{tabular}
% }
% {produces in {\tt my\_map.kml} and in {\tt my\_map.png} an image of the input map that can be seen with
% Google Sky.
% To do so, start GoogleEarth or GoogleSky and open {\tt my\_map.kml}. Under
% Mac\-OSX, simply type {\tt open my\_map.kml} on the command line.
% }
% \end{example}

\newpage
