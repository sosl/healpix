

\sloppy


\title{\healpix C Subroutines Overview}
\docid{vec2ang} \section[vec2ang]{ }
\label{csub:vec2ang}
\docrv{Version 2.0}
\author{E. Hivon}
\abstract{This document describes the \healpix C subroutine VEC2ANG.}

\begin{facility}
{Routine to convert the 3D position vector $(x,y,z)$ of point into its position
  angles  $(\theta,\phi)\myhtmlimage{}$ on the sphere with
$x = \sin\theta\cos\phi\myhtmlimage{}$, $y=\sin\theta\sin\phi\myhtmlimage{}$, $z=\cos\theta\myhtmlimage{}$.
}
{src/C/subs/vec2ang.c}
\end{facility}

\begin{Cfunction}
{void vec2ang(double *vector, double *theta, double *phi);}
\end{Cfunction}

\begin{arguments}
{
\begin{tabular}{p{0.3\hsize} p{0.10\hsize} p{0.05\hsize} p{0.45\hsize}} \hline  
\textbf{name~\&~dimensionality} & \textbf{kind} & \textbf{in/out} & \textbf{description} \\ \hline
                   &   &   &                           \\ %%% for presentation
vector(3) & double & IN & three dimensional cartesian position vector
                   $(x,y,z)$. The north pole is $(0,0,1)$\\
theta & double & OUT & colatitude in radians measured southward from north pole (in
    [0,$\pi$]). \\
phi   & double & OUT & longitude in radians measured eastward (in [0, $2\pi$]).\\
\end{tabular}
}
\end{arguments}

% \begin{example}
% {
% call vec2ang(vector,theta,phi) \\
% }
% {
% }
% \end{example}
\begin{related}
  \begin{sulist}{} %%%% NOTE the ``extra'' brace here %%%%
  \item[\htmlref{ang2vec}{csub:ang2vec}] converts the position angles of a point on the sphere 
into its 3D position vector.
  \end{sulist}
\end{related}

\rule{\hsize}{2mm}

