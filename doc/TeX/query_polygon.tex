
\sloppy


\title{\healpix Fortran Subroutines Overview}
\docid{query\_polygon} \section[query\_polygon]{ }
\label{sub:query_polygon}
\docrv{Version 1.3}
\author{Eric Hivon}
\abstract{This document describes the \healpix Fortran90 subroutine QUERY\_POLYGON.}

\begin{facility}
{Routine to find the index of all pixels enclosed in a polygon. The polygon should be convex, 
or have only one concave vertex. The edges should not intersect each other. 
The output indices can be either in the RING or NESTED scheme} 
{\modPixTools}
\end{facility}

\begin{f90format}
{\mylink{sub:query_polygon:nside}{nside}%
, \mylink{sub:query_polygon:vlist}{vlist}%
, \mylink{sub:query_polygon:nv}{nv}%
, \mylink{sub:query_polygon:listpix}{listpix}%
, \mylink{sub:query_polygon:nlist}{nlist}%
 [, \mylink{sub:query_polygon:nest}{nest}%
, \mylink{sub:query_polygon:inclusive}{inclusive}%
]}
\end{f90format}

\begin{arguments}
{
\begin{tabular}{p{0.25\hsize} p{0.05\hsize} p{0.1\hsize} p{0.5\hsize}} \hline  
\textbf{name~\&~dimensionality} & \textbf{kind} & \textbf{in/out} & \textbf{description} \\ \hline
                   &   &   &                           \\ %%% for presentation
nside\mytarget{sub:query_polygon:nside} & I4B & IN & the $\nside$ parameter of the map. \\
vlist\mytarget{sub:query_polygon:vlist}(1:3,0:*) & DP & IN & cartesian vector pointing at polygon
                   respective vertices. \\
nv\mytarget{sub:query_polygon:nv} & I4B & IN & number of vertices, should be equal to 3 or larger. \\
listpix\mytarget{sub:query_polygon:listpix}(0:*) & I4B/ I8B & OUT & the index for all pixels enclosed in the triangle. Make sure that the size of the array is big enough to contain all pixels. \\ 
nlist\mytarget{sub:query_polygon:nlist} & I4B/ I8B & OUT & The number of pixels listed in {\tt listpix}. \\
nest\mytarget{sub:query_polygon:nest}\ \ (OPTIONAL) & I4B & IN &  The pixel indices are in the NESTED numbering scheme if nest=1, and in RING scheme otherwise. \\
inclusive\mytarget{sub:query_polygon:inclusive}\ \ (OPTIONAL) & I4B & IN & If set to 1, all the pixels overlapping
                   (even partially)
                   with the polygon are listed, otherwise only those whose
                   center lies within the polygon are listed. \\
\end{tabular}
}
\end{arguments}

\begin{example}
{
use \htmlref{healpix\_modules}{sub:healpix_modules} \\
real(dp), dimension(1:3,0:9) :: vertices \\
vertices(:,0) = (/0.,0.,1./)  ! +z \\
vertices(:,1) = (/1.,0.,0./)  ! +x \\
vertices(:,2) = (/1.,1.,-1./) ! x+y-z \\
vertices(:,3) = (/0.,1.,0./)  ! +y \\
 \\
call query\_polygon(256,vertices,4,listpix,nlist,nest=0)  \\
}
{
Returns the RING pixel index of all pixels in the polygon with vertices of
cartesian coordinates (0,0,1), (1,0,0), (1,1,-1) and (0,1,0) in a $\nside=256$ map.
}
\end{example}
%\newpage
\begin{modules}
  \begin{sulist}{} %%%% NOTE the ``extra'' brace here %%%%
 \item[isort] routine to sort integer number
 \item[\htmlref{query\_triangle}{sub:query_triangle}] render the list of pixels enclosed
  in a given triangle
 \item[\htmlref{surface\_triangle}{sub:surface_triangle}] computes the surface of a spherical triangle defined by 3 vertices
 \item[\htmlref{vect\_prod}{sub:vect_prod}] routine to compute the vectorial product of two 3D vectors
  \end{sulist}
\end{modules}

\begin{related}
  \begin{sulist}{} %%%% NOTE the ``extra'' brace here %%%%
  \item[\htmlref{pix2ang}{sub:pix_tools}, \htmlref{ang2pix}{sub:pix_tools}] convert between angle and pixel number.
  \item[\htmlref{pix2vec}{sub:pix_tools}, \htmlref{vec2pix}{sub:pix_tools}] convert between a cartesian vector and pixel number.
  \item[\htmlref{query\_disc}{sub:query_disc}, query\_polygon,]
  \item[\htmlref{query\_strip}{sub:query_strip}, \htmlref{query\_triangle}{sub:query_triangle}] render the list of pixels enclosed
  respectively in a given disc, polygon, latitude strip and triangle
  \item[\htmlref{surface\_triangle}{sub:surface_triangle}] computes the surface
in steradians of a spherical triangle defined by 3 vertices

  \end{sulist}
\end{related}

\rule{\hsize}{2mm}

\newpage
