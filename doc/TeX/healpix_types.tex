
\sloppy


\title{\healpix Fortran Subroutines Overview}
\docid{healpix\_types} \section[healpix\_types module]{ }
\label{sub:healpix_types}
\docrv{Version 2.0}
\author{Eric Hivon}
\abstract{This document describes the \healpix Fortran90 module HEALPIX\_TYPES.}

\begin{facility}
{This module defines a set of parameters used by most other
\healpix modules.}
{\modHealpixTypes}
\end{facility}


%---------------------
\newenvironment{mytable}[1]{%
\begin{minipage}[b]{\linewidth}{%
\renewcommand{\thefootnote}{\fnsymbol{footnote}}
\renewcommand{\footnoterule}{}
{#1}
}%
\end{minipage}
}
%---------------------

The parameters defined in \thedocid\ include

\begin{itemize}
\item
'kind' parameters, used when defining the type of a variable,

\begin{mytable}{%
\begin{tabularx}{\linewidth}{lcc X}
name & type & value\footnote{actual value may depend on hardware or compiler} & definition \\
\hline
I1B & integer & 1 & number of bytes in the hardware-supported signed integers covering the range -99 to
99 with the least margin\\
I2B & integer & 2 & same as above for the range -9999 to 9999 (ie, 4 digits)\\
I4B & integer & 4 & same as above for 9 digits \\
I8B & integer & 8 & same as above for 16 digits\footnote{may not be supported by
  some hardware or compiler; on those systems, the user should set the
preprocessing variable {\tt NO64BITS} to 1 during compilation to demote
automatically {\tt I8B} to {\tt I4B}} \\
SP & integer & 4 & number of bytes in the hardware-supported floating-point
numbers covering the range $10^{-30}$ to $10^{30}$ with the least margin
(hereafter single precision)\\
DP & integer & 8 & same as above for the range $10^{-200}$ to $10^{200}$
(double precision)\\
SPC & integer & 4 & number of bytes in real ({\em or} imaginary) part of single precision complex numbers\\
DPC & integer & 8 & same as above for double precision complex numbers\\
LGT & integer & 4 & number of bytes in logical variables \\
\hline
\end{tabularx}
}%
\end{mytable}


\item
largest accessible numbers,

\begin{mytable}{%
\begin{tabularx}{\linewidth}{lcc X}
name & type or kind & value\footnote{actual value may depend on hardware or compiler} & definition \\
\hline
MAX\_I1B & integer & 127 & largest number accessible to integers of kind {\tt I1B}\\
MAX\_I2B & integer & $32767$ &
same as above for {\tt I2B} integers\\
MAX\_I4B & integer & $2^{31}-1 \simeq 2.1\ 10^9$& same as above for {\tt I4B} integers \\
MAX\_I8B & I8B & $2^{63}-1 \simeq 9.2\ 10^{18}$& same as above for {\tt I4B} integers \\
MAX\_SP & SP & $\simeq 3.40\ 10^{38}$ & same as above for {\tt SP} floating-point\\
MAX\_DP & DP & $\simeq 1.80\ 10^{308}$ & same as above for {\tt DP} floating-point\\
\hline
\end{tabularx}
}
\end{mytable}

\item
mathematical definitions, \\
\begin{mytable}{%
\begin{tabularx}{\linewidth}{lcc X}
name & kind & value & definition \\
\hline
QUARTPI & DP & $\pi/4$ & \\
HALFPI & DP & $\pi/2$ & \\
PI & DP & $\pi$ & \\
TWOPI & DP & $2\pi$ & \\
FOURPI & DP & $4\pi$ & \\
SQRT2 & DP & $\sqrt{2}$ & \\
EULER & DP & $\gamma \simeq 0.577\ldots$ & Euler constant \\
SQ4PI\_INV & DP & $1/\sqrt{4\pi}$ & \\
TWOTHIRD & DP & $2/3$ & \\
DEG2RAD & DP & $\pi/180$ & Degrees to Radians conversion factor\\
RAD2DEG & DP & $180/\pi$ & Radians to Degrees conversion factor\\
\hline
\end{tabularx}
}
\end{mytable}

\item
and \healpix specific definitions, \\
\begin{mytable}{%
\begin{tabularx}{\linewidth}{lcc X}
name & type or kind & value & definition \\
\hline
HPX\_SBADVAL & SP & $-1.6375\ 10^{30}$ & default sentinel value given to missing
pixels in single precision data sets \\
HPX\_DBADVAL & DP & $-1.6375\ 10^{30}$ & same as above for double precision data
sets\\
FILENAMELEN & integer & 1024 & default length in character of file names. \\
%% HPX\_MXL0 & I4B & $40$ & parameter $m_0$ of $(l,m)$ range of spherical harmonics computed. \\
%% HPX\_MXL1 & DP & $1.35$ & parameter $r$ in equation above\\
\hline
\end{tabularx}
}
\end{mytable}

\end{itemize}

\begin{example}
{
use healpix\_types \\
real(kind=DP) :: dx \\
print*,' pi = ',PI
}
{
The value of {\tt PI}, as well as all other \thedocid\ parameters are made known
to the code
}
\end{example}


\rule{\hsize}{2mm}

\newpage
