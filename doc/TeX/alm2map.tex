
\sloppy


\docid{alm2map*}\section[alm2map*]{ }
\label{sub:alm2map}
\docrv{Version 2.1}
\author{Eric Hivon, Frode K.~Hansen}
\abstract{This document describes the \healpix Fortran90 subroutine ALM2MAP*.}

\begin{facility}
{This routine is a wrapper to 10 other routines: alm2map\_sc\_X,
  alm2map\_sc\_pre\_X, alm2map\_pol\_X, alm2map\_pol\_pre1\_X,
  alm2map\_pol\_pre2\_X, where X stands for either s or d. These routines
  synthesize a \healpix {\em RING ordered} temperature map (and if specified, polarisation maps) 
from input $a_{lm}^T$ (and if specified $a_{lm}^E$ and $a_{lm}^B$) values. 
The different routines are called dependent on what parameters are passed. 
Some routines synthesize maps with or without precomputed harmonics and some
  with or without polarisation. 
The routines accept both single and double precision arrays for alm\_TGC and
  map\_TQU. The precision of these arrays should match.}
{src/f90/mod/alm\_tools.F90}
\end{facility}

\begin{f90format}
{nsmax, nlmax, nmmax, alm\_TGC, map\_TQU [, plm]}
\end{f90format}

\begin{arguments}
{
\begin{tabular}{p{0.4\hsize} p{0.05\hsize} p{0.1\hsize} p{0.35\hsize}} \hline  
\textbf{name~\&~dimensionality} & \textbf{kind} & \textbf{in/out} & \textbf{description} \\ \hline
                   &   &   &                           \\ %%% for presentation
nsmax & I4B & IN & the $N_{side}$ value of the map to synthesize. \\
nlmax & I4B & IN & the maximum $\ell$ value used for the $a_{lm}$. \\
nmmax & I4B & IN & the maximum $m$ value used for the $a_{lm}$. \\
alm\_TGC(1:p, 0:nlmax, 0:nmmax) & SPC or DPC & IN & The $a_{lm}$ values to make
                   the map from. $p$ is 3 or 1 depending on wether polarisation is
                   respectively included or not. In the former case, the first
                   index runs from 1 to 3 corresponding to (T,E,B). \\
\end{tabular}

\begin{tabular}{p{0.4\hsize} p{0.05\hsize} p{0.1\hsize} p{0.35\hsize}}  \hline  
map\_TQU(0:12*nsmax**2-1)      & SP or DP & OUT & if only a temperature map is
to be synthesized, the map-array should be passed with this rank.  \\ 
map\_TQU(0:12*nsmax**2-1, 1:3) & SP or DP & OUT & if both temperature an 
polarisation maps are to be synthesized, the map array should have this rank, 
where the second index is (1,2,3) corresponding to (T,Q,U). \\ 
plm(0:n\_plm-1), \hskip 3cm OPTIONAL & DP & IN & If this optional matrix is passed with
this rank, precomputed $P_{lm}(\theta)$ are used instead of recursion. (
n\_plm = nsmax*(nmmax+1)*(2*nlmax-nmmax+2)
\\ 
plm(0:n\_plm-1,1:3), \hskip 3cm OPTIONAL & DP & IN & If this optional matrix is 
passed with this rank, precomputed $P_{lm}(\theta)$ AND precomputed tensor
harmonics are used instead of recursion. (n\_plm =
nsmax*(nmmax+1)*(2*nlmax-nmmax+2) 
\end{tabular}
}
\end{arguments}

\begin{example}
{
use healpix\_types \\
use pix\_tools, only : nside2npix \\
use alm\_tools, only : alm2map \\
integer(I4B) :: nside, lmax, mmax, npix, n\_plm\\
real(SP), dimension(:,:), allocatable :: map \\
complex(SPC), dimension(:,:,:), allocatable :: alm \\
real(DP), dimension(:,:), allocatable :: plm \\
\ldots \\
nside=256 ; lmax=512 ; mmax=lmax\\
npix=nside2npix(nside)\\
n\_plm=nside*(mmax+1)*(2*lmax-mmax+2)\\
allocate(alm(1:3,0:lmax,0:mmax))\\
allocate(map(0:npix-1,1:3))\\
allocate(plm(0:n\_plm-1,1:3))\\
\ldots \\
call alm2map(nside, lmax, mmax, alm, map, plm)  \\
}
{
Make temperature and polarisation maps from the scalar and tensor $a_{lm}$
passed in alm. The maps have $N_{side}$ of 256, and are constructed from
$a_{lm}$ values up to 512 in $\ell$ and $m$. Since the optional plm array is
passed with both precomputed $P_{lm}(\theta)$ AND tensor harmonics, there will
be no recursions in the routine and execution will be faster. 
}
\end{example}

\begin{modules}
  \begin{sulist}{} %%%% NOTE the ``extra'' brace here %%%%
  \item[\htmlref{ring\_synthesis}{sub:ring_synthesis}] Performs FFT over $m$ for synthesis of the rings.
  \item[compute\_lam\_mm, get\_pixel\_layout, ]
  \item[gen\_lamfac,gen\_mfac, gen\_normpol, ] 
  \item[gen\_recfac, init\_rescale, l\_min\_ylm] Ancillary routines used
  for $Y_{\ell m}$ recursion
  \item[\textbf{misc\_utils}] module, containing:
  \item[assert\_alloc] routine to print error message, when an array can not be
  allocated properly
  \end{sulist}
\end{modules}

\begin{related}
  \begin{sulist}{} %%%% NOTE the ``extra'' brace here %%%%
   \item[\htmlref{alm2map\_der}{sub:alm2map_der}] routine generating a map and
   its derivatives from its $a_{\ell m}$
   \item[\htmlref{alm2map\_spin}{sub:alm2map_spin}] routine generating maps of
arbitrary spin from their  ${_s}a_{\ell m}$
  \item[smoothing] executable using \thedocid\ to smooth maps
  \item[synfast] executable using \thedocid\ to synthesize maps.
  \item[\htmlref{map2alm}{sub:map2alm}] routine performing the inverse transform
  of \thedocid.
  \item[\htmlref{create\_alm}{sub:create_alm}] routine to generate randomly
  distributed $a_{\ell m}$ coefficients according to a given power spectrum
  \item[\htmlref{pixel\_window}{sub:pixel_window},
\htmlref{generate\_beam}{sub:generate_beam}] return the $l$-space \healpix-pixel and beam window function respectively
  \item[\htmlref{alter\_alm}{sub:alter_alm}] modifies $a_{lm}$ to emulate effect
of real space filtering
  \end{sulist}
\end{related}

\rule{\hsize}{2mm}

\newpage
