
\sloppy

\title{\healpix Fortran Subroutines Overview}
\docid{convert\_inplace*} \section[convert\_inplace*]{ }
\label{sub:convert_inplace}
\docrv{Version 2.0}
\author{Eric Hivon}
\abstract{This document describes the \healpix Fortran90 subroutine CONVERT\_INPLACE.}


\begin{facility}
{Routine to convert a \healpix map from NESTED to RING scheme or vice
  versa. The conversion is done in place, meaning that it doesn't require memory
  for a temporary map, like the
  $\htmlref{convert\_nest2ring}{sub:convert_nest2ring}$ or
  $\htmlref{convert\_ring2nest}{sub:convert_ring2nest}$
  routines. But for that reason, this routine is slower and not parallelized. The routine is a
  wrapper for 6 different routines and can threfore process
  integer, single precision and double precision maps as well as mono or bi
  dimensional arrays.}
{\modPixTools}
\end{facility}

\begin{f90format}
{\mylink{sub:convert_inplace:subcall}{subcall}%
, \mylink{sub:convert_inplace:map}{map}%
}
\end{f90format}

\begin{arguments}
{
\begin{tabular}{p{0.4\hsize} p{0.05\hsize} p{0.1\hsize} p{0.35\hsize}} \hline  
\textbf{name~\&~dimensionality} & \textbf{kind} & \textbf{in/out} & \textbf{description} \\ \hline
                   &   &   &                           \\ %%% for presentation
subcall\mytarget{sub:convert_inplace:subcall} & --- & IN & routine to be called by convert\_inplace\_real. Set this to \htmlref{ring2nest}{sub:pix_tools} or \htmlref{nest2ring}{sub:pix_tools} dependent on wether the conversion is RING to NESTED or vice versa. \\
map\mytarget{sub:convert_inplace:map}(0:npix-1) & I4B/ SP/ DP & INOUT & mono-dimensional full sky map to be converted, the routine finds the size itself. \\
map(0:npix-1,1:nd) & I4B/ SP/ DP & INOUT & bi-dimensional (nd$>0$) full sky map to be
                   converted, the routine finds both dimensions
                   itself. Processing a bidimensional map with nd$>1$ should be
                   faster than each of the nd 1D-maps consecutively.\\

\end{tabular}
}
\end{arguments}

\begin{example}
{
call convert\_inplace(\htmlref{ring2nest}{sub:pix_tools},map)  \\
}
{
Converts an map from RING to NESTED scheme.
}
\end{example}
%%\newpage
\begin{modules}
  \begin{sulist}{} %%%% NOTE the ``extra'' brace here %%%%
 \item[\htmlref{nest2ring}{sub:pix_tools}] routine to convert a NESTED pixel index to RING pixel number.
 \item[\htmlref{ring2nest}{sub:pix_tools}] routine to convert a RING pixel index to NESTED pixel number.	
  \end{sulist}
\end{modules}

\begin{related}
  \begin{sulist}{} %%%% NOTE the ``extra'' brace here %%%%
  \item[\htmlref{convert\_nest2ring}{sub:convert_nest2ring}] convert from NESTED to RING scheme using a temporary array. Requires more space then convert\_inplace, but is faster.
  \item[\htmlref{convert\_ring2nest}{sub:convert_ring2nest}]
  convert from RING to NESTED scheme using a temporary array. Requires more space then convert\_inplace, but is faster.
  \end{sulist}
\end{related}

\rule{\hsize}{2mm}

\newpage
