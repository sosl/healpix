
\sloppy


\title{\healpix Fortran Subroutines Overview}
\docid{udgrade\_nest*} \section[udgrade\_nest*]{ }
\label{sub:udgrade_nest}
\docrv{Version 2.0}
\author{Eric Hivon}
\abstract{This document describes the \healpix Fortran90 subroutine UDGRADE\_NEST.}


\begin{facility}
{Routine to degrade or prograde the pixel size of a \healpix map indexed with
  the NESTED scheme. The degradation/progradation is done assuming an
intensive quantity (like temperature) that does NOT scale with surface area. \newline
In case of degradation, a big pixel that contains one or several bad pixels will
take the average of the valid small pixels, unless a 'pessimistic' behavior
is assumed in which case the big pixel will take the bad pixel sentinel value.
In case of progradation, a bad pixel only spawns bad pixels. \newline
The routine accepts both mono and bi-dimensional maps.
}
{\modUdgradeNr}
\end{facility}

\begin{f90format}
{map\_in, nside\_in, map\_out, nside\_out \optional{[, fmissval, pessimistic]}}
\end{f90format}
\aboutoptional

\begin{arguments}
{
\begin{tabular}{p{0.3\hsize} p{0.05\hsize} p{0.1\hsize} p{0.45\hsize}} \hline  
\textbf{name~\&~dimensionality} & \textbf{kind} & \textbf{in/out} & \textbf{description} \\ \hline
                   &   &   &                           \\ %%% for presentation
map\_in(0:12*nside\_in**2-1) & SP/ DP & IN & mono-dimensional full sky map to be
                   prograded or degraded. \\
map\_in(0:12*nside\_in**2-1,1:nd) & SP/ DP & IN & bi-dimensional full sky map to be
                   prograded or degraded. The routine finds the second
                   dimension (nd) by itself. \\
nside\_in & I4B & IN & the $\nside$ resolution parameter of the input
                   map. Must be a power of 2.\\
map\_out(0:12*nside\_out**2-1) & SP/ DP & OUT & mono-dimensional full sky map after
                   degradation or progradation. \\
map\_out(0:12*nside\_out**2-1,1:nd) & SP/ DP & OUT & bi-dimensional full sky map after
                   degradation or progradation. The second dimension
                   (nd) should match that of the input map.\\
nside\_out & I4B & IN & the $\nside$ resolution parameter of the output
                   map. Must be a power of 2. If nside\_out $>$ nside\_in, the
                   map is prograded (ie, more and smaller pixels) with each
                   pixel having the same value as its parent; otherwise, the
                   map in degraded (ie, fewer larger pixels), with each pixel
                   being the average of its $($nside\_in/nside\_out$)^2$ components.\\
\optional{fmissval}  & SP/ DP & IN & sentinel value given to bad pixels in input and output
                   maps.%
\default{\htmlref{\tt HPX\_SBADVAL}{sub:healpix_types} or %
\htmlref{\tt HPX\_DBADVAL}{sub:healpix_types}%
} \\
\optional{pessimistic}   & LGT & IN & if set to {\tt .true.}, during a degradation, a big pixel containing at least a small
                   bad pixel will be returned as bad as well, instead of taking
                   the average of the remaing valid pixels. \default{\tt .false.}
\end{tabular}
}
\end{arguments}

\begin{example}
{
use udgrade\_nr \\
call udgrade\_nest(map\_hi, 256, map\_low, 64)  \\
}
{
Degrades a NESTED ordered map with $\nside=256$ into a NESTED map with $\nside=64$
}
\end{example}

%% \begin{modules}
%%   \begin{sulist}{} %%%% NOTE the ``extra'' brace here %%%%
%%  \item[\htmlref{ring2nest}{sub:pix_tools}] routine to udgrade a RING pixel index to NESTED pixel number.
%%   \end{sulist}
%% \end{modules}
%%%%\newpage
\begin{related}
  \begin{sulist}{} %%%% NOTE the ``extra'' brace here %%%%
  \item[\htmlref{udgrade\_ring}{sub:udgrade_ring}] prograde or degrade a RING
  ordered map.
%%   \item[\htmlref{udgrade\_inplace}{sub:udgrade_inplace}] udgrade between 
%%     RING and NESTED schemes inplace. This routine is slower than \thedocid, but doesn't require as much memory.
  \end{sulist}
\end{related}

\rule{\hsize}{2mm}

\newpage
