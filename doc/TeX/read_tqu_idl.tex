% -*- LaTeX -*-

% PLEASE USE THIS FILE AS A TEMPLATE FOR THE DOCUMENTATION OF YOUR OWN
% FACILITIES: IN PARTICULAR, IT IS IMPORTANT TO NOTE COMMENTS MADE IN
% THE TEXT AND TO FOLLOW THIS ORDERING. THE FORMAT FOLLOWS ONE USED BY
% THE COBE-DMR PROJECT.	
% A.J. Banday, April 1999.

\renewcommand{\facname}{{read\_tqu }}
\renewcommand{\FACNAME}{{READ\_TQU }}
\sloppy

\title{\healpix IDL Facility User Guidelines}
\docid{read\_tqu} \section[read\_tqu]{ }
\label{idl:read_tqu}
\docrv{Version 1.2}
\author{Eric Hivon}
\abstract{This document describes the \healpix facility \thedocid.}

\begin{facility}
{This IDL facility reads a temperature+polarization Healpix map
(T,Q,U) from a binary table FITS file, 
with optionally the error (dT,dQ,dU) and correlation (dQU, dTU, dTQ)
from separate extensions
}
{src/idl/fits/read\_tqu.pro}
\end{facility}

\begin{IDLformat}
{\FACNAME, \mylink{idl:read_tqu:File}{File}%
, \mylink{idl:read_tqu:TQU}{TQU}%
, [\mylink{idl:read_tqu:Extension}{Extension=}%
, \mylink{idl:read_tqu:Hdr}{Hdr=}%
, \mylink{idl:read_tqu:Xhdr}{Xhdr=}%
, \mylink{idl:read_tqu:HELP}{/HELP}%
, \mylink{idl:read_tqu:Nside}{Nside=}%
, \mylink{idl:read_tqu:Ordering}{Ordering=}%
, \mylink{idl:read_tqu:Coordsys}{Coordsys=}%
]}
\end{IDLformat}

\begin{qualifiers}
  \begin{qulist}{} %%%% NOTE the ``extra'' brace here %%%%
 	\item[{File}] \mytarget{idl:read_tqu:File}
          name of a FITS file from which the maps are to be read

   \item[{TQU}]\mytarget{idl:read_tqu:TQU} : 
array of Healpix maps of size ($\npix$,3,{\tt n\_ext}) where $\npix$ is the total
   number of Healpix pixels on the sky, and {\tt n\_ext} $\le$ 3 is
   the number of extensions read\\
     Three maps are available in each extension of the FITS file : \\
      -the temperature+polarization Stokes parameters maps (T,Q,U) in
   extension 0 \\
      -the error maps (dT,dQ,dU) in extension 1 (if applicable)\\
      -the correlation maps (dQU, dTU, dTQ) in extension 2 (if applicable)

       \item[{Extension=}] \mytarget{idl:read_tqu:Extension}
		(optional), \\
	extension unit to be read from FITS file: 
 either its 0-based ID number (ie, 0 for first extension {\em after} primary array) 
 or the case-insensitive value of its EXTNAME keyword.
	If absent, all available extensions are read.

       \item[{Hdr=}] \mytarget{idl:read_tqu:Hdr}
		  (optional), \\
		string variable containing on output  the contents of the primary header. (If already present, FITS reserved
		  keywords will be automatically updated).

       \item[{Xhdr=}] \mytarget{idl:read_tqu:Xhdr}
		  (optional), \\
		string variable containing on output the contents of the
		  extension header. If 
                  several extensions are read, then the extension 
                  headers are returned appended into one string array.

	 \item[{Nside=}] \mytarget{idl:read_tqu:Nside}
		(optional), \\
	        returns on output the \healpix resolution parameter, as read
		from the FITS header. Set to -1 if not found

	 \item[{Ordering=}] \mytarget{idl:read_tqu:Ordering}
	        (optional), \\
	        returns on output the pixel ordering, as read from the FITS
	        header. Either 'RING' or 'NESTED' or ' ' (if not found).

	 \item[{Coordsys=}] \mytarget{idl:read_tqu:Coordsys}
	        (optional), \\
	        returns on output the astrophysical coordinate system used, 
		as read from FITS header (value of keywords COORDSYS or SKYCOORD)

  \end{qulist}
\end{qualifiers}

\begin{keywords}
  \begin{kwlist}{} %%% extra brace
	\item[{/HELP}]  \mytarget{idl:read_tqu:HELP} if set, an extensive help is displayed and no
	file is read
   \end{kwlist}
\end{keywords}

\begin{codedescription}
{\thedocid{} reads out Stokes parameters (T,Q,U) maps for the whole
sky into a FITS file. It is also possible to read the error per pixel for each
map and the correlation between fields, as subsequent extensions of the same FITS
file (see qualifiers above). Therefore the file may have up to three extensions with three
maps in each. Extensions can be written together or one by one (in
their physical order) using the Extension option.\\
For more information on the FITS file format supported in \healpix, 
including the one implemented in \facname,
see \url{\hpxfitsdoc}.}
\end{codedescription}



\begin{related}
  \begin{sulist}{} %%%% NOTE the ``extra'' brace here %%%%
  \item[idl] version \idlversion or more is necessary to run \thedocid
  \item[synfast] This \healpix f90 facility can be used to generate
  temperature+polarization maps that can be read with \thedocid
  \item[\htmlref{write\_tqu}{idl:write_tqu}] This \healpix IDL facility can be used to write
  out temperature+polarization that can be read by \thedocid.
  \item[%
\htmlref{read\_fits\_cut4}{idl:read_fits_cut4},
\htmlref{read\_fits\_partial}{idl:read_fits_partial},
\htmlref{read\_fits\_map}{idl:read_fits_map}]
  \item[%
\htmlref{read\_tqu}{idl:read_tqu},
\htmlref{read\_fits\_s}{idl:read_fits_s}]
\healpix IDL routines to read cut-sky maps and partial maps, full-sky maps, polarized full-sky maps and
arbitrary data sets from FITS files

  \item[\htmlref{read\_fits\_s}{idl:read_fits_s}] This general purpose \healpix IDL facility can be used to read
  into an IDL structure maps contained in binary table FITS files.
  \item[sxpar] This IDL routine (included in \healpix package) can be
  used to extract FITS keywords from the header(s) HDR or XHDR read with \thedocid.
  \end{sulist}
\end{related}


\begin{example}
{
\begin{tabular}{l} %%%% use this tabular format %%%%
read\_tqu, 'map\_polarization.fits', TQU, xhdr=xhdr\\
\end{tabular}
}
{
Reads into {\tt TQU} the polarization maps contained in the FITS file
'map\_polarization.fits'.
The variable {\tt xhdr} will contain the extension(s) header.
}
\end{example}

