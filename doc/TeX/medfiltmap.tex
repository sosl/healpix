
\sloppy


\title{\healpix Fortran Subroutines Overview}
\docid{medfiltmap*} \section[medfiltmap*]{ }
\label{sub:medfiltmap}
\docrv{Version 2.1}
\author{Eric Hivon}
\abstract{This document describes the \healpix Fortran90 subroutine MEDFILTMAP*.}

\begin{facility}
{This routine performs the median filtering of a \healpix full sky map for a
  given neighborhood radius }
{\modPixTools}
\end{facility}

\begin{f90format}
{\mylink{sub:medfiltmap:in_map}{in\_map}%
, \mylink{sub:medfiltmap:radius}{radius}%
, \mylink{sub:medfiltmap:med_map}{med\_map}%
 \hfill [,~\mylink{sub:medfiltmap:nest}{nest}%
, \mylink{sub:medfiltmap:fmissval}{fmissval}%
, \mylink{sub:medfiltmap:fill_holes}{fill\_holes}%
]}
\end{f90format}

\begin{arguments}
{
\begin{tabular}{p{0.30\hsize} p{0.05\hsize} p{0.05\hsize} p{0.50\hsize}} \hline  
\textbf{name~\&~dimensionality} & \textbf{kind} & \textbf{in/out} & \textbf{description} \\ \hline
                   &   &   &                           \\ %%% for presentation
in\_map\mytarget{sub:medfiltmap:in_map}(0:npix-1) & SP/ DP & IN & Full sky \healpix map to filter. {\tt npix}
                   should be valid \healpix pixel number. \\
radius\mytarget{sub:medfiltmap:radius} & DP & IN & Radius in RADIANS of the disk on which the median is
                   computed. \\
med\_map\mytarget{sub:medfiltmap:med_map}(0:npix-1) & SP/ DP & OUT & Median filtered map: each pixel is the
                   median of the input map valid neighboring pixels contained
                   within a distance {\tt radius} \\
nest\mytarget{sub:medfiltmap:nest} \hskip 1cm OPTIONAL & I4B & IN & set to 1 if the map ordering is NESTED, set to 0 if
                   it is RING. \\
fmissval\mytarget{sub:medfiltmap:fmissval} \hskip 1cm OPTIONAL & SP/ DP & IN & sentinel value given to missing or
                   non-valid pixels. Default: {\tt HPX\_SBADVAL} or {\tt
                   HPX\_DBADVAL} $ = -1.6375\ 10^{30}$ \\
fill\_holes\mytarget{sub:medfiltmap:fill_holes} \hskip 1cm OPTIONAL & LGT & IN & if set to {\tt .true.} will replace
                   non-valid pixels by median of neighbors; if set to {\tt .false.}
                   will leave non-valid pixels unchanged. Default: {\tt .false.}
\end{tabular}
}
\end{arguments}
%%\newpage

\begin{example}
{
use healpix\_types \\
use pix\_tools \\
... \\
call medfiltmap(map, 0.5*DEG2RAD, med)  \\
}
{
Output in {\tt med} the median filter of {\tt map}, using a filter radius of 0.5 Deg
}
\end{example}

\begin{modules}
  \begin{sulist}{} %%%% NOTE the ``extra'' brace here %%%%
  \item[\textbf{statistics}] module, containing:
  \item[\htmlref{median}{sub:median}] routine to compute the median of a data set		
  \item[\textbf{pix\_tools}] module, containing:
  \item[\htmlref{pix2vec\_ring, pix2vec\_nest}{sub:pix_tools}] routines to find
  the location of a pixel on the sky
  \item[\htmlref{query\_disc}{sub:query_disc}] routine to find pixels lying
  within a radius of a given point
  \end{sulist}
\end{modules}

%% \begin{related}
%%   \begin{sulist}{} %%%% NOTE the ``extra'' brace here %%%%
%%   \item[anafast] executable using \thedocid to analyse maps.
%%   \item[\htmlref{alm2map}{sub:alm2map}] routine performing the inverse transform of \thedocid.
%%   \end{sulist}
%% \end{related}

\rule{\hsize}{2mm}

\newpage
