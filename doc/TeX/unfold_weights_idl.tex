% -*- LaTeX -*-


\renewcommand{\facname}{{unfold\_weights }}
\renewcommand{\FACNAME}{{UNFOLD\_WEIGHTS }}

\sloppy



\title{\healpix IDL Facility User Guidelines}
\docid{unfold\_weights} \section[unfold\_weights]{ }
\label{idl:unfold_weights}
\docrv{Version 1.0}
\author{Eric Hivon}
\abstract{This document describes the \healpix IDL facility \facname.}


\begin{facility}
{This IDL function returns the full sky map of the weights to be applied to a \healpix map in order to improve the quadrature.
The input weights can be either ring-based or pixel-based, and read from file with user provided path, or from files with standardized name and location (ie, \texttt{!healpix.path.data+'weight\_ring\_n?????.fits'} and 
\texttt{!healpix.path.data+'weight\_pixel\_n?????.fits'}}
{src/idl/toolkit/unfold\_weights.pro}
\end{facility}

\begin{IDLformats}
{\mylink{idl:unfold_weights:weight_map}{weight\_map} = %
\facname(%
\mylink{idl:unfold_weights:nside}{File}, 
[\mylink{idl:unfold_weights:dim}{Dim},
\mylink{idl:unfold_weights:help}{/HELP},
\mylink{idl:unfold_weights:silent}{/SILENT}])}%
%
{\mylink{idl:unfold_weights:weight_map}{weight\_map} = %
\facname(%
\mylink{idl:unfold_weights:nside}{Nside}, 
[\mylink{idl:unfold_weights:dim}{Dim},
\mylink{idl:unfold_weights:pixel}{/PIXEL},
\mylink{idl:unfold_weights:ring}{/RING},
\mylink{idl:unfold_weights:scheme}{SCHEME=},
\mylink{idl:unfold_weights:directory}{DIRECTORY=},
\mylink{idl:unfold_weights:help}{/HELP},
\mylink{idl:unfold_weights:silent}{/SILENT}])}%
%
\end{IDLformats}

\begin{qualifiers}
  \begin{qulist}{} %%%% NOTE the ``extra'' brace here %%%%
    \item[Nside\mytarget{idl:unfold_weights:nside}] \healpix resolution parameter (scalar integer),
         should be a valid Nside (power of 2 in $\{1,\ldots,2^{29}\}$)
    \item[File\mytarget{idl:unfold_weights:file}] Input weight file to be read. If not provided, 
the function will try to guess the relevant file path based on 
\mylink{idl:unfold_weights:nside}{Nside}, 
the optional \mylink{idl:unfold_weights:directory}{DIRECTORY}, 
and the weighting scheme which \textbf{must} be set, with either
\mylink{idl:unfold_weights:ring}{RING}, 
\mylink{idl:unfold_weights:pixel}{PIXEL} or 
\mylink{idl:unfold_weights:scheme}{SCHEME}
    \item[Dim\mytarget{idl:unfold_weights:dim}] dimension of output, either 1 or 2. \default{1}
    \item[weight\_map\mytarget{idl:unfold_weights:weight_map}] output: vector of size $\npix=12\nside^2$ if Dim=1, array of size $(\npix,3)$ if Dim=2 (in the latter case, all three columns are identical).
  \end{qulist}
\end{qualifiers}

\begin{keywords}
  \begin{kwlist}{} %%% extra brace
    \item[DIRECTORY=\mytarget{idl:unfold_weights:directory}] 
    directory in which to look for the weight file \default{\htmlref{!healpix.path.data}{idl:init_healpix}}
    \item[/HELP\mytarget{idl:unfold_weights:help}] if set on input, the documentation header 
  is printed out and the function exits
    \item[/PIXEL\mytarget{idl:unfold_weights:pixel}] if set, the code will look for the pixel-based weight file corresponding the the \mylink{idl:unfold_weights:nside}{Nside} provided, in the default or provided 
   \mylink{idl:unfold_weights:directory}{Directory}
    \item[/RING\mytarget{idl:unfold_weights:ring}] if set, the code will look for the ring-based weight file corresponding the the \mylink{idl:unfold_weights:nside}{Nside} provided, in the default or provided
   \mylink{idl:unfold_weights:directory}{Directory}
    \item[SCHEME=\mytarget{idl:unfold_weights:scheme}] can be either 'PIXEL' or 'RING', setting the type of weight file the code will look for.
    \item[/SILENT\mytarget{idl:unfold_weights:silent}] if set on input, the function works silently
  \end{kwlist}
\end{keywords}  

\begin{codedescription}
{\facname reads a list of weights, stored in a compact form in a FITS file, and centered on 0,
either ring-based (uniform weights on each iso-latitude rings, defined on $2\nside$ rings),
or pixel-based (defined on $N_w \simeq 0.75 \nside^2 \simeq \npix/16$) and turns them into a 
full sky \healpix map of quadrature weights, with RING indexing and with values centered on 1.}
\end{codedescription}



\begin{related}
  \begin{sulist}{} %%%% NOTE the ``extra'' brace here %%%%
    \item[idl] version \idlversion or more is necessary to run \facname.	
  \item[\htmlref{nside2npweights}{idl:nside2npweights}] 
 returns the number of non-redundant pixel-based weights used for disc storage
  \end{sulist}
\end{related}

\begin{example}
{
\begin{tabular}{l} %%%% use this tabular format %%%%
\htmlref{mollview}{idl:mollview}, \mylink{idl:mollview:hist_equal}{/hist},  \\
\hspace{2em} 	unfold\_weights(256, \mylink{idl:unfold_weights:ring}{/ring}), 
 title='Ring-based weights @ Nside=256'\\
\htmlref{mollview}{idl:mollview}, \mylink{idl:mollview:hist_equal}{/hist},  \\
\hspace{2em} 	unfold\_weights(256, \mylink{idl:unfold_weights:pixel}{/pixel}), 
 title='Pixel-based weights @ Nside=256' \\
\end{tabular}
}
{will plot the full sky map of the ring-based and pixel-based quadrature weights for $\nside=256$.
}
\end{example}

