% -*- LaTeX -*-

% PLEASE USE THIS FILE AS A TEMPLATE FOR THE DOCUMENTATION OF YOUR OWN
% FACILITIES: IN PARTICULAR, IT IS IMPORTANT TO NOTE COMMENTS MADE IN
% THE TEXT AND TO FOLLOW THIS ORDERING. THE FORMAT FOLLOWS ONE USED BY
% THE COBE-DMR PROJECT.	
% A.J. Banday, April 1999.

\documentclass[12pt,twoside]{article}
%%\usepackage{healrut}
\usepackage{healpix,html,makeidx}
\usepackage{ae,lmodern}% load vectorial font, keep PDF small *and* good quality when in T1 font
\usepackage[T1]{fontenc}% underscore searchable in PDF, but larger PDF http://latex-community.org/forum/viewtopic.php?t=8891
\begin{htmlonly}
\renewcommand{\ell}{l}
\renewcommand{\lq}{`}
% -*- LaTeX -*-
% This LaTeX file sets the Healpix version
% as it will appear in the documentation
% implement it with: % -*- LaTeX -*-
% This LaTeX file sets the Healpix version
% as it will appear in the documentation
% implement it with: % -*- LaTeX -*-
% This LaTeX file sets the Healpix version
% as it will appear in the documentation
% implement it with: \input{hpxversion}
% \newcommand{\hpxversion}{3.31}
% \newcommand{\hpxverstex}{3\_31}
% \newcommand{\hpxversion}{3.40}
% \newcommand{\hpxverstex}{3\_40}
% \newcommand{\hpxversion}{3.41}
% \newcommand{\hpxverstex}{3\_41}
\newcommand{\hpxversion}{3.50}
\newcommand{\hpxverstex}{3\_50}

% \newcommand{\hpxversion}{3.31}
% \newcommand{\hpxverstex}{3\_31}
% \newcommand{\hpxversion}{3.40}
% \newcommand{\hpxverstex}{3\_40}
% \newcommand{\hpxversion}{3.41}
% \newcommand{\hpxverstex}{3\_41}
\newcommand{\hpxversion}{3.50}
\newcommand{\hpxverstex}{3\_50}

% \newcommand{\hpxversion}{3.31}
% \newcommand{\hpxverstex}{3\_31}
% \newcommand{\hpxversion}{3.40}
% \newcommand{\hpxverstex}{3\_40}
% \newcommand{\hpxversion}{3.41}
% \newcommand{\hpxverstex}{3\_41}
\newcommand{\hpxversion}{3.50}
\newcommand{\hpxverstex}{3\_50}

\end{htmlonly}
\newcommand{\nside}{N_\mathrm{side}}
\newcommand{\npix}{N_\mathrm{pix}}
%\newcommand{\myhtmlimage}[1]{\htmlimage{#1}}
\newcommand{\myhtmlimage}[1]{ }

\hypersetup{%
	pdftitle={HEALPix C Subroutines Overview},%
	pdfauthor={E. Hivon et al},%
	pdfkeywords={HEALPix, C},%
	colorlinks=true}%
%\includeonly{}

\sloppy
\setcounter{secnumdepth}{0}
\begin{document}

\title{\healpix C Subroutines Overview}
\label{sub:subroutines}
\docrv{Version \hpxversion}
% \docrv{Version 1.20}
\author{Eric Hivon, Anthony J.~Banday, Matthias Bartelmann, Frode K.~Hansen,
Krzysztof M.~G\'orski, Martin Reinecke and Benjamin D.~Wandelt}
\abstract{This document is an overview of the \healpix C subroutines.}
% -*- LaTeX -*-
% This LaTeX file sets the Healpix website
% as it will appear in the documentation
% implement it with: % -*- LaTeX -*-
% This LaTeX file sets the Healpix website
% as it will appear in the documentation
% implement it with: % -*- LaTeX -*-
% This LaTeX file sets the Healpix website
% as it will appear in the documentation
% implement it with: \input{hpxwebsite}
%
% DEPRECATED !!!! replaced with hpxwebsites.tex
%

%
% DEPRECATED !!!! replaced with hpxwebsites.tex
%

%
% DEPRECATED !!!! replaced with hpxwebsites.tex
%

\date{\today}
\frontpage
%%   \pagestyle{myheadings}
%%   \markboth{left}{right}

\tableofcontents
\newpage
\section[Conventions]{{\Large Conventions}}
Here we list some conventions which are used in this document.
\\
\hrule
\begin{tabular}{@{}p{0.3\hsize}@{\hspace{1ex}}
                        p{0.7\hsize}@{}}  &  \\


$\mathbf{N_{side}}$ & \healpix resolution parameter --- see the
\healpix Primer.\\
\\
% $\mathbf{map}$ & We use the word ``map'' referring to a function,
% defined on the set of all \healpix pixels. %'' ``
\\
$\mathbf{\theta}$ & The polar angle or colatitude on the sphere,
ranging from 0 at the North Pole to $\pi$ at the South Pole.\\
\\
$\mathbf{\phi}$ & The azimuthal angle on the sphere, $\phi\in[0,2\pi[$.\\
\\
\end{tabular}

\section[Compilation and Installation]{{\Large Compilation and Installation}}
A tentative compilation and installation script is provided in {\tt
src/C/doinstall}.
If it does not work, you can try editing the {\tt
src/C/subs/Makefile} by hand.

\section[Usage]{{\Large Usage}}
To use in your 'C' code, include the line 
 
{\tt \#include "chealpix.h" } 
 
in your code and link with something like 
 
{\tt gcc -o myprog myprog.c -I<incdir> -L<libdir> -lchealpix}

where {\tt <incdir>} is where you've installed the '.h' files and 
{\tt <libdir>} is where you've installed the libraries (See the header of
the 'subs/Makefile').

You will also need the 'cfitsio' library. See \hfill \\
\htmladdnormallink{{\tt http://heasarc.gsfc.nasa.gov/docs/software/fitsio/}}
			{http://heasarc.gsfc.nasa.gov/docs/software/fitsio/}

\section[Note on the C routines]{{\Large Note on the C routines}}
This small set of C routines is provided as a start up kit to users wanting to
link the \healpix routines with some other languages (C, C++, IDL, perl, ...),
and it was actually mainly provided by various users (see individual routines
for details). As for the rest of the
\healpix package, all interested persons are welcome to contribute to this effort.

\newpage


\sloppy


\title{\healpix C Subroutines Overview}
\docid{ang2vec} \section[ang2vec]{ }
\label{csub:ang2vec}
\docrv{Version 2.0}
\author{E. Hivon}
\abstract{This document describes the \healpix C subroutine ANG2VEC.}

\begin{facility}
{Routine to convert the position angles $(\theta,\phi)\myhtmlimage{}$ of a point on the sphere 
into its 3D position vector $(x,y,z)$ with
$x = \sin\theta\cos\phi\myhtmlimage{}$, $y=\sin\theta\sin\phi\myhtmlimage{}$, $z=\cos\theta\myhtmlimage{}$. 
}
{src/C/subs/chealpix.c}
\end{facility}

\begin{Cfunction}
{void vec2ang(double theta, double phi, double *vector);}
\end{Cfunction}

\begin{arguments}
{
%% \begin{tabular}{p{0.4\hsize} p{0.05\hsize} p{0.1\hsize} p{0.35\hsize}} \hline  
\begin{tabular}{p{0.3\hsize} p{0.10\hsize} p{0.05\hsize} p{0.45\hsize}} \hline  
\textbf{name~\&~dimensionality} & \textbf{kind} & \textbf{in/out} & \textbf{description} \\ \hline
                   &   &   &                           \\ %%% for presentation
theta & double & IN & colatitude in radians measured southward from north pole (in
    [0,$\pi$]). \\
phi   & double & IN & longitude in radians measured eastward (in [0, $2\pi$]).\\
vector(3) & double & OUT & three dimensional cartesian position vector
                   $(x,y,z)$. The north pole is $(0,0,1)$\\
\end{tabular}
}
\end{arguments}


% \begin{example}
% {
% call ang2vec(theta,phi,vector) \\
% }
% {
% }
% \end{example}

\begin{related}
  \begin{sulist}{} %%%% NOTE the ``extra'' brace here %%%%
  \item[\htmlref{vec2ang}{csub:vec2ang}] converts the 3D position vector of point into its position
  angles on the sphere.
  \end{sulist}
\end{related}

\rule{\hsize}{2mm}



\sloppy


\title{\healpix Fortran Subroutines Overview}
\docid{get\_fits\_size} \section[get\_fits\_size]{ }
\label{csub:get_fits_size}
\docrv{Version 1.2}
\author{Eric Hivon}
\abstract{This document describes the \healpix C subroutine GET\_FITS\_SIZE.}

\begin{facility}
{This routine reads the number of pixels, the resolution parameter and the pixel ordering of a FITS file containing a \healpix map.}
{src/C/subs/get\_fits\_size.c}
\end{facility}

\begin{Cfunction}
{long \thedocid(char *filename, long *nside, char *ordering)}
\end{Cfunction}

\begin{arguments}
{
\begin{tabular}{p{0.3\hsize} p{0.05\hsize} p{0.05\hsize} p{0.5\hsize}} \hline  
\textbf{name\&dimensionality} & \textbf{kind} & \textbf{in/out} & \textbf{description} \\ \hline
                   &   &   &                           \\ %%% for presentation
\thedocid & long & OUT & number of pixels the FITS file \\
filename & char & IN & filename of the FITS-file containing the \healpix map. \\
ordering  & char & OUT & pixel ordering, either 'RING' or 'NESTED' \\
nside  & long & OUT & Healpix resolution parameter Nside
\end{tabular}
}
\end{arguments}

\begin{example}
{
long npix, nside ; \\
char file[180]=''map.fits'' ;\\
char order[10] ; \\
npix= get\_fits\_size(file, \&nside, order)  \\
}
{
Returns in npix the number of pixel in the file 'map.fits', and read in nside or
order its resolution parameter or ordering scheme}
\end{example}
% \newpage
% \begin{modules}
%   \begin{sulist}{} %%%% NOTE the ``extra'' brace here %%%%
%   \item[\textbf{fitstools}] module, containing:
%   \item[printerror] routine for printing FITS error messages.
%   \item[\textbf{cfitsio}] library for FITS file handling.		
%   \end{sulist}
% \end{modules}

\begin{related}
  \begin{sulist}{} %%%% NOTE the ``extra'' brace here %%%%
  \item[\htmlref{read\_healpix\_map}{csub:read_healpix_map}] subroutine to read \healpix maps
  \item[\htmlref{write\_healpix\_map}{csub:write_healpix_map}] subroutine to write \healpix maps
  \end{sulist}
\end{related}

\rule{\hsize}{2mm}

\newpage



\sloppy


\title{\healpix C Subroutines Overview}
\docid{npix2nside} \section[npix2nside]{ }
\label{csub:npix2nside}
\docrv{Version 1.0}
\author{E. Hivon}
\abstract{This document describes the \healpix C subroutine NPIX2NSIDE.}

\begin{facility}
{Function to provide the resolution parameter $\nside$ corresponding to the number of pixels $\npix$ over the full sky. 
}
{src/C/subs/chealpix.c}
\end{facility}

\begin{Cfunction}
{long npix2nside(const long npix)}
\end{Cfunction}

\begin{arguments}
{
\begin{tabular}{p{0.3\hsize} p{0.05\hsize} p{0.1\hsize} p{0.45\hsize}} \hline  
\textbf{name\&dimensionality} & \textbf{kind} & \textbf{in/out} & \textbf{description} \\ \hline
                   &   &   &                           \\ %%% for presentation
npix & long & IN & the number of pixels $\npix$ of the map . \\
\thedocid  & long & OUT & returns the $\nside$ parameter of the map such that $\npix=12\nside^2$.\\
\end{tabular}
}
\end{arguments}

\begin{example}
{
nside= npix2nside(786432);  \\
}
{
Returns the resolution parameter (256) corresponding to 786432 \healpix pixels.
}
\end{example}
% \begin{related}
%   \begin{sulist}{} %%%% NOTE the ``extra'' brace here %%%%
%   \item[\htmlref{npix2nside}{sub:npix2nside}] returns resolution parameter corresponding to the number of pixels.
% %   \item[pix2xxx] conversion between pixel index and position on the sky.
%   \end{sulist}
% \end{related}

\begin{related}
  \begin{sulist}{} %%%% NOTE the ``extra'' brace here %%%%
  \item[\htmlref{ang2vec}{csub:ang2vec}] converts $(\theta,\phi)$ spherical coordinates into $(x,y,z)$ cartesian coordinates.
  \item[\htmlref{vec2ang}{csub:vec2ang}] converts $(x,y,z)$ cartesian coordinates into $(\theta,\phi)$ spherical coordinates.
  \item[\htmlref{nside2npix}{csub:nside2npix}] converts number of full sky pixels $\npix$ into resolution parameter $\nside$
%  \item[\htmlref{npix2nside}{csub:npix2nside}] converts $\nside$ into number of
%full sky pixels $\npix$.
  \end{sulist}
\end{related}

\rule{\hsize}{2mm}




\sloppy


\title{\healpix C Subroutines Overview}
\docid{nside2npix} \section[nside2npix]{ }
\label{csub:nside2npix}
\docrv{Version 1.0}
\author{E. Hivon}
\abstract{This document describes the \healpix C subroutine NSIDE2NPIX.}

\begin{facility}
{Function to provide the number of pixels $\npix$ over the full sky corresponding
to resolution parameter $\nside$. 
}
{src/C/subs/nside2npix.c}
\end{facility}

\begin{Cfunction}
{long nside2npix(const long nside)}
\end{Cfunction}

\begin{arguments}
{
\begin{tabular}{p{0.3\hsize} p{0.05\hsize} p{0.1\hsize} p{0.45\hsize}} \hline  
\textbf{name\&dimensionality} & \textbf{kind} & \textbf{in/out} & \textbf{description} \\ \hline
                   &   &   &                           \\ %%% for presentation
nside & long & IN & the $N_{side}$ parameter of the map. \\
\thedocid  & long & OUT & returns the number of pixels $N_{pix}$ of the map $\npix=12\nside^2$.\\
\end{tabular}
}
\end{arguments}

\begin{example}
{
npix= nside2npix(256);  \\
}
{
Returns the pixel the number of \healpix pixels (786432) for the resolution
parameter 256.
}
\end{example}
% \begin{related}
%   \begin{sulist}{} %%%% NOTE the ``extra'' brace here %%%%
%   \item[\htmlref{npix2nside}{sub:npix2nside}] returns resolution parameter corresponding to the number of pixels.
% %   \item[pix2xxx] conversion between pixel index and position on the sky.
%   \end{sulist}
% \end{related}

\rule{\hsize}{2mm}




\sloppy

\title{\healpix C Subroutines Overview}
% \docid{pix2xxx,ang2xxx,vec2xxx, nest2ring,ring2nest} \section[pix2xxx,ang2xxx,vec2xxx, nest2ring,ring2nest]{ }
\docid{pix2xxx,~ang2xxx,~vec2xxx, nest2ring,~ring2nest} \section[pix2xxx,~ang2xxx,~vec2xxx,~nest2ring,~ring2nest]{ }
\label{csub:pix_tools}
\docrv{Version 1.1}
\author{Eric Hivon}
% \abstract{This document describes the \healpix C subroutines in the subdirectory
% pix\_tools.}

\begin{facility}
{These subroutines can be used to convert between pixel number in the
\healpix map and $(\theta,\phi)$ coordinates on the sphere. This is only a
subset of the routines equivalent in Fortran90 or in IDL. }
{src/C/subs/chealpix.c}
\end{facility}

% Note: These routines are based on the translation of the original F77 routines
% to C++ and then to C, 
% by Reza Ansari (ansari@lal.in2p3.fr), Alex Kim (akim@lilys.lbl.gov), Guy
% Le Meur (lemeur@lal.in2p3.fr), Benoit Revenu (revenu@iap.fr) and Ken Ganga (kmg@ipac.caltech.edu).

\begin{arguments}
{
\begin{tabular}{p{0.28\hsize} p{0.10\hsize} p{0.05\hsize} p{0.47\hsize}} \hline  
\textbf{name~\&~dimensionality} & \textbf{type} & \textbf{in/out} & \textbf{description} \\ \hline
                   &   &   &                           \\ %%% for presentation
nside & long & IN & $\nside$ parameter for the \healpix map. \\
ipnest & long & --- & pixel identification number in NESTED scheme over the range \{0,$\npix-1$\}. \\
ipring & long & --- & pixel identification number in RING scheme over the range \{0,$\npix-1$\}. \\
theta & double & --- & colatitude in radians measured southward from north pole in [0,$\pi$]. \\
phi & double & --- & longitude in radians, measured eastward in [0,$2\pi$]. \\
vector & double & --- & 3D cartesian position vector $(x,y,z)$. The north pole is $(0,0,1)$. An output vector is normalised to unity.
\end{tabular}
}
\end{arguments}
\newpage

\rule{\hsize}{0.7mm}
\textsc{\large{\textbf{ROUTINES: }}}\hfill\newline
{\tt void  pix2ang\_ring(long nside, long ipring, double *theta, double *phi);} 

 \begin{tabular}{@{}p{0.3\hsize}@{\hspace{1ex}}
                        p{0.7\hsize}@{}}
                                         & renders {\em theta} and {\em phi} coordinates of the nominal pixel center given the pixel number {\em ipring} and a map resolution parameter {\em nside}. \\
     \end{tabular}\\\\
{\tt void  pix2vec\_ring(long nside, long ipring, double *vector);} 

  \begin{tabular}{@{}p{0.3\hsize}@{\hspace{1ex}}
                         p{0.7\hsize}@{}}
                                          & renders cartesian vector coordinates of the nominal pixel center given the pixel number {\em ipring} and a map resolution parameter {\em nside}. Optionally renders cartesian vector coordinates of the considered pixel four vertices.\\
      \end{tabular}\\\\
{\tt void  ang2pix\_ring(long nside, double theta, double phi, long *ipring);} 

 \begin{tabular}{@{}p{0.3\hsize}@{\hspace{1ex}}
                        p{0.7\hsize}@{}}
                                         & renders the pixel number {\em ipring} for a pixel which, given the map resolution parameter {\em nside}, contains the point on the sphere at angular coordinates {\em theta} and {\em phi}. \\
     \end{tabular}\\\\
{\tt void  vec2pix\_ring(long nside, double *vector, long *ipring);} 

 \begin{tabular}{@{}p{0.3\hsize}@{\hspace{1ex}}
                        p{0.7\hsize}@{}}
                                         & renders the pixel number {\em ipring} for a pixel which, given the map resolution parameter {\em nside}, contains the point on the sphere at cartesian coordinates {\em vector}. \\
     \end{tabular}\\\\
{\tt void  pix2ang\_nest(long nside, long ipnest, double *theta, double *phi);} 

 \begin{tabular}{@{}p{0.3\hsize}@{\hspace{1ex}}
                        p{0.7\hsize}@{}}
                                         & renders {\em theta} and {\em phi} coordinates of the nominal pixel center given the pixel number {\em ipnest} and a map resolution parameter {\em nside}. \\
     \end{tabular}\\\\
{\tt void  pix2vec\_nest(long nside, long ipnest, double *vector);} 

 \begin{tabular}{@{}p{0.3\hsize}@{\hspace{1ex}}
                        p{0.7\hsize}@{}}
                                         & renders cartesian vector coordinates of the nominal pixel center given the pixel number {\em ipnest} and a map resolution parameter {\em nside}. Optionally renders cartesian vector coordinates of the considered pixel four vertices.\\
     \end{tabular}\\\\
{\tt void  ang2pix\_nest(long nside, double theta, double phi, long *ipnest);} 

 \begin{tabular}{@{}p{0.3\hsize}@{\hspace{1ex}}
                        p{0.7\hsize}@{}}
                                         & renders the pixel number {\em ipnest} for a pixel which, given the map resolution parameter {\em nside}, contains the point on the sphere at angular coordinates {\em theta} and {\em phi}. \\
     \end{tabular}\\\\
{\tt void  vec2pix\_nest(long nside, double *vector, long *ipnest)} 

 \begin{tabular}{@{}p{0.3\hsize}@{\hspace{1ex}}
                        p{0.7\hsize}@{}}
                                         & renders the pixel number
                        {\em ipnest} for a pixel which, given the map
                        resolution parameter {\em nside}, contains the
                        point on the sphere at cartesian coordinates
                        {\em vector} . \\
     \end{tabular}\\\\

{\tt void  nest2ring(long nside, long ipnest, long *ipring);} 

 \begin{tabular}{@{}p{0.3\hsize}@{\hspace{1ex}}
                        p{0.7\hsize}@{}}
                                         & performs conversion from NESTED to RING pixel number. \\
     \end{tabular}\\\\
{\tt void  ring2nest(long nside, long ipring, long *ipnest);} 

 \begin{tabular}{@{}p{0.3\hsize}@{\hspace{1ex}}
                        p{0.7\hsize}@{}}
                                         & performs conversion from RING to NESTED pixel number. \\
     \end{tabular}\\\\

\begin{modules}
  \begin{sulist}{} %%%% NOTE the ``extra'' brace here %%%%
 \item[mk\_pix2xy, mk\_xy2pix] routines used in the conversion between pixel values and ``cartesian'' coordinates on the Healpix face.
  \end{sulist}
\end{modules}

\begin{related}
  \begin{sulist}{} %%%% NOTE the ``extra'' brace here %%%%
%%   \item[\htmlref{neighbours\_nest}{sub:neighbours_nest}] find neighbouring pixels.
  \item[\htmlref{ang2vec}{csub:ang2vec}] converts $(\theta,\phi)$ spherical coordinates into $(x,y,z)$ cartesian coordinates.
  \item[\htmlref{vec2ang}{csub:vec2ang}] converts $(x,y,z)$ cartesian coordinates into $(\theta,\phi)$ spherical coordinates.
  \item[\htmlref{nside2npix}{csub:nside2npix}] converts number of full sky
pixels $\npix$ into resolution parameter $\nside$
  \item[\htmlref{npix2nside}{csub:npix2nside}] converts $\nside$ into number of
full sky pixels $\npix$.
  \end{sulist}
\end{related}

\rule{\hsize}{2mm}

\newpage


\sloppy


\title{\healpix Fortran Subroutines Overview}
\docid{read\_healpix\_map} \section[read\_healpix\_map]{ }
\label{csub:read_healpix_map}
\docrv{Version 1.2}
\author{Eric Hivon}
\abstract{This document describes the \healpix C subroutine READ\_HEALPIX\_MAP.}

\begin{facility}
{This routine reads a full sky \healpix map from a FITS file}
{src/C/subs/read\_healpix\_map.c}
\end{facility}

\begin{Cfunction}
{float *read\_healpix\_map(char *infile, long *nside, char *coordsys, char *ordering)}
\end{Cfunction}

\begin{arguments}
{
\begin{tabular}{p{0.3\hsize} p{0.05\hsize} p{0.05\hsize} p{0.5\hsize}} \hline  
\textbf{name\&dimensionality} & \textbf{kind} & \textbf{in/out} & \textbf{description} \\ \hline
                   &   &   &                           \\ %%% for presentation
\thedocid & float & OUT & array containing the map read from the file \\
infile   & char & IN & FITS file containing a full sky to be read \\
nside    & long & OUT & \healpix resolution parameter of the map \\
coordsys & char & OUT & astronomical coordinate system of pixelisation 
	(either 'C', 'E' or 'G' standing respectively for Celestial=equatorial,
		  Ecliptic or Galactic)\\
ordering & char & OUT & \healpix pixel ordering (either 'RING' or 'NESTED')
\end{tabular}
}
\end{arguments}

% \begin{example}
% {
% npix= read\_healpix\_map('map.fits', nmaps=nmaps, ordering=ordering,obs\_npix=obs\_npix, nside=nside, mlpol=mlpol, type=type, polarisation=polarisation)  \\
% }
% {
% Returns 1 or 3 in nmaps, dependent on wether 'map.fits' contain only
% temperature or both temperature and polarisation maps. The pixel ordering number is found by reading the keyword ORDERING in the FITS file. If this keyword does not exist, 0 is returned.
% }
% \end{example}

%\newpage

% \begin{modules}
%   \begin{sulist}{} %%%% NOTE the ``extra'' brace here %%%%
%   \item[\textbf{fitstools}] module, containing:
%   \item[printerror] routine for printing FITS error messages.
%   \item[\textbf{cfitsio}] library for FITS file handling.		
%   \end{sulist}
% \end{modules}

\begin{related}
  \begin{sulist}{} %%%% NOTE the ``extra'' brace here %%%%
  \item[anafast] executable that reads a \healpix map and analyses it. 
  \item[synfast] executable that generate full sky \healpix maps
%   \item[\htmlref{read\_healpix\_map}{csub:read_healpix_map}] subroutine to read \healpix maps
  \item[\htmlref{write\_healpix\_map}{csub:write_healpix_map}] subroutine to write \healpix maps
  \item[\htmlref{get\_fits\_size}{csub:get_fits_size}] subroutine to determine
  the size of a map
  \end{sulist}
\end{related}

\rule{\hsize}{2mm}

\newpage



\sloppy


\title{\healpix C Subroutines Overview}
\docid{vec2ang} \section[vec2ang]{ }
\label{csub:vec2ang}
\docrv{Version 2.0}
\author{E. Hivon}
\abstract{This document describes the \healpix C subroutine VEC2ANG.}

\begin{facility}
{Routine to convert the 3D position vector $(x,y,z)$ of point into its position
  angles  $(\theta,\phi)\myhtmlimage{}$ on the sphere with
$x = \sin\theta\cos\phi\myhtmlimage{}$, $y=\sin\theta\sin\phi\myhtmlimage{}$, $z=\cos\theta\myhtmlimage{}$.
}
{src/C/subs/chealpix.c}
\end{facility}

\begin{Cfunction}
{void vec2ang(double *vector, double *theta, double *phi);}
\end{Cfunction}

\begin{arguments}
{
\begin{tabular}{p{0.3\hsize} p{0.10\hsize} p{0.05\hsize} p{0.45\hsize}} \hline  
\textbf{name~\&~dimensionality} & \textbf{kind} & \textbf{in/out} & \textbf{description} \\ \hline
                   &   &   &                           \\ %%% for presentation
vector(3) & double & IN & three dimensional cartesian position vector
                   $(x,y,z)$. The north pole is $(0,0,1)$\\
theta & double & OUT & colatitude in radians measured southward from north pole (in
    [0,$\pi$]). \\
phi   & double & OUT & longitude in radians measured eastward (in [0, $2\pi$]).\\
\end{tabular}
}
\end{arguments}

% \begin{example}
% {
% call vec2ang(vector,theta,phi) \\
% }
% {
% }
% \end{example}
\begin{related}
  \begin{sulist}{} %%%% NOTE the ``extra'' brace here %%%%
  \item[\htmlref{ang2vec}{csub:ang2vec}] converts the position angles of a point on the sphere 
into its 3D position vector.
  \end{sulist}
\end{related}

\rule{\hsize}{2mm}



\sloppy


\title{\healpix Fortran Subroutines Overview}
\docid{write\_healpix\_map} \section[write\_healpix\_map]{ }
\label{csub:write_healpix_map}
\docrv{Version 1.2}
\author{Eric Hivon}
\abstract{This document describes the \healpix C subroutine WRITE\_HEALPIX\_MAP.}

\begin{facility}
{This routine writes a full sky \healpix map into a FITS file}
{src/C/subs/write\_healpix\_map.c}
\end{facility}

\begin{Cfunction}
{int write\_healpix\_map( float *signal, long nside, char *filename, char nest, char *coordsys)}
\end{Cfunction}

\begin{arguments}
{
\begin{tabular}{p{0.3\hsize} p{0.05\hsize} p{0.05\hsize} p{0.5\hsize}} \hline  
\textbf{name\&dimensionality} & \textbf{kind} & \textbf{in/out} & \textbf{description} \\ \hline
                   &   &   &                           \\ %%% for presentation
\thedocid & int & OUT & returns a non zero value in case of error \\
signal    & float & IN & full sky map to be written \\
nside     & long & IN & \healpix resolution parameter of the map (the map should
                   have 12 * nside * nside pixels).\\
filename  & char & IN & FITS file in which to write the full sky map \\
nest      & char & IN & flag specifing the \healpix pixel ordering of the
                   map. 0: 'RING' and 1: 'NESTED' \\
coordsys  & char & IN & astronomical coordinate system of map
	(must be either 'C', 'E' or 'G' standing respectively for Celestial=equatorial,
		  Ecliptic or Galactic)
\end{tabular}
}
\end{arguments}

% \begin{example}
% {
% npix= write\_healpix\_map('map.fits', nmaps=nmaps, ordering=ordering,obs\_npix=obs\_npix, nside=nside, mlpol=mlpol, type=type, polarisation=polarisation)  \\
% }
% {
% Returns 1 or 3 in nmaps, dependent on wether 'map.fits' contain only
% temperature or both temperature and polarisation maps. The pixel ordering number is found by writeing the keyword ORDERING in the FITS file. If this keyword does not exist, 0 is returned.
% }
% \end{example}

%\newpage

% \begin{modules}
%   \begin{sulist}{} %%%% NOTE the ``extra'' brace here %%%%
%   \item[\textbf{fitstools}] module, containing:
%   \item[printerror] routine for printing FITS error messages.
%   \item[\textbf{cfitsio}] library for FITS file handling.		
%   \end{sulist}
% \end{modules}

\begin{related}
  \begin{sulist}{} %%%% NOTE the ``extra'' brace here %%%%
  \item[anafast] executable that reads a \healpix map and analyses it. 
  \item[synfast] executable that generate full sky \healpix maps
  \item[\htmlref{read\_healpix\_map}{csub:read_healpix_map}] subroutine to read \healpix maps
%   \item[\htmlref{write\_healpix\_map}{csub:write_healpix_map}] subroutine to write \healpix maps
  \item[\htmlref{get\_fits\_size}{csub:get_fits_size}] subroutine to determine the size of a map
  \end{sulist}
\end{related}

\rule{\hsize}{2mm}

\newpage


\end{document}
