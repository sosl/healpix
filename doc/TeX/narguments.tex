\sloppy

\title{\healpix Fortran Subroutines Overview}
\docid{nArguments} \section[nArguments]{ }
\label{sub:narguments}
\docrv{Version 1.0}
\author{Eric Hivon}
\abstract{This document describes the \healpix Fortran90 subroutine
narguments.}

\begin{facility}
{This function emulates the C routine {\tt iargc}, which returns the number of
command line arguments provided.}
{src/f90/mod/extension.F90}
\end{facility}

\begin{f90function}
{\ }
\end{f90function}

\begin{arguments}
{
\begin{tabular}{p{0.3\hsize} p{0.05\hsize} p{0.1\hsize} p{0.45\hsize}} \hline  
\textbf{name\&dimensionality} & \textbf{kind} & \textbf{in/out} & \textbf{description} \\ \hline
                   &   &   &                           \\ %%% for presentation
var & I4B & OUT & number of command line arguments

\end{tabular}}
\end{arguments}

% \begin{example}
% {
% use extension \\
% character(len=128) :: healpixdir \\
% call getargument('HEALPIX', healpixdir) \\
% print*,healpixdir
% }
% {
% Will return the value of the {\tt \$HEALPIX} system variable (if it is defined)
% }
% \end{example}

\begin{related}
  \begin{sulist}{} %%%% NOTE the ``extra'' brace here %%%%
  \item[\htmlref{getEnvironment}{sub:getenvironment}] returns value of
  environment variable
  \item[\htmlref{getArgument}{sub:getargument}] returns list of command line arguments
%  \item[\htmlref{nArguments}{sub:narguments}] returns number of command line arguments
  \end{sulist}
\end{related}

\rule{\hsize}{2mm}

\newpage
