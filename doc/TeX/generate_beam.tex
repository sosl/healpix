
\sloppy


\title{\healpix Fortran Subroutines Overview}
\docid{generate\_beam} \section[generate\_beam]{ }
\label{sub:generate_beam}
\docrv{Version 2.0}
\author{Eric Hivon}
\abstract{This document describes the \healpix Fortran90 subroutine GENERATE\_BEAM.}

\begin{facility}
{This routine generates the beam window function in multipole space. It is
  either a gaussian parametrized by its FWHM in arcmin in real space, or it is
  read from an external file.}
{\modAlmTools}
\end{facility}

\begin{f90format}
{fwhm\_arcmin, lmax, beam \optional{[, beam\_file]}}
\end{f90format}
\aboutoptional

\begin{arguments}
{
\begin{tabular}{p{0.4\hsize} p{0.05\hsize} p{0.1\hsize} p{0.35\hsize}} \hline  
\textbf{name~\&~dimensionality} & \textbf{kind} & \textbf{in/out} & \textbf{description} \\ \hline
                   &   &   &                           \\ %%% for presentation
fwhm\_arcmin & DP & IN & fwhm size of the gaussian beam in arcminutes. \\
lmax & I4B & IN & maximum $\ell$ value of the window function.   \\
beam(0:lmax,1:p) & DP & OUT & beam window function generated. The second index runs form 1:1 for temperature only, and 1:3 for polarisation. In the latter case, 1=T, 2=E, 3=B.\\
\optional{beam\_file}(LEN=\filenamelen) (\nobreak{OPTIONAL})& CHR & IN & name of the file containing
the (non necessarily gaussian) window function $B_\ell$ of a circular beam. If present, it will override
the argument {\tt fwhm\_arcmin}. If fewer columns than requested are found in
the file, missing colums will duplicate the existing ones (based on the
assumption that $B_\ell$ is the same in T, E and B). Supports the {\tt fitsio} 'Extended Filename
Syntax' (see \htmlref{examples below}{sub:ex:generate_beam}).
\end{tabular}
}
\end{arguments}

% \begin{examples}{1}
% {
% call generate\_beam(5.0\_dp, 1024, beam)  \\
% }
% {%
% Generates the window function of a gaussian beam of FWHM = 5 arcmin, for $\ell
% \leq 1024$.
% }
% \end{examples}

\begin{example}
{
use \htmlref{healpix\_modules}{sub:healpix_modules} \\
real(dp), dimension(0:1024, 1:3) :: gb0, b1, b2, b3\\
call generate\_beam(5.0\_dp, 1024, gb0)\\
call generate\_beam(0\_dp, 1024, b1, beam\_file='file.fits')\\
call generate\_beam(0\_dp, 1024, b2, beam\_file='file.fits[col 1]')\\
call generate\_beam(0\_dp, 1024, b3, beam\_file='file.fits[col 1; 2=0; 3=0]')\\
}
{%
{\tt gb0} will contain the window function of a gaussian beam of FWHM = 5 arcmin, for $\ell
\leq 1024$.
{\tt b1} will contain the first 3 columns of {\tt file.fits}; {\tt b2} will be
based on a virtual FITS file containg only the first column of {\tt file.fits},
and we will have {\tt b2(:,3) = b2(:,2) = b2(:,1)}; finally {\tt b3} will read a
virtual FITS file in which the first column is the same as in {\tt file.fits},
while the column 2 and 3 are set to 0. Therefore {\tt b3(:,3) = b3(:,2) = 0}.
\label{sub:ex:generate_beam}
}
\end{example}



\begin{modules}
  \begin{sulist}{} %%%% NOTE the ``extra'' brace here %%%%
  \item[\textbf{alm\_tools}] module, containing:
	\item[\htmlref{gaussbeam}{sub:gaussbeam}] routine to generate a gaussian beam
  \end{sulist}
\end{modules}

\begin{related}
  \begin{sulist}{} %%%% NOTE the ``extra'' brace here %%%%
  \item[\htmlref{create\_alm}{sub:create_alm}] Routine to create $a_{\ell m}$
  coefficients using \thedocid.
  \item[\htmlref{alter\_alm}{sub:alter_alm}] Routine to alter $a_{\ell m}$
  coefficients  using \thedocid.
  \item[\htmlref{pixel\_window}{sub:pixel_window}] Routine returning a pixel
  window function.
  \end{sulist}
\end{related}

\rule{\hsize}{2mm}

\newpage
