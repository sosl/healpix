
\sloppy


\title{\healpix Fortran Subroutines Overview}
\docid{same\_shape\_pixels\_nest, same\_shape\_pixels\_ring} \section[same\_shape\_pixels\_nest, same\_shape\_pixels\_ring]{ }
\label{sub:same_shape_pixels_xxx}
\docrv{Version 1.0}
\author{E. Hivon}
\abstract{This document describes the \healpix Fortran90 subroutines
  SAME\_SHAPE\_PIXELS\_RING and SAME\_SHAPE\_PIXELS\_NEST.}

\begin{facility}
{These routines provide the ordered list of all \healpix pixels having the same shape
  as a given template, for a resolution parameter $\nside$. Depending on the
  template considered the number of such pixels is either 8, 16, 4$\nside$ or
  8$\nside$.

%% Any pixel can be {\em matched in shape}
%%   to a single of these templates by a combination of  a rotation around the polar axis with 
%%   reflexion(s) around a meridian and/or the equator. 

The template pixels are all located in the Northern Hemisphere, or on the
 Equator.
They are chosen to have their center located at
\begin{eqnarray}
     z=\cos(\theta)\ge 2/3,  &\ &    0< \phi \leq \pi/2,   \nonumber\\            %[Nside*(Nside+2)/4]
     2/3 > z \geq 0,  &\ & \phi=0, \quad{\rm or}\quad  \phi=\frac{\pi}{4\nside}.  \nonumber %[Nside]
\myhtmlimage{}
\end{eqnarray}
 They are numbered continuously from 0, starting at the North Pole, with the index
 increasing in $\phi$, and then increasing for decreasing $z$.
}
{src/f90/mod/pix\_tools.F90}
\end{facility}

\docid{same\_shape\_pixels\_nest}
\begin{f90format}
{ nside, template [,~list, reflexion, nrep]}
\end{f90format}
\docid{same\_shape\_pixels\_ring}
\begin{f90format}
{ nside, template [,~list, reflexion, nrep]}
\end{f90format}

\begin{arguments}
{
\begin{tabular}{p{0.28\hsize} p{0.05\hsize} p{0.1\hsize} p{0.47\hsize}} \hline  
\textbf{name~\&~dimensionality} & \textbf{kind} & \textbf{in/out} & \textbf{description} \\ \hline
                   &   &   &                           \\ %%% for presentation
nside & I4B & IN & the \healpix $\nside$ parameter. \\
template & I4B/ I8B & IN & identification number of the
                   template pixel (the numbering
                   scheme of the pixel templates is the same for both routines). \\
list(0:nrep-1) \hskip 3cm OPTIONAL & I4B/ I8B & OUT & pointer containing the ordered list of NESTED/RING scheme
                   identification numbers (in \{0,$12\nside^2-1$\})
  of all pixels having the same shape as the template provided. The routines
                   will allocate the {\tt list} array if it is not allocated
                   upon calling. \\
reflexion(0:nrep-1) \hskip 3cm OPTIONAL & I4B & OUT & pointer containing the transformation(s) (in
                   \{0, 3\}) to
                   apply to each of the returned pixels to match exactly in
                   shape and position its respective template. 0: rotation around the polar axis only,
                   1: rotation + East-West swap (ie, reflexion around meridian),
                   2: rotation + North-South swap (ie, reflexion around
                   Equator), 3: rotation + East-West and North-South swaps. The routines
                   will allocate the {\tt list} array if it is not allocated
                   upon calling. \\
nrep \hskip 4cm OPTIONAL & I4B/ I8B  & OUT & number of pixels having the same template (either 8, 16, 4$\nside$ or
  8$\nside$).
\end{tabular}
}
\end{arguments}

\begin{example}
{
call same\_shape\_pixels\_ring(256, 1234, list, reflexion, np)  \\
}
{
Returns in {\tt list} the RING-scheme index of the all the pixels having
the same shape as the template \#1234 for $\nside=256$. Upon return {\tt reflexion} will
contain the rotation/reflexions to apply to each pixel returned to match the template,
and {\tt np} will contain the number of pixels having that same shape (16 in that case).
}
\end{example}
\begin{related}
  \begin{sulist}{} %%%% NOTE the ``extra'' brace here %%%%
  \item[\htmlref{nside2templates}{sub:nside2ntemplates}] returns the
  number of template pixel shapes available for a given $\nside$.
  \item[\htmlref{template\_pixel\_ring}{sub:template_pixel_xxx}] 
  \item[\htmlref{template\_pixel\_nest}{sub:template_pixel_xxx}] 
  return
  the template shape matching the pixel provided
  \end{sulist}
\end{related}

\rule{\hsize}{2mm}

