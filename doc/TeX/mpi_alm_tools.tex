
\sloppy


\title{\healpix Fortran Subroutines Overview}
\docid{mpi\_alm\_tools*} \section[mpi\_alm\_tools*]{ }
\label{sub:mpi_alm_tools}
\docrv{Version 1.0}
\author{Hans K. Eriksen}
\abstract{This document describes the \healpix Fortran 90 module MPI\_ALM\_TOOLS*.}

\begin{facility}
{This module implements MPI parallelization of the alm2map and map2alm routines. 
It is not compiled by default during installation, but rather intended for
users who need massive parallelization in their own programming. Typical
applications are Monte Carlo simulations and Markov chain type
analyses.

The routines can be called in two modes, either simple or
advanced. The former mimics the interface of the standard routines,
but with an additional MPI handle as a first argument, and is intended
for applications which requires only one or a few transforms. The
latter interface provides both more flexibility (in particular the
option of pre-computation of the Legendre polynomials) and a simpler
interface when multiple transforms are required. This interface is
particularly well suited for Monte Carlo simulations and Markov chain
type analyses.  } 
{src/f90/mod/mpi\_alm\_tools.f90}
\end{facility}

\begin{example}
{ 
\begin{itemize}

\item Simple one-line interfaces:
\begin{itemize}
\item mpi\_map2alm\_simple
\item mpi\_alm2map\_simple
\end{itemize}

\item Three-step advanced interfaces:
\begin{enumerate}
   \item Initialization: \\mpi\_initialize\_alm\_tools
   \item Execution of spherical harmonics transforms
   \begin{itemize}
	   \item mpi\_map2alm (root processor)
	   \item mpi\_alm2map (root processor)
	   \item mpi\_map2alm\_slave (slave processor)
	   \item mpi\_alm2map\_slave (slave processor)
    \end{itemize}
    \item Finalizing: \\ mpi\_cleanup\_alm\_tools
\end{enumerate}
\end{itemize}
}
{
}
\end{example}

\rule{\hsize}{2mm}

\newpage
