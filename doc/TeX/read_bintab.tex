
\sloppy


\title{\healpix Fortran Subroutines Overview}
\docid{read\_bintab*} \section[read\_bintab*]{ }
\label{sub:read_bintab}
\docrv{Version 2.1}
\author{Frode K.~Hansen, Eric Hivon}
\abstract{This document describes the \healpix Fortran90 subroutine READ\_BINTAB.}

\begin{facility}
{This routine reads a \healpix map from a binary FITS-file. The routine can read a temperature map or both temperature and polarisation maps (T,Q,U)}
{\modFitstools}
\end{facility}

\begin{f90format}
{\mylink{sub:read_bintab:filename}{filename}%
, \mylink{sub:read_bintab:map}{map}%
, \mylink{sub:read_bintab:npixtot}{npixtot}%
, \mylink{sub:read_bintab:nmaps}{nmaps}%
, \mylink{sub:read_bintab:nullval}{nullval}%
, \mylink{sub:read_bintab:anynull}{anynull}%
 \optional{[,\mylink{sub:read_bintab:header}{header}%
, \mylink{sub:read_bintab:units}{units}%
, \mylink{sub:read_bintab:extno}{extno}%
]}}
\end{f90format}
\aboutoptional

\begin{arguments}
{
\begin{tabular}{p{0.4\hsize} p{0.05\hsize} p{0.05\hsize} p{0.40\hsize}} \hline  
\textbf{name~\&d~imensionality} & \textbf{kind} & \textbf{in/out} & \textbf{description} \\ \hline
                   &   &   &                           \\ %%% for presentation
filename\mytarget{sub:read_bintab:filename}(LEN=\filenamelen) & CHR & IN & filename of FITS-file containing the map(s). \\
npixtot\mytarget{sub:read_bintab:npixtot} & I4B & IN & Number of pixels to be read from map.\\
nmap & I4B & IN & number of maps to be read, 1 for temperature only, and 3 for (T,Q,U). \\
map\mytarget{sub:read_bintab:map}(0:npixtot-1,1:nmap) & SP/ DP & OUT & the map read from the FITS-file.\\
nullval\mytarget{sub:read_bintab:nullval} & SP/ DP & OUT & value of missing pixels in the map. \\
anynull\mytarget{sub:read_bintab:anynull} & LGT & OUT & {\tt .TRUE.}, if there are missing pixels, and {\tt .FALSE.}
                   otherwise. \\
\optional{header\mytarget{sub:read_bintab:header}}(LEN=80)(1:)\hskip 3cm (OPTIONAL) & CHR & OUT & character string array
                   containing the FITS header read from the file. Its
                   dimension has to be defined prior to calling the
                   routine \\
\optional{units\mytarget{sub:read_bintab:units}}(LEN=*)(1:nmaps) & CHR & OUT & character string array
                   containing the physical units of each map read \\
\optional{extno\mytarget{sub:read_bintab:extno}} & I4B & IN & extension number to read the data from
                   (0 based).\default 0 (the first extension is read) 
\end{tabular}
}
\end{arguments}
\newpage

\begin{example}
{
call read\_bintab ('map.fits', map, 12*32**2, 1, nullval, anynull)  \\
}
{
Reads a \healpix temperature map from the file `map.fits' to the array
map(0:12*32**2-1,1:1). The pixel number 12*32**2 is the number of pixels in a
$\nside=32$ \healpix map. 
If there are missing pixels in the input file (with
value {\tt NaN} (Not a Number), $\pm${\tt Infinity}, or matching the FITS
keyword {\tt BAD\_DATA}) then {\tt
anynull} is {\tt .TRUE.} and these pixels get the value returned in {\tt nullval}. 
}
\end{example}

\begin{modules}
  \begin{sulist}{} %%%% NOTE the ``extra'' brace here %%%%
  \item[\textbf{fitstools}] module, containing:
  \item[printerror] routine for printing FITS error messages.
  \item[\textbf{cfitsio}] library for FITS file handling.		
  \end{sulist}
\end{modules}

\begin{related}
  \begin{sulist}{} %%%% NOTE the ``extra'' brace here %%%%
  \item[\htmlref{input\_map}{sub:input_map}] Routine which reads a map using \thedocid\ and fills missing pixels with a given value.
  \item[\htmlref{map2alm}{sub:map2alm}] Routine which analyse a map and returns the $a_{lm}$
  coefficients.
  \item[\htmlref{read\_fits\_cut4}{sub:read_fits_cut4}] Routine to read cut sky \healpix FITS maps
  \item[\htmlref{write\_plm}{sub:write_plm}, \htmlref{write\_bintab}{sub:write_bintab}] Routines to write \healpix FITS maps
  \end{sulist}
\end{related}

\rule{\hsize}{2mm}

\newpage
