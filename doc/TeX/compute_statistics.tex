
\sloppy


\title{\healpix Fortran Subroutines Overview}
\docid{compute\_statistics*} \section[compute\_statistics*]{ }
\label{sub:compute_statistics}
\docrv{Version 2.0}
\author{Eric Hivon}
\abstract{This document describes the \healpix Fortran90 subroutine COMPUTE\_STATISTICS*.}

\newcommand{\myskip}{\hskip 1cm}

\begin{facility}
{This routine computes the min, max, absolute deviation and first four order moment of a data set}
{\modStatistics}
\end{facility}

\begin{f90format}
{\mylink{sub:compute_statistics:data}{data}%
 ,\mylink{sub:compute_statistics:stats}{stats}%
 \optional{[,~\mylink{sub:compute_statistics:badval}{badval}%
]}}
\end{f90format}
\aboutoptional

\begin{arguments}
{
\begin{tabular}{p{0.30\hsize} p{0.05\hsize} p{0.05\hsize} p{0.50\hsize}} \hline  
\textbf{name~\&~dimensionality} & \textbf{kind} & \textbf{in/out} & \textbf{description} \\ \hline
                   &   &   &                           \\ %%% for presentation
data\mytarget{sub:compute_statistics:data}(:) & SP/ DP & IN & data set $\{x_i\}$\\
stats\mytarget{sub:compute_statistics:stats}   & tstats & OUT & structure containing the statistics of the
                   data. The respective fields (stats\%{\em field}) are:\\
\myskip ntot & I8B & -- & total number of data points \\
\myskip nvalid & I8B & -- & number $n$ of valid data points \\
\myskip mind, maxd & DP & -- & minimum and maximum valid data \\
\myskip average & DP & -- & average of valid points $m= \sum_i x_i / n$\\
\myskip absdev & DP & -- & absolute deviation $a= \sum_i|x_i-m|/n$\\
\myskip var & DP & -- & variance $\sigma^2 = \sum(x_i-m)^2/ (n-1)$\\
\myskip rms & DP & -- & standard deviation $\sigma$ \\
\myskip skew & DP & -- & skewness factor $s = \sum(x_i-m)^3 / (n\sigma^3)$\\
\myskip kurt & DP & -- & kurtosis factor $k = \sum(x_i-m)^4 / (n\sigma^4) - 3$\\
\ & \ & \ & \\
\optional{badval\mytarget{sub:compute_statistics:badval}} \hskip 3cm (OPTIONAL) & SP/ DP & IN & sentinel value given to bad data points. Data points with this
                   value will be ignored during calculation of the statistics. If
                   not set, all points will be considered. {\bf Do not set to 0!}.
\end{tabular}
}
\end{arguments}
%%\newpage

\begin{example}
{
use statistics, only: compute\_statistics, print\_statistics, tstats \\
type(tstats) :: stats \\
... \\
compute\_statistics(map, stats)  \\
print*,stats\%average, stats\%rms\\
print\_statistics(stats) \\
}
{
Computes the statistics of {\tt map}, prints its average and {\em rms} and
prints the whole list of statistical measures.
}
\end{example}

%% \begin{modules}
%%   \begin{sulist}{} %%%% NOTE the ``extra'' brace here %%%%
%%   \item[\textbf{}] 
%%   \item[] 
%%   \end{sulist}
%% \end{modules}

\begin{related}
  \begin{sulist}{} %%%% NOTE the ``extra'' brace here %%%%
  \item[\htmlref{median}{sub:median}] routine to compute median of a data set
  \end{sulist}
\end{related}

\rule{\hsize}{2mm}

\newpage
