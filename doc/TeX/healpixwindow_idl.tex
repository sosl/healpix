% -*- LaTeX -*-


\renewcommand{\facname}{{healpixwindow}}
\renewcommand{\FACNAME}{{HEALPIXWINDOW}}
\renewcommand{\l}{{$\ell$ }}

\sloppy



\title{\healpix IDL Facility User Guidelines}
\docid{\facname} \section[\facname]{ }
\label{idl:healpixwindow}
\docrv{Version 1.1}
\author{Eric Hivon}
\abstract{This document describes the \healpix IDL facility \facname.}




\begin{facility}
{This IDL facility provides the window function in $\ell$ associated with the
Healpix pixel of resolution Nside.}
{src/idl/misc/healpixwindow.pro}
\end{facility}

\begin{IDLformat}
{\mylink{idl:healpixwindow:wpix}{wpix}=\FACNAME(\mylink{idl:healpixwindow:nside}{Nside} 
[, \mylink{idl:healpixwindow:Dim}{Dim}, 
\mylink{idl:healpixwindow:directory}{DIRECTORY=},
\mylink{idl:healpixwindow:help}{HELP=},
\mylink{idl:healpixwindow:lmax}{LMAX=}])}
\end{IDLformat}

\begin{qualifiers}
  \begin{qulist}{} %%%% NOTE the ``extra'' brace here %%%%
    \item[Nside\mytarget{idl:healpixwindow:nside}] resolution parameter
    \item[Wpix\mytarget{idl:healpixwindow:wpix}] the pixel window function, computed for the multipoles $\ell$ in \{0,...,\mylink{idl:healpixwindow:lmax}{\texttt{LMAX}}\}
    \item[Dim\mytarget{idl:healpixwindow:Dim}] scalar integer, optional. \\
     If absent or set to 0 or 1,
          the output has size (\mylink{idl:healpixwindow:lmax}{\texttt{LMAX}}+1) and is the temperature
    window function;\\
     if set to $2 \le$ Dim $\le 4$ ,
          the output has size (\mylink{idl:healpixwindow:lmax}{\texttt{LMAX}}+1,Dim)
          and contains in that order : \\
	  the TEMPERATURE window function,\\
          the GRAD/ELECTRIC polarization one\\
          the CURL/MAGNETIC polarization one\\
          the TEMPERATURE*GRAD one.
   \item[DIRECTORY=\mytarget{idl:healpixwindow:directory}] directory in which the precomputed pixel window file is looked for.\\
          \default{\htmlref{\texttt{!healpix.path.data}}{idl:init_healpix}}
   \item[HELP=\mytarget{idl:healpixwindow:help}] if set, a documentatin header is printed out, 
    and the routine exits
   \item[LMAX=\mytarget{idl:healpixwindow:lmax}] maximum multipole included in Wpix. 
	Must be in [0, 4 \mylink{idl:healpixwindow:nside}{Nside}].
        Negative values are ignored.
	\default{4 Nside}.
  \end{qulist}
\end{qualifiers}

% \begin{keywords}
%   \begin{kwlist}{} %%% extra brace
%   \end{kwlist}
% \end{keywords}  

\begin{codedescription}
{\thedocid{} computes the $\ell$ space window function due to the finite size of the
\healpix pixels. The typical size of a pixel (square root of its uniform surface
area) is $\myhtmlimage{}\sqrt{3/\pi}\ 3600/\nside$ arcmin.
If a unpixelated sky has a power spectrum $C(\ell)$, the same
sky pixelated with a resolution parameter Nside 
will have the power spectrum $C(\ell)_{\mathrm{pix}} = C(\ell)
W(\ell)^2$ where $W(\ell)$ is given by \thedocid(Nside). The polarized
pixel window function is also provided (when Dim $>$ 1).
This routine reads some FITS files located in the subdirectory {\tt data/} of the
\healpix distribution, unless the keyword {\tt Directory} is set otherwise.}
\end{codedescription}



\begin{related}
  \begin{sulist}{} %%%% NOTE the ``extra'' brace here %%%%
    \item[idl] version \idlversion or more is necessary to run \facname
    \item[\htmlref{gaussbeam}{idl:gaussbeam}] computes the $\ell$ space window function associated with
    a gaussian beam
    \item[synfast] F90 code to generate CMB maps of given power spectrum at a
    given resolution (=pixel size)
    \item[anafast] F90 code to compute the power spectrum of a \healpix sky map
  \end{sulist}
\end{related}

\begin{example}
{
\begin{tabular}{ll} %%%% use this tabular format %%%%
wpix = \thedocid(256)
\end{tabular}
}
{
wpix contains the window function in \{0,...,1024\} of the \healpix pixel with resolution
parameter 256 (pixel size of 13.7 arcmin)}
\end{example}


