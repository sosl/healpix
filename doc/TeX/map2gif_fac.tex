
\sloppy


\title{\healpix Fortran Facility User Guidelines}
\docid{map2gif} \section[map2gif]{\nosectionname}
\label{fac:map2gif}
\docrv{Draft 0.7}
\author{Anthony J.~Banday}
\abstract{This document describes the \healpix facility map2gif.}

\begin{facility}
{This Fortran facility provides a means to generate a
gif image from an input \healpix sky map. It is intended to allow some
primitive visualisation for those with limited or no access to IDL or python. 
It is also useful for image generation in a pipeline environment.}
{src/f90/map2gif/map2gif.f90}
\end{facility}

\begin{f90facility}
{map2gif 
\mylink{fac:map2gif:inp}{-inp} FITS\_file 
\mylink{fac:map2gif:out}{-out} GIF\_file [options]}
\end{f90facility}

\begin{qualifiers}
  \begin{qulist}{} %%%% NOTE the ``extra'' brace here %%%%
%
    \item[{-inp FITS\_file}]\mytarget{fac:map2gif:inp} 
The file name of the input FITS sky map \nodefault.\\
In \thedocid\ (and \thedocid\ only) it may be necessary to put the file name between quotes if it contains symbols
such as \texttt{\&}, \texttt{[}, \texttt{]}, \texttt{?}, \texttt{=}  or blanks
%
    \item[{-out GIF\_file}]\mytarget{fac:map2gif:out} 
The file name of the output gif image \nodefault.\\
Prepending the output file name with \texttt{$\backslash$!} 
(see \mylink{fac:map2gif:example}{example below}) will allow the overwriting of the
file if it already exists
%
    \item[{-add offset}]\mytarget{fac:map2gif:add} 
Real value to add to the signal before performing any
      other operation to it (like taking the logarithm etc.)
      \default{0.0}
%
    \item[{-ash flag}]\mytarget{fac:map2gif:ash} 
Logical to use the hyperbolic arc sine of the signal when
      plotting. Cannot be true when \mylink{fac:map2gif:log}{\texttt{-log}} is true.
      \default{.false.}
%
    \item[{-bar flag}]\mytarget{fac:map2gif:bar}
 Logical which determines whether a color bar is
      displayed \default{.false.}. 
%
    \item[{-col table}]\mytarget{fac:map2gif:col} 
The number of the color table to utilise (in [1,5]) \default{5}.
%
    \item[{-hlp}]\mytarget{fac:map2gif:hlp}
 Print on-line help for the facility.
%
    \item[{-lat lat0}]\mytarget{fac:map2gif:lat}
 Latitude (Deg) of central point \default{0}
%
    \item[{-log flag}]\mytarget{fac:map2gif:log}
 Logical to use the log of the signal when plotting 
      \default{.false.}
%
    \item[{-lon lon0}]\mytarget{fac:map2gif:lon}
 Longitude (Deg) of central point \default{0}
%
    \item[{-max maxval}]\mytarget{fac:map2gif:max}
 Set the maximum value for the plotted signal
      \default{is to use the actual signal maximum}.
%
    \item[{-min minval}]\mytarget{fac:map2gif:min}
 Set the minimum value for the plotted signal
     \default{is to use the actual signal minimum}.
%
    \item[{-mul factor}]\mytarget{fac:map2gif:mul}
 Real value to multiply the signal with directly after
      adding the offset (see above).
      \default{1.0}
%
    \item[{-pro projection}]\mytarget{fac:map2gif:pro}
 Select the projection scheme 
	(among \texttt{mol} or \texttt{MOL} for Mollweide and \texttt{gno} or \texttt{GNO} for gnomic)
	\default{MOL}.
%
    \item[{-res reso}]\mytarget{fac:map2gif:res}
 Resolution in projected plan in arcmin (only for gnomic projection) \default{2}
%
    \item[{-sig number}]\mytarget{fac:map2gif:sig}
 The identifier of the signal to plot: for a
      polarisation map, then the mapping is 1 = I; 2 = Q; 3 = U
      \default{1}.
%
    \item[{-ttl title}]\mytarget{fac:map2gif:ttl}
 A string specifying the title for the plot (see note on quotes for 
\mylink{fac:map2gif:inp}{\texttt{-inp}}) \nodefault.
%
    \item[{-xsz xsize}]\mytarget{fac:map2gif:xsz}
 The x-dimension of the image in pixels \default{800}.
  \end{qulist}
\end{qualifiers}

\begin{codedescription}
{map2gif reads in a \healpix sky map in FITS format and generates an
image in GIF format. map2gif allows the selection of the projection
scheme (Mollweide for the full sky or Gnomonic for small patches of the sky), color
table, color bar inclusion, linear or log scaling, maximum and 
minimum range for the plot and plot-title. 
Flagged, undefined and NaN-valued input pixels will take a grey color on output. 
The facility utilises a command-line interface.}
\end{codedescription}

\begin{datasets}
{
\begin{tabular}{p{0.3\hsize} p{0.35\hsize}} \hline  
  \textbf{Dataset} & \textbf{Description} \\ \hline
                   &                      \\ %%% for presentation
  None required & \\ 
                   &                      \\ \hline %%% for presentation
\end{tabular}
} 
\end{datasets}

\begin{support}
  \begin{sulist}{} %%%% NOTE the ``extra'' brace here %%%%
  \item[display, open] a facility is required to view the
            gif image generated by map2gif (a browser can also 
            be used).
  \item[\htmlref{synfast}{fac:synfast}] This \healpix facility will generate the FITS format 
            sky map to be input to map2gif.
  \end{sulist}
\end{support}


\begin{examples}{1}
{
\begin{tabular}{ll} %%%% use this tabular format %%%%
\mytarget{fac:map2gif:example}
map2gif & \mylink{fac:map2gif:inp}{-inp} planck100GHZ-LFI.fits $\backslash$ \\
        & \mylink{fac:map2gif:out}{-out} $\backslash$!planck100GHZ-LFI.gif $\backslash$ \\
        & \mylink{fac:map2gif:bar}{-bar} .true. $\backslash$ \\
        & \mylink{fac:map2gif:min}{-min} -100 $\backslash$ \\
        & \mylink{fac:map2gif:max}{-max} 100 $\backslash$ \\
        & \mylink{fac:map2gif:ttl}{-ttl} 'Simulated Planck LFI Sky Map at 100GHz' \\
\end{tabular}
}
{map2gif reads in the map 'planck100GHZ-LFI.fits' and outputs
its Mollweide projection in a gif image with name 'planck100GHZ-LFI.gif' (overwriting it if
necessary) in which
the temperature scale has been set to lie between $\pm$ 100 ($\mu$K), 
a color bar has been drawn and the title 'Simulated Planck
LFI Sky Map at 100GHz' appended to the image.
}
\end{examples}

\begin{release}
  \begin{relist}
    \item Initial release (\healpix 1.00)
%    \item Updated with new rotation and projection facilities (\healpix
%    1.01) %%%% This is what we would invisage for additional releases %%%%
  \end{relist}
\end{release}

\begin{messages}
{
\begin{tabular}{p{0.25\hsize} p{0.1\hsize} p{0.35\hsize}} \hline  
  \textbf{Message} & \textbf{Severity} & \textbf{Text} \\ \hline
                   &                   &   \\ %%% for presentation
   None at present &                   &   \\ 
                   &                   &   \\ \hline %%% for presentation
\end{tabular}
} 
\end{messages}

\rule{\hsize}{2mm}

\newpage
