%   !=========================================================================
%    subroutine size_holes_nest(nside, mask, nholes, nph, &
%         &                     tags, sizeholes, listpix)
%      !=========================================================================
%      ! SIZE_HOLES_NEST: subroutine to determine size (=number of pixels) of holes
%      ! A hole is the set of all adjacent pixels initially set to 0.
%      ! 2 pixels are adjacent if they have at least one point in common.
%      !
%      ! Nside: integer, IN, resolution parameter
%      ! Mask(0:): integer 1D array, IN, each pixel must be either 1 (=valid) or 0 (=invalid)
%      ! Nholes: integer, OUT, number of holes
%      ! Nph:    integer, OUT, total number of pixels in holes
%      ! Tags(0:npix-1): integer 1D array, OPTIONAL, OUT:
%      !                   invalid pixels belonging to largest hole have value -1,
%      !                   invalid pixels belonging to second largest hole: -2,
%      !                   and so on, while valid pixels have value 1
%      ! Sizeholes(0:Nholes-1): integer pointer, OPTIONAL, OUT: respective size of each hole
%      ! Listpix(0:Nph+Nholes): integer pointer, OPTIONAL, OUT: list of pixels in each hole
%      !
%      !========================================================================
\sloppy
\docid{size\_holes\_nest}\section[size\_holes\_nest]{ }
\label{sub:size_holes_nest}
\docrv{Version 1.0}
\author{Eric Hivon}
\abstract{This document describes the \healpix Fortran90 subroutine SIZE\_HOLES\_NEST.}

\begin{facility}
{For a input binary mask in NESTED ordering, \thedocid\ identifies the pixels
located on the inner boundary of {\em invalid} regions
}
{\modMaskTools}
\end{facility}

\begin{f90format}
{\mylink{sub:size_holes_nest:nside}{nside}%
, \mylink{sub:size_holes_nest:mask}{mask}%
, \mylink{sub:size_holes_nest:nholes}{nholes}%
, \mylink{sub:size_holes_nest:nph}{nph}%
,  \optional{[\mylink{sub:size_holes_nest:tags}{tags}%
, \mylink{sub:size_holes_nest:sizeholes}{sizeholes}%
, \mylink{sub:size_holes_nest:listpix}{listpix}%
]}}
\end{f90format}
\aboutoptional

\begin{arguments}
{
\begin{tabular}{p{0.25\hsize} p{0.05\hsize} p{0.08\hsize} p{0.50\hsize}} \hline  
\textbf{name~\&~dim.} & \textbf{kind} & \textbf{in/out} & \textbf{description} \\ \hline
                   &   &   &                           \\ %%% for presentation
nside\mytarget{sub:size_holes_nest:nside} & I4B & IN & The $nside$ value of the input mask. \\
mask\mytarget{sub:size_holes_nest:mask}(0:Npix-1) & I4B & IN & Input binary NESTED-ordered mask. Npix =
12*nside*nside\\
nholes\mytarget{sub:size_holes_nest:nholes} & I4B & OUT & Number of holes found \\
nph\mytarget{sub:size_holes_nest:nph} & I4B & OUT & Number of pixels in holes 
\\
\optional{tags\mytarget{sub:size_holes_nest:tags}}(0:Npix-1) \hskip 2cm  (OPTIONAL) & I4B & OUT & Pointer allocated by \thedocid, containing
a sky map in which {\em invalid} pixels belonging to the largest hole have
value -1, those belonging to the second largest hole have value -2, and so on,
while valid pixels keep value +1.
\\
\optional{sizeholes\mytarget{sub:size_holes_nest:sizeholes}}(0:nholes-1) &I4B & OUT & Pointer allocated by \thedocid,
containing on output the respective size of each hole (in decreasing order).
Eg, {\tt sizeholes}(0) is the number of pixels in the largest hole (taking value -1 in
{\tt tags}).
\\
\optional{listpix\mytarget{sub:size_holes_nest:listpix}}(0:nholes+nph) & I4B & OUT & Pointer allocated by \thedocid,
containing on output the indexed list of pixels in each hole. Pixels located in the first (and largest)
hole are given by {\tt listpix(listpix(0):listpix(1)-1)}
\end{tabular}
}
\end{arguments}

\begin{example}
{
use healpix\_types \\
use healpix\_modules \\
% use pix\_tools, only : nside2npix \\
% use alm\_tools, only : size\_holes\_nest \\
% integer(I4B) :: nside, lmax, mmax, npix, spin\\
% real(SP), dimension(:,:), allocatable :: map \\
% complex(SPC), dimension(:,:,:), allocatable :: alm \\
% \ldots \\
% nside=256 ; lmax=512 ; mmax=lmax ; spin=4\\
% npix=nside2npix(nside)\\
% allocate(alm(1:2,0:lmax,0:mmax))\\
% allocate(map(0:npix-1,1:2))\\
\ldots \\
call \thedocid(nside, mask, nholes, nph)  \\
}
{???
}
\end{example}

\begin{modules}
  \begin{sulist}{} %%%% NOTE the ``extra'' brace here %%%%
  \item[\textbf{mask\_tools}] mask processing module (see related routines below)
  \end{sulist}
\end{modules}

\begin{related}
  \begin{sulist}{} %%%% NOTE the ``extra'' brace here %%%%
	\maskToolsRelated
  \end{sulist}
\end{related}

\rule{\hsize}{2mm}

\newpage
