
\sloppy


\title{\healpix Fortran Subroutines Overview}
\docid{gaussbeam} \section[gaussbeam]{ }
\label{sub:gaussbeam}
\docrv{Version 2.0}
\author{Eric Hivon}
\abstract{This document describes the \healpix Fortran90 subroutine GAUSSBEAM.}

\begin{facility}
{This routine generates the beam window function in multipole space of a
  gaussian beam parametrized by its FWHM. The
polarization beam is also provided assuming a perfectly
co-polarized beam (eg, Challinor et al 2000, astro-ph/0008228)}
{\modAlmTools}
\end{facility}

\begin{f90format}
{fwhm\_arcmin, lmax, beam}
\end{f90format}

\begin{arguments}
{
\begin{tabular}{p{0.35\hsize} p{0.05\hsize} p{0.05\hsize} p{0.45\hsize}} \hline  
\textbf{name~\&~dimensionality} & \textbf{kind} & \textbf{in/out} & \textbf{description} \\ \hline
                   &   &   &                           \\ %%% for presentation
fwhm\_arcmin & DP & IN & FWHM of the gaussian beam in arcminutes. \\
lmax & I4B & IN & maximum $\ell$ value of the window function.   \\
beam(0:lmax,1:p) & DP & OUT & beam window function generated. The second index runs form 1:1 for temperature only, and 1:3 for polarisation. In the latter case, 1=T, 2=E, 3=B.\\
\end{tabular}
}
\end{arguments}

\begin{example}
{
call gaussbeam(5.0\_dp, 1024, beam)  \\
}
{
Generates the window function of a gaussian beam of FWHM = 5 arcmin, for $\ell
\leq 1024$.
}
\end{example}

%% \begin{modules}
%%   \begin{sulist}{} %%%% NOTE the ``extra'' brace here %%%%
%%   \end{sulist}
%% \end{modules}

\begin{related}
  \begin{sulist}{} %%%% NOTE the ``extra'' brace here %%%%
  \item[\htmlref{generate\_beam}{sub:generate_beam}] Routine returning a beam
  window function.
  \item[\htmlref{pixel\_window}{sub:pixel_window}] Routine returning a pixel
  window function.
  \end{sulist}
\end{related}

\rule{\hsize}{2mm}

\newpage
