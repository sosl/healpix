
\sloppy


\title{\healpix IDL Facility User Guidelines}
\docid{nest2uniq} \section[nest2uniq]{ }
\label{idl:nest2uniq}
\docrv{Version 1.0}
\author{E. Hivon}
\abstract{This document describes the \healpix facility
  NEST2UNIQ.}

\begin{facility}
{This IDL facility turns $\nside$ and (NESTED) pixel index into the Unique Identifier  . 
}
{src/idl/toolkit/nest2uniq.pro}
\end{facility}

\begin{IDLformat}
{nest2uniq, 
\mylink{idl:nest2uniq:nside}{Nside},
\mylink{idl:nest2uniq:pnest}{Pnest},
\mylink{idl:nest2uniq:puniq}{Puniq} 
[,%
\mylink{idl:nest2uniq:help}{/HELP}]}
\end{IDLformat}


\begin{qualifiers}
  \begin{qulist}{} %%%% NOTE the ``extra'' brace here %%%%

\item[{Nside}] \mytarget{idl:nest2uniq:nside} (IN, scalar or vector Integer) The \healpix $\nside$ parameter(s)
\item[{Pnest}] \mytarget{idl:nest2uniq:pnest} (IN, scalar or vector Integer) (NESTED scheme) pixel identification number(s) in the range \{0,$12\nside^2-1$\}. If \mylink{idl:nest2uniq:nside}{{\tt Nside}} is a scalar, \mylink{idl:nest2uniq:pnest}{{\tt Pnest}} can a be a scalar or a vector, if \mylink{idl:nest2uniq:nside}{{\tt Nside}} is a vector,
\mylink{idl:nest2uniq:pnest}{{\tt Pnest}} must be a vector of the same size
\item[{Puniq}] \mytarget{idl:nest2uniq:puniq} (OUT, same size as \mylink{idl:nest2uniq:pnest}{{\tt Pnest}}) The \healpix Unique pixel identifier(s). 
  \end{qulist}
\end{qualifiers}

\begin{keywords}
  \begin{kwlist}{} %%%% NOTE the ``extra'' brace here %%%%

\item[{/HELP}] \mytarget{idl:nest2uniq:help} If set, a documentation header is printed out, and the routine exits
  \end{kwlist}
\end{keywords}


\begin{codedescription}
{\thedocid\ turns the parameter $\nside$ (a power of 2) and the pixel index $p$ into the Unique ID number $u = p + 4 \nside^2$. See \htmlref{''The Unique Identifier scheme''}{intro:unique} in 
\linklatexhtml{''\healpix Introduction Document''}{intro.pdf}{intro.htm} for more details.
}
\end{codedescription}


\begin{example}
{
\begin{tabular}{ll} %%%% use this tabular format %%%%
nest2uniq, [1, 2, 4], [0, 0, 0], puniq\\
print, puniq
\end{tabular}
}
{
\begin{minipage}{11cm}
returns  \\
   4 \hskip 1cm 16 \hskip 1cm 64 \\
since the first pixels ($p=0$) at $\nside=$ 1, 2 and 4 are respectively the pixels with Unique ID numbers 4, 16 and 64.
\end{minipage}
}
\end{example}


\begin{related}
  \begin{sulist}{} %%%% NOTE the ``extra'' brace here %%%%
  \item[\htmlref{uniq2nest}{idl:uniq2nest}] Transforms Unique \healpix pixel ID number into Nside and Nested pixel number
  \item[\htmlref{pix2xxx,...}{idl:pix_tools}] to turn NESTED pixel index into sky coordinates and back
  \end{sulist}
\end{related}

\rule{\hsize}{2mm}

