% -*- LaTeX -*-

\sloppy

\title{\healpix IDL Facility User Guidelines}
\docid{iprocess\_mask} \section[iprocess\_mask]{ }
\label{idl:iprocess_mask}
\docrv{Version 1.0}
\author{Eric Hivon}
\abstract{This document describes the \healpix IDL facility \thedocid.}

\begin{facility}
{This IDL facility provides an interface to F90 '\htmlref{process\_mask}{fac:process_mask}' facility. For a
given input binary mask, it can determine the angular distance in Radians of each valid (1 valued)
pixel to the closest invalid (0 valued) pixel, with the option of ignoring small
clusters of invalid pixels. The distance map can then be used to generate an
apodized mask.}
{src/idl/interfaces/iprocess\_mask.pro}
\end{facility}

\begin{IDLformat}
{IPROCESS\_MASK, 
\mylink{idl:iprocess_mask:mask_in}{mask\_in},  
\mylink{idl:iprocess_mask:distance_map}{distance\_map},[
\mylink{idl:iprocess_mask:binpath}{binpath=},
\mylink{idl:iprocess_mask:filled_mask}{filled\_mask=},
\mylink{idl:iprocess_mask:help}{/help},
\mylink{idl:iprocess_mask:hole_arcmin2}{hole\_arcmin2=},
\mylink{idl:iprocess_mask:hole_pixels}{hole\_pixels=},
\mylink{idl:iprocess_mask:keep_tmp_files}{keep\_tmp\_files=}, 
\mylink{idl:iprocess_mask:nested}{/nested},
\mylink{idl:iprocess_mask:ordering}{ordering=}, 
\mylink{idl:iprocess_mask:ring}{/ring},
\mylink{idl:iprocess_mask:silent}{/silent},
\mylink{idl:iprocess_mask:tmpdir}{tmpdir=}]}
\end{IDLformat}

\begin{qualifiers}
  \begin{qulist}{} %%%% NOTE the ``extra'' brace here %%%%
   \item[mask\_in] \mytarget{idl:iprocess_mask:mask_in} required input: input binary
mask. It can be a FITS file, or a memory array containing the mask to process.
    \item[distance\_map] \mytarget{idl:iprocess_mask:distance_map} optional
output: double precision angular distance map in Radians. It can be a FITS file, or a
   memory array. It will have the same ordering as the input mask.
  \end{qulist}
\end{qualifiers}

\begin{keywords}
  \begin{kwlist}{} %%% extra brace
 \item[binpath=] \mytarget{idl:iprocess_mask:binpath} full path to back-end routine \default {\$HEXE/process\_mask, then \$HEALPIX/bin/process\_mask}\\
              -- a binpath starting with / (or $\backslash$), $~$ or \$ is interpreted as absolute\\
              -- a binpath starting with ./ is interpreted as relative to current directory\\
              -- all other binpathes are relative to \$HEALPIX

 \item[filled\_mask=] \mytarget{idl:iprocess_mask:filled_mask} optional output
mask with holes smaller than 
\mylink{idl:iprocess_mask:hole_arcmin2}{{\tt hole\_arcmin2}} or 
\mylink{idl:iprocess_mask:hole_pixels}{{\tt hole\_pixels}} filled in.
     Will have the same ordering as the input mask

 \item[/help]      \mytarget{idl:iprocess_mask:help} if set, prints extended help

 \item[hole\_arcmin2]   \mytarget{idl:iprocess_mask:hole_arcmin2} Minimal size
(in arcmin$^2$) of invalid regions to be kept 
    (can be used together with \mylink{idl:iprocess_mask:hole_pixels}{{\tt hole\_pixels}}, 
     the result will be the largest of the two). \default{0.0}

 \item[hole\_pixels]   \mytarget{idl:iprocess_mask:hole_pixels} Minimal size (in pixels) of invalid regions to be kept 
    (can be used together with \mylink{idl:iprocess_mask:hole_arcmin2}{{\tt hole\_arcmin2}}, 
     the result will be the largest of the two). \default{0}

\item[/keep\_tmp\_files] \mytarget{idl:iprocess_mask:keep_tmp_files} if set,
temporary files are not discarded at the end of the run

\item[/nested] \mytarget{idl:iprocess_mask:nested} if set, signals that the mask read online is in
   NESTED scheme (does not apply to FITS file), see also
\mylink{idl:iprocess_mask:ring}{{\tt /ring}} and 
\mylink{idl:iprocess_mask:ordering}{{\tt Ordering}}

\item[ordering=] \mytarget{idl:iprocess_mask:ordering} either 'RING' or 'NESTED', ordering of online mask,
 see 
\mylink{idl:iprocess_mask:ring}{{\tt /ring}} and 
\mylink{idl:iprocess_mask:nested}{{\tt /nested}}

\item[/ring] \mytarget{idl:iprocess_mask:ring} see 
\mylink{idl:iprocess_mask:nested}{{\tt /nested}} and 
\mylink{idl:iprocess_mask:ordering}{{\tt Ordering}} above

\item[/silent]    \mytarget{idl:iprocess_mask:silent} if set, works silently

\item[tmpdir=]      \mytarget{idl:iprocess_mask:tmpdir} directory in which are written temporary files 
\default {IDL\_TMPDIR (see IDL documentation)}

  \end{kwlist}
\end{keywords}  

\begin{codedescription}
{\thedocid\ is an interface to '\htmlref{process\_mask}{fac:process_mask}' F90 facility. It
requires some disk space on which to write the parameter file and the other
temporary files. Most data can be provided/generated as an external FITS
file, or as a memory array.}
\end{codedescription}



\begin{related}
  \begin{sulist}{} %%%% NOTE the ``extra'' brace here %%%%
    \item[idl] version \idlversion or more is necessary to run \thedocid.
    \item[process\_mask] F90 facility called by \thedocid.
    \item[\htmlref{ialteralm}{idl:ialteralm}] IDL Interface to F90 \htmlref{alteralm}{fac:alteralm}
    \item[\htmlref{ianafast}{idl:ianafast}] IDL Interface to F90 \htmlref{anafast}{fac:anafast} and C++ anafast\_cxx
%    \item[\htmlref{iprocess\_mask}{idl:iprocess_mask}] IDL Interface to F90 \htmlref{process\_mask}{fac:process_mask}
    \item[\htmlref{ismoothing}{idl:ismoothing}] IDL Interface to F90 \htmlref{smoothing}{fac:smoothing}
    \item[\htmlref{isynfast}{idl:isynfast}] IDL Interface to F90 \htmlref{synfast}{fac:synfast}
  \end{sulist}
\end{related}

\begin{example}
{
\begin{tabular}{l} %%%% use this tabular format %%%%
 npix = nside2npix(256) \\
 mask = replicate(1, npix) \& mask[randomu(seed,100)*npix] = 0 \\
 \thedocid, mask, distance, /ring, /silent \\
 mollview, distance
\end{tabular}
}
{
A binary mask in which 100 randomly located pixels are 0-valued
(=invalid) is generated. Then the distance (in Radians) of the valid pixels to
the closest invalid pixels is computed and plotted.
}
\end{example}


