

\sloppy


\title{\healpix Fortran Subroutines Overview}
\docid{scan\_directories} \section[scan\_directories]{ }
\label{sub:scan_directories}
\docrv{Version 1.0}
\author{E. Hivon}
\abstract{This document describes the \healpix Fortran90 subroutine SCAN\_DIRECTORIES.}

\begin{facility}
{Function to scan a set of directories for a given file
}
{\modParamfileIo}
\end{facility}

\begin{f90function}
{\mylink{sub:scan_directories:directories}{directories}%
, \mylink{sub:scan_directories:filename}{filename}%
, \mylink{sub:scan_directories:fullpath}{fullpath}%
}
\end{f90function}

\begin{arguments}
{
\begin{tabular}{p{0.3\hsize} p{0.05\hsize} p{0.1\hsize} p{0.45\hsize}} \hline  
\textbf{name\&dimensionality} & \textbf{kind} & \textbf{in/out} & \textbf{description} \\ \hline
                   &   &   &                           \\ %%% for presentation
directories\mytarget{sub:scan_directories:directories} & CHR & IN & contains the set of directories (up to 20), separated by an ASCII
                   character of value $<$ 32  (see {\tt{\htmlref{concatnl}{sub:concatnl}}}). During the
                   search, it is assumed that the
                   given directories and filename can be separated by nothing,
                   a $/$ (slash) or a $\backslash$ (backslash)\\
filename\mytarget{sub:scan_directories:filename} & CHR & IN & the file to be found. \\
fullpath\mytarget{sub:scan_directories:fullpath} & CHR & OUT & returns the full path to the first occurrence of the
                   file among the directories provided. Empty if the file is not
                   found. The search is not recursive.  \\
var & LGT & OUT & set to true if the file is found, to false otherwise.\\
\end{tabular}
}
\end{arguments}

\begin{example}
{
use paramfile\_io \\
character(len=filenamelen) :: dirs, full \\
logical(lgt) :: found \\
dirs = concatnl('dir1','$/$dir2','$/$dir2$/$subdir1$/$') ! build directories list. \\
found = \thedocid(dirs, 'myfile', full) ! do the search \\
if (found) print*,trim(full)
}
{{Search for 'myfile' in the directories  'dir1', '$/$dir2', '$/$dir2$/$subdir1$/$'}}
\end{example}

\begin{related}
  \begin{sulist}{} %%%% NOTE the ``extra'' brace here %%%%
  \item[\htmlref{parse\_xxx}{sub:parse_xxx}] parse an ASCII file for parameters definition
  \item[\htmlref{concatnl}{sub:concatnl}] concatenates a set of substrings into one string, interspaced
  with LineFeed character
  \end{sulist}
\end{related}

\rule{\hsize}{2mm}

