
\sloppy


\title{\healpix Fortran Subroutines Overview}
\docid{dump\_alms*} \section[dump\_alms*]{ }
\label{sub:dump_alms}
\docrv{Version 2.1}
\author{Frode K.~Hansen}
\abstract{This document describes the \healpix Fortran90 subroutine DUMP\_ALMS.}

\begin{facility}
{This routine stores  $a_{lm}$  values in a binary FITS file. The FITS file created will contain one integer column with $index=\ell^2+\ell+m+1$ and 2 single precision columns with real/imaginary  $a_{lm}$  values. One can store temperature $a_{lm}$ or polarisation, $a^E_{lm}$ or $a^B_{lm}$. If temperature is specified, a FITS file is created. If polarisation is specified, an old FITS file is opened and extra extensions is created.}
{src/f90/mod/fitstools.F90}
\end{facility}

\begin{f90format}
{filename, alms, nlmax, header, nlheader, extno}
\end{f90format}

\begin{arguments}
{
\begin{tabular}{p{0.4\hsize} p{0.05\hsize} p{0.1\hsize} p{0.35\hsize}} \hline  
\textbf{name~\&~dimensionality} & \textbf{kind} & \textbf{in/out} & \textbf{description} \\ \hline
                   &   &   &                           \\ %%% for presentation
filename(LEN=\filenamelen) & CHR & IN & filename for the FITS-file to store the $a_{lm}$ in. \\
nlmax & I4B & IN & maximum $\ell$ value to store. \\
alms(0:nlmax,0:nlmax) & SPC/ DPC & IN & array with $a_{lm}$, in the format used
by eg. \htmlref{map2alm}{sub:map2alm}, so {\tt alms(l,m)} corresponds to  $a_{lm}$  \\
extno & I4B & IN & extension number. If 0 is specified, a FITS file is created and $a_{lm}$ is stored in the first FITS extension as temperature $a_{lm}$. If 1 or 2 is specified, an already existing file is opened and a 2nd or 3rd extension is created, treating $a_{lm}$ as $a_{lm}^E$ or $a_{lm}^B$. \\
nlheader & I4B & IN & number of header lines to write to the file. \\
header(LEN=80) (1:nlheader) & CHR & IN & the header to the FITS-file. \\ 
\end{tabular}
}
\end{arguments}

\begin{example}
{
call dump\_alms ('alms.fits', alms, 64, header, 100, 1)  \\
}
{
Opens an already existing FITS file which contains temperature $a_{lm}$. An extra extension is added to the file where the $a_{lm}$ array are written in a three-column format as described above. 100 header lines are written to the file from the array header(1:80). 
}
\end{example}

\begin{modules}
  \begin{sulist}{} %%%% NOTE the ``extra'' brace here %%%%
  \item[\textbf{fitstools}] module, containing:
  \item[printerror] routine for printing FITS error messages.
  \item[\textbf{cfitsio}] library for FITS file handling.		
  \end{sulist}
\end{modules}

\begin{related}
  \begin{sulist}{} %%%% NOTE the ``extra'' brace here %%%%
  \item[\htmlref{fits2alms}{sub:fits2alms}, \htmlref{read\_conbintab}{sub:read_conbintab}] routines to read $a_{lm}$ from a FITS-file 
  \item[\htmlref{alms2fits}{sub:alms2fits}] has the same function as \thedocid\ but is more general.
  \end{sulist}
\end{related}

\rule{\hsize}{2mm}

\newpage
