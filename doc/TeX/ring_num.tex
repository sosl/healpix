

\sloppy

\title{\healpix Fortran Subroutines Overview}
\docid{ring\_num} \section[ring\_num]{ }
\label{sub:ring_num}
\docrv{Version 1.1}
\author{Frode K.~Hansen}
\abstract{This document describes the \healpix Fortran90 function RING\_NUM.}


\begin{facility}
{This function returns the ring number for a z coordinate.}
{src/f90/mod/pix\_tools.f90}
\end{facility}

\begin{f90function}
{nside, z}
\end{f90function}

\begin{arguments}
{
\begin{tabular}{p{0.4\hsize} p{0.05\hsize} p{0.1\hsize} p{0.35\hsize}} \hline  
\textbf{name\&dimensionality} & \textbf{kind} & \textbf{in/out} & \textbf{description} \\ \hline
                   &   &   &                           \\ %%% for presentation
nside & I4B & IN & the $N_{side}$ parameter of the map. \\
z & DP & IN & the z coordinate to find the ring number for. \\

\end{tabular}
}
\end{arguments}

\begin{example}
{
print*,ring\_num(256, 0.5)  \\
}
{
Prints the ring number of the ring at position $z=0.5$.
}
\end{example}

\begin{modules}
  \begin{sulist}{} %%%% NOTE the ``extra'' brace here %%%%
 \item[None]	
  \end{sulist}
\end{modules}
\newpage
\begin{related}
  \begin{sulist}{} %%%% NOTE the ``extra'' brace here %%%%
 \item[\htmlref{in\_ring}{sub:in_ring}] Returns the pixels in a slice on a given ring.
  \end{sulist}
\end{related}

\rule{\hsize}{2mm}

\newpage
