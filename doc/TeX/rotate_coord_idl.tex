% -*- LaTeX -*-


\renewcommand{\facname}{{rotate\_coord}}
\renewcommand{\FACNAME}{{ROTATE\_COORD}}

\sloppy



\title{\healpix IDL Facility User Guidelines}
\docid{\facname} \section[\facname]{ }
\label{idl:rotate_coord}
\docrv{Version 1.0}
\author{Eric Hivon}
\abstract{This document describes the \healpix IDL facility \facname.}




\begin{facility}
{This IDL facility provides a means to rotate a set of 3D position
vectors (and their Stokes parameters Q and U) between to astrophysical coordinate systems
or by an arbitrary rotation.
}
{src/idl/misc/\facname.pro}
\end{facility}

\begin{IDLformat}
{%
\mylink{idl:rotate_coord:outvec}{Outvec} = \FACNAME(%
\mylink{idl:rotate_coord:invec}{Invec} [, 
\mylink{idl:rotate_coord:help}{/Help}, %
\mylink{idl:rotate_coord:euler_matrix}{Euler\_Matrix=}, %
\mylink{idl:rotate_coord:inco}{Inco=}, %
\mylink{idl:rotate_coord:outco}{Outco=}, % 
\mylink{idl:rotate_coord:stokes_parameters}{Stokes\_Parameters=}%
] )}
\end{IDLformat}

\begin{qualifiers}
  \begin{qulist}{} %%%% NOTE the ``extra'' brace here %%%%
    \item[Invec] \mytarget{idl:rotate_coord:invec}
      input,  array of size (n,3) : set of 3D position vectors
    \item[Outvec] \mytarget{idl:rotate_coord:outvec}
     output, array of size (n,3) : rotated 3D vectors
    \item[Euler\_Matrix=] \mytarget{idl:rotate_coord:euler_matrix}
       input, array of size (3,3). Euler Matrix
       describing the rotation to apply to vectors.
       \default{unity : no rotation}.\\
       Can not be used together with a change in coordinates.
    \item[Inco=] \mytarget{idl:rotate_coord:inco}
       input, character string (either 'Q' or 'C': equatorial,
    'G': galactic or 'E': ecliptic) describing the input coordinate system 
    \item[Outco=] \mytarget{idl:rotate_coord:outco}
        input, character string (see above) describing the output
          coordinate system.\\
    Can not be used together with Euler\_Matrix
    \item[Stokes\_Parameters=]\mytarget{idl:rotate_coord:stokes_parameters}
       input and output, array of size (n, 2) :
      values of the Q and U Stokes parameters on the sphere for each of
      the input position vector. Q and U are defined wrt the local
      parallel and meridian and are therefore transformed in a non
      trivial way in case of rotation
  \end{qulist}
\end{qualifiers}

\begin{keywords}
  \begin{kwlist}{} %%% extra brace
    \item[/Help] \mytarget{idl:rotate_coord:help}
     if set, the documentation header is printed and the routine exits	
  \end{kwlist}
\end{keywords}  

\begin{codedescription}
{\thedocid \ is a generalisation of the Astro library routine {\tt skyconv}. It allows
a rotation of 3D position vectors between two standard astronomic coordinates
system but also an arbitrary rotation described by its Euler Matrix.
It can also be applied to compute the effect of a rotation on the
linear polarization Stokes parameters (Q and U) expressed in local
coordinates system at the location of each of the input 3D vectors.}
\end{codedescription}



\begin{related}
  \begin{sulist}{} %%%% NOTE the ``extra'' brace here %%%%
    \item[idl] version \idlversion or more is necessary to run \thedocid.
    \item[\htmlref{euler\_matrix\_new}{idl:euler_matrix_new}] constructs the Euler Matrix for a set of
    three angles and three axes of rotation
  \end{sulist}
\end{related}

% \begin{example}
% {
% \begin{tabular}{ll} %%%% use this tabular format %%%%
% \end{tabular}
% }
% {
% }
% \end{example}

