% -*- LaTeX -*-


\renewcommand{\facname}{{rotate\_coord}}
\renewcommand{\FACNAME}{{ROTATE\_COORD}}

\sloppy



\title{\healpix IDL Facility User Guidelines}
\docid{\facname} \section[\facname]{ }
\label{idl:rotate_coord}
\docrv{Version 1.0}
\author{Eric Hivon}
\abstract{This document describes the \healpix IDL facility \facname.}




\begin{facility}
{This IDL facility provides a means to rotate a set of 3D position
vectors (and their Stokes parameters $Q$ and $U$) between to astrophysical coordinate systems
or by an arbitrary rotation.
}
{src/idl/misc/\facname.pro}
\end{facility}

\begin{IDLformat}
{%
\mylink{idl:rotate_coord:outvec}{Outvec} = \facname(%
\mylink{idl:rotate_coord:invec}{Invec}  
[, \mylink{idl:rotate_coord:delta_psi}{Delta\_Psi=}, %
\mylink{idl:rotate_coord:euler_matrix}{Euler\_Matrix=}, %
\mylink{idl:rotate_coord:inco}{Inco=}, %
\mylink{idl:rotate_coord:outco}{Outco=}, % 
\mylink{idl:rotate_coord:stokes_parameters}{Stokes\_Parameters=}, %
\mylink{idl:rotate_coord:free_norm}{/free\_norm}, %
\mylink{idl:rotate_coord:help}{/help}%
] )}
\end{IDLformat}

\begin{qualifiers}
  \begin{qulist}{} %%%% NOTE the ``extra'' brace here %%%%
%
    \item[Invec] \mytarget{idl:rotate_coord:invec}
      input,  array of size (n,3) : set of 3D position vectors
%
    \item[Outvec] \mytarget{idl:rotate_coord:outvec}
     output, array of size (n,3) : rotated 3D vectors, with the same norms as the input vectors
% 
   \item[Delta\_Psi]\mytarget{idl:rotate_coord:delta_psi}
       output, vector of size (n) containing the change in azimuth $\Delta\psi$ in Radians
	resulting from the rotation 
       (\htmlref{measured with respect to the local meridian, from South to East}{intro:fig:reftqu}),
        so that for field of spin $s$ the output $Q',U'$ is related to the input $Q,U$ via
                  $Q' = Q \cos (s\Delta\psi)  -  U \sin (s\Delta\psi),\ $
                  $U' = U \cos (s\Delta\psi)  +  Q \sin (s\Delta\psi), $
       with $s=2$ for polarization Stokes parameters 
(for which the specific \mylink{idl:rotate_coord:stokes_parameters}{Stokes\_Parameters} is also available).
%
   \item[Euler\_Matrix=] \mytarget{idl:rotate_coord:euler_matrix}
       input, array of size (3,3). Euler Matrix
       describing the rotation to apply to vectors.
       \default{identity : no rotation}.\\
       Can \emph{not} be used together with a change in coordinates.
%
    \item[Inco=] \mytarget{idl:rotate_coord:inco}
       input, character string (either 'Q' or 'C': equatorial,
    'G': galactic or 'E': ecliptic) describing the input coordinate system 
%
    \item[Outco=] \mytarget{idl:rotate_coord:outco}
        input, character string (see above) describing the output
          coordinate system.\\
    Can not be used together with \mylink{idl:rotate_coord:euler_matrix}{Euler\_Matrix}
%
    \item[Stokes\_Parameters=]\mytarget{idl:rotate_coord:stokes_parameters}
       input and output, array of size (n, 2) :
      values of the $Q$ and $U$ Stokes parameters on the sphere for each of
      the input position vector. \htmlref{$Q$ and $U$ are defined wrt the local
      meridian and parallel}{intro:fig:reftqu} and are therefore transformed in a
      non-trivial way in case of rotation
%
%
  \end{qulist}
\end{qualifiers}

\begin{keywords}
  \begin{kwlist}{} %%% extra brace
%
     \item[/free\_norm]\mytarget{idl:rotate_coord:free_norm} if set 
         (and \mylink{idl:rotate_coord:stokes_parameters}{Stokes\_Parameters} and/or 
         \mylink{idl:rotate_coord:delta_psi}{Delta\_Psi} are present)
        the input (and output) coordinate vectors are \emph{not} assumed to be normalized to 1.
        Using this option is therefore safer, but 20 to 30\% slower.
        (Note that 3D vectors produced by 
        \htmlref{\texttt{ang2vec}}{idl:ang2vec}, 
	\htmlref{\texttt{pix2vec\_nest}}{idl:pix_tools} and 
        \htmlref{\texttt{pix2vec\_ring}}{idl:pix_tools} \emph{are} properly normalized).
        Ignored when \mylink{idl:rotate_coord:stokes_parameters}{Stokes\_Parameters} and
         \mylink{idl:rotate_coord:delta_psi}{Delta\_Psi} are both absent.
%
    \item[/help] \mytarget{idl:rotate_coord:help}
     if set, the documentation header is printed and the routine exits
%	
  \end{kwlist}
\end{keywords}  

\begin{codedescription}
{\thedocid{} is a generalisation of the Astro library routine {\tt skyconv}. It allows
a rotation of 3D position vectors between two standard astronomic coordinates
system but also an arbitrary active rotation described by its Euler Matrix.
It can also compute how the linear polarization Stokes parameters ($Q$ and $U$, 
expressed in local coordinates system)
of each input location are affected by the solid body rotation, or equivalently
it can output the corresponding change in azimuth.}
\end{codedescription}



\begin{related}
  \begin{sulist}{} %%%% NOTE the ``extra'' brace here %%%%
    \item[idl] version \idlversion or more is necessary to run \thedocid.
    \item[\htmlref{euler\_matrix\_new}{idl:euler_matrix_new}] constructs the Euler Matrix for a set of
    three angles and three axes of rotation
    \item[\htmlref{\texttt{ang2vec}}{idl:ang2vec}, \htmlref{\texttt{pix2vec\_*}}{idl:pix_tools}] can be used to generate the input 3D vectors
  \end{sulist}
\end{related}

% \begin{example}
% {
% \begin{tabular}{ll} %%%% use this tabular format %%%%
% \end{tabular}
% }
% {
% }
% \end{example}

