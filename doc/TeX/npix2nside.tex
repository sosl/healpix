

\sloppy


\title{\healpix Fortran Subroutines Overview}
\docid{npix2nside} \section[npix2nside]{ }
\label{sub:npix2nside}
\docrv{Version 1.1}
\author{E. Hivon}
\abstract{This document describes the \healpix Fortran90 subroutine NPIX2NSIDE.}

\begin{facility}
{Function to provide the resolution parameter $\nside$ correspoonding to $\npix$
pixels over the full sky. 
}
{\modPixTools}
\end{facility}

\begin{f90function}
{npix}
\end{f90function}

\begin{arguments}
{
\begin{tabular}{p{0.3\hsize} p{0.05\hsize} p{0.1\hsize} p{0.45\hsize}} \hline  
\textbf{name~\&~dimensionality} & \textbf{kind} & \textbf{in/out} & \textbf{description} \\ \hline
                   &   &   &                           \\ %%% for presentation
npix & I4B/ I8B & IN & the number $\npix$ of pixels over the whole sky. \\
var & I4B & OUT & the parameter $\nside$. If $\npix$ is valid (12 times a power of 2 in
$\{1,\ldots,2^{28}\}$), $\nside=\sqrt{\npix/12}$ is returned; if not, an error message is
issued and -1 is returned.\\
\end{tabular}
}
\end{arguments}

\begin{example}
{
use \htmlref{healpix\_modules}{sub:healpix_modules} \\
integer(\htmlref{I4B}{sub:healpix_types}) :: nside \\
nside= npix2nside(786432)  \\
}
{
Returns the resolution parameter $\nside$ (256) corresponding to 786432 pixels
on the sky.
}
\end{example}
\begin{related}
  \begin{sulist}{} %%%% NOTE the ``extra'' brace here %%%%
  \item[\htmlref{nside2npix}{sub:nside2npix}] returns the number of pixels $\npix$ correspondinng to
  resolution parameter $\nside$
  \end{sulist}
\end{related}

\rule{\hsize}{2mm}

