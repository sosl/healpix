% -*- LaTeX -*-


\sloppy

\title{\healpix IDL Facility User Guidelines}
\docid{beam2bl} \section[beam2bl]{ }
\label{idl:beam2bl}
\docrv{Version 1.0}
\author{Eric Hivon}
\abstract{This document describes the \healpix IDL facility beam2bl.}

\begin{facility}
{This IDL facility computes a transfer (or window) function $b(l)$ for a circular beam profile $b(\theta)$.
}
{src/idl/misc/beam2bl.pro}
\end{facility}

\begin{IDLformat}
{%\mylink{idl:beam2bl:beam}
{bl}=\thedocid(
\mylink{idl:beam2bl:beam}{beam}, 
\mylink{idl:beam2bl:theta}{theta}, 
\mylink{idl:beam2bl:lmax}{lmax}, 
[%
\mylink{idl:beam2bl:arcmin}{/ARCMIN} ,
\mylink{idl:beam2bl:degrees}{/DEGREES}, 
\mylink{idl:beam2bl:help}{/HELP}, 
\mylink{idl:beam2bl:radians}{/RADIANS}%
])}
\end{IDLformat}

\begin{qualifiers}
  \begin{qulist}{} %%%% NOTE the ``extra'' brace here %%%%
%     \item[beam]\mytarget{idl:beam2bl:bl}%
% 	output (circular) beam profile $b(\theta)$
    \item[beam] \mytarget{idl:beam2bl:beam}%
      input beam profile $b(\theta)$ 
    \item[theta] \mytarget{idl:beam2bl:theta}%
    angles $\theta$ (in arcmin, degrees or radians) 
  at which the input beam $b(\theta)$ is defined
    \item[lmax] \mytarget{idl:beam2bl:lmax}%
    maximum multipole on which the output $b(l)$ is to be computed
  \end{qulist}
\end{qualifiers}

\begin{keywords}
  \begin{kwlist}{} %%% extra brace
    \item[/ARCMIN] \mytarget{idl:beam2bl:arcmin}%
	if set, \mylink{idl:beam2bl:theta}{$\theta$} is in arcmin
    \item[/DEGREES] \mytarget{idl:beam2bl:degrees}%
	if set, \mylink{idl:beam2bl:theta}{$\theta$} is in degrees
    \item[/HELP] \mytarget{idl:beam2bl:help}%
	if set, prints out the help header and exits
    \item[/RADIANS] \mytarget{idl:beam2bl:radians}%
	if set, \mylink{idl:beam2bl:theta}{$\theta$} is in radians
  \end{kwlist}
\end{keywords}  

\begin{codedescription}
{%
Since the SH Transform of an arbitrary beam is
\begin{equation}
	b_{lm} = \int d{\bf{r}}\ b({\bf{r}})\ Y_{lm}^*({\bf{r}})
\end{equation}
 then, for a circular beam
\begin{eqnarray}
	b(l) &=& b_{l0}  \sqrt{\frac{4 \pi}{2l+1}} \nonumber \\
             &=& \int  b(\theta) P_l(\theta) \sin(\theta)\ d\theta\ 2\pi
\end{eqnarray}
where $P_l$ is the Legendre Polynomial, $b(l)$ is the beam window (or transfer)
function returned by \thedocid\ and $b(\theta)$ is the beam radial
profile expected as input of \thedocid.\\
IDL's routine {\tt{INT\_TABULATED}} is used to perform the integration.}
\end{codedescription}



\begin{related}
  \begin{sulist}{} %%%% NOTE the ``extra'' brace here %%%%
    \item[idl] version \idlversion or more is necessary to run \thedocid.
    \item[\htmlref{bl2beam}{idl:bl2beam}] facility to performs the inverse
transform to \thedocid.
    \item[\htmlref{bl2fits}{idl:bl2fits}] facility to write a $b(l)$ window function into a FITS file.
    \item[\htmlref{fits2cl}{idl:fits2cl}] facility to read a $b(l)$ window
function from a FITS file
  \end{sulist}
\end{related}

\begin{example}
{
\begin{tabular}{l} %%%% use this tabular format %%%%
bl = \htmlref{gaussbeam}{idl:gaussbeam}(15.d0, 4000, 1) \\
theta = dindgen(4000)/100. \\
beam = \htmlref{bl2beam}{idl:bl2beam}(bl, theta, /arcmin)\\
bl1 = \thedocid(beam, theta, 4000, /arcmin) \\
plot, bl1-bl\\
\end{tabular}
}
{
the example above generates a beam window function (defined for
all $l$ in $\{0,\ldots,4000\}$) for a 15arcmin-FWHM gaussian beam, computes the
beam profile for angles in $[0,40]$ arcmin, computes back the beam window
function from the beam profile and finally plots the difference between the beam
window functions.%
}
\end{example}



