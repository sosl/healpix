% -*- LaTeX -*-

% PLEASE USE THIS FILE AS A TEMPLATE FOR THE DOCUMENTATION OF YOUR OWN
% FACILITIES: IN PARTICULAR, IT IS IMPORTANT TO NOTE COMMENTS MADE IN
% THE TEXT AND TO FOLLOW THIS ORDERING. THE FORMAT FOLLOWS ONE USED BY
% THE COBE-DMR PROJECT.	
% A.J. Banday, April 1999.




\sloppy



\title{\healpix IDL Facility User Guidelines}
\docid{azeqview} \section[azeqview]{ }
\label{idl:\thedocid}
\docrv{Version 1.2}
\author{Eric Hivon}
\abstract{This document describes the \healpix facility azeqview.}



\begin{facility}
{This IDL facility provides a means to visualise an azimuthal equidistant projection 
 of
\healpix and COBE Quad-Cube maps in an IDL environment. 
It also offers the possibility to
generate GIF, JPEG, PNG and Postscript color-coded images of the projected map.
The projected (but not color-coded) data can also be output in FITS files and
IDL arrays.}
{src/idl/visu/azeqview.pro}
\end{facility}

\begin{IDLformat}
{AZEQVIEW, 
\normalsize{
\mylink{idl:mollview:file}{File}, 
[ \mylink{idl:mollview:select}{Select}, ]
[ \mylink{idl:mollview:asinh}{/ASINH}, 
\mylink{idl:mollview:charsize}{CHARSIZE=}, 
\mylink{idl:mollview:charsize}{CHARTHICK=}, 
\mylink{idl:mollview:colt}{COLT=}, 
\mylink{idl:mollview:coord}{COORD=}, 
\mylink{idl:mollview:crop}{/CROP}, 
\mylink{idl:mollview:execute}{EXECUTE=}, 
\mylink{idl:mollview:factor}{FACTOR=}, 
\mylink{idl:mollview:fits}{FITS=}, 
\mylink{idl:mollview:flip}{/FLIP}, 
\mylink{idl:mollview:gal_cut}{GAL\_CUT=}, 
\mylink{idl:mollview:gif}{GIF=}, 
\mylink{idl:mollview:glsize}{GLSIZE=}, 
\mylink{idl:mollview:graticule}{GRATICULE=}, 
\mylink{idl:mollview:half_sky}{/HALF\_SKY}, 
\mylink{idl:mollview:hbound}{HBOUND=}, 
\mylink{idl:mollview:help}{/HELP}, 
\mylink{idl:mollview:hist_equal}{/HIST\_EQUAL}, 
\mylink{idl:mollview:hxsize}{HXSIZE=}, 
\mylink{idl:mollview:iglsize}{IGLSIZE=}, 
\mylink{idl:mollview:igraticule}{IGRATICULE=}, 
\mylink{idl:mollview:log}{/LOG}, 
\mylink{idl:mollview:map_out}{MAP\_OUT=}, 
\mylink{idl:mollview:max}{MAX=}, 
\mylink{idl:mollview:min}{MIN=}, 
\mylink{idl:mollview:nested}{/NESTED}, 
\mylink{idl:mollview:no_dipole}{/NO\_DIPOLE}, 
\mylink{idl:mollview:no_monopole}{/NO\_MONOPOLE}, 
\mylink{idl:mollview:nobar}{/NOBAR}, 
\mylink{idl:mollview:nolabels}{/NOLABELS}, 
\mylink{idl:mollview:noposition}{/NOPOSITION}, 
\mylink{idl:mollview:offset}{OFFSET=}, 
\mylink{idl:mollview:outline}{OUTLINE=}, 
\mylink{idl:mollview:png}{PNG=}, 
\mylink{idl:mollview:polarization}{POLARIZATION=}, 
\mylink{idl:mollview:preview}{/PREVIEW}, 
\mylink{idl:mollview:ps}{PS=}, 
\mylink{idl:mollview:pxsize}{PXSIZE=}, 
\mylink{idl:mollview:pysize}{PYSIZE=}, 
\mylink{idl:mollview:reso_arcmin}{RESO\_ARCMIN=}, 
\mylink{idl:mollview:retain}{RETAIN=}, 
\mylink{idl:mollview:rot}{ROT=}, 
\mylink{idl:mollview:save}{/SAVE}, 
\mylink{idl:mollview:shaded}{/SHADED}, 
\mylink{idl:mollview:silent}{/SILENT}, 
\mylink{idl:mollview:subtitle}{SUBTITLE=}, 
\mylink{idl:mollview:titleplot}{TITLEPLOT=}, 
\mylink{idl:mollview:transparent}{TRANSPARENT=}, 
\mylink{idl:mollview:truecolors}{TRUECOLORS=}, 
\mylink{idl:mollview:units}{UNITS=}, 
\mylink{idl:mollview:window}{WINDOW=}, 
\mylink{idl:mollview:xpos}{XPOS=}, 
\mylink{idl:mollview:ypos}{YPOS=}]
}
}
\end{IDLformat}

\begin{qualifiers}
  \begin{qulist}{} %%%% NOTE the ``extra'' brace here %%%%
\item [{\  }] For a full list of qualifiers see \htmlref{mollview}{idl:mollview}
  \end{qulist}
\end{qualifiers}

\begin{keywords}
  \begin{kwlist}{} %%%% NOTE the ``extra'' brace here %%%%
\item [{\  }] For a full list of keywords see \htmlref{mollview}{idl:mollview}
  \end{kwlist}
\end{keywords}


\begin{codedescription}
{\thedocid\ reads in a \healpix sky map in FITS format and generates an
azimuthal equidistant projection of it, that can be visualized on the screen or
exported in a GIF, JPEG, PNG or Postscript file. azeqview allows the selection of
the coordinate system, point of projection, map size, color table, color bar inclusion,
linear or log scaling, histogram equalised
color scaling, maximum and 
minimum range for the plot, plot-title {\it etc}. It also allows the representation of the
polarization field. }
\end{codedescription}

%
% defines the field 'RELATED' for mollview, gnomview, orthview,
% cartview

\begin{related}
  \begin{sulist}{} %%%% NOTE the ``extra'' brace here %%%%
  \item[idl] version \idlversion or more is necessary to run \thedocid
  \item[ghostview] ghostview or a similar facility is required to view
	  the Postscript image generated by \thedocid.
  \item[xv] xv or a similar facility is required to view the
            GIF/PNG image generated by \thedocid  (a browser can also 
            be used).
  \item[synfast] This \healpix facility will generate the FITS format 
            sky map to be input to \thedocid.
  \item[{\htmlref{cartview}{idl:cartview}}] 
	IDL facility to generate a Cartesian projection of
  	a \healpix map.
  \item[{\htmlref{cartcursor}{idl:cartcursor}}] 
	interactive cursor to be used with cartview
  \item[{\htmlref{gnomview}{idl:gnomview}}] 
	IDL facility to generate a gnomonic projection of
  	a \healpix map.
  \item[{\htmlref{gnomcursor}{idl:gnomcursor}}] 
	interactive cursor to be used with gnomview
  \item[{\htmlref{mollview}{idl:mollview}}] 
	IDL facility to generate a Mollweide projection of
  	a \healpix map.
  \item[{\htmlref{mollcursor}{idl:mollcursor}}] interactive cursor to be used with mollview
  \item[{\htmlref{orthview}{idl:orthview}}] 
	IDL facility to generate an orthographic projection of
  	a \healpix map.
  \item[{\htmlref{orthcursor}{idl:orthcursor}}] 
	interactive cursor to be used with orthview
  \end{sulist}
\end{related}

\begin{related}
  \begin{sulist}{} %%%% NOTE the ``extra'' brace here %%%%
  \item[{\ }] see \htmlref{mollview}{idl:mollview}
  \item[\htmlref{hpx2dm}{idl:hpx2dm}] turns Healpix maps into DomeMaster images
using \thedocid.
  \end{sulist}
\end{related}

% \begin{example}
% {
% \begin{tabular}{ll} %%%% use this tabular format %%%%
% azeqview, & \lq planck100GHZ-LFI.fits', rot=[160,-30], reso\_arcmin=2., \$ \\
%           & pxsize = 500., title='Simulated Planck LFI Sky Map at 100GHz' \\
% \end{tabular}
% }
% {azeqview reads in the map $\lq$ planck100GHZ-LFI.fits' and generates
% an output 500$\times$500 image, with a resolution of 2 arcmin/pixel at
% the center, in which
% the temperature scale has been set to lie between $\pm$ 100 ($\mu$K), 
% and the title $\lq$ Simulated Planck
% LFI Sky Map at 100GHz' appended to the image. The map is centered on 
% ($l=$160, $b=$-30) }
% \end{example}

% \begin{example}
% {
% \begin{tabular}{l} %%%% use this tabular format %%%%

% map  = findgen(48) \\
% triangle= create\_struct('coord','G','ra',[0,80,0],'dec',[40,45,65]) \\
% azeqview,map,/online,res=45,graticule=[45,30],rot=[10,20,30],pysize=250,\$ \\
% $\quad$	   title='Cartesian cylindrical (full sky)',subtitle='azeqview', \$ \\
% $\quad$           outline=triangle \\
% \end{tabular}
% }
% {makes a cartesian cylindrical projection of map (see Figure~\ref{fig:plot_visu}a on
% page~\pageref{page:plot_visu}) after an arbitrary rotation, with a graticule grid
% (with a 45$^o$ step in longitude and 30$^o$ in latitude) and an arbitrary triangular outline}
% \end{example}

\newpage
