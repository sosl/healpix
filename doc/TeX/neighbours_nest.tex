
\sloppy

\title{\healpix Fortran Subroutines Overview}
\docid{neighbours\_nest} \section[neighbours\_nest]{ }
\label{sub:neighbours_nest}
\docrv{Version 2.0}
\author{}
\abstract{This document describes the \healpix Fortran90 subroutine NEIGHBOURS\_NEST.}

\begin{facility}
{This subroutine returns the number and locations (in terms of pixel
numbers) of the topological neighbours of a central pixel. The pixels
are ordered in a clockwise sense about the central pixel with the
southernmost pixel in first element. For the 4 pixels in the southern corners of the
equatorial faces which have two equally southern neighbours the
routine returns the southwestern pixel first and proceeds clockwise.}
{\modPixTools}
\end{facility}

\begin{f90format}
{\mylink{sub:neighbours_nest:nside}{nside}%
, \mylink{sub:neighbours_nest:ipix}{ipix}%
, \mylink{sub:neighbours_nest:list}{list}%
, \mylink{sub:neighbours_nest:nneigh}{nneigh}%
}
\end{f90format}

\begin{arguments}
{
\begin{tabular}{p{0.4\hsize} p{0.05\hsize} p{0.1\hsize} p{0.35\hsize}} \hline  
\textbf{name~\&~dimensionality} & \textbf{kind} & \textbf{in/out} & \textbf{description} \\ \hline
                   &   &   &                           \\ %%% for presentation
nside\mytarget{sub:neighbours_nest:nside} & I4B & IN & The $\nside$ parameter of the map. \\
ipix\mytarget{sub:neighbours_nest:ipix} & I4B/ I8B & IN & The NESTED pixel index of the central pixel. \\
list\mytarget{sub:neighbours_nest:list}(8) & I4B/ I8B & OUT & On exit, the vector of neighbouring pixels. This
                   contains {\tt nneigh} relevant elements.\\
nneigh\mytarget{sub:neighbours_nest:nneigh} & I4B & OUT & The number of neighbours (mostly 8, except at
                   8 sites, where there are only 7 neighbours).\\
\end{tabular}
}
\end{arguments}

\begin{example}
{
use pix\_tools \\
integer :: nneigh, list(1:8) \\
call neighbours\_nest(4, 1, list, nneigh)  \\
print*,nneigh \\
print*,list(1:nneigh)
}
{
This returns {\tt nneigh}$=8$ and a vector {\tt list}, which contains the pixel
numbers ( 90,  0,  2,  3,  6,  4,  94,  91).}
\end{example}

\begin{modules}
  \begin{sulist}{} %%%% NOTE the ``extra'' brace here %%%%
 \item[mk\_xy2pix, mk\_pix2xy] precomputing arrays for the conversion
 of NESTED pixel numbers to Cartesian coords in each face.
 \item[pix2xy\_nest, xy2pix\_nest] Conversion between NESTED pixel numbers to Cartesian coords in each face.
 \item[\textbf{bit\_manipulation}] module, containing:
 \item[invMSB, invLSB,swapLSBMSB,invswapLSBMSB] functions which manipulate the bit vector which
 represents the NESTED pixel numbers. They correspond to
 NorthWest<->SouthEast, SouthWest<->NorthEast, East<->West and
 North-South flips of the diamond faces, respectively.
  \end{sulist}
\end{modules}

\begin{related}
  \begin{sulist}{} %%%% NOTE the ``extra'' brace here %%%%
  \item[\htmlref{pix2ang}{sub:pix_tools}, \htmlref{ang2pix}{sub:pix_tools}] convert between angle and pixel number.
  \item[\htmlref{pix2vec}{sub:pix_tools}, \htmlref{vec2pix}{sub:pix_tools}] convert between a cartesian vector and pixel number.
  \end{sulist}
\end{related}

\rule{\hsize}{2mm}


