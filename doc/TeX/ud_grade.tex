
\sloppy


\title{\healpix Fortran Facility User Guidelines}
\docid{ud\_grade} \section[ud\_grade]{\nosectionname}
\label{fac:ud_grade}
\docrv{Version 1.1}
\author{Frode K.~Hansen}
\abstract{This document describes the \healpix facility UD\_GRADE.}

\begin{facility}
{This program can upgrade or degrade the resolution of a \healpix map.} 
{src/f90/ud\_grade/ud\_grade.f90}
\end{facility}

\begin{f90facility}
{ud\_grade [options] [parameter\_file]}
\end{f90facility}

\begin{options}
  \begin{optionlistwide}{} %%%% NOTE the ``extra'' brace here %%%%
    \item[{\tt -d}]
    \item[{\tt -}{\tt -}{\tt double}] double precision mode (see Notes on double/single precision modes on page~\pageref{page:ioprec})
    \item[{\tt -s}]
    \item[{\tt -}{\tt -}{\tt single}] single precision mode (default)
  \end{optionlistwide}
\end{options}

\begin{qualifiers}
  \begin{qulist}{} %%%% NOTE the ``extra'' brace here %%%%
    \item[{nside\_out = }] Defines the resolution parameter for the output map. 
	(default= 64)
\item[{infile = }] Defines the name of the file containing the map to be 
up/degraded.
(default='map.fits')
\item[{outfile = }] Defines the filename for the output up/degraded map.
(default='outmap.fits')
  \end{qulist}
\end{qualifiers}

\begin{codedescription}
{
This facility transforms the resolution of an input \healpix map.
At each step of map resolution upgrade the four output map pixels nested 
in  one pixel of 
the input map are given  the values of the input pixel. 
At each step of map resolution degradation
the four input map pixels nested in one output map pixel
are averaged to produce the pixel
value in the output map.
{\bf Caution} Beware that, at this stage, the parallel tranport of the polarization
(Q and U Stokes vectors) that would be necessary to describe the change
in local coordinates is {\bf not} implemented.
}
\end{codedescription}

\begin{datasets}
{
\begin{tabular}{p{0.3\hsize} p{0.35\hsize}} \hline  
  \textbf{Dataset} & \textbf{Description} \\ \hline
                   &                      \\ %%% for presentation
 None required & \\
                   &                      \\ \hline %%% for presentation
\end{tabular}
} 
\end{datasets}

\begin{support}
  \begin{sulist}{} %%%% NOTE the ``extra'' brace here %%%%
  \item[mollview] IDL routine to view an up/downgraded map.
  \item[\htmlref{anafast}{fac:anafast}] This \healpix facility can analyse an up/downgraded map.
  \end{sulist}
\end{support}

\begin{examples}{1}
{
\begin{tabular}{ll} %%%% use this tabular format %%%%
ud\_grade  \\
\end{tabular}
}
{
ud\_grade runs in interactive mode, self-explanatory.
}
\end{examples}

\vfill\newpage

\begin{examples}{2}
{
\begin{tabular}{ll} %%%% use this tabular format %%%%
ud\_grade  filename \\
\end{tabular}
}
{When `filename' is present, ud\_grade enters the non-interactive mode and parses
its inputs from the file `filename'. This has the following
structure: the first entry is a qualifier which announces to the parser
which input immediately follows. If this input is omitted in the
input file, the parser assumes the default value.
If the equality sign is omitted, then the parser ignores the entry.
In this way comments may also be included in the file.
In this example, the file contains the following qualifiers:\hfill\newline
\fileparam{nside\_out = 64}
\fileparam{infile= map.fits}
\fileparam{outfile = outmap.fits}
}

Ud\_grade transforms the \healpix map in 'map.fits' 
to resolution $Nside=64$, no matter what the input map resolution was. 
The up/downgraded map is stored in 'outmap.fits'.
\end{examples}

\begin{release}
  \begin{relist}
    \item (Initial release \healpix 0.90)
    \item Extension to multi-dimensional maps (\healpix 1.20)
  \end{relist}
\end{release}

\begin{messages}
{
\begin{tabular}{p{0.25\hsize} p{0.1\hsize} p{0.35\hsize}} \hline  
  \textbf{Message} & \textbf{Severity} & \textbf{Text} \\ \hline
                   &                   &   \\ %%% for presentation
can not allocate memory for array xxx &  Fatal & You do not have
                   sufficient system resources to run this
                   facility at the map resolution you required. 
  Try a lower map resolution.  \\ 
                   &                   &   \\ \hline %%% for presentation
\end{tabular}
} 
\end{messages}

\rule{\hsize}{2mm}

\newpage
