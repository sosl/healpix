\sloppy
\docid{alm2map\_der*}\section[alm2map\_der*]{ }
\label{sub:alm2map_der}
\docrv{Version 2.0}
\author{Eric Hivon}
\abstract{This document describes the \healpix Fortran90 subroutine ALM2MAP\_DER*.}

\begin{facility}
{This routine is a wrapper to four other routines that synthesize a \healpix
  temperature (and polarisation) map(s), its (their) first derivatives, and optionally
  its (their) second derivatives.
The routines accept both single and double precision arrays for alm, map, der1 and
der2. The precision of these arrays should match. All maps produced are RING
ordered.

See \linklatexhtml{''Fortran
Facilities''}{facilities.pdf}{facilities.htm} Appendix for a note on a bug
affecting the calculation of polarisation derivatives on past versions of this routine.
}
{\modAlmTools}
\end{facility}

\begin{f90format}
{\mylink{sub:alm2map_der:nsmax}{nsmax}%
, \mylink{sub:alm2map_der:nlmax}{nlmax}%
, \mylink{sub:alm2map_der:nmmax}{nmmax}%
, \mylink{sub:alm2map_der:alm}{alm}%
, \mylink{sub:alm2map_der:map}{map}%
, \mylink{sub:alm2map_der:der1}{der1}%
 [, \mylink{sub:alm2map_der:der2}{der2}%
]}
\end{f90format}

\begin{arguments}
{
\begin{tabular}{p{0.35\hsize} p{0.05\hsize} p{0.1\hsize} p{0.40\hsize}} \hline  
\textbf{name~\&~dimensionality} & \textbf{kind} & \textbf{in/out} & \textbf{description} \\ \hline
                   &   &   &                           \\ %%% for presentation
nsmax\mytarget{sub:alm2map_der:nsmax} & I4B & IN & the $N_{side}$ value of the map to synthesize. \\
nlmax\mytarget{sub:alm2map_der:nlmax} & I4B & IN & the maximum $\ell$ value used for the $a_{lm}$. \\
nmmax\mytarget{sub:alm2map_der:nmmax} & I4B & IN & the maximum $m$ value used for the $a_{lm}$. \\
alm\mytarget{sub:alm2map_der:alm}(1:p, 0:nlmax, 0:nmmax) & SPC/ DPC & IN & The $a_{lm}$ values to make the map
                   from. p is either 1 (temperature only) or 3 (temperature+polarisation).\\
%% \end{tabular}
%% \begin{tabular}{p{0.4\hsize} p{0.05\hsize} p{0.1\hsize} p{0.35\hsize}}  \hline  
map\mytarget{sub:alm2map_der:map}(0:12*nsmax**2-1) \hskip 3cm {\em or\ \ }(0:12*nsmax**2-1,1:3)      & SP/ DP & OUT & temperature
map\mytarget{sub:alm2map_der:map} $T(p)$ or temperature + polarisation maps $T(p)$, $Q(p)$, $U(p)$ to be synthesized.  \\ 
der1\mytarget{sub:alm2map_der:der1}(0:12*nsmax**2-1, 1:2*p) & SP/ DP & OUT &  contains on output the first
derivatives of T: $\left({\partial T}/{\partial \theta}, {\partial T}/{\partial \phi}/\sin\theta \right)\myhtmlimage{}
$ or the interleaved derivatives of T, Q, and U: $\left({\partial T}/{\partial
  \theta}, {\partial Q}/{\partial \theta}, {\partial U}/{\partial \theta};\right.\myhtmlimage{}$
$\left.{\partial T}/{\partial \phi}/\sin\theta, \ldots \right)\myhtmlimage{}
$\\ 
der2\mytarget{sub:alm2map_der:der2}(0:12*nsmax**2-1,1:3*p), \hskip 3cm OPTIONAL & SP/ DP & OUT & If this optional
matrix is passed with this rank, it will contain on output the second derivatives
$\left({\partial^2 T}/{\partial \theta^2}, {\partial^2 T}/{\partial
  \theta\partial\phi}/\sin\theta,\right.\myhtmlimage{}$ 
$\left.{\partial^2 T}/{\partial \phi^2}/\sin^2\theta \right)\myhtmlimage{}$ or
$\left({\partial^2 T}/{\partial \theta^2}, {\partial^2 Q}/{\partial \theta^2},
{\partial^2 Q}/{\partial \theta^2}, \ldots \right)\myhtmlimage{}$ 

\end{tabular}
}
\end{arguments}

\begin{example}
{
use healpix\_types \\
use pix\_tools, only : nside2npix \\
use alm\_tools, only : alm2map\_der \\
integer(I4B) :: nside, lmax, mmax, npix, n\_plm\\
real(SP), dimension(:), allocatable :: map \\
real(SP), dimension(:,:), allocatable :: der1, der2 \\
complex(SPC), dimension(:,:,:), allocatable :: alm \\
\ldots \\
nside=256 ; lmax=512 ; mmax=lmax\\
npix=nside2npix(nside)\\
allocate(alm(1:1,0:lmax,0:mmax))\\
allocate(map(0:npix-1))\\
allocate(der1(0:npix-1,1:2), der2(0:npix-1,1:3))\\
\ldots \\
call alm2map\_der(nside, lmax, mmax, alm, map, der1, der2)  \\
}
{
Make temperature maps and its derivatives from the $a_{lm}$ passed in alm. The maps have $N_{side}$ of 256, and are constructed from $a_{lm}$ values up to 512 in $\ell$ and $m$.
}
\end{example}

\begin{modules}
  \begin{sulist}{} %%%% NOTE the ``extra'' brace here %%%%
  \item[\htmlref{ring\_synthesis}{sub:ring_synthesis}] Performs FFT over $m$ for synthesis of the rings.
  \item[compute\_lam\_mm, get\_pixel\_layout, ]
  \item[gen\_lamfac\_der, gen\_mfac,  ] 
  \item[gen\_recfac, init\_rescale, l\_min\_ylm] Ancillary routines used
  for $\ {_s}Y_{\ell m}$ recursion
  \item[\textbf{misc\_utils}] module, containing:
  \item[assert\_alloc] routine to print error message, when an array can not be
  allocated properly
  \end{sulist}
\end{modules}

\begin{related}
  \begin{sulist}{} %%%% NOTE the ``extra'' brace here %%%%
   \item[\htmlref{alm2map}{sub:alm2map}] routine generating maps of temperature
   and polarisation from their  $a_{\ell m}$
   \item[\htmlref{alm2map\_spin}{sub:alm2map_spin}] routine generating maps of
arbitrary spin from their  ${_s}a_{\ell m}$
%%   \item[smoothing] executable using \thedocid\ to smooth maps
  \item[synfast] executable using \thedocid\ to synthesize maps.
%%   \item[\htmlref{map2alm}{sub:map2alm}] routine performing the inverse transform
%%   of alm2map.
  \item[\htmlref{create\_alm}{sub:create_alm}] routine to generate randomly
  distributed $a_{\ell m}$ coefficients according to a given power spectrum
  \end{sulist}
\end{related}

\rule{\hsize}{2mm}

\newpage
