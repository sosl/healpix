
\sloppy


\title{\healpix Fortran Subroutines Overview}
\docid{in\_ring} \section[in\_ring]{ }
\label{sub:in_ring}
\docrv{Version 1.3}
% \author{Frode K.~Hansen}
\author{Eric Hivon}
\abstract{This document describes the \healpix Fortran90 subroutine IN\_RING.}

\begin{facility}
{Routine to find the pixel index of all pixels on a slice of a given
ring. The output indices can be either in the RING or NESTED scheme,
depending on the {\tt nest} keyword.}
{\modPixTools}
\end{facility}

\begin{f90format}
{\mylink{sub:in_ring:nside}{nside}%
, \mylink{sub:in_ring:iz}{iz}%
, \mylink{sub:in_ring:phi0}{phi0}%
, \mylink{sub:in_ring:dphi}{dphi}%
, \mylink{sub:in_ring:listir}{listir}%
, \mylink{sub:in_ring:nir}{nir}%
, \mylink{sub:in_ring:nest}{nest}%
}
\end{f90format}

\begin{arguments}
{
\begin{tabular}{p{0.4\hsize} p{0.05\hsize} p{0.1\hsize} p{0.35\hsize}} \hline  
\textbf{name~\&~dimensionality} & \textbf{kind} & \textbf{in/out} & \textbf{description} \\ \hline
                   &   &   &                           \\ %%% for presentation
nside\mytarget{sub:in_ring:nside} & I4B & IN & the $N_{side}$ parameter of the map. \\
iz\mytarget{sub:in_ring:iz} & I4B & IN & ring number, counted southwards from the north pole. \\
phi0\mytarget{sub:in_ring:phi0} & DP & IN & central $\phi$ position in the slice. \\
dphi\mytarget{sub:in_ring:dphi} & DP & IN & defines the size of the slice. The slice has length $2\times dphi$ along the ring with center at $phi0$. \\ 
listir\mytarget{sub:in_ring:listir}(0:4*nside-1) & I4B/ I8B & OUT & The pixel indexes in the slice. \\
nir\mytarget{sub:in_ring:nir} & I4B & OUT & the number of pixels in the slice. {\tt nir}$\le 4\nside$\\
nest\mytarget{sub:in_ring:nest}\ \ (OPTIONAL) & I4B & IN &  The pixel indexes are in the NESTED numbering
scheme if {\tt nest}=1, and in RING scheme otherwise. \\
\end{tabular}
}
\end{arguments}

\begin{example}
{
call in\_ring(256, 10, 0, 0.1, listir, nir, nest=1)  \\
}
{
Returns the NESTED pixel index of all pixels within 0.1 radians on each side of $\phi=0$ on the 10th ring.
}
\end{example}

\newpage
\begin{modules}
  \begin{sulist}{} %%%% NOTE the ``extra'' brace here %%%%
 \item[\htmlref{ring2nest}{sub:pix_tools}] conversion from RING scheme pixel index to NESTED scheme pixel index
 \item[next\_in\_line\_nest] returns NESTED index of pixel lying to the East of the
 current pixel and on the same ring
  \end{sulist}
\end{modules}

\begin{related}
  \begin{sulist}{} %%%% NOTE the ``extra'' brace here %%%%
  \item[\htmlref{pix2ang}{sub:pix_tools}, \htmlref{ang2pix}{sub:pix_tools}] convert between angle and pixel number.
  \item[\htmlref{pix2vec}{sub:pix_tools}, \htmlref{vec2pix}{sub:pix_tools}] convert between a cartesian vector and pixel number.
  \item[\htmlref{getdisc\_ring}{sub:getdisc_ring}] find all pixels within a certain radius.
  \end{sulist}
\end{related}

\rule{\hsize}{2mm}

\newpage
