% -*- LaTeX -*-


\renewcommand{\facname}{{getsize\_fits }}
\renewcommand{\FACNAME}{{GETSIZE\_FITS }}
\sloppy



\title{\healpix IDL Facility User Guidelines}
\docid{\facname} \section[\facname]{ }
\label{idl:getsize_fits}
\docrv{Version 1.3}
\author{Eric Hivon}
\abstract{This document describes the \healpix facility \facname.}

%%\newcommand{\nside}{N_{\rm side}}

\begin{facility}
{This IDL function reads the number of maps and/or the pixel ordering of a FITS file containing a \healpix map.}
{src/idl/fits/getsize\_fits.pro}
\end{facility}

\begin{IDLformat}
{var = \FACNAME(File, [Nmaps=, Nside=, Mlpol=, Ordering=, Obs\_Npix=, Type=, Header=, Extension=, /Help])
}
\end{IDLformat}

\begin{qualifiers}
  \begin{qulistwide}{} %%%% NOTE the ``extra'' brace here %%%%
 	\item[{File}] 
          name of a FITS file containing the \healpix map(s).

	\item[{var}] {contains on output the number of pixels stored in a map FITS file.
     Each pixel is counted only once 
     (even if several information is stored on each of them, see nmaps).
     Depending on the data storage format, result may be : \\
       -- equal or smaller to the number Npix of Healpix pixels available 
          over the sky for the given resolution (Npix =
     12*nside*nside) \\
       -- equal or larger to the number of non blank pixels 
         (obs\_npix)}

	
 	\item[{Nmaps=}] contains on output the number of maps in the file
	

       \item[{Nside=}] contains on output the \healpix resolution parameter $\nside$
		  

       \item[{Mlpol=}] contains on output the maximum multipole used to generate the map 

	\item[{Ordering=}] contains on output the pixel ordering
	scheme: either 'RING' or 'NESTED'
		  
	\item[{Obs\_Npix=}] contains on output the number of non blanck pixels. It is set to -1 if it can not be determined from header
		
	\item[{Type=}] {Healpix/FITS file type\\
             $<$0 : file not found, or not valid\\
             0  : image only fits file, deprecated Healpix format
                   (var $=12\nside^2$) \\
             1  : ascii table, generally used for C(l) storage \\
             2  : binary table : with implicit pixel indexing (full sky)
                   (var $=12\nside^2$) \\
             3  : binary table : with explicit pixel indexing (generally cut sky)
                   (var $\le 12\nside^2$) \\
           999  : unable to determine the type }

	\item[{Header=}] contains on output the FITS extension header
		
       \item[{Extension=}]
	extension unit to be read from FITS file: 
 either its 0-based ID number (ie, 0 for first extension {\em after} primary array) 
 or the case-insensitive value of its EXTNAME keyword.

  \end{qulistwide}
\end{qualifiers}

\begin{keywords}
  \begin{kwlist}{} %%% extra brace
   \item[{HELP=}] if set, an extensive help is displayed and no
	file is read
   \end{kwlist}
\end{keywords}

\begin{codedescription}
{\facname gets the number of pixels in a FITS file. If the file
follows the \healpix standard, the routine can also get the resolution
parameter Nside, the ordering scheme, ..., and can determine the type
of data set contained in the file.}
\end{codedescription}



\begin{related}
  \begin{sulist}{} %%%% NOTE the ``extra'' brace here %%%%
  \item[idl] version \idlversion or more is necessary to run \facname
  \item[\htmlref{read\_fits\_map}{idl:read_fits_map}] This \healpix IDL facility can be used to read in maps
  written by \facname.
  \item[sxaddpar] This IDL routine (included in \healpix package) can be used to update
  or add FITS keywords to {\tt Header}
  \item[\htmlref{reorder}{idl:reorder}] This \healpix IDL routine can be used to reorder a map from
  NESTED scheme to RING scheme and vice-versa.
  \item[\htmlref{write\_fits\_sb}{idl:write_fits_sb}] routine to write multi-column binary FITS table
  \end{sulist}
\end{related}


\begin{example}
{
\begin{tabular}{l} %%%% use this tabular format %%%%
 npix = getsize\_fits(!healpix.directory+'/test/map.fits', nside=nside, \$ \\
$\quad$       mlpol=lmax, type=filetype)\\
 print, npix, nside, lmax, filetype
\end{tabular}
}
{\parbox[t]{\hsize}{ should produce something like \\
   {\em 196608 \ \        128 \ \         256  \ \      2} \\
meaning that the map contained in that file has 196608 pixels, the resolution parameter is
nside=128, the maximum multipole was 256, and this a full sky map
(type 2).
}}
\end{example}

