
\sloppy


\title{\healpix Fortran Subroutines Overview}
\docid{query\_triangle} \section[query\_triangle]{ }
\label{sub:query_triangle}
\docrv{Version 1.3}
\author{Eric Hivon}
\abstract{This document describes the \healpix Fortran90 subroutine QUERY\_TRIANGLE.}

\begin{facility}
{Routine to find the index of all pixels enclosed in a spherical triangle described by its three vertices. The output indices can be either in the RING or NESTED scheme} 
{src/f90/mod/pix\_tools.F90}
\end{facility}

\begin{f90format}
{nside, v1, v2, v3, listpix, nlist [, nest, inclusive]}
\end{f90format}

\begin{arguments}
{
\begin{tabular}{p{0.28\hsize} p{0.05\hsize} p{0.1\hsize} p{0.47\hsize}} \hline 
\textbf{name\&dimensionality} & \textbf{kind} & \textbf{in/out} & \textbf{description} \\ \hline
                   &   &   &                           \\ %%% for presentation
nside & I4B & IN & the $\nside$ parameter of the map. \\
v1(3) & DP & IN & cartesian vector pointing at the triangle first vertex. \\
v2(3) & DP & IN & cartesian vector pointing at the triangle second vertex. \\
v3(3) & DP & IN & cartesian vector pointing at the triangle third vertex. \\
listpix(0:*) & I4B/ I8B & OUT & the index for all pixels enclosed in the triangle. Make sure that the size of the array is big enough to contain all pixels. \\ 
nlist & I4B/ I8B & OUT & The number of pixels listed in {\tt listpix}. \\
nest\ \ (OPTIONAL) & I4B & IN &  The pixel indices are in the NESTED numbering scheme if nest=1, and in RING scheme otherwise. \\
inclusive\ \ (OPTIONAL) & I4B & IN & If set to 1, all the pixels overlapping
                   (even partially)
                   with the triangle are listed, otherwise only those whose
                   center lies within the triangle are listed. \\
\end{tabular}
}
\end{arguments}

\begin{example}
{
call query\_triangle(256,(/1,0,0/),(/0,1,0/),(/0,0,1/),listpix,nlist)  \\
}
{
Returns the RING pixel index of the (98560) pixels in the octant ($x,y,z>0$) in a $\nside=256$ map.
}
\end{example}
%\newpage
\begin{modules}
  \begin{sulist}{} %%%% NOTE the ``extra'' brace here %%%%
 \item[\htmlref{in\_ring}{sub:in_ring}] routine to find the pixels in a certain slice of a given ring.		
 \item[intrs\_intrv] routine to compute the intersection of 2 intervals on a circle
 \item[\htmlref{ring\_num}{sub:ring_num}] function to return the ring number corresponding to the coordinate $z$
 \item[\htmlref{vect\_prod}{sub:vect_prod}] routine to compute the vectorial product of two 3D vectors
  \end{sulist}
\end{modules}

\begin{related}
  \begin{sulist}{} %%%% NOTE the ``extra'' brace here %%%%
  \item[\htmlref{pix2ang}{sub:pix_tools}, \htmlref{ang2pix}{sub:pix_tools}] convert between angle and pixel number.
  \item[\htmlref{pix2vec}{sub:pix_tools}, \htmlref{vec2pix}{sub:pix_tools}] convert between a cartesian vector and pixel number.
  \item[\htmlref{query\_disc}{sub:query_disc}, \htmlref{query\_polygon}{sub:query_polygon},]
  \item[\htmlref{query\_strip}{sub:query_strip}, query\_triangle] render the list of pixels enclosed
  respectively in a given disc, polygon, latitude strip and triangle
  \item[\htmlref{surface\_triangle}{sub:surface_triangle}] computes the surface
in steradians of a spherical triangle defined by 3 vertices

  \end{sulist}
\end{related}

\rule{\hsize}{2mm}

\newpage
