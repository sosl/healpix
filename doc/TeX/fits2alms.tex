
\sloppy


\title{\healpix Fortran Subroutines Overview}
\docid{fits2alms*} \section[fits2alms*]{ }
\label{sub:fits2alms}
\docrv{Version 2.0}
\author{Frode K.~Hansen, Eric Hivon}
\abstract{This document describes the \healpix Fortran90 subroutine FITS2ALMS.}

\begin{facility}
{This routine reads  $a_{lm}$  values from a binary FITS file. Each FITS file
  extension is supposed to contain one integer column with
  $index=\ell^2+\ell+m+1$  and 2 or 4 single (or double) precision columns 
with real/imaginary  $a_{lm}$  values and real/imaginary   standard deviation. 
One can read temperature $a_{lm}$ or temperature and polarisation, $a^T_{lm}$, $a^E_{lm}$ and $a^B_{lm}$.}
{\modFitstools}
\end{facility}

\begin{f90format}
{filename, nalms, alms, ncl, header, nlheader, next}
\end{f90format}

\begin{arguments}
{
\begin{tabular}{p{0.4\hsize} p{0.05\hsize} p{0.05\hsize} p{0.40\hsize}} \hline  
\textbf{name~\&~dimensionality} & \textbf{kind} & \textbf{in/out} & \textbf{description} \\ \hline
                   &   &   &                           \\ %%% for presentation
filename(LEN=\filenamelen) & CHR & IN & filename of the FITS-file to read the $a_{lm}$ from. \\
nalms & I4B & IN & number of $a_{lm}$ to read. \\
ncl & I4B & IN & number of columns to read in the FITS file. If an standard
               deviation is to be read, this number is 5, otherwise it is 3. \\ 
next & I4B & IN & the number of extensions to read. 1 for temperature only, 3
                   for temperature and polarisation. \\ 
\end{tabular}
\begin{tabular}{p{0.4\hsize} p{0.05\hsize} p{0.05\hsize} p{0.40\hsize}} \hline  
alms(1:nalms,1:(ncl+1),1:next) & SP/ DP & OUT & the $a_{lm}$ to read from the
          file. alms(i,1,j) and alms(i,2,j) contain the $\ell$ and $m$ values
          for the ith  $a_{lm}$  (j=1,2,3 for (T,E,B)). alms(i,3,j) and
          alms(i,4,j) contain the real and imaginary value of the ith
          $a_{lm}$. Finally, the   standard deviation for the ith  $a_{lm}$  is
          contained in alms(i,5,j) (real) and alms(i,6,j) (imaginary). \\  
nlheader & I4B & IN & number of header lines to read from the file. \\
header(LEN=80) (1:nlheader, 1:next) & CHR & OUT & the header(s) read from the FITS-file. \\ 
\end{tabular}
}
\end{arguments}

\begin{example}
{
call fits2alms ('alms.fits', 65*66/2, alms, 3, header, 80, 3)  \\
}
{
Reads a FITS file with the $a_{lm}^T$, $a_{lm}^E$ and $a_{lm}^B$ values read into alms(1:65*66/2,1:4,1:3). The last index specifies (T,E,B). The second index gives l, m, real( $a_{lm}$ ), imaginary( $a_{lm}$ ) for each of the $a_{lm}$. The number 65*66/2 is the number of  $a_{lm}$  values up to an $\ell$ value of 64. 80 lines is read from the header in each extension and returned in header(1:80,1:3).
}
\end{example}

\begin{modules}
  \begin{sulist}{} %%%% NOTE the ``extra'' brace here %%%%
  \item[read\_alms] routine called by \thedocid\ for each extension.
  \item[\textbf{fitstools}] module, containing:
  \item[printerror] routine for printing FITS error messages.
  \item[\textbf{cfitsio}] library for FITS file handling.		
  \end{sulist}
\end{modules}
\newpage
\begin{related}
  \begin{sulist}{} %%%% NOTE the ``extra'' brace here %%%%
  \item[\htmlref{alms2fits}{sub:alms2fits}, \htmlref{dump\_alms}{sub:dump_alms}] routines to store $a_{lm}$ in a FITS-file 
  \item[\htmlref{read\_conbintab}{sub:read_conbintab}] has the same function as
  \thedocid\ but with parameters passed differently.
  \item[\htmlref{number\_of\_alms}{sub:number_of_alms},[\htmlref{getsize\_fits}{sub:getsize_fits}]
  can be used to find out the number of $a_{lm}$ available in the file.
  \end{sulist}
\end{related}

\rule{\hsize}{2mm}

\newpage
