% -*- LaTeX -*-


\renewcommand{\facname}{{pix2xxx, ang2xxx,... }}
\renewcommand{\FACNAME}{{PIX2XXX, ANG2XXX,...}}

\sloppy



\title{\healpix IDL Facility User Guidelines}
\docid{pix2xxx, ang2xxx, vec2xxx, nest2ring, ring2nest} \section[pix2xxx,
ang2xxx, vec2xxx, nest2ring,ring2nest]{ }
\label{idl:pix_tools}
\docrv{Version 1.1}
\author{Eric Hivon} % same layout as Frode pix_tools
\abstract{This document describes the \healpix IDL facilities \facname}




\begin{facility}
{These routines provide conversion between pixel number in the \healpix map and $(\theta,\phi)$ or $(x,y,z)$ coordinates on the sphere. Some of these routines are listed here.
}
{src/idl/toolkit/}
\end{facility}

% \begin{IDLformat}
% {Npix=\FACNAME(Nside [,ERROR=])}
% \end{IDLformat}

\begin{qualifiers}
{
\begin{tabular}{p{0.15\hsize} p{0.15\hsize} p{0.1\hsize} p{0.50\hsize}} \hline  
\textbf{name (dim.)} & \textbf{type} & \textbf{in/out} & \textbf{description} \\ \hline
                   &   &   &                           \\ %%% for presentation
nside & scalar integer & IN & $N_{side}$ parameter for the \healpix map. \\
ipnest(n) & vector integer & --- & pixel identification number in NESTED scheme over the range \{0,$N_{pix}-1$\}. \\
ipring(n) & vector integer & --- & pixel identification number in RING scheme over the range \{0,$N_{pix}-1$\}. \\
theta(n) & vector double & --- & colatitude in radians measured southward from
                   north pole in \{0,$\pi$\}\\
phi(n) & vector double & --- & longitude in radians, measured eastward in \{0,$2\pi$\}. \\ 
vector(n,3) & array double & --- & three dimensional cartesian position vector
                   $(x,y,z)$. The north pole is $(0,0,1)$. An output vector is
                   normalised to unity. The coordinates are ordered as follows
                   $x(0),\ldots,x(n-1),\ y(0),\ldots,y(n-1),\ z(0),\ldots,z(n-1)$
                   \\
vertex(n,3,4) & array double & optional OUT & three dimensional cartesian position vector
                   $(x,y,z)$. Contains the location of the four vertices
                   (=corners) of a
                   pixel in the order North, West, South, East. The coordinates
                   are ordered as follows
                   $x_N(0),\ldots,x_N(n-1),\ y_N(0),\ldots,y_N(n-1),\ z_N(0),\ldots,z_N(n-1)$,
                   $x_W(0),\ldots,x_W(n-1),\ y_W(0),\ldots,y_W(n-1),\
                   z_W(0),\ldots,z_W(n-1)$,
			and so on with South and East vertices
\end{tabular}
}
\end{qualifiers}

\rule{\hsize}{0.7mm}
\textsc{\large{\textbf{ROUTINES: }}}\hfill\newline
{\tt pix2ang\_ring, nside, ipring, theta, phi} 

 \begin{tabular}{@{}p{0.3\hsize}@{\hspace{1ex}}
                        p{0.7\hsize}@{}}
                                         & renders $theta$ and $phi$ coordinates of the nominal pixel center given the pixel number $ipring$ and a map resolution parameter $nside$. \\
     \end{tabular}\\\\
{\tt pix2vec\_ring, nside, ipring, vector [,vertex]} 

 \begin{tabular}{@{}p{0.3\hsize}@{\hspace{1ex}}
                        p{0.7\hsize}@{}}
                                         & renders cartesian vector coordinates of
                        the nominal pixel center given the pixel number $ipring$
                        and a map resolution parameter $nside$. Optionally returns 
                        the location of the 4 vertices for the pixel(s) under
                        consideration \\
     \end{tabular}\\\\
{\tt ang2pix\_ring, nside, theta, phi, ipring} 

 \begin{tabular}{@{}p{0.3\hsize}@{\hspace{1ex}}
                        p{0.7\hsize}@{}}
                                         & renders the pixel number $ipring$ for a pixel which, given the map resolution parameter $nside$, contains the point on the sphere at angular coordinates $theta$ and $phi$. \\
     \end{tabular}\\\\
{\tt vec2pix\_ring, nside, vector, ipring} 

 \begin{tabular}{@{}p{0.3\hsize}@{\hspace{1ex}}
                        p{0.7\hsize}@{}}
                                         & renders the pixel number $ipring$ for a pixel which, given the map resolution parameter $nside$, contains the point on the sphere at cartesian coordinates $vector$. \\
     \end{tabular}\\\\
{\tt pix2ang\_nest, nside, ipnest, theta, phi} 

 \begin{tabular}{@{}p{0.3\hsize}@{\hspace{1ex}}
                        p{0.7\hsize}@{}}
                                         & renders $theta$ and $phi$ coordinates of the nominal pixel center given the pixel number $ipnest$ and a map resolution parameter $nside$. \\
     \end{tabular}\\\\
{\tt pix2vec\_nest, nside, ipnest, vector [,vertex]} 

 \begin{tabular}{@{}p{0.3\hsize}@{\hspace{1ex}}
                        p{0.7\hsize}@{}}
                                         & renders cartesian vector coordinates of
                        the nominal pixel center given the pixel number $ipnest$
                        and a map resolution parameter $nside$. Optionally returns
                        the location of the 4 vertices for the pixel(s) under consideration\\
     \end{tabular}\\\\
{\tt ang2pix\_nest, nside, theta, phi, ipnest} 

 \begin{tabular}{@{}p{0.3\hsize}@{\hspace{1ex}}
                        p{0.7\hsize}@{}}
                                         & renders the pixel number $ipnest$ for a pixel which, given the map resolution parameter $nside$, contains the point on the sphere at angular coordinates $theta$ and $phi$. \\
     \end{tabular}\\\\
{\tt vec2pix\_nest, nside, vector, ipnest} 

 \begin{tabular}{@{}p{0.3\hsize}@{\hspace{1ex}}
                        p{0.7\hsize}@{}}
                                         & renders the pixel number $ipnest$ for a pixel which, given the map resolution parameter $nside$, contains the point on the sphere at cartesian coordinates $vector$. \\
     \end{tabular}\\\\
{\tt nest2ring, nside, ipnest, ipring} 

 \begin{tabular}{@{}p{0.3\hsize}@{\hspace{1ex}}
                        p{0.7\hsize}@{}}
                                         & performs conversion from NESTED to RING pixel number. \\
     \end{tabular}\\\\
{\tt ring2nest, nside, ipring, ipnest} 

 \begin{tabular}{@{}p{0.3\hsize}@{\hspace{1ex}}
                        p{0.7\hsize}@{}}
                                         & performs conversion from RING to NESTED pixel number. \\
     \end{tabular}\\\\

% \begin{keywords}
%   \begin{kwlist}{} %%% extra brace
%     \item[ERROR =] error flag, set to 1 on output if Nside is NOT valid, or
%     stays to 0 otherwise.
%   \end{kwlist}
% \end{keywords}  

% \begin{codedescription}
% {\facname checks that the given Nside is valid (power of 2 in
% $\{1,\ldots,8192\}$) and then computes the
% corresponding number of pixels $N_{\rm pix} = 12N_{\rm side}^2$.}
% \end{codedescription}



\begin{related}
  \begin{sulist}{} %%%% NOTE the ``extra'' brace here %%%%
    \item[idl] version \idlversion or more is necessary to run \facname.	
    \item[\htmlref{npix2nside}{idl:npix2nside}] computes Nside (resolution) corresponding to Npix (total
    pixel number)
    \item[\htmlref{nside2npix}{idl:nside2npix}] computes Npix corresponding to Nside
    \item[\htmlref{ang2vec}{idl:ang2vec}, \htmlref{vec2ang}{idl:vec2ang}] geometrical conversion between position angles and position vector
  \end{sulist}
\end{related}

\begin{example}
{
\begin{tabular}{ll} %%%% use this tabular format %%%%
pix2ang\_ring, 256, [17,1000], theta, phi \\
print,theta,phi
\end{tabular}
}
{
\begin{minipage}{11cm}
returns  \\
    0.0095683558  \hskip 1cm   0.070182078 \\
       2.8797933  \hskip 1cm         5.4620872 \\
position of 2 pixels 17 and 1000 in the RING scheme with parameter 256.
\end{minipage}
}
\end{example}


