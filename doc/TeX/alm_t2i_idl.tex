% -*- LaTeX -*-


\sloppy

\title{\healpix IDL Facility User Guidelines}
\docid{alm\_t2i} \section[alm\_t2i]{ }
\label{idl:alm_t2i}
\docrv{Version 1.0}
\author{Eric Hivon}
\abstract{This document describes the \healpix IDL facility alm\_t2i.}

\begin{facility}
{This IDL facility turns a tabular (real or complex) a(l,m) array into an
indexed list of alm that can be written into a FITS file with \htmlref{alm2fits}{idl:alm2fits}%
}
{src/idl/misc/alm\_t2i.pro}
\end{facility}

\begin{IDLformat}
{\thedocid, %
\mylink{idl:alm_t2i:alm_table}{Alm\_table},
\mylink{idl:alm_t2i:index}{Index}, 
\mylink{idl:alm_t2i:alm_vec}{Alm\_vec},
[%
\mylink{idl:alm_t2i:help}{/HELP},
\mylink{idl:alm_t2i:mfirst}{/MFIRST}%
])}
\end{IDLformat}

\begin{qualifiers}
  \begin{qulist}{} %%%% NOTE the ``extra'' brace here %%%%
    \item[Alm\_table] \mytarget{idl:alm_t2i:alm_table}
Input real or complex array, containing all the $a^s_{lm}$ for $l$
         in [0,$l_{\rm max}$] and $m$ in [0,$m_{\rm max}$] 
(and $s$ in [0,$s_{\rm max}$] if applicable)\\
       if REAL    it has 3 (or 4) dimensions,\\
       if COMPLEX is has 2 (or 3) dimensions

    \item[Index] \mytarget{idl:alm_t2i:index}
Output integer vector of size ni containing the index $i$ of the 
            of $a_{lm}$ coefficients, related to $\{l,m\}$ by 
             $i = l^2 + l + m + 1$
    \item[Alm\_vec] \mytarget{idl:alm_t2i:alm_vec}%
Output array of $a_{lm}$ coefficients, with dimension (ni, 2 [,$s_{\rm max}+1$])
     where\\
           ni   = number of $i$ indices\\
           2 for real and imaginary parts of alm coefficients\\
           $s_{\rm max}+1$ = number of signals (usually 1 for any of T E B
                  or 3 for T,E,B together)
\end{qulist}
\end{qualifiers}

\begin{keywords}
  \begin{kwlist}{} %%% extra brace
    \item[/HELP] \mytarget{idl:alm_t2i:help}%
	if set, prints out the help header and exits
    \item[/MFIRST] \mytarget{idl:alm_t2i:mfirst}%
 if set, the input array is a(m,l) instead of a(l,m)
  \end{kwlist}
\end{keywords}  

\begin{codedescription}
{%
\thedocid\ turns a real or complex tabular array of a(l,m) (or a(m,l) is \htmlref{{\tt MFIRST}}{idl:alm_t2i:mfirst} is
set) into a real list of $a_{lm}$ (with the real and imaginary parts separated)
and its index $i=l^2+l+m+1$. The unphysical $m>l$ elements of the input
table are dropped from the output list.
% returns a real or complex array, containing the $a_{lm}$ with
% $0 \le l \le l_{\rm max}$ and $0 \le m \le m_{\rm max}$. The negative $m$ are
% therefore ignored.
}
\end{codedescription}



\begin{related}
  \begin{sulist}{} %%%% NOTE the ``extra'' brace here %%%%
    \item[idl] version \idlversion or more is necessary to run \thedocid.
    \item[\htmlref{alm\_i2t}{idl:alm_i2t}] this function is complementary to
\thedocid and
turns an indexed list of alm (as generated by
\htmlref{fits2alm}{idl:fits2alm}) into a tabular (real or complex) a(l,m) array
for easier manipulation
  \item[\htmlref{alm2fits}{idl:alm2fits}, \htmlref{fits2alm}{idl:fits2alm}]
routines to read and write $a_{lm}$ indexed lists from and to FITS files.
  \end{sulist}
\end{related}

\begin{example}
{%
}%
{%
See \mylink{idl:alm_i2t:example}{alm\_i2t example}
}
\end{example}


