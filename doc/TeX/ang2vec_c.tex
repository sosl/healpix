

\sloppy


\title{\healpix C Subroutines Overview}
\docid{ang2vec} \section[ang2vec]{ }
\label{csub:ang2vec}
\docrv{Version 2.0}
\author{E. Hivon}
\abstract{This document describes the \healpix C subroutine ANG2VEC.}

\begin{facility}
{Routine to convert the position angles $(\theta,\phi)\myhtmlimage{}$ of a point on the sphere 
into its 3D position vector $(x,y,z)$ with
$x = \sin\theta\cos\phi\myhtmlimage{}$, $y=\sin\theta\sin\phi\myhtmlimage{}$, $z=\cos\theta\myhtmlimage{}$. 
}
{src/C/subs/chealpix.c}
\end{facility}

\begin{Cfunction}
{void vec2ang(double theta, double phi, double *vector);}
\end{Cfunction}

\begin{arguments}
{
%% \begin{tabular}{p{0.4\hsize} p{0.05\hsize} p{0.1\hsize} p{0.35\hsize}} \hline  
\begin{tabular}{p{0.3\hsize} p{0.10\hsize} p{0.05\hsize} p{0.45\hsize}} \hline  
\textbf{name~\&~dimensionality} & \textbf{kind} & \textbf{in/out} & \textbf{description} \\ \hline
                   &   &   &                           \\ %%% for presentation
theta & double & IN & colatitude in radians measured southward from north pole (in
    [0,$\pi$]). \\
phi   & double & IN & longitude in radians measured eastward (in [0, $2\pi$]).\\
vector(3) & double & OUT & three dimensional cartesian position vector
                   $(x,y,z)$. The north pole is $(0,0,1)$\\
\end{tabular}
}
\end{arguments}


% \begin{example}
% {
% call ang2vec(theta,phi,vector) \\
% }
% {
% }
% \end{example}

\begin{related}
  \begin{sulist}{} %%%% NOTE the ``extra'' brace here %%%%
  \item[\htmlref{vec2ang}{csub:vec2ang}] converts the 3D position vector of point into its position
  angles on the sphere.
  \end{sulist}
\end{related}

\rule{\hsize}{2mm}

