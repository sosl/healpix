
\sloppy


\title{\healpix Fortran Subroutines Overview}
\docid{write\_bintabh} \section[write\_bintabh*]{ }
\label{sub:write_bintabh}
\docrv{Version 1.2}
\author{Eric Hivon, Frode K.~Hansen}
\abstract{This document describes the \healpix Fortran90 subroutine WRITE\_BINTABH.}

\begin{facility}
{This routine is designed to write large (or huge) arrays into a binary table
extension of a FITS file. The user can
choose to write the array piece by piece. This is designed to deal with Time
Ordered Data set (tod).}
{\modFitstools}
\end{facility}

\begin{f90format}
{tod, npix, ntod, header, nlheader, filename, \optional{[extno, firstpix, repeat]}}
\end{f90format}
\aboutoptional

\begin{arguments}
{
\begin{tabular}{p{0.35\hsize} p{0.05\hsize} p{0.08\hsize} p{0.45\hsize}} \hline  
\textbf{name~\&~dimensionality} & \textbf{kind} & \textbf{in/out} & \textbf{description} \\ \hline
                   &   &   &                           \\ %%% for presentation
tod(0:npix-1,1:ntod) & SP/ DP & IN & The map or tod
  to write to the FITS file. It will be written in the file at the location
                   corresponding to pixels (or time samples)
                   {\tt firstpix} to {\tt firtpix + npix} -1.\\
npix & I8B & IN & Number of pixels or time samples in the map or TOD. See Note below.\\
ntod & I4B & IN & Number of maps or tods to be written. Each of them will be in a different column of the FITS binary table.\\
header(LEN=80) (1:nlheader) & CHR & IN & The header for the FITS file. \\
nlheader & I4B & IN & number of header lines to write to the file. \\
filename(LEN=\filenamelen) & CHR & IN & The array is written into a FITS file with this filename. \\
\optional{extno} & I4B & IN & extension number in which to write the data (0
                   based).  \default 0 \\
\optional{firstpix} & I8B & IN & 0 Location in the FITS file of the first
                   pixel (or time sample) to be written (0 based). \default 
                   0. See Note below.
                   \\
\optional{repeat} & I4B & IN & \parbox[t]{0.99\hsize}{ 
		Length of the element vector used in the binary
                   table. \default{1024 if {\tt npix} $\propto 1024$; 12000 if
                   {\tt npix} $> 12000$ and 1 otherwise}. \\
 		Choosing a large {\tt
                   repeat} for multi-column tables ({\tt ntod} $>1$) generally
                   speeds up the I/O. It also helps bringing the number of rows
                   of the table under $2^{31}$, which is a hard limit of
                   cfitsio. \\
		   If the number of samples or pixels of each map or TOD is not a multiple of 
		{\tt repeat}, then the last element vector will be padded with sentinel values 
\mylink{sub:healpix_types:hpx_sbadval}{\tt HPX\_SBADVAL} or
\mylink{sub:healpix_types:hpx_dbadval}{\tt HPX\_DBADVAL}.}
\end{tabular}
{\bf Note :} Indices and number of data elements larger than
                   $2^{31}$ are only accessible in FITS files on computers with 64 bit
                   enabled compilers and with some specific compilation options of
                   cfitsio (see cfitsio documentation).
}
\end{arguments}

\begin{example}
{
use \htmlref{healpix\_types}{sub:healpix_types} \\
use fitstools, only : \thedocid \\
character(len=80), dimension(1:128) :: hdr \\
real(SP), dimension(0:49,1) :: tod \\
character(len=\mylink{sub:healpix_types:filenamelen}{FILENAMELEN}) :: fname='tod.fits' \\
hdr(:) = ' ' \\
tod(:,1) = 1. \\
call \thedocid (tod, 50\_i8b, 1, hdr, 128, fname, firstpix=0\_i8b, repeat=10)  \\
tod = tod * 3. \\
call \thedocid (tod, 20\_i8b, 1, hdr, 128, fname, firstpix=40\_i8b)  
}
{
Writes into the FITS file `tod.fits' a 1 column binary table, where the first 40
data samples have the value $1.$ and the next 20 have the value $3.$ (Note that
in this example the
second call to \thedocid \ overwrites some of the pixels written by the first call). The samples will be
written in element vectors of length 10. The header for the FITS file is given in the
string array {\tt hdr} and its number of lines is 128. 
}
\end{example}

\begin{modules}
  \begin{sulist}{} %%%% NOTE the ``extra'' brace here %%%%
  \item[\textbf{fitstools}] module, containing:
  \item[printerror] routine for printing FITS error messages.
  \item[\textbf{cfitsio}] library for FITS file handling.		
  \end{sulist}
\end{modules}

\begin{related}
  \begin{sulist}{} %%%% NOTE the ``extra'' brace here %%%%
  \item[\htmlref{input\_tod*}{sub:input_tod}] routine that reads a file created by \thedocid. 
  \item[\htmlref{input\_map}{sub:input_map},
  \htmlref{read\_bintab}{sub:read_bintab}] routines to read \healpix sky map,
  \item[\htmlref{write\_minimal\_header}{sub:write_minimal_header}] routine to write minimal FITS header
  \end{sulist}
\end{related}

\rule{\hsize}{2mm}

\newpage
