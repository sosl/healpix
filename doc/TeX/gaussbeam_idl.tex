% -*- LaTeX -*-


\renewcommand{\facname}{{gaussbeam}}
\renewcommand{\FACNAME}{{GAUSSBEAM}}
\renewcommand{\l}{{$\ell$ }}

\sloppy



\title{\healpix IDL Facility User Guidelines}
\docid{\facname} \section[\facname]{ }
\label{idl:gaussbeam}
\docrv{Version 1.2}
\author{Eric Hivon}
\abstract{This document describes the \healpix IDL facility \facname.}




\begin{facility}
{This IDL facility provides the window function in $\ell$ space for a
gaussian axisymmetric beam of given FWHM.
}
{src/idl/misc/gaussbeam.pro}
\end{facility}

\begin{IDLformat}
{beam=\FACNAME(%
	\mylink{idl:gaussbeam:fwhm}{Fwhm}, 
	\mylink{idl:gaussbeam:lmax}{Lmax}
 [,~\mylink{idl:gaussbeam:dim}{Dim},
	\mylink{idl:gaussbeam:help}{HELP=}]%
)}
\end{IDLformat}

\begin{qualifiers}
  \begin{qulist}{} %%%% NOTE the ``extra'' brace here %%%%
    \item[Fwhm\mytarget{idl:gaussbeam:fwhm}] Full Width Half Maximum of the gaussian beam, in arcmin (scalar real)
    \item[Lmax\mytarget{idl:gaussbeam:lmax}] the window function is computed for the multipoles $\ell$ in \{0,...,Lmax\}
    \item[Dim\mytarget{idl:gaussbeam:dim}] scalar integer, optional. \\
     If absent or set to 0 or 1,
          the output has size (Lmax+1) and is the temperature beam;\\
     if set to $2 \le$ Dim $\le 4$ ,
          the output has size (Lmax+1,Dim)
          and contains in that order : \\
	  the TEMPERATURE beam,\\
          the GRAD/ELECTRIC polarization beam\\
          the CURL/MAGNETIC polarization beam\\
          the TEMPERATURE*GRAD beam
    \item[HELP=\mytarget{idl:gaussbeam:help}]%
	if set, prints out the help header and exits
  \end{qulist}
\end{qualifiers}

% \begin{keywords}
%   \begin{kwlist}{} %%% extra brace
%   \end{kwlist}
% \end{keywords}  

\begin{codedescription}
{\facname{} computes the $\ell$ space window function of a gaussian beam of FWHM
Fwhm. For a sky of underlying power spectrum $C(\ell)$ observed with beam of
given FWHM, the measured power spectrum will be $C(\ell)_{\mathrm{meas}} = C(\ell)
B(\ell)^2$ where $B(\ell)$ is given by gaussbeam(Fwhm,Lmax). The
polarization beam is also provided (when Dim $>$ 1) assuming a perfectly
co-polarized beam (eg, Challinor et al 2000, 
\htmladdnormallink{astro-ph/0008228}{https://arxiv.org/abs/astro-ph/0008228v2})}
\end{codedescription}



\begin{related}
  \begin{sulist}{} %%%% NOTE the ``extra'' brace here %%%%
    \item[idl] version \idlversion or more is necessary to run \facname
    \item[\htmlref{healpixwindow}{idl:healpixwindow}] computes the \l space window function associated with
    a \healpix pixel size
    \item[synfast] f90 code to generate CMB maps of given power spectrum convolved with a gaussian beam
    \item[smoothing] f90 code to smooth existing \healpix maps with a gaussian beam
    \item[anafast] f90 code to compute the power spectrum of a \healpix sky map
  \end{sulist}
\end{related}

\begin{example}
{
\begin{tabular}{ll} %%%% use this tabular format %%%%
beam = gaussbeam(5.,1200)
\end{tabular}
}
{
beam contains the window function in \{0,...,1200\} of a gaussian beam of fwhm 5 arcmin}
\end{example}


