% -*- LaTeX -*-

\sloppy

\title{\healpix IDL Facility User Guidelines}
\docid{ianafast} \section[ianafast]{ }
\label{idl:ianafast}
\docrv{Version 1.0}
\author{Eric Hivon}
\abstract{This document describes the \healpix IDL facility \thedocid.}

\begin{facility}
{This IDL facility provides an interface to 'anafast' F90 and 'anafast\_cxx' C++
facilities. It can be used to produce the Spherical Harmonics coefficients
($a_{lm}$ of a \healpix map (or pair of maps) and/or the resulting auto (or
cross) power spectra $C(l)$.}
{src/idl/interfaces/ianafast.pro}
\end{facility}

\begin{IDLformat}
{IANAFAST,  map1\_in [, cl\_out,
       alm1\_out=, alm2\_out=, binpath=, cxx=, double=, help=, healpix\_data=, iter\_order=, keep\_tmp\_files=, 
       map2\_in=, maskfile=, nested=, nlmax=, nmmax=, ordering=, plmfile=, polarisation=, 
       regression=, ring=, show\_cl=, simul\_type=, silent=, theta\_cut\_deg=, tmpdir=, 
       weighted=, won=, w8file=, w8dir=]}
\end{IDLformat}

\begin{qualifiers}
  \begin{qulist}{} %%%% NOTE the ``extra'' brace here %%%%
   \item[map1\_in] required input: 1st input map, can be a FITS file, or a memory array containing the
        map to analyze
    \item[cl\_out] optional output: auto or cross power spectrum $C(l)$, can be a FITS
file or a memory array
  \end{qulist}
\end{qualifiers}

\begin{keywords}
  \begin{kwlist}{} %%% extra brace
\item[alm1\_out=]   output alm of 1st map, must be a FITS file          \default {alm not kept}

\item[alm2\_out=]   output alm of 2nd map (if any, must be a FITS file) \default {alm not kept}

\item[binpath=] full path to back-end routine \default {\$HEXE/anafast, then \$HEALPIX/bin/anafast 
                or \$HEALPIX/src/cxx/\$HEALPIX\_TARGET\-/bin/anafast\_cxx, then \$HEALPIX/src/cxx/generic\_gcc\-/bin/anafast\_cxx if cxx is set}\\
              -- a binpath starting with / (or $\backslash$), $~$ or \$ is interpreted as absolute\\
              -- a binpath starting with ./ is interpreted as relative to current directory\\
              -- all other binpathes are relative to \$HEALPIX

\item[/cxx] if set, the C++ back-end anafast\_cxx is invoked instead of F90 anafast,
           AND the parameter file is written accordingly

\item[/double]    if set, I/O is done in double precision \default {single precision I/O}

\item[/help]      if set, prints extended help

\item[healpix\_data=] directory with Healpix precomputed files (only for C++ back\_end when weighted=1)
     \default {\$HEALPIX/data}

\item[iter\_order=] order of iteration in the analysis \default {0}

\item[/keep\_tmp\_files] if set, temporary files are not discarded at the end of the
                  run

\item[map2\_in=] 2nd input map (FITS file or array), if provided, Cl\_out will
  contain the cross power spectra of the 2 maps \default {no 2nd map}

\item[maskfile=] pixel mask (FITS file or array)   \default {no mask}

\item[/nested=] if set, signals that *all* maps and mask read online are in
   NESTED scheme (does not apply to FITS file), see also /ring and Ordering

\item[nlmax=]   maximum multipole of analysis, *required* for C++ anafast\_cxx,
      optional for F90 anafast

\item[nmmax=]   maximum degree m, only valid for C++ anafast\_cxx \default {nlmax}

\item[ordering=] either 'RING' or 'NESTED', ordering of online maps and masks,
 see /nested and /ring

\item[plmfile=] FITS file containing precomputed Spherical Harmonics (deprecated) \default {no file}

\item[/polarisation] if set analyze temperature + polarization (same as simul\_type = 2)

\item[regression=] 0, 1 or 2, regress out best fit monopole and/or dipole before
    alm analysis
  \default {0, analyze raw map} 

\item[/ring] see /nested and ordering above

\item[/show\_cl] if set, and {\tt cl\_out} is defined, the produced $l (l+1) C(l)/2\pi$ will
be plotted

\item[simul\_type=] 1 or 2, analyze temperature only or temperature + polarization

\item[/silent]    if set, works silently

\item[theta\_cut\_deg=] cut around the equatorial plane 

\item[tmpdir=]      directory in which are written temporary files 
\default {IDL\_TMPDIR (see IDL documentation)}

\item[/weighted]     same as won
    \default {apply weighting}

\item[/won]     if set, a weighting scheme is used to improve the quadrature
    \default {apply weighting}

\item[w8file=]    FITS file containing weights 
     \default {determined automatically by back-end routine}.
   Do not set this keyword unless you really know what you are doing

\item[w8dir=]     directory where the weights are to be found 
        \default {determined automatically by back-end routine}

  \end{kwlist}
\end{keywords}  

\begin{codedescription}
{\thedocid\ is an interface to 'anafast' F90 and 'anafast\_cxx' C++
facilities. It
requires some disk space on which to write the parameter file and the other
temporary files. Most data can be provided/generated as an external FITS
file, or as a memory array.}
\end{codedescription}



\begin{related}
  \begin{sulist}{} %%%% NOTE the ``extra'' brace here %%%%
    \item[idl] version \idlversion or more is necessary to run \thedocid.
    \item[anafast] F90 facility called by \thedocid.
    \item[anafast\_cxx] C++ called by \thedocid.
    \item[\htmlref{ialteralm}{idl:ialteralm}] IDL Interface to F90 alteralm
%    \item[\htmlref{ianafast}{idl:ianafast}] IDL Interface to F90 anafast and C++ anafast\_cxx
    \item[\htmlref{iprocess\_mask}{idl:iprocess_mask}] IDL Interface to F90 process\_mask
    \item[\htmlref{ismoothing}{idl:ismoothing}] IDL Interface to F90 smoothing
    \item[\htmlref{isynfast}{idl:isynfast}] IDL Interface to F90 synfast
  \end{sulist}
\end{related}

\begin{example}
{
\begin{tabular}{l} %%%% use this tabular format %%%%
 whitenoise = randomn(seed, \htmlref{nside2npix}{idl:nside2npix}(256))  \\
 \thedocid, whitenoise, cl, /ring, /silent  \\
 plot, cl[*,0]  
\end{tabular}
}
{
 will plot the power spectrum of a white noise map
}
\end{example}


