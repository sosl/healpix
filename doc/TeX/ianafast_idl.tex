% -*- LaTeX -*-

\sloppy

\title{\healpix IDL Facility User Guidelines}
\docid{ianafast} \section[ianafast]{ }
\label{idl:ianafast}
\docrv{Version 1.0}
\author{Eric Hivon}
\abstract{This document describes the \healpix IDL facility \thedocid.}

\begin{facility}
{This IDL facility provides an interface to '\htmlref{anafast}{fac:anafast}' F90 and 'anafast\_cxx' C++
facilities. It can be used to produce the Spherical Harmonics coefficients
($a_{\ell m}$ of a \healpix map (or pair of maps) and/or the resulting auto (or
cross) power spectra $C(\ell)$.}
{src/idl/interfaces/ianafast.pro}
\end{facility}

\begin{IDLformat}
{IANAFAST%
,  \mylink{idl:ianafast:map1_in}{map1\_in}%
 [, \mylink{idl:ianafast:cl_out}{cl\_out}%
,
       \mylink{idl:ianafast:alm1_out}{alm1\_out=}%
, \mylink{idl:ianafast:alm2_out}{alm2\_out=}%
, \mylink{idl:ianafast:binpath}{binpath=}%
, \mylink{idl:ianafast:cxx}{cxx=}%
, \mylink{idl:ianafast:double}{double=}%
, \mylink{idl:ianafast:help}{help=}%
, \mylink{idl:ianafast:healpix_data}{healpix\_data=}%
, \mylink{idl:ianafast:iter_order}{iter\_order=}%
, \mylink{idl:ianafast:keep_tmp_files}{keep\_tmp\_files=}%
, 
       \mylink{idl:ianafast:map2_in}{map2\_in=}%
, \mylink{idl:ianafast:maskfile}{maskfile=}%
, \mylink{idl:ianafast:nested}{nested=}%
, \mylink{idl:ianafast:nlmax}{nlmax=}%
, \mylink{idl:ianafast:nmmax}{nmmax=}%
, \mylink{idl:ianafast:ordering}{ordering=}%
, \mylink{idl:ianafast:plmfile}{plmfile=}%
, \mylink{idl:ianafast:polarisation}{polarisation=}%
, 
       \mylink{idl:ianafast:regression}{regression=}%
, \mylink{idl:ianafast:ring}{ring=}%
, \mylink{idl:ianafast:show_cl}{show\_cl=}%
, \mylink{idl:ianafast:simul_type}{simul\_type=}%
, \mylink{idl:ianafast:silent}{silent=}%
, \mylink{idl:ianafast:theta_cut_deg}{theta\_cut\_deg=}%
, \mylink{idl:ianafast:tmpdir}{tmpdir=}%
, 
       \mylink{idl:ianafast:weighted}{weighted=}%
, \mylink{idl:ianafast:won}{won=}%
, \mylink{idl:ianafast:w8file}{w8file=}%
, \mylink{idl:ianafast:w8dir}{w8dir=}%
]}
\end{IDLformat}

\begin{qualifiers}
  \begin{qulist}{} %%%% NOTE the ``extra'' brace here %%%%
   \item[map1\_in\mytarget{idl:ianafast:map1_in}%
] required input: 1st input map, can be a FITS file, or a memory array containing the
        map to analyze
    \item[cl\_out\mytarget{idl:ianafast:cl_out}%
] optional output: auto or cross power spectrum $C(\ell)$, can be a FITS
file or a memory array
  \end{qulist}
\end{qualifiers}

\begin{keywords}
  \begin{kwlist}{} %%% extra brace
\item[alm1\_out\mytarget{idl:ianafast:alm1_out}%
=]   output alm of 1st map, must be a FITS file          \default {alm not kept}

\item[alm2\_out\mytarget{idl:ianafast:alm2_out}%
=]   output alm of 2nd map (if any, must be a FITS file) \default {alm not kept}

\item[binpath\mytarget{idl:ianafast:binpath}%
=] full path to back-end routine \default {\$HEXE/anafast, then \$HEALPIX\-/bin/anafast 
%%%%                or \$HEALPIX\-/src/cxx/\$HEALPIX\_TARGET\-/bin/anafast\_\-cxx, then \$HEALPIX\-/src/cxx/generic\_gcc\-/bin/anafast\_\-cxx if cxx is set}\\
                or \$HEALPIX\-/bin/anafast\_\-cxx if \mylink{idl:ianafast:cxx}{cxx} is set}\\
              -- a binpath starting with / (or $\backslash$), $~$ or \$ is interpreted as absolute\\
              -- a binpath starting with ./ is interpreted as relative to current directory\\
              -- all other binpaths are relative to \$HEALPIX

\item[/cxx\mytarget{idl:ianafast:cxx}%
] if set, the C++ back-end anafast\_cxx is invoked instead of F90 \htmlref{anafast}{fac:anafast},
           AND the parameter file is written accordingly

\item[/double\mytarget{idl:ianafast:double}%
]    if set, I/O is done in double precision \default {single precision I/O}

\item[/help\mytarget{idl:ianafast:help}%
]      if set, prints extended help

\item[healpix\_data\mytarget{idl:ianafast:healpix_data}%
=] same as \mylink{idl:ianafast:w8dir}{w8dir}

\item[iter\_order\mytarget{idl:ianafast:iter_order}%
=] order of iteration in the analysis \default {0}

\item[/keep\_tmp\_files\mytarget{idl:ianafast:keep_tmp_files}%
] if set, temporary files are not discarded at the end of the
                  run

\item[map2\_in\mytarget{idl:ianafast:map2_in}%
=] 2nd input map (FITS file or array), if provided, Cl\_out will
  contain the cross power spectra of the 2 maps \default {no 2nd map}

\item[maskfile\mytarget{idl:ianafast:maskfile}%
=] pixel mask (FITS file or array)   \default {no mask}

\item[/nested\mytarget{idl:ianafast:nested}%
=] if set, signals that *all* maps and mask read online are in
   NESTED scheme (does not apply to FITS file), see also /ring and Ordering

\item[nlmax\mytarget{idl:ianafast:nlmax}%
=]   maximum multipole of analysis, *required* for C++ anafast\_cxx,
      optional for F90 \htmlref{anafast}{fac:anafast}

\item[nmmax\mytarget{idl:ianafast:nmmax}%
=]   maximum degree m, only valid for C++ anafast\_cxx \default {nlmax}

\item[ordering\mytarget{idl:ianafast:ordering}%
=] either 'RING' or 'NESTED', ordering of online maps and masks,
 see /nested and /ring

\item[plmfile\mytarget{idl:ianafast:plmfile}%
=] FITS file containing precomputed Spherical Harmonics (deprecated) \default {no file}

\item[/polarisation\mytarget{idl:ianafast:polarisation}%
] if set analyze temperature + polarization (same as simul\_type = 2)

\item[regression\mytarget{idl:ianafast:regression}%
=] 0, 1 or 2, regress out best fit monopole and/or dipole before
    alm analysis
  \default {0, analyze raw map} 

\item[/ring\mytarget{idl:ianafast:ring}%
] see /nested and ordering above

\item[/show\_cl\mytarget{idl:ianafast:show_cl}%
] if set, and {\tt cl\_out} is defined, the produced $\ell (\ell+1) C(\ell)/2\pi$ will
be plotted

\item[simul\_type\mytarget{idl:ianafast:simul_type}%
=] 1 or 2, analyze temperature only or temperature + polarization

\item[/silent\mytarget{idl:ianafast:silent}%
]    if set, works silently

\item[theta\_cut\_deg\mytarget{idl:ianafast:theta_cut_deg}%
=] cut around the equatorial plane 

\item[tmpdir\mytarget{idl:ianafast:tmpdir}%
=]      directory in which are written temporary files 
\default {IDL\_TMPDIR (see IDL documentation)}

\item[weighted\mytarget{idl:ianafast:weighted}%
=]     same as \mylink{idl:ianafast:won}{won}
    \default {see won}

\item[won\mytarget{idl:ianafast:won}%
=]     if set to 0, no weighting applied, if set to 1, a ring-based quadrature weighting scheme is applied,
              if set to 2, a pixel-based quadrature weighting scheme is applied.
    \default {1: apply ring-based weighting}

\item[w8file\mytarget{idl:ianafast:w8file}%
=]    In F90: FITS file containing weights 
     \default {determined automatically by back-end routine}.
   Do not set this keyword unless you really know what you are doing \\
      In C++ (\mylink{idl:ianafast:cxx}{/cxx} flag):  
      must be set to full path of weight file, consistent with value 
     of \mylink{idl:ianafast:won}{won} (or weighted)

\item[w8dir\mytarget{idl:ianafast:w8dir}%
=]     In F90 only: directory where the weights are to be found 
        \default {determined automatically by back-end routine}

  \end{kwlist}
\end{keywords}  

\begin{codedescription}
{\thedocid\ is an interface to '\htmlref{anafast}{fac:anafast}' F90 and 'anafast\_cxx' C++
facilities. It
requires some disk space on which to write the parameter file and the other
temporary files. Most data can be provided/generated as an external FITS
file, or as a memory array.}
\end{codedescription}



\begin{related}
  \begin{sulist}{} %%%% NOTE the ``extra'' brace here %%%%
    \item[idl] version \idlversion or more is necessary to run \thedocid.
    \item[\htmlref{anafast}{fac:anafast}] F90 facility called by \thedocid.
    \item[anafast\_cxx] C++ called by \thedocid.
    \item[\htmlref{ialteralm}{idl:ialteralm}] IDL Interface to F90 \htmlref{alteralm}{fac:alteralm}
%    \item[\htmlref{ianafast}{idl:ianafast}] IDL Interface to F90 \htmlref{anafast}{fac:anafast} and C++ anafast\_cxx
    \item[\htmlref{iprocess\_mask}{idl:iprocess_mask}] IDL Interface to F90 \htmlref{process\_mask}{fac:process_mask}
    \item[\htmlref{ismoothing}{idl:ismoothing}] IDL Interface to F90 \htmlref{smoothing}{fac:smoothing}
    \item[\htmlref{isynfast}{idl:isynfast}] IDL Interface to F90 \htmlref{synfast}{fac:synfast}
  \end{sulist}
\end{related}

\begin{example}
{
\begin{tabular}{l} %%%% use this tabular format %%%%
 whitenoise = randomn(seed, \htmlref{nside2npix}{idl:nside2npix}(256))  \\
 \thedocid, whitenoise, cl, /ring, /silent  \\
 plot, cl[*,0]  
\end{tabular}
}
{
 will plot the power spectrum of a white noise map
}
\end{example}


