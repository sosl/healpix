
\sloppy


\title{\healpix Fortran Subroutines Overview}
\docid{query\_disc} \section[query\_disc]{ }
\label{sub:query_disc}
\docrv{Version 1.3}
\author{Eric Hivon}
\abstract{This document describes the \healpix Fortran90 subroutine QUERY\_DISC.}

\begin{facility}
{Routine to find the index of all pixels within an angular distance radius from a defined
center. The output indices can be either in the RING or NESTED scheme} 
{\modPixTools}
\end{facility}

\begin{f90format}
{\mylink{sub:query_disc:nside}{nside}%
, \mylink{sub:query_disc:vector0}{vector0}%
, \mylink{sub:query_disc:radius}{radius}%
, \mylink{sub:query_disc:listpix}{listpix}%
, \mylink{sub:query_disc:nlist}{nlist}%
 [, \mylink{sub:query_disc:nest}{nest}%
, \mylink{sub:query_disc:inclusive}{inclusive}%
]}
\end{f90format}

\begin{arguments}
{
\begin{tabular}{p{0.28\hsize} p{0.05\hsize} p{0.1\hsize} p{0.47\hsize}} \hline 
\textbf{name~\&~dimensionality} & \textbf{kind} & \textbf{in/out} & \textbf{description} \\ \hline
                   &   &   &                           \\ %%% for presentation
nside\mytarget{sub:query_disc:nside} & I4B & IN & the $\nside$ parameter of the map. \\
vector0\mytarget{sub:query_disc:vector0}(3) & DP & IN & cartesian vector pointing at the disc center. \\
radius\mytarget{sub:query_disc:radius} & DP & IN & disc radius in radians. \\
listpix\mytarget{sub:query_disc:listpix}(0:*) & I4B/ I8B & OUT & the index for all pixels within {\tt radius}. Make sure that the size of the array is big enough to contain all pixels. \\ 
nlist\mytarget{sub:query_disc:nlist} & I4B/ I8B & OUT & The number of pixels listed in {\tt listpix}. \\
nest\mytarget{sub:query_disc:nest}\ \ (OPTIONAL) & I4B & IN &  The pixel indices are in the NESTED numbering
                   scheme if nest=1, and in RING scheme otherwise. \\
inclusive\mytarget{sub:query_disc:inclusive}\ \ (OPTIONAL) & I4B & IN & If set to 1, all the pixels overlapping
                   (even partially)
                   with the disc are listed, otherwise only those whose
                   center lies within the disc are listed. \\

\end{tabular}
}
\end{arguments}

\begin{example}
{
use \htmlref{healpix\_modules}{sub:healpix_modules} \\
call query\_disc(256,(/0,0,1/),pi/2,listpix,nlist,nest=1)  \\
}
{
Returns the NESTED pixel index of all pixels north of the equatorial line in a $\nside=256$ map.
}
\end{example}
% \newpage
\begin{modules}
  \begin{sulist}{} %%%% NOTE the ``extra'' brace here %%%%
 \item[\htmlref{in\_ring}{sub:in_ring}] routine to find the pixels in a certain slice of a given ring.		
 \item[\htmlref{ring\_num}{sub:ring_num}] function to return the ring number corresponding to the coordinate $z$
  \end{sulist}
\end{modules}

\begin{related}
  \begin{sulist}{} %%%% NOTE the ``extra'' brace here %%%%
  \item[\htmlref{pix2ang}{sub:pix_tools}, \htmlref{ang2pix}{sub:pix_tools}] convert between angle and pixel number.
  \item[\htmlref{pix2vec}{sub:pix_tools}, \htmlref{vec2pix}{sub:pix_tools}] convert between a cartesian vector and pixel number.
  \item[query\_disc, \htmlref{query\_polygon}{sub:query_polygon},]
  \item[\htmlref{query\_strip}{sub:query_strip}, \htmlref{query\_triangle}{sub:query_triangle}] render the list of pixels enclosed
  respectively in a given disc, polygon, latitude strip and triangle
  \item[\htmlref{surface\_triangle}{sub:surface_triangle}] computes the surface
in steradians of a spherical triangle defined by 3 vertices

  \end{sulist}
\end{related}

\rule{\hsize}{2mm}

\newpage
