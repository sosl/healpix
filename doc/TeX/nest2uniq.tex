
\sloppy


\title{\healpix Fortran Subroutines Overview}
\docid{nest2uniq} \section[nest2uniq]{ }
\label{sub:nest2uniq}
\docrv{Version 1.0}
\author{E. Hivon}
\abstract{This document describes the \healpix Fortran90 subroutines
  NEST2UNIQ.}

\begin{facility}
{This F90 facility turns the
parameter $\nside$ (a power of 2) and the pixel index $p$ into the Unique ID number $u = p + 4 \nside^2$.
See \htmlref{''The Unique Identifier scheme''}{intro:unique} in 
\linklatexhtml{''\healpix Introduction Document''}{intro.pdf}{intro.htm} for more details.
}
{\modPixTools}
\end{facility}

\begin{f90format}
{%
\mylink{sub:nest2uniq:nside}{nside}, 
\mylink{sub:nest2uniq:pnest}{pnest},
\mylink{sub:nest2uniq:puniq}{puniq}}
\end{f90format}

\begin{arguments}
{
\begin{tabular}{p{0.10\hsize} p{0.1\hsize} p{0.1\hsize} p{0.60\hsize}} \hline  
\textbf{name} & \textbf{kind} & \textbf{in/out} & \textbf{description} \\ \hline
                   &   &   &                           \\ %%% for presentation
nside \mytarget{sub:nest2uniq:nside} & I4B     & IN & The \healpix $\nside$ parameter. \\
pnest \mytarget{sub:nest2uniq:pnest} & I4B/I8B & IN & (NESTED scheme) pixel identification number over the range \{0,$12\nside^2-1$\}.\\
puniq \mytarget{sub:nest2uniq:puniq} & I4B/I8B & OUT & The \healpix Unique pixel identifier. 
\end{tabular}
}
\end{arguments}

\begin{example}
{use \htmlref{healpix\_modules}{sub:healpix_modules}\\
integer(I4B) :: puniq \\
call nest2uniq(1, 0, puniq)\\
print*,puniq
}
{
\begin{minipage}{11cm}
returns  \\
     4 \\
since the first pixel ($p=0$) at $\nside=$ 1 is the pixel with Unique ID number 4.
\end{minipage}
}
\end{example}

\begin{related}
  \begin{sulist}{} %%%% NOTE the ``extra'' brace here %%%%
  \item[\htmlref{uniq2nest}{sub:uniq2nest}] ] Transforms  Unique \healpix pixel ID number into Nside and Nested pixel number
  \item[\htmlref{pix2xxx, ...}{sub:pix_tools}] to turn NESTED pixel index into sky coordinates and back
  \end{sulist}
\end{related}

\rule{\hsize}{2mm}

