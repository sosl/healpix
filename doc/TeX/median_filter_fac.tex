
\sloppy


\title{\healpix Fortran Facility User Guidelines}
\docid{median\_filter} \section[median\_filter]{\nosectionname}
\label{fac:median_filter}
\docrv{Version 1.2}
\author{Eric Hivon}
\abstract{This document describes the \healpix facility MEDIAN\_FILTER.}

\begin{facility}
{This program produces the median filtered map of an input \healpix map
  (polarised or unpolarised). The
  neighborhood on the which the median is computed is defined as a disk of
  user-defined radius}
{src/f90/median\_filter/median\_filter.f90}
\end{facility}

\begin{f90facility}
{median\_filter [options] [parameter\_file]}
\end{f90facility}

\begin{options}
  \begin{optionlistwide}{} %%%% NOTE the ``extra'' brace here %%%%
    \item[{\tt -d}]
    \item[{\tt -}{\tt -}{\tt double}] double precision mode (see 
  \htmlref{Notes on double/single precision modes}{fac:subsec:ioprec}%
\latexhtml{ on page~\pageref{page:ioprec}}{})
    \item[{\tt -s}]
    \item[{\tt -}{\tt -}{\tt single}] single precision mode (default)
  \end{optionlistwide}
\end{options}
%% \begin{options}
%%   \begin{optionlistwide}{} %%%% NOTE the ``extra'' brace here %%%%
%%     \item[{\tt -d}]
%%     \item[{\tt --double}] If present, the input and output arrays will be double
%%     precision reals, and the output data will be written on disk at double precision (see Notes on I/O precision on page~\pageref{page:ioprec})
%%     \item[{\tt -s}]
%%     \item[{\tt --single}] If present, the input and output arrays will be single
%%     precision reals, and the output data will be written on disk at single precision (default)
%%   \end{optionlistwide}
%% \end{options}

\begin{qualifiers}
  \begin{qulistwide}{} %%%% NOTE the ``extra'' brace here %%%%
%
    \item[{simul\_type = }]\mytarget{fac:median_filter:simul_type}%
 Either 1 or 2. If set to 1, only the temperature component of the
      input map will be filtered. If set to 2, all the Stokes components available
      in the input file will be filtered (default = 1)
%
    \item[{infile = }]\mytarget{fac:median_filter:infile}%
 Name of the FITS file containing the map to be filtered
      (default = '', no default input file).
%
    \item[{mf\_radius\_arcmin = }]\mytarget{fac:median_filter:mf_radius_arcmin}%
 Radius in arcmin of the disk over which the
      median is computed (default = $3\theta_{\mathrm{pix}}$ where $\theta_{\mathrm{pix}}$
      is the input map pixel size).
%
    \item[{fill\_holes = }]\mytarget{fac:median_filter:fill_holes}%
 If set to true, flagged pixels take for value the median of
      the valid pixels surrounding them (if any). Otherwise they are left
      unchanged.
      (default = .false.). Note that {\tt y, yes,t, true, .true.} and 1 are
      interpreted as {\em true}, while {\tt n, no, f, false, .false.} and 0 stand
      for {\em false}.
%
    \item[{mffile = }]\mytarget{fac:median_filter:mffile}%
 Name of the FITS file containing the median filtered map
%
  \end{qulistwide}
\end{qualifiers}

\begin{codedescription}
{%
Median\_Filter produces a median filtered map in which the value of each pixel
is the median of the input map valid pixels found within a disk of given
  radius centered on that pixel. A pixel flagged as 'non-valid' in the input map
  can either be left unchanged or 'filled in' with the same scheme, if at
  least one valid pixel is found among its neighbors. \\
If the map is polarized, each of the three Stokes components is filtered separately.
}
\end{codedescription}

\begin{support}
  \begin{sulist}{} %%%% NOTE the ``extra'' brace here %%%%
  \item[\htmlref{anafast}{fac:anafast}] This \healpix Fortran facility can
     	       analyse a \healpix map.
  \item[\htmlref{synfast}{fac:synfast}] This \healpix facility can generate a
  \healpix map from a power spectrum $C_\ell$.
		
  \end{sulist}
\end{support}

\begin{examples}{1}
{
\begin{tabular}{ll} %%%% use this tabular format %%%%
median\_filter  [option] \\
\end{tabular}
}
{
Median\_Filter runs in interactive mode, self-explanatory.
}
\end{examples}

%% \vfill\newpage

\begin{examples}{2}
{
\begin{tabular}{ll} %%%% use this tabular format %%%%
median\_filter  [option] filename \\
\end{tabular}
}
{When 'filename' is present, median\_filter enters the non-interactive mode and parses
its inputs from the file 'filename'. This has the following
structure: the first entry is a qualifier which announces to the parser
which input immediately follows. If this input is omitted in the
input file, the parser assumes the default value.
If the equality sign is omitted, then the parser ignores the entry.
In this way comments may also be included in the file.
In this example, the file contains the following qualifiers:\hfill\newline
\fileparam{\mylink{fac:median_filter:simul_type}{simul\_type}%
 = 1}
\fileparam{\mylink{fac:median_filter:infile}{infile}%
 = map.fits}
\fileparam{\mylink{fac:median_filter:mf_radius_arcmin}{mf\_radius\_arcmin}%
 = 20.0}
\fileparam{\mylink{fac:median_filter:mffile}{mffile}%
 = med.fits}

Median\_Filter reads the sky map from 'map.fits'. Since \hfill\newline
\fileparam{\mylink{fac:median_filter:fill_holes}{fill\_holes}%
 }
has its default value, ..., The median will be computed on a disk of 20 arcmin
in radius, and the result will be written in 'med.fits'.
}
\end{examples}

\begin{release}
  \begin{relist}
    \item Initial release (\healpix 2.00)
  \end{relist}
\end{release}

\begin{messages}
{
\begin{tabular}{p{0.25\hsize} p{0.1\hsize} p{0.35\hsize}} \hline  
  \textbf{Message} & \textbf{Severity} & \textbf{Text} \\ \hline
                   &                   &   \\ %%% for presentation
can not allocate memory for array xxx &  Fatal & You do not have
                   sufficient system resources to run this
                   facility at the map resolution you required. 
  Try a lower map resolution.  \\ 
                   &                   &   \\ \hline %%% for presentation

this is not a binary table & & the fitsfile you have specified is not 
of the proper format \\
                   &                   &   \\ \hline %%% for presentation
there are undefined values in the table! & & the fitsfile you have specified is not 
of the proper format \\
                  &                   &   \\ \hline %%% for presentation
the header in xxx is too long & & the fitsfile you have specified is not 
of the proper format \\
                  &                   &   \\ \hline %%% for presentation
XXX-keyword not found & & the fitsfile you have specified is not 
of the proper format \\
                  &                   &   \\ \hline %%% for presentation
found xxx in the file, expected:yyyy & & the specified fitsfile does not
contain the proper amount of data. \\
                   &                   &   \\ \hline %%% for presentation

\end{tabular}
} 
\end{messages}

\rule{\hsize}{2mm}

\newpage
