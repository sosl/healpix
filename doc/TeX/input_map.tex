
\sloppy


\title{\healpix Fortran Subroutines Overview}
\docid{input\_map*} \section[input\_map*]{ }
\label{sub:input_map}
\docrv{Version 1.3}
\author{Eric Hivon \& Frode K.~Hansen}
\abstract{This document describes the \healpix Fortran90 subroutine INPUT\_MAP.}

\begin{facility}
{This routine reads a \healpix map from a FITS file. This can deal with full sky
as well as cut sky maps}
{src/f90/mod/fitstools.F90}
\end{facility}

\begin{f90format}
{filename, map, npixtot, nmaps \optional{[, fmissval, header, units, extno]}}
\end{f90format}
\aboutoptional

\begin{arguments}
{
\begin{tabular}{p{0.3\hsize} p{0.05\hsize} p{0.05\hsize} p{0.5\hsize}} \hline  
\textbf{name~\&~dimensionality} & \textbf{kind} & \textbf{in/out} & \textbf{description} \\ \hline
                   &   &   &                           \\ %%% for presentation
filename(len=\filenamelen) & CHR & IN & FITS file to be read from,
                   containing a full sky or cut sky map \\
map(0:npixtot-1,1:nmaps)    & SP/ DP & OUT & full sky map(s) constructed
                   from the data present in the file, missing pixels are filled
                   with fmissval \\
npixtot                    & I4B/ I8B & IN & number of pixels in the full sky map \\
nmaps     & I4B & IN &  number of maps in the file  \\
                   &   &   &                           \\ %%% for presentation
\optional{fmissval}  & SP/ DP & IN &  value to be given to missing pixels, its default
                   value is 0 \\
\optional{header}(LEN=80)(1:)     & CHR & OUT &   FITS extension header \\
\optional{units}(LEN=20)(1:nmaps)  & CHR & OUT &  maps units \\
\optional{extno}  & I4B & IN & extension number to read the data from
                   (0 based).\default 0 (the first extension is read) 
\end{tabular}
}
\end{arguments}

\begin{example}
{
use pix\_tools, only: nside2npix \\
use fitstools, only: getsize\_fits, input\_map \\
\ldots \\
npixtot = getsize\_fits('map.fits',nmaps=nmaps, nside=nside) \\
npix = nside2npix(nside) \\
allocate(map(0:npix-1,1:nmaps)) \\
call input\_map('map.fits', map, npix, nmaps)  \\
}
{
Reads into {\tt map} the content of the FITS file 'map.fits'
}
\end{example}
%%\newpage
\begin{modules}
  \begin{sulist}{} %%%% NOTE the ``extra'' brace here %%%%
  \item[\textbf{fitstools}] module, containing:
  \item[printerror] routine for printing FITS error messages.
  \item[\htmlref{read\_bintab}{sub:read_bintab}] routine to read a binary table
  from a FITS file
  \item[\htmlref{read\_fits\_cut4}{sub:read_fits_cut4}] routine to read cut sky
  map from a FITS file
  \item[\textbf{cfitsio}] library for FITS file handling.
  \end{sulist}
\end{modules}

\begin{related}
  \begin{sulist}{} %%%% NOTE the ``extra'' brace here %%%%
  \item[anafast] executable that reads a \healpix map and analyses it. 
  \item[synfast] executable that generate full sky \healpix maps
  \item[\htmlref{getsize\_fits}{sub:getsize_fits}] subroutine to know the size of a FITS file.
  \item[\htmlref{output\_map}{sub:output_map}] subroutine to write a FITS file
  from a \healpix map
  \item[\htmlref{write\_bintabh}{sub:write_bintabh}] subroutine to write a large
  array into a FITS file piece by piece
  \item[\htmlref{input\_tod*}{sub:input_tod}] subroutine to read an arbitrary subsection of
  a large binary table
  \end{sulist}
\end{related}

\rule{\hsize}{2mm}

\newpage
