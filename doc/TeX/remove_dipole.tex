
\sloppy


\title{\healpix Fortran Subroutines Overview}
\docid{remove\_dipole*} \section[remove\_dipole*]{ }
\label{sub:remove_dipole}
\docrv{Version 2.1}
\author{Eric Hivon}
\abstract{This document describes the \healpix Fortran90 subroutine
REMOVE\_DIPOLE.}
\newcommand{\vecf}{{\rm{ \bf f}}}
\newcommand{\vecb}{{\rm{ \bf b}}}
\newcommand{\matA}{{\rm{ \bf A}}}
\newcommand{\calP}{\cal{P}}

\begin{facility}
{This routine provides a means to fit and remove the dipole and monopole
from a \healpix map. The fit is obtained by solving the linear system
\begin{equation}
	\sum_{j=0}^{d^2-1}\ A_{ij}\ f_j = b_i
\end{equation}
 with, $d=1$ or 2, and
\begin{equation}
	b_i = \sum_{p \in \calP} s_i(p) w(p) m(p),
\end{equation}
\begin{equation}
	A_{ij} = \sum_{p \in \calP} s_i(p) w(p) s_j(p),
\end{equation}
 where $\calP$ is the set of
valid, unmasked pixels, $m$ is the input map, $w$ is pixel weighting, while
$s_0(p) = 1$ and $s_1(p)=x,\ s_2(p)=y,\ s_3(p)=z$ are
respectively the monopole and dipole templates. The output map is then
\begin{equation}
	m'(p) = m(p) - \sum_{i=0}^{d^2-1} f_i s_i(p).
\end{equation}
}
{src/f90/mod/pix\_tools.f90}
\end{facility}

\begin{f90format}
{nside, map, ordering, degree, multipoles, zbounds [, fmissval, mask, weights]}
\end{f90format}

\begin{arguments}
{
\begin{tabular}{p{0.32\hsize} p{0.05\hsize} p{0.08\hsize} p{0.45\hsize}} \hline  
\textbf{name~\&~dimensionality} & \textbf{kind} & \textbf{in/out} & \textbf{description} \\ \hline
                  &   &   &                           \\ %%% for presentation
nside & I4B & IN & value of $\nside$ resolution parameter for input map\\
map(0:12*nside*nside-1) & SP/ DP & INOUT & \healpix map from which the monopole and dipole will be
                   removed. Those are removed from {\em all unflagged pixels},
                   even those excluded by the cut {\tt zounds} or the {\tt mask}. \\
ordering & I4B & IN & \healpix\ scheme 1:RING, 2: NESTED \\
degree   & I4B & IN & multipoles to fit and remove. It is either 0 (nothing done),
                   1 (monopole only) or 2 (monopole and dipole). \\
multipoles(0:degree*degree-1) & DP & OUT & values of best fit monopole and
                   dipole. The monopole is described as a scalar in the same
                   units as the input map, the dipole as a 3D cartesian vector, in the same units. \\
zbounds(1:2) & DP & IN & section of the map on which to perform the
                   fit, expressed in terms of $z=\sin({\rm latitude}) =
                   \cos(\theta)$. If zbounds(1)$<$zbounds(2), the fit is
                   performed {\em on} the strip zbounds(1)$<z<$zbounds(2); if not, the
                   fit is performed {\em outside} of the strip
                   zbounds(2)$<z<$zbounds(1). \\
fmissval  \hskip 4cm (OPTIONAL) & SP/ DP & IN & value used to flag bad pixel on input
                   \default{-1.6375e30}. Pixels with that value are ignored
                   during the fit, and left unchanged on output.\\
mask(0:12*nside*nside-1)  \hskip 4cm (OPTIONAL)& SP/ DP & IN & mask of valid pixels. 
                       Pixels with $|$mask$|<10^{-10}$ are not used for fit. Note:
                   the map is {\em not} multiplied by the mask. \\
weights(0:12*nside*nside-1)  \hskip 4cm (OPTIONAL)& SP/ DP & IN & weight to be
given to each map pixel before doing the fit. By default pixels are given
a uniform weight of 1. Note:
                   the output map is {\em not} multiplied by the weights. \\

\end{tabular}
}
\end{arguments}

\newpage
\begin{example}
{
s = sin(15.0\_dp * PI / 180.0\_dp) \\
call \thedocid (128, map, 1, 2, multipoles, ($\backslash$ s, -s $\backslash$) )  \\
}
{
Will compute and remove the best fit monopole and dipole from a map with
$\nside=128$ in RING ordering scheme. The fit is performed on pixels with $|b|>15^o$.
}
\end{example}

\begin{modules}
  \begin{sulist}{} %%%% NOTE the ``extra'' brace here %%%%
  \item[\textbf{pix\_tools}] module, containing:
%  \item[\textbf{pix\_tools}] module, containing:
  \end{sulist}
\end{modules}

\begin{related}
  \begin{sulist}{} %%%% NOTE the ``extra'' brace here %%%%
  \item[\htmlref{add\_dipole}{sub:add_dipole}] routine to add a dipole and
  monopole to a map.
  \end{sulist}
\end{related}

\rule{\hsize}{2mm}

\newpage
