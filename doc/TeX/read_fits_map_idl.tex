% -*- LaTeX -*-

% PLEASE USE THIS FILE AS A TEMPLATE FOR THE DOCUMENTATION OF YOUR OWN
% FACILITIES: IN PARTICULAR, IT IS IMPORTANT TO NOTE COMMENTS MADE IN
% THE TEXT AND TO FOLLOW THIS ORDERING. THE FORMAT FOLLOWS ONE USED BY
% THE COBE-DMR PROJECT.	
% A.J. Banday, April 1999.




\renewcommand{\facname}{{read\_fits\_map }}
\renewcommand{\FACNAME}{{READ\_FITS\_MAP }}
\sloppy



\title{\healpix IDL Facility User Guidelines}
\docid{read\_fits\_map} \section[read\_fits\_map]{ }
\label{idl:read_fits_map}
\docrv{Version 1.1}
\author{Eric Hivon}
\abstract{This document describes the \healpix facility \thedocid.}




\begin{facility}
{This IDL facility reads in a \healpix map from a FITS file.}
{src/idl/fits/read\_fits\_map.pro}
\end{facility}

\begin{IDLformat}
{\FACNAME, \mylink{idl:read_fits_map:File}{File}%
, \mylink{idl:read_fits_map:T_sky}{T\_sky}%
, [\mylink{idl:read_fits_map:Hdr}{Hdr}%
, \mylink{idl:read_fits_map:Exthdr}{Exthdr}%
, \mylink{idl:read_fits_map:PIXEL}{PIXEL=}%
, \mylink{idl:read_fits_map:SILENT}{SILENT=}%
, \mylink{idl:read_fits_map:NSIDE}{NSIDE=}%
, \mylink{idl:read_fits_map:ORDERING}{ORDERING=}%
,
\mylink{idl:read_fits_map:COORDSYS}{COORDSYS=}%
, \mylink{idl:read_fits_map:EXTENSION}{EXTENSION=}%
, \mylink{idl:read_fits_map:HELP}{HELP=}%
]
}
\end{IDLformat}

\begin{qualifiers}
  \begin{qulist}{} %%%% NOTE the ``extra'' brace here %%%%
 	\item[{File}]   \mytarget{idl:read_fits_map:File}
          name of a FITS file containing 
               the \healpix map in an extension or in the image field 

 	\item[{T\_sky}]  \mytarget{idl:read_fits_map:T_sky}
	variable containing on output the \healpix map

       \item[{Hdr}] \mytarget{idl:read_fits_map:Hdr}
		  (optional), \\
		string variable containing on output
		  the FITS primary header

       \item[{Exthdr}] \mytarget{idl:read_fits_map:Exthdr}
		  (optional), \\
		string variable containing on output
		  the FITS extension header

  	\item[{PIXEL=}]  \mytarget{idl:read_fits_map:PIXEL}
		(optional), \\
               pixel number to read from or pixel range to read
                 (in the order of appearance in the file), starting from 0. \\
               if $\ge$ 0 scalar        : read from pixel to the end of the file \\
               if two elements array : reads from pixel[0] to pixel[1]
		(included) \\
               if absent             : read the whole file

	 \item[{NSIDE=}] \mytarget{idl:read_fits_map:NSIDE}
		(optional), \\
	        returns on output the \healpix resolution parameter, as read
		from the FITS header. Set to -1 if not found

	 \item[{ORDERING=}] \mytarget{idl:read_fits_map:ORDERING}
	        (optional), \\
	        returns on output the pixel ordering, as read from the FITS
	        header. Either 'RING' or 'NESTED' or ' ' (if not found).

	 \item[{COORDSYS=}] \mytarget{idl:read_fits_map:COORDSYS}
	        (optional), \\
	        returns on output the astrophysical coordinate system used, 
		as read from FITS header (value of keywords COORDSYS or SKYCOORD)

       \item[{EXTENSION=}] \mytarget{idl:read_fits_map:EXTENSION}
		(optional), \\
	extension unit to be read from FITS file: 
 either its 0-based ID number (ie, 0 for first extension {\em after} primary array) 
 or the case-insensitive value of its EXTNAME keyword.
	If absent, all available extensions are read.
 

% 		Either row number in the binary table to read data from,
% 		  starting from row one, or a two element array containing a
% 		  range of row numbers to read.  If not passed, then the entire
% 		  file is read in.
  \end{qulist}
\end{qualifiers}

\begin{keywords}
  \begin{kwlist}{} %%% extra brace
   \item[{HELP=}] \mytarget{idl:read_fits_map:HELP} if set, an extensive help is displayed and no
	file is read
  \item[{SILENT=}] \mytarget{idl:read_fits_map:SILENT} if set, no message is issued during normal execution
   \end{kwlist}
\end{keywords}

\begin{codedescription}
{\thedocid\ reads in a \healpix sky map from a FITS file, and outputs
the variable {\tt T\_sky}, where the optional variables {\tt Hdr} 
and {\tt Exthdr} contain
respectively the primary and extension headers. According to \healpix
convention, the map should be is stored as a FITS file binary table
extension. Note:the routine \htmlref{read\_tqu}{idl:read_tqu} which requires less
memory is recommended when reading {\em large polarized} maps.}
\end{codedescription}



\begin{related}
  \begin{sulist}{} %%%% NOTE the ``extra'' brace here %%%%
  \item[idl] version \idlversion or more is necessary to run \thedocid
  \item[%
\htmlref{read\_fits\_cut4}{idl:read_fits_cut4},
\htmlref{read\_fits\_partial}{idl:read_fits_partial},
\htmlref{read\_fits\_map}{idl:read_fits_map}]
  \item[%
\htmlref{read\_tqu}{idl:read_tqu},
\htmlref{read\_fits\_s}{idl:read_fits_s}]
\healpix IDL routines to read cut-sky maps and partial maps, full-sky maps, polarized full-sky maps and
arbitrary data sets from FITS files

    \item[sxpar] This IDL routine (included in \healpix package) can be
  used to extract FITS keywords from the header(s) Hdr or Xhdr read with \thedocid.
  \item[synfast] This \healpix facility will generate the FITS format 
            sky map that can be read by \thedocid.
  \item[\htmlref{write\_fits\_map}{idl:write_fits_map}] This \healpix IDL facility can be used to generate the FITS format 
            sky maps complient with \healpix convention and readable by \thedocid.
  \end{sulist}
\end{related}


\begin{example}
{
\begin{tabular}{l} %%%% use this tabular format %%%%
\thedocid, 'planck100GHZ-LFI.fits', map, hdr, xhdr, /silent \\
\end{tabular}
}
{\thedocid\ reads in the file 'planck100GHZ-LFI.fits' and outputs the
\healpix map in {\tt map}, the primary header in {\tt hdr} and the extension
header in {\tt xhdr}.
}
\end{example}

