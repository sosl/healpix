
\sloppy


\title{\healpix Fortran Subroutines Overview}
\docid{read\_conbintab*} \section[read\_conbintab*]{ }
\label{sub:read_conbintab}
\docrv{Version 2.0}
\author{Frode K.~Hansen, Eric Hivon}
\abstract{This document describes the \healpix Fortran90 subroutine READ\_CONBINTAB.}

\begin{facility}
{This routine reads a FITS file containing  $a_{\ell m}$  values for constained
  realisation. The FITS file is supposed to contain one integer column with
  $index=\ell^2+\ell+m+1$ and 2 or 4 single (or double) precision columns with
  real/imaginary  $a_{\ell m}$  values and real/imaginary   standard deviation on
  these $a_{\ell m}$. It is supposed to contain either 1 or 3 extension(s) containing
  respectively the $a_{\ell m}$ for T and if applicable E and B.}
{\modFitstools}
\end{facility}

\begin{f90format}
{\mylink{sub:read_conbintab:filename}{filename}%
, \mylink{sub:read_conbintab:alms}{alms}%
, \mylink{sub:read_conbintab:nalms}{nalms}%
 [, \mylink{sub:read_conbintab:units}{units}%
, \mylink{sub:read_conbintab:extno}{extno}%
]}
\end{f90format}

\begin{arguments}
{
\begin{tabular}{p{0.39\hsize} p{0.05\hsize} p{0.06\hsize} p{0.40\hsize}} \hline  
\textbf{name\&dimensionality} & \textbf{kind} & \textbf{in/out} & \textbf{description} \\ \hline
                   &   &   &                           \\ %%% for presentation
filename\mytarget{sub:read_conbintab:filename}(LEN=\filenamelen) & CHR & IN & filename of FITS file containing $a_{\ell m}$. \\
nalms\mytarget{sub:read_conbintab:nalms} & I4B & IN & Number of  $a_{\ell m}$  values to read from the file. \\
alms\mytarget{sub:read_conbintab:alms}(0:nalms-1,1:6) & SP/ DP & OUT & the $a_{\ell m}$ read from the file. alms(i,1)
                   and alms(i,2) contain the $\ell$ and $m$ values for the ith
                   $a_{\ell m}$ . alms(i,3) and alms(i,4) contain the real and
                   imaginary value of the ith  $a_{\ell m}$ . Finally, the
                   standard deviation for the ith  $a_{\ell m}$  is contained in
                   alms(i,5) (real) and alms(i,6) (imaginary). \\
units\mytarget{sub:read_conbintab:units}(len=20)(1:) \hskip 6cm (OPTIONAL)& CHR & OUT & character string containing the units of the
                   $a_{\ell m}$ \\
extno\mytarget{sub:read_conbintab:extno} \hskip 8cm (OPTIONAL) & I4B & IN & extension (0 based) of the FITS file to be read

\end{tabular}
}
\end{arguments}
\newpage

\begin{example}
{
call read\_conbintab ('alms.fits',alms,65*66/2)  \\
}
{
Read 65*66/2 (the number of $a_{\ell m}$ needed to fill the whole range from l=0 to l=64)  $a_{\ell m}$  values from the file `alms.fits' into the array alms(0:65*66/2-1,1:6). 
}
\end{example}

\begin{modules}
  \begin{sulist}{} %%%% NOTE the ``extra'' brace here %%%%
  \item[\textbf{fitstools}] module, containing:
  \item[printerror] routine for printing FITS error messages.
  \item[\textbf{cfitsio}] library for FITS file handling.		
  \end{sulist}
\end{modules}

\begin{related}
  \begin{sulist}{} %%%% NOTE the ``extra'' brace here %%%%
  \item[\htmlref{alms2fits}{sub:alms2fits}, \htmlref{dump\_alms}{sub:dump_alms}] routines to write $a_{\ell m}$ to a FITS-file 
  \item[\htmlref{fits2alms}{sub:fits2alms}] has the same function as read\_conbintab but is more general.
  \item[\htmlref{number\_of\_alms}{sub:number_of_alms}, \htmlref{getsize\_fits}{sub:getsize_fits}]
  can be used to find out the number of $a_{\ell m}$ available in the file.
  \end{sulist}
\end{related}

\rule{\hsize}{2mm}

\newpage
