% -*- LaTeX -*-

\sloppy

\title{\healpix IDL Facility User Guidelines}
\docid{isynfast} \section[isynfast]{ }
\label{idl:isynfast}
\docrv{Version 1.0}
\author{Eric Hivon}
\abstract{This document describes the \healpix IDL facility \thedocid.}

\begin{facility}
{This IDL facility provides an interface to F90 'synfast' facility. It can be
used to generate sky maps and/or $a_{lm}$ from power spectra ($C(l)$), synthesize maps from
$a_{lm}$ or simulate maps from $C(l)$ and constraining $a_{lm}$.}
{src/idl/interfaces/isynfast.pro}
\end{facility}

\begin{IDLformat}
{ISYNFAST, cl\_in [, map\_out,
      alm\_in=,  alm\_out=, apply\_windows=, beam\_file=, binpath=, double=, fwhm\_arcmin=, help=,
      iseed=, keep\_tmp\_files=, lmax=, nlmax=, nside=, nsmax=, plmfile=,
      simul\_type=, silent=, tmpdir=, windowfile=, winfiledir=]}
\end{IDLformat}

\begin{qualifiers}
  \begin{qulist}{} %%%% NOTE the ``extra'' brace here %%%%
   \item[cl\_in] input power spectrum, can be a FITS file, or a memory array containing the
        $C(l)$, used to generate a map or a set of gaussian alm \\
   If empty quotes ('') or a zero (0) are provided, it will be interpreted as "No input C(l)", in
   which case some input alm's (alm\_in) are required.
    \item[map\_out] optional output: {\em RING ordered} map synthetised from the power spectrum or from constraining alm
  \end{qulist}
\end{qualifiers}

\begin{keywords}
  \begin{kwlist}{} %%% extra brace
 \item[alm\_in=]    optional input (constraining) alm (must be a FITS file)           \default {no alm}

 \item[alm\_out=]   contains on output the effective alm (must be a FITS file)

 \item[/apply\_windows] if set, beam and pixel windows are applied to input alm\_in
(if any)

 \item[beam\_file=] beam window function, either a FITS file or an array

 \item[binpath=] full path to back-end routine \default {\$HEXE/synfast, then \$HEALPIX/bin/synfast}\\
              -- a binpath starting with / (or $\backslash$), $~$ or \$ is interpreted as absolute\\
              -- a binpath starting with ./ is interpreted as relative to current directory\\
              -- all other binpathes are relative to \$HEALPIX

 \item[/double]    if set, I/O is done in double precision \default {single precision I/O}

 \item[fwhm\_arcmin=] gaussian beam FWHM in arcmin \default {0}

 \item[/help]      if set, prints extended help

 \item[iseed=] integer seed of radom sequence \default {0}

 \item[/keep\_tmp\_files] if set, temporary files are not discarded at the end of the
                   run

 \item[lmax=, nlmax=]   maximum multipole simulation \default {2*$N_{\rm side}$}

 \item[nside=, nsmax=]  Healpix resolution parameter $N_{\rm side}$

 \item[plmfile=] FITS file containing precomputed Spherical Harmonics (deprecated) \default {no file}

 \item[simul\_type=] 
        1) Temperature only \\
        2) Temperature + polarisation \\
        3) Temperature + 1st derivatives \\
        4) Temperature + 1st \& 2nd derivatives \\
        5) T+P + 1st derivatives \\
        6) T+P + 1st \& 2nd derivates
	\default {2: T+P}

 \item[/silent]    if set, works silently

 \item[tmpdir=]      directory in which are written temporary files 
 \default {IDL\_TMPDIR (see IDL documentation)}

 \item[windowfile=]    FITS file containing pixel window 
        \default {determined automatically by back-end routine}.
      Do not set this keyword unless you really know what you are doing

  \item[winfiledir=]     directory where the pixel windows are to be found 
        \default {determined automatically by back-end routine}.
      Do not set this keyword unless you really know what you are doing

  \end{kwlist}
\end{keywords}  

\begin{codedescription}
{\thedocid\ is an interface to F90 'synfast' F90 facility. It
requires some disk space on which to write the parameter file and the other
temporary files. Most data can be provided/generated as an external FITS
file, or as a memory array.}
\end{codedescription}



\begin{related}
  \begin{sulist}{} %%%% NOTE the ``extra'' brace here %%%%
    \item[idl] version \idlversion or more is necessary to run \thedocid.
    \item[synfast] F90 facility called by \thedocid.
%    \item[\htmlref{isynfast}{idl:isynfast}] IDL Interface to F90 synfast
    \item[\htmlref{ianafast}{idl:ianafast}] IDL Interface to F90 anafast and C++ anafast\_cxx
    \item[\htmlref{iprocess\_mask}{idl:iprocess_mask}] IDL Interface to F90 process\_mask
    \item[\htmlref{ismoothing}{idl:ismoothing}] IDL Interface to F90 smoothing
  \end{sulist}
\end{related}

\begin{example}
{
\begin{tabular}{l} %%%% use this tabular format %%%%
\thedocid, '\$HEALPIX/test/cl.fits', map, fwhm=30, nside=256, /silent  \\
\htmlref{mollview}{idl:mollview}, map, 1, title='I'  \\
\htmlref{mollview}{idl:mollview}, map, 2, title='Q'  \\
\end{tabular}
}
{
will synthetize and plot I and Q  maps consistent with WMAP-1yr best fit power
spectrum and observed with a circular gaussian 30 arcmin beam.
}
\end{example}


