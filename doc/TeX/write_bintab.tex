
\sloppy


\title{\healpix Fortran Subroutines Overview}
\docid{write\_bintab*} \section[write\_bintab*]{ }
\label{sub:write_bintab}
\docrv{Version 1.2}
\author{Frode K.~Hansen, Eric Hivon}
\abstract{This document describes the \healpix Fortran90 subroutine WRITE\_BINTAB.}

\begin{facility}
{This routine creates a binary FITS-file from a \healpix map. The routine can save a temperature map or both temperature and polarisation maps (T,Q,U) to the file.}
{\modFitstools}
\end{facility}

\begin{f90format}
{\mylink{sub:write_bintab:map}{map}%
, \mylink{sub:write_bintab:npix}{npix}%
, \mylink{sub:write_bintab:nmap}{nmap}%
, \mylink{sub:write_bintab:header}{header}%
, \mylink{sub:write_bintab:nlheader}{nlheader}%
, \mylink{sub:write_bintab:filename}{filename}%
 \optional{[, \mylink{sub:write_bintab:extno}{extno}%
]}}
\end{f90format}
\aboutoptional

\begin{arguments}
{
\begin{tabular}{p{0.30\hsize} p{0.05\hsize} p{0.08\hsize} p{0.49\hsize}} \hline  
\textbf{name~\&~dimensionality} & \textbf{kind} & \textbf{in/out} & \textbf{description} \\ \hline
                   &   &   &                           \\ %%% for presentation
map\mytarget{sub:write_bintab:map}(0:npix-1,1:nmap) & SP/ DP & IN & the map to write to the FITS-file.\\
npix\mytarget{sub:write_bintab:npix} & I4B/ I8B & IN & Number of pixels in the map.\\
nmap\mytarget{sub:write_bintab:nmap} & I4B & IN & number of maps to be written, 1 for temperature only, and 3 for (T,Q,U). \\
header\mytarget{sub:write_bintab:header}(LEN=80) (1:nlheader) & CHR & IN & The header for the FITS-file. \\
nlheader\mytarget{sub:write_bintab:nlheader} & I4B & IN & number of header lines to write to the file. \\
filename\mytarget{sub:write_bintab:filename}(LEN=*) & CHR & IN & the map(s) is (are) written to a FITS-file with this filename. \\
\optional{extno\mytarget{sub:write_bintab:extno}}	& I4B & IN & extension number in which to write the data (0
                   based). \default {0}
\end{tabular}
}
\end{arguments}

\begin{example}
{
call write\_bintab (map,12*32**2,3,header,120,'map.fits')  \\
}
{
Makes a binary FITS-file called `map.fits' from the \healpix maps (T,Q,U) in the array map(0:12*32**2-1,1:3). The number of pixels 12*32**2 corresponds to the number of pixels in a $N_{side}=32$ \healpix map. The header for the FITS-file is given in the string array header and the number of lines in the header is 120. 
}
\end{example}
\begin{modules}
  \begin{sulist}{} %%%% NOTE the ``extra'' brace here %%%%
  \item[\textbf{fitstools}] module, containing:
  \item[printerror] routine for printing FITS error messages.
  \item[\textbf{cfitsio}] library for FITS file handling.		
  \end{sulist}
\end{modules}

\begin{related}
  \begin{sulist}{} %%%% NOTE the ``extra'' brace here %%%%
  \item[\htmlref{input\_map}{sub:input_map}, \htmlref{read\_bintab}{sub:read_bintab}] routines which read a file created by \thedocid. 
  \item[\htmlref{map2alm}{sub:map2alm}] subroutine which analyse a map and returns the $a_{lm}$ coeffecients.
  \item[\htmlref{output\_map}{sub:output_map}] subroutine which calls \thedocid
  \item[\htmlref{write\_bintabh}{sub:write_bintabh}] subroutine to write a large
array into a FITS file piece by piece
  \item[\htmlref{input\_tod*}{sub:input_tod}] subroutine to read an arbitrary subsection of
  a large binary table
  \item[\htmlref{write\_minimal\_header}{sub:write_minimal_header}] routine to write minimal FITS header\end{sulist}
\end{related}

\rule{\hsize}{2mm}

\newpage
