% -*- LaTeX -*-


\renewcommand{\facname}{{query\_disc }}
\renewcommand{\FACNAME}{{QUERY\_DISC }}
\sloppy



\title{\healpix IDL Facility User Guidelines}
\docid{\facname} \section[\facname]{ }
\label{idl:query_disc}
\docrv{Version 1.1}
\author{Eric Hivon}
\abstract{This document describes the \healpix IDL facility \facname.}




\begin{facility}
{This IDL facility provides a means to find the index of all pixels within an
angular distance {\tt Radius} from a defined center.
}
{src/idl/toolkit/query\_disc.pro}
\end{facility}

\begin{IDLformat}
{\facname, 
\mylink{idl:query_disc:nside}{Nside}, 
\mylink{idl:query_disc:vector0}{Vector0}, 
\mylink{idl:query_disc:radius}{Radius}, 
\mylink{idl:query_disc:listpix}{Listpix}, 
[\mylink{idl:query_disc:nlist}{Nlist}, 
\mylink{idl:query_disc:deg}{/DEG}, 
\mylink{idl:query_disc:nested}{/NESTED}, 
\mylink{idl:query_disc:inclusive}{/INCLUSIVE}]}
\end{IDLformat}

\begin{qualifiers}
  \begin{qulist}{} %%%% NOTE the ``extra'' brace here %%%%
    \item[Nside \mytarget{idl:query_disc:nside}] \healpix resolution parameter used to index the pixel list (scalar integer)
    \item[Vector0 \mytarget{idl:query_disc:vector0}] position vector of the disc center (3 elements vector)
          NB : the norm of Vector0 does not have to be one, what is
          consider is the intersection of the sphere with the line of
          direction Vector0.
    \item[Radius \mytarget{idl:query_disc:radius}] radius of the disc (in radians, unless \mylink{idl:query_disc:deg}{DEG} is set), (scalar
    real)
    \item[Listpix \mytarget{idl:query_disc:listpix}] on output: list of ordered index for the pixels found 
    within a radius Radius of the position defined by vector0. The RING numbering
    scheme is used unless the keyword \mylink{idl:query_disc:nested}{{\tt NESTED}} is set.
     (=-1 if the radius is too small and no pixel is found)
    \item[Nlist \mytarget{idl:query_disc:nlist}] on output: number of pixels in Listpix (=0 if no pixel is found).
  \end{qulist}
\end{qualifiers}

\begin{keywords}
  \begin{kwlist}{} %%% extra brace
    \item[/DEG \mytarget{idl:query_disc:deg}] if set \mylink{idl:query_disc:radius}{{\tt Radius}} is in degrees instead of radians
    \item[/NESTED \mytarget{idl:query_disc:nested}] if set, the output list uses the NESTED numbering scheme
    instead of the default RING
    \item[/INCLUSIVE \mytarget{idl:query_disc:inclusive}] if set, all the pixels overlapping (even partially)
                   with the disc are listed, otherwise only those whose
                   center lies within the disc are listed
  \end{kwlist}
\end{keywords}  

\begin{codedescription}
{\facname finds the pixels within the given disc in a selective way WITHOUT
scanning all the sky pixels. The numbering scheme of the output list and the
inclusiveness of the disc can be changed}
\end{codedescription}



\begin{related}
  \begin{sulist}{} %%%% NOTE the ``extra'' brace here %%%%
    \item[idl] version \idlversion or more is necessary to run \facname.
    \item[ang2pix, pix2ang] conversion between angles and pixel index
    \item[vec2pix, pix2vec] conversion between vector and pixel index
    \item[\htmlref{query\_disc}{idl:query_disc}, \htmlref{query\_polygon}{idl:query_polygon},]
    \item[\htmlref{query\_strip}{idl:query_strip}, \htmlref{query\_triangle}{idl:query_triangle}] render the list of pixels enclosed
  respectively in a given disc, polygon, latitude strip and triangle
  \end{sulist}
\end{related}

\begin{example}
{
\begin{tabular}{l} %%%% use this tabular format %%%%
\facname, 256L, [.5,.5,0.], 10., listpix, nlist, /Deg, /Nest
\end{tabular}
}
{
On return listpix contains the index of the (5982) pixels within 10 degrees from
the point on the sphere having the direction [.5,.5,0.].
The pixel indices correspond to the Nested scheme with resolution 256.
}
\end{example}


