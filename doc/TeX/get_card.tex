
\sloppy

\title{\healpix Fortran Subroutines Overview}
\docid{get\_card} \section[get\_card]{ }
\label{sub:get_card}
\docrv{Version 1.2}
\author{Eric Hivon}
\abstract{This document describes the \healpix Fortran90 subroutine GET\_CARD.}

\begin{facility}
{This routine reads a keyword of any kind from a FITS header. It is a wrapper to
other routines that read keywords of different kinds.}
{\modHeadFits}
\end{facility}

\begin{f90format}
{header, kwd, value, comment}
\end{f90format}

\begin{arguments}
{
\begin{tabular}{p{0.4\hsize} p{0.05\hsize} p{0.1\hsize} p{0.35\hsize}} \hline  
\textbf{name~\&~dimensionality} & \textbf{kind} & \textbf{in/out} & \textbf{description} \\ \hline
                   &   &   &                           \\ %%% for presentation
header(LEN=80) DIMENSION(:) & CHR & IN & The header to read the keyword from. \\
kwd(LEN=8) & CHR & IN & the FITS keyword to read (NOT case sensitive). \\
value & any & OUT & the value read for the keyword. 
The type of the fortran variable 'value' (double, real, integer, logical or
                   character) should match the type under which the
                   value is written in the FITS file, except if
                   'value' is a character string, in which case it can read any
                   keyword value, or if 'value' if real or double, in which case
                   it can read any numerical value\\
comment(LEN=*) & CHR & OUT & comment read for the keyword. \\ 
\end{tabular}
}
\end{arguments}

\begin{example}
{
call get\_card(header,'NsIdE',nside,comment)  \\
}
{
if {\tt nside} is defined as an integer, it
will contain on output the value of NSIDE (say 256) found in header
}
\end{example}

\begin{example}
{
call get\_card(header,'ORDERING',ordering,comment)  \\
}
{
if {\tt ordering} is defined as an character string, it
will contain on output the value of ORDERING (say 'RING') found in header
}
\end{example}

\begin{modules}
  \begin{sulist}{} %%%% NOTE the ``extra'' brace here %%%%
  \item[\textbf{cfitsio}] library for FITS file handling.		
  \end{sulist}
\end{modules}

\begin{related}
  \begin{sulist}{} %%%% NOTE the ``extra'' brace here %%%%
  \item[\htmlref{add\_card}{sub:add_card}] general purpose routine to write any keywords into a FITS
  file header
  \item[\htmlref{del\_card}{sub:del_card}] routine to discard a keyword from a FITS header
  \item[\htmlref{read\_par}{sub:read_par}, \htmlref{number\_of\_alms}{sub:number_of_alms}] routines to read specific keywords from a
  header in a FITS file.
  \item[\htmlref{getsize\_fits}{sub:getsize_fits}] function returning the size of the data set in a fits
  file and reading some other useful FITS keywords
  \item[\htmlref{merge\_headers}{sub:merge_headers}] routine to merge two FITS headers
  \end{sulist}
\end{related}

\rule{\hsize}{2mm}

\newpage
