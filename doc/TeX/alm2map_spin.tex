\sloppy
\docid{alm2map\_spin*}\section[alm2map\_spin*]{ }
\label{sub:alm2map_spin}
\docrv{Version 2.0}
\author{Eric Hivon}
\abstract{This document describes the \healpix Fortran90 subroutine ALM2MAP\_SPIN*.}

\begin{facility}
{This routine produces the maps of arbitrary spin $s$ and $-s$ given their alm
coefficients.
%
A (complex) map $S$ of spin $s$ is a linear combination of the spin weighted harmonics ${_s}Y_{\ell m}$
\begin{equation}
	{_s}S(p) = \sum_{\ell m} {_s}a_{\ell m}\ \ {_s}Y_{\ell m}(p)
\end{equation}
% \begin{eqnarray}
% 	{_s}S(p) = \sum_{\ell m} {_s}a_{\ell m}\ \ {_s}Y_{\ell m}(p) \myhtmlimage{} \label{spinmap}
% \end{eqnarray}
for $\ell \ge |m|, \ell \ge |s|$,
and is such that ${_s}S^* = {_{-s}}S$.\\
The 
\linklatexhtml{usual phase convention for the spin weighted harmonics}%
{https://en.wikipedia.org/wiki/Spin-weighted_spherical_harmonics\#Calculating}%
{https://en.wikipedia.org/wiki/Spin-weighted_spherical_harmonics\#Calculating}
is
${_s}Y_{\ell m}^* = (-1)^{s+m} {_{-s}}Y_{\ell -m}$
and therefore 
${_s}a_{\ell m}^* = (-1)^{s+m} {_{-s}}a_{\ell -m}$.


%
\thedocid\ expects the alm coefficients to be provided as
	%typo correction on spin sign of second term of RHS, 2009-09-03
\begin{eqnarray}
	{_{|s|}}a^{+}_{\ell m} &\myequal& - ( {_{|s|}}a_{\ell m} + (-1)^s {_{-|s|}}a_{\ell m} )/2 \\
	{_{|s|}}a^{-}_{\ell m} &\myequal& - ( {_{|s|}}a_{\ell m} - (-1)^s {_{-|s|}}a_{\ell m} )/(2i)
\end{eqnarray}
for $m\ge 0$, knowing that, just as for spin 0 maps, the
coefficients for $m<0$ are given by 
\begin{eqnarray}
{_{|s|}}a^{+}_{\ell-m} &\myequal& (-1)^m {_{|s|}}a^{+*}_{\ell m}, \\
{_{|s|}}a^{-}_{\ell-m} &\myequal& (-1)^m {_{|s|}}a^{-*}_{\ell m}.
\end{eqnarray}
%
The two (real) maps produced by \thedocid\ are defined respectively as
\begin{eqnarray}
	{_{|s|}}S^+ &\myequal& ({_{|s|}}S + {_{-|s|}}S)/2 \\
	{_{|s|}}S^- &\myequal& ({_{|s|}}S - {_{-|s|}}S)/(2i).
\end{eqnarray}
%
With these definitions, ${_2}a^{+}$, ${_2}a^{-}$, ${_2}S^+$ and ${_2}S^-$
match \healpix polarization $a^E, a^B, Q$ and $U$ respectively. However, for
$s=0$, $\ _{0}a^+_{\ell m} = -a^T_{\ell m}$, $\ _{0}a^-_{\ell m} = 0$, $\ {_0}S^+ = T$, $\ {_0}S^- = 0.$}
{\modAlmTools}
\end{facility}

\begin{f90format}
{\mylink{sub:alm2map_spin:nsmax}{nsmax}%
, \mylink{sub:alm2map_spin:nlmax}{nlmax}%
, \mylink{sub:alm2map_spin:nmmax}{nmmax}%
, \mylink{sub:alm2map_spin:spin}{spin}%
, \mylink{sub:alm2map_spin:alm}{alm}%
, \mylink{sub:alm2map_spin:map}{map}%
[, \mylink{sub:alm2map_spin:zbounds}{zbounds}]}
\end{f90format}

\begin{arguments}
{
\begin{tabular}{p{0.35\hsize} p{0.05\hsize} p{0.1\hsize} p{0.40\hsize}} \hline  
\textbf{name~\&~dimensionality} & \textbf{kind} & \textbf{in/out} & \textbf{description} \\ \hline
                   &   &   &                           \\ %%% for presentation
nsmax\mytarget{sub:alm2map_spin:nsmax} & I4B & IN & the $\nside$ value of the map to synthesize. \\
nlmax\mytarget{sub:alm2map_spin:nlmax} & I4B & IN & the maximum $\ell$ value used for the $a_{\ell m}$. \\
nmmax\mytarget{sub:alm2map_spin:nmmax} & I4B & IN & the maximum $m$ value used for the $a_{\ell m}$. \\
spin\mytarget{sub:alm2map_spin:spin} & I4B & IN & spin $s$ of the maps to be generated (only its absolute value
is relevant). \\
alm\mytarget{sub:alm2map_spin:alm}(1:2, 0:nlmax, 0:nmmax) & SPC/ DPC & IN & The ${_{|s|}}a^+_{\ell m}$ and ${_{|s|}}a^-_{\ell m}$ values to make the map
                   from.\\
map\mytarget{sub:alm2map_spin:map}(0:12*nsmax**2-1, 1:2) & SP/ DP & OUT & ${_{|s|}}S^+$ and ${_{|s|}}S^-$ output maps\\
zbounds\mytarget{sub:alm2map_spin:zbounds}(1:2), \hskip 4cm OPTIONAL & DP & IN & section of the sphere on which to perform the map synthesis, expressed in terms of $z=\sin(\mathrm{latitude}) = \cos(\theta).$ If zbounds(1)$<$zbounds(2), it is
performed {\em on} the strip zbounds(1)$<z<$zbounds(2); if not,
it is performed {\em outside} of the strip
%  zbounds(2)$<z<$zbounds(1). % OLD
zbounds(2)$\le z \le$zbounds(1). % NEW ??
% The whole sphere is treated if \texttt{zbounds=(/-1.0\_dp, 1.0\_dp/)} or if 
% it is absent.
If absent, the whole map is processed.

\end{tabular}
}
\end{arguments}

\begin{example}
{
use healpix\_types \\
use pix\_tools, only : nside2npix \\
use alm\_tools, only : alm2map\_spin \\
integer(I4B) :: nside, lmax, mmax, npix, spin\\
real(SP), dimension(:,:), allocatable :: map \\
complex(SPC), dimension(:,:,:), allocatable :: alm \\
\ldots \\
nside=256 ; lmax=512 ; mmax=lmax ; spin=4\\
npix=nside2npix(nside)\\
allocate(alm(1:2,0:lmax,0:mmax))\\
allocate(map(0:npix-1,1:2))\\
\ldots \\
call alm2map\_spin(nside, lmax, mmax, spin, alm, map)  \\
}
{
Make spin-4 maps from the $a_{\ell m}$ passed in alm. The maps have $\nside$ of 256, and are constructed from $a_{\ell m}$ values up to 512 in $\ell$ and $m$.
}
\end{example}

\begin{modules}
  \begin{sulist}{} %%%% NOTE the ``extra'' brace here %%%%
  \item[\htmlref{ring\_synthesis}{sub:ring_synthesis}] Performs FFT over $m$ for synthesis of the rings.
  \item[compute\_lam\_mm, get\_pixel\_layout, ]
  \item[gen\_lamfac\_der, gen\_mfac, gen\_mfac\_spin, do\_lam\_lm\_spin, ] 
  \item[gen\_recfac, gen\_recfac\_spin, init\_rescale, l\_min\_ylm] Ancillary routines used
  for $Y_{\ell m}$ recursion
  \item[\textbf{misc\_utils}] module, containing:
  \item[\htmlref{assert\_alloc}{sub:assert}] routine to print error message, when an array can not be
  allocated properly
  \end{sulist}
%Note: Starting with \htmlref{version 2.20}{sub:new2p20}, {\tt libpsht} routines will be called if $0<|s|\le100$.
Note: Starting with \htmlref{version 3.10}{sub:new3p10}, {\tt libsharp} routines will be called if $0<|s|\le100$.
\end{modules}

\begin{related}
  \begin{sulist}{} %%%% NOTE the ``extra'' brace here %%%%
   \item[\htmlref{alm2map}{sub:alm2map}] routine generating maps of temperature
   and polarisation from their  $a_{\ell m}$
   \item[\htmlref{alm2map\_der}{sub:alm2map_der}] routine generating maps of temperature
   and polarisation, and their spatial derivatives, from their  $a_{\ell m}$
   \item[\htmlref{map2alm\_spin}{sub:map2alm_spin}] routine performing the inverse transform
   of alm2map.
  \item[\htmlref{create\_alm}{sub:create_alm}] routine to generate randomly
  distributed $a_{\ell m}$ coefficients according to a given power spectrum
  \end{sulist}
\end{related}

\rule{\hsize}{2mm}

\newpage
