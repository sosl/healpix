% -*- LaTeX -*-


\renewcommand{\facname}{{read\_fits\_cut4 }}
\renewcommand{\FACNAME}{{READ\_FITS\_CUT4 }}
\sloppy



\title{\healpix IDL Facility User Guidelines}
\docid{\facname} \section[\facname]{ }
\label{idl:read_fits_cut4}
\docrv{Version 1.2}
\author{Eric Hivon}
\abstract{This document describes the \healpix facility \facname.}




\begin{facility}
{This IDL facility reads a cut sky \healpix map from a FITS file according to
the \healpix convention. The format used for the
FITS file follows the one used for Boomerang98 and is adapted from
COBE/DMR. This routine can also be used to read polarized cut sky map, where
each Stokes parameter is stored in a different extension of the same FITS file.}
{src/idl/fits/read\_fits\_cut4.pro}
\end{facility}

\begin{IDLformat}
{\FACNAME, \mylink{idl:read_fits_cut4:File}{File}%
, \mylink{idl:read_fits_cut4:Pixel}{Pixel}%
, \mylink{idl:read_fits_cut4:Signal}{Signal}%
 [, \mylink{idl:read_fits_cut4:N_Obs}{N\_Obs}%
, \mylink{idl:read_fits_cut4:Serror}{Serror}%
, \mylink{idl:read_fits_cut4:EXTENSION}{EXTENSION=}%
, \mylink{idl:read_fits_cut4:HDR}{HDR=}%
, \mylink{idl:read_fits_cut4:XHDR}{XHDR=}%
, \mylink{idl:read_fits_cut4:NSIDE}{NSIDE=}%
,
    \mylink{idl:read_fits_cut4:ORDERING}{ORDERING=}%
, \mylink{idl:read_fits_cut4:COORDSYS}{COORDSYS=}%
]
}
\end{IDLformat}

\begin{qualifiers}
  \begin{qulist}{} %%%% NOTE the ``extra'' brace here %%%%
 	\item[{File}] 
          name of a FITS file in which the map is to be written

 	\item[{Pixel}] 
	 (OUT, LONG vector), \\ index of observed (or valid) pixels

 	\item[{Signal}] 
	 (OUT, FLOAT vector), \\ value of signal in each observed pixel

 	\item[{N\_Obs}] 
	 (OUT, LONG or INT vector, Optional), \\ number of observation per pixel

 	\item[{Serror}] 
	 (OUT, FLOAT vector, Optional), \\{\em rms} of signal in pixel. For white noise,
                   this is $\propto 1/\sqrt{{\rm n\_obs}}$

  \end{qulist}
\end{qualifiers}

\begin{keywords}
  \begin{kwlist}{} %%% extra brace
    	\item[EXTENSION\mytarget{idl:read_fits_cut4:EXTENSION}%
 =] 	
		(IN, optional), \\
		0 based number of extension to read. Extension 0 contains the
temperature information, while extensions 1 and 2 contain respectively the Q
and U Stokes parameters related information. \default{0} 

    	\item[HDR\mytarget{idl:read_fits_cut4:HDR}%
 =] 	
		(OUT, optional), \\
		String array containing the primary header. 

    	\item[XHDR\mytarget{idl:read_fits_cut4:XHDR}%
 =] 	
		(OUT, optional), \\
		String array containing the extension header. 

	 \item[{NSIDE=}]
		(OUT, optional), \\
	        returns on output the \healpix resolution parameter, as read
		from the FITS header. Set to -1 if not found

	 \item[{ORDERING=}]
	        (OUT, optional), \\
	        returns on output the pixel ordering, as read from the FITS
	        header. Either 'RING' or 'NESTED' or ' ' (if not found).

	 \item[{COORDSYS=}]
	        (OUT, optional), \\
	        returns on output the astrophysical coordinate system used, 
		as read from FITS header (value of keywords COORDSYS or SKYCOORD)

  \end{kwlist}
\end{keywords}

%% \begin{keywords}
%%   \begin{kwlist}{} %%% extra brace
%% 	\item[{Nested}] if set, specifies that the map is in the NESTED ordering
%% 	scheme\\
%% 	\seealso Ordering and Ring 
%% 	\item[{Ring}] if set, specifies that the map is in the RING ordering
%% 	scheme\\
%% 	\seealso Ordering and Nested
%%    \end{kwlist}
%% \end{keywords}

\begin{codedescription}
% {\parbox[t]{\hsize}{\facname writes out the information contained in {\tt Prim\_stc} and {\tt
% Exten\_stc} in the primary unit and extension of the FITS file
% {\tt File} respectively . Coordinate systems can also be specified by {\tt Coordsys}. Specifying the
% ordering scheme is compulsary and can be done either in {\tt Header} or by setting {\tt
% Ordering} or {\tt Nested} or {\tt Ring} to the correct value. If {\tt
% Ordering} or {\tt Nested} or {\tt Ring} is set, its value overrides what is
% given in {\tt Header}. \\

% The data is assumed to represent a full sky data set with 
% the number of data points npix = 12*Nside*Nside
% unless   
% \\Partial is set OR the input fits header contains OBJECT =
%                'PARTIAL' \\
%        AND \\
%          the Nside qualifier is given a valid value OR the FITS header contains
%                  a NSIDE}}
\end{codedescription}



\begin{related}
  \begin{sulist}{} %%%% NOTE the ``extra'' brace here %%%%
  \item[idl] version \idlversion or more is necessary to run \facname
  \item[\htmlref{write\_fits\_cut4}{idl:write_fits_cut4}] This \healpix IDL
facility can be used to generate the FITS format {\em cut-sky} maps complient
with \healpix convention and readable by \facname.
  \item[%
\htmlref{read\_fits\_cut4}{idl:read_fits_cut4},
\htmlref{read\_fits\_partial}{idl:read_fits_partial},
\htmlref{read\_fits\_map}{idl:read_fits_map}]
  \item[%
\htmlref{read\_tqu}{idl:read_tqu},
\htmlref{read\_fits\_s}{idl:read_fits_s}]
\healpix IDL routines to read cut-sky maps and partial maps, full-sky maps, polarized full-sky maps and
arbitrary data sets from FITS files

    \item[sxpar] This IDL routine (included in \healpix package) can be
  used to extract FITS keywords from the header(s) HDR or XHDR read with \facname.
  \end{sulist}
\end{related}


% \begin{example}
% {
% \begin{tabular}{ll} %%%% use this tabular format %%%%
% \end{tabular}
% }
% {
% }
% \end{example}


