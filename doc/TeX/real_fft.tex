\sloppy

\title{\healpix Fortran Subroutines Overview}
\docid{real\_fft} \section[real\_fft]{ }
\label{sub:real_fft}
\docrv{Version 1.1}
\author{Martin Reinecke}
\abstract{This document describes the \healpix Fortran90 subroutine
real\_fft.}

\begin{facility}
{This routine performs a forward or backward Fast Fourier Transformation
on its argument {\tt data}.}
{\modHealpixFft}
\end{facility}

\begin{f90format}
{data, backward}
\end{f90format}

\begin{arguments}
{
\begin{tabular}{p{0.3\hsize} p{0.05\hsize} p{0.1\hsize} p{0.45\hsize}} \hline  
\textbf{name~\&~dimensionality} & \textbf{kind} & \textbf{in/out} & \textbf{description} \\ \hline
                   &   &   &                           \\ %%% for presentation
data(:) & XXX & INOUT &
  array containing the input and output data.
  Can be of type real(sp) or real(dp) \\
backward & LGT & IN & Optional argument. If present and true, perform backward transformation, else forward 
\end{tabular}}
\end{arguments}

\begin{example}
{
use healpix\_fft \\
call real\_fft (data, backward=.true.)
}
{
Performs a backward FFT on data.
}
\end{example}

\begin{related}
  \begin{sulist}{} %%%% NOTE the ``extra'' brace here %%%%
  \item[\htmlref{complex\_fft}{sub:complex_fft}] routine for FFT of complex data
  \end{sulist}
\end{related}

\rule{\hsize}{2mm}

\newpage
