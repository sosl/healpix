% -*- LaTeX -*-




\sloppy



\title{\healpix IDL Facility User Guidelines}
\docid{cl2fits} \section[cl2fits]{ }
\label{idl:cl2fits}
\docrv{Version 1.2}
\author{Anthony J.~Banday, Eric Hivon}
\abstract{This document describes the \healpix IDL facility \thedocid.}




\begin{facility}
{This IDL facility provides a means to
write into a FITS file as an ascii table extension the power 
spectrum coefficients passed to the routine. Adds additional
headers if required. The facility is primarily intended to allow the
user to write a theoretical power spectrum into a FITS file in 
the correct format to be ingested by the \healpix simulation facility 
\textbf{synfast}.
}
{src/idl/fits/cl2fits.pro}
\end{facility}

\begin{IDLformat}
{{CL2FITS}%
, \mylink{idl:cl2fits:cl_array}{cl\_array}%
, \mylink{idl:cl2fits:fitsfile}{fitsfile}%
, [\mylink{idl:cl2fits:HDR}{HDR=}%
, \mylink{idl:cl2fits:HELP}{/HELP}%
, \mylink{idl:cl2fits:XHDR}{XHDR=}%
, \mylink{idl:cl2fits:CMBFAST}{/CMBFAST}%
, \mylink{idl:cl2fits:UNITS}{UNITS=}%
]}
\end{IDLformat}

\begin{qualifiers}
  \begin{qulist}{} %%%% NOTE the ``extra'' brace here %%%%
    \item[cl\_array\mytarget{idl:cl2fits:cl_array}%
] real or double array of Cl coefficients to be written to
      file. This has dimension either (lmax+1,6) given in the sequence T E B
      TxE TxB ExB {\bf or} (lmax+1,4) given in the sequence T E B
      TxE {\bf or} (lmax+1) for T alone. The convention for the power spectrum is that it is not
      normalised by the Harrison-Zeldovich (flat) spectrum.
    \item[fitsfile\mytarget{idl:cl2fits:fitsfile}%
] String containing the name of the file to be written.
  \end{qulist}
\end{qualifiers}

\begin{keywords}
  \begin{kwlist}{} %%% extra brace
    \item[HDR\mytarget{idl:cl2fits:HDR}%
 =] String array containing the (non-trivial) primary header
      for the FITS file. 
    \item[/HELP\mytarget{idl:cl2fits:HELP}%
] If set, a help message is printed out, no file is written

    \item[XHDR\mytarget{idl:cl2fits:XHDR}%
=] String array containing the (non-trivial) extension header
      for the FITS file. 
    \item[/CMBFAST\mytarget{idl:cl2fits:CMBFAST}%
] if set, the routine will add the keyword 'POLNORM =
    CMBFAST' in the FITS header, meaning that the polarization power spectra
    have the same convention as CMBFAST (and Healpix 1.2). 
    If this keyword is not present in the input FITS file, {\tt synfast} will issue a
    warning when simulating a polarization map from that power spectrum, but no
    attempt to renormalize the power spectra will be made. To actually perform
    the renormalization, see \htmlref{convert\_oldhpx2cmbfast}{idl:convert_oldhpx2cmbfast}
    \item[UNITS\mytarget{idl:cl2fits:UNITS}%
=] String scalar containing units of power spectrum (eg, uK{\^{}}2,
         Kelvin**2, ...), to be put in keywords 'TUNIT*' of the extension header. 
         If provided, will override the values present in XHDR (if any).
    \item[ ] NOTE: optional header strings should NOT include the
      header keywords explicitly written by this routine.

  \end{kwlist}
\end{keywords}  

\begin{codedescription}
{\thedocid\ writes the input power spectrum coefficients into a FITS
file containing an ascii table extension. Optional headers conforming
to the FITS convention can also be written to the output file. All
required FITS header keywords (like SIMPLE, BITPIX, ...) are automatically generated by the
routine and should NOT be duplicated in the optional header inputs
(they would be ignored anyway).
The one/four/six column(s) are automatically named 
{\tt TEMPERATURE}, {\tt GRAD}, {\tt CURL}, {\tt G-T}, {\tt C-T} and {\tt C-G}
respectively.
If the power spectrum is provided in a double precision array, the output format
will automatically feature more decimal places.
The current implementation is much faster than the one available in
Healpix 1.10 thanks to replacing an internal loop by vector operations.}
\end{codedescription}



\begin{related}
  \begin{sulist}{} %%%% NOTE the ``extra'' brace here %%%%
    \item[idl] version \idlversion or more is necessary to run \thedocid.
    \item[\htmlref{fits2cl}{idl:fits2cl}] provides the complimentary routine to read in a
      power spectrum from a FITS file.
    \item[\htmlref{convert\_oldhpx2cmbfast}{idl:convert_oldhpx2cmbfast}] convert an
    existing power spectrum FITS file from the polarization convention used in
    Healpix 1.1 to the one used in Healpix 1.2 (and CMBFAST).
    \item[\htmlref{bl2fits}{idl:bl2fits}] facility to write a window function into a FITS file.
    \item[\htmlref{fits2alm}{idl:fits2alm}, \htmlref{alm2fits}{idl:alm2fits}] routines to read and write $a_{lm}$ coefficients
    \item[synfast] utilises the output file generated by \thedocid.
  \end{sulist}
\end{related}

\begin{example}
{
\begin{tabular}{l} %%%% use this tabular format %%%%
\thedocid, pwrsp, 'spectrum.fits', HDR = hdr, XHDR = xhdr
\end{tabular}
}
{
\thedocid writes the power spectrum stored in the variable {\tt pwrsp}
to the output FITS file {\tt spectrum.fits} with optional headers
passed by the string variables {\tt hdr} and {\tt xhdr}.
}
\end{example}


