
\sloppy


\title{\healpix Fortran Subroutines Overview}
\docid{surface\_triangle} \section[surface\_triangle]{ }
\label{sub:surface_triangle}
\docrv{Version 1.2}
\author{Eric Hivon}
\abstract{This document describes the \healpix Fortran90 subroutine SURFACE\_TRIANGLE.}

\begin{facility}
{Returns the surface in steradians of the spherical triangle described by its
three vertices} 
{src/f90/mod/pix\_tools.f90}
\end{facility}

\begin{f90format}
{v1, v2, v3, surface}
\end{f90format}

\begin{arguments}
{
\begin{tabular}{p{0.25\hsize} p{0.05\hsize} p{0.1\hsize} p{0.5\hsize}} \hline 
\textbf{name\&dimensionality} & \textbf{kind} & \textbf{in/out} & \textbf{description} \\ \hline
                   &   &   &                           \\ %%% for presentation
v1(3) & DP & IN & cartesian vector pointing at the triangle first vertex. \\
v2(3) & DP & IN & cartesian vector pointing at the triangle second vertex. \\
v3(3) & DP & IN & cartesian vector pointing at the triangle third vertex. \\
surface & DP & OUT & surface of the triangle in steradians.
\end{tabular}
}
\end{arguments}

\begin{example}
{
use healpix\_types \\
use pix\_tools,    only : surface\_triangle \\
real(DP) :: surface, one = 1.0\_dp \\
call surface\_triangle((/1,0,0/)*one, (/0,1,0/)*one, (/0,0,1/)*one, surface)  \\
print*, surface
}
{
Returns the surface in steradians of the triangle defined by the octant ($x,y,z>0$) : 1.5707963267948966
}
\end{example}
% \newpage
% \begin{modules}
%   \begin{sulist}{} %%%% NOTE the ``extra'' brace here %%%%
%  \item[\htmlref{in\_ring}{sub:in_ring}] routine to find the pixels in a certain slice of a given ring.		
%  \item[\htmlref{ring\_num}{sub:ring_num}] function to return the ring number corresponding to the coordinate $z$
%   \end{sulist}
% \end{modules}

\begin{related}
  \begin{sulist}{} %%%% NOTE the ``extra'' brace here %%%%
  \item[\htmlref{pix2ang}{sub:pix_tools}, \htmlref{ang2pix}{sub:pix_tools}] convert between angle and pixel number.
  \item[\htmlref{pix2vec}{sub:pix_tools}, \htmlref{vec2pix}{sub:pix_tools}] convert between a cartesian vector and pixel number.
  \item[\htmlref{query\_disc}{sub:query_disc}, \htmlref{query\_polygon}{sub:query_polygon},]
  \item[\htmlref{query\_strip}{sub:query_strip}, \htmlref{query\_triangle}{sub:query_triangle}] render the list of pixels enclosed
  respectively in a given disc, polygon, latitude strip and triangle

  \end{sulist}
\end{related}

\rule{\hsize}{2mm}

\newpage
