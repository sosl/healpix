% -*- LaTeX -*-


\sloppy

\title{\healpix IDL Facility User Guidelines}
\docid{angulardistance} \section[angulardistance]{ }
\label{idl:angulardistance}
\docrv{Version 1.0}
\author{Eric Hivon}
\abstract{This document describes the \healpix IDL facility angulardistance.}

\begin{facility}
{This IDL facility computes the angular distance (in RADIANS) between pairs of vectors.
}
{src/idl/toolkit/angulardistance.pro}
\end{facility}

\begin{IDLformat}
{%\mylink{idl:angulardistance:beam}
{distance}=\thedocid(%
\mylink{idl:angulardistance:v}{V}, 
\mylink{idl:angulardistance:w}{W},
[\mylink{idl:angulardistance:help}{/HELP}])}
\end{IDLformat}

\begin{qualifiers}
  \begin{qulist}{} %%%% NOTE the ``extra'' brace here %%%%
    \item[V] \mytarget{idl:angulardistance:v}%
      3D-vector (of shape (3) or (1,3)) or list of n 3D-vectors (of shape (n,3))
    \item[W] \mytarget{idl:angulardistance:w}%
      3D-vector (of shape (3) or (1,3)) or list of n 3D-vectors (of shape
(n,3))\\
	It is {\bf not} necessary for {\tt V} and {\tt W} vectors to be normalised to 1
        upon calling the function\\
% 	If {\tt V} and {\tt W} both are lists of vectors, 
%         they should be of the same length
If {\tt V} (and/or {\tt W}) has the form (n,3,4) (like the pixel {\em corners} returned by
 \htmlref{pix2vec\_*}{idl:pix_tools}), it should be preprocessed with
 {\tt V = reform( transpose(V, [0,2,1]), n\_elements(V)/3, 3)}
  before being passed to \thedocid.  
\end{qulist}
\end{qualifiers}

\begin{keywords}
  \begin{kwlist}{} %%% extra brace
    \item[/HELP] \mytarget{idl:angulardistance:help}%
	if set, prints out the help header and exits
  \end{kwlist}
\end{keywords}  

\newcommand{\vecV}{\ensuremath{\bf V}}
\newcommand{\vecW}{\ensuremath{\bf W}}
\begin{codedescription}
{%
After renormalizing the vectors, \thedocid\ computes the angular distance using
$\cos^{-1}(\vecV.\vecW)$ in general, or
$2 \sin^{-1}\left(||\vecV-\vecW||\right)/2$ when
 $\vecV$ and $\vecW$ are almost aligned.\\
If $\vecV$ (resp. $\vecW$) is a single vector, while $\vecW$ (resp. $\vecW$) is a list of vectors,
then the result is a list of distances
 $d_i = {\rm dist}(\vecV,{\vecW}_i)$ 
(resp. $d_i = {\rm dist}({\vecV}_i,{\vecW})$).\\
If both $\vecV$ and $\vecW$ are lists of vector {\em of the same length},
then the result is a list of distances
 $d_i = {\rm dist}({\vecV}_i,{\vecW}_i)$.\\
}
\end{codedescription}



\begin{related}
  \begin{sulist}{} %%%% NOTE the ``extra'' brace here %%%%
    \item[idl] version \idlversion or more is necessary to run \thedocid.
%    \item[vect\_prod] facility to compute the vectorial product of 2 vectors
  \end{sulist}
\end{related}

\begin{example}
{%
\begin{tabular}{l}   %%%% use this tabular format %%%%
    nside=8\\
      \htmlref{pix2vec\_ring}{idl:pix_tools}, nside, lindgen(\htmlref{nside2npix}{idl:nside2npix}(nside)), vpix \\
      \htmlref{mollview}{idl:mollview}, angulardistance( vpix, [1,1,1])
\end{tabular}
}%
{%
will plot the angular distance between the Healpix pixels center for
$\nside=8$ , and the vector $(x,y,z) = (1,1,1)/\sqrt{3}$%
}
\end{example}


