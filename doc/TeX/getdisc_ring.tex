
\sloppy


\title{\healpix Fortran Subroutines Overview}
\docid{getdisc\_ring} \section[getdisc\_ring]{ }
\label{sub:getdisc_ring}
\docrv{Version 1.3}
% \author{Frode K.~Hansen}
\author{Eric Hivon}
\abstract{This document describes the \healpix Fortran90 subroutine GETDISC\_RING.}

\begin{facility}
{ %Routine to find the pixel index of all pixels within an angular distance radius from a defined center.
{\bf This routine is obsolete, use \htmlref{query\_disc}{sub:query_disc} instead} }
{src/f90/mod/pix\_tools.F90}
\end{facility}

% \begin{f90format}
% {nside, vector0, radius, listpix, nlist}
% \end{f90format}

% \begin{arguments}
% {
% \begin{tabular}{p{0.4\hsize} p{0.05\hsize} p{0.1\hsize} p{0.35\hsize}} \hline  
% \textbf{name\&dimensionality} & \textbf{kind} & \textbf{in/out} & \textbf{description} \\ \hline
%                    &   &   &                           \\ %%% for presentation
% nside & I4B & IN & the $N_{side}$ parameter of the map. \\
% vector0(3) & DP & IN & cartesian vector pointing at the central position. \\
% radius & DP & IN & radius from central position in radians. \\
% listpix(0:*) & I4B & OUT & the pixel indexs for all pixels inside $radius$. Make sure that the size of the array is big enough to contain all pixels. \\ 
% nlist & I4B & OUT & The number of pixels listed in $listpix$. \\
% \end{tabular}
% }
% \end{arguments}

% \begin{example}
% {
% call getdisc\_ring(256,(0,0,1),pi/2,listpix,nlist)  \\
% }
% {
% Returns the pixel indexs of all pixels north of the equatorial line in a $N_{side}=256$ map.
% }
% \end{example}
% \newpage
% \begin{modules}
%   \begin{sulist}{} %%%% NOTE the ``extra'' brace here %%%%
%  \item[\htmlref{in\_ring}{sub:in_ring}] routine to find the pixels in a certain slice of a given ring.		
%   \end{sulist}
% \end{modules}

% \begin{related}
%   \begin{sulist}{} %%%% NOTE the ``extra'' brace here %%%%
%   \item[\htmlref{pix2ang}{sub:pix_tools}, \htmlref{ang2pix}{sub:pix_tools}] convert between angle and pixel number.
%   \item[\htmlref{pix2vec}{sub:pix_tools}, \htmlref{vec2pix}{sub:pix_tools}] convert between a cartesian vector and pixel number.
%   \end{sulist}
% \end{related}

\rule{\hsize}{2mm}

\newpage
