
\sloppy


\title{\healpix Fortran Facility User Guidelines}
\docid{anafast} \section[anafast]{\nosectionname}
\label{fac:anafast}
\docrv{Version 1.2}
\author{Frode K.~Hansen, Benjamin D.~Wandelt, K.M.~G\'orski, Eric Hivon}
\abstract{This document describes the \healpix facility ANAFAST.}

\begin{facility}
{This program performs harmonic analysis of the \healpix maps 
up to a user
specified maximum spherical harmonic order $\lmax$.
The integrals are computed on the whole sphere, unless the user 
chooses  a provided option 
to excise from the input map(s) a simple, constant latitude, symmetric cut, and/or
apply an arbitray cut read from an external file.
Scalar, or scalar and tensor, spherical harmonic coefficients are evaluated
from the map(s) if the input provides, respectively, only the temperature,
or temperature and polarisation maps.
The total operation count scales as 
${\cal {O}}(\npix^{1/2}\lmax^2 )$. %
\\ %
Anafast reads one (two) file(s) containing the map(s) and produces 
a file containing the temperature auto- (or cross-) power spectrum
$C^{TT}_\ell$ and, if requested, 
also the polarisation power spectra $C^{EE}_\ell$, $C^{BB}_\ell$, $C^{TE}_\ell$,
$C^{TB}_\ell$, $C^{EB}_\ell$ (as well as  $C^{ET}_\ell$,
$C^{BT}_\ell$, $C^{BE}_\ell$ if two maps are provided).
The $a_{\ell m}$  coefficients computed during the execution also can be  
written to a (two) file(s) if requested. %
\\ %
Anafast executes an approximate, discrete point-set quadrature on 
a sphere
sampled at the \healpix pixel centers.
Spherical harmonic transforms are computed 
using recurrence relations for Legendre polynomials on co-latitude, 
$\theta$, 
and  Fast Fourier Transforms on longitude, $\phi$. %
\\%
Anafast is provided with an option to use precomputed Legendre Polynomials;
please note that since version 2.20 this will most likely reduce performance
instead of increasing it. %
\\%
Anafast permits two execution options 
which allow a significant improvement of accuracy of 
the approximate quadrature performed by this facility:%
\\ %
$\bullet$ Improved analyses using either the provided ring weights, 
which correct the quadrature on iso-latitude rings, or pixel-based weights which 
improve the quadrature on every pixels, and/or  %
\\%
$\bullet$ An iterative scheme using in 
succession several backward and forward harmonic transforms 
of the maps.}%
{src/f90/anafast/anafast.f90}
\end{facility}

\begin{recommend}
{Execution of \thedocid\ requires a user to specify the maximum 
spherical harmonic order $\lmax$ up to which the harmonic 
decomposition of the input maps will be performed.
Since there are no formal limits on parameter
$\lmax$ enforced by \thedocid, the user should make his/her choices
judiciously. 
Hereafter it is convenient to specify $\lmax$  
in terms of the \healpix
map resolution parameter $\nside$ (called $nsmax$ in some other contexts).

If the function to be analysed is strictly  band-width
limited, or nearly band-width limited (as in the case of 
a Gaussian beam smoothed signal discretized at a rate of a few pixels
per beam area), it is  sufficient to run \thedocid\ with
$\lmax \approx 2\cdot \nside$, with a very good $C_\ell$ 
error performance 
already in the raw (i.e. uncorrected quadrature) harmonic transform mode. 
If quadrature 
corrections are still desired in this case, it should be sufficient to use, at no
extra cost in execution time, the ring-weighted quadrature scheme. 
This is the recommended mode of operation of \thedocid\ for essentially
error and worry free typical applications, e.g. CPU-intensive 
Monte Carlo studies. 

A new set of pixel-based quadrature weights was introduced in \healpix 3.40.
Pre-computed to inforce a (near) ideal integration of the spherical harmonics $Y_{\ell m}$ on the pixelized sphere (ie $\frac{4\pi}{\npix} \sum_p w(p) Y_{\ell m}(p) = \sqrt{4 \pi} \delta_{\ell 0}\delta_{m 0}$)
for $|m| \le \ell \le 3 \nside$, they can be used to insure that the $a_{\ell m}$ and $C_\ell$ computed by \thedocid\
are perfectly accurate (almost to machine precision) without the need for iterations, 
but \textbf{only} for band-width 
limited input signal with $\lmax \le 1.5 \nside$.

If more aggressive attempts are undertaken to extract from a map 
the spectral coefficients at $\ell > 2\cdot \nside$ (for example, as 
in a possible case of an attempt to analyse an existing map, which was 
irreversibly binned  at a suboptimal resolution) 
the following should be kept in mind:

$\bullet$ Spherical harmonics discretized using \healpix 
(either sampled at
pixel centers, or avaraged over pixel areas) form a linearly independent
system up to $\lmax = 3 \cdot \nside -1$. Hence, the functions which are
strictly band-width limited to $\lmax = 3 \cdot \nside - 1 $ 
can be fully
spectrally resolved with \thedocid, albeit with integration errors
in the uncorrected quadrature mode, which grow up to 
$\myhtmlimage{}\delta C_\ell \propto \epsilon \cdot C_\ell$, with $\epsilon <0.1$, 
at the highest values of $\ell$. These integration  errors 
can be efficiently 
reduced
using \thedocid\ in the iterative mode. Although this $\lmax$ range
--- $2 \cdot \nside < \lmax < 3 \cdot \nside - 1$ --- is easily
manageable with \thedocid\ used on strictly band-width limited functions,
it should be used with caution in basic and automated applications, e.g.
Monte Carlo simulations.

$\bullet$ As with any discrete Fourier transform, \thedocid\ application to
functions which are not band-width limited results with aliasing
of power, which can not be remedied. If the particular case of interest
may result in such a band-width violation (i.e. there is significant power
in the function at $\ell > 3 \cdot \nside -1$), the function should
be smoothed before the application of \thedocid, or discretized and
then analysed, on a refined \healpix grid (with larger $\nside$).

$\bullet$ REMEMBER: 
A peculiar property of the sphere, which usually surprises those 
whose intuition is built on experience with FFTs on a segment, or 
on a Euclidean \phantom{xxxxxxxxxxxxxxxxxxxxxxxxx}
}\end{recommend}\begin{recommend_contd}{
multidimensional
domain, is the lack of 
a regular and uniform point-set at arbitrary resolution, 
and the resulting non-commutativity of the forward and
backward discrete Fourier transforms on nearly-uniform point-sets,
e.g. {\healpixns}. Hence,
as in any case of attempting an extreme application of an off-the-shelf
software, use caution and understand your problem well {\it before}
executing \thedocid\ under such circumstances!
}
\end{recommend_contd}

\begin{f90facility}
{anafast [options] [parameter\_file]}
\end{f90facility}

%% \begin{options}
%%   \begin{optionlistwide}{} %%%% NOTE the ''extra'' brace here %%%%
%%     \item[{\tt -d}]
%%     \item[{\tt --double}] If present, all internal variables and arrays will be in double precision, and the output data will be written on disk at double precision (see Notes on I/O precision on page~\pageref{page:ioprec})
%%     \item[{\tt -s}]
%%     \item[{\tt --single}] If present, most internal variables and arrays will be in single precision, except for those involved in accuracy critical calculations such as spherical harmonics recursion, and the output data will be written on disk at single precision (default)
%%   \end{optionlistwide}
%% \end{options}
\begin{options}
  \begin{optionlistwide}{} %%%% NOTE the ''extra'' brace here %%%%
    \item[{\tt -d}]
    \item[{\tt -}{\tt -}{\tt double}] double precision mode (see 
  \htmlref{Notes on double/single precision modes}{fac:subsec:ioprec}%
\latexhtml{ on page~\pageref{page:ioprec}}{})
    \item[{\tt -s}]
    \item[{\tt -}{\tt -}{\tt single}] single precision mode (default)
  \end{optionlistwide}
\end{options}

\begin{qualifiers}
  \begin{qulist}{} %%%% NOTE the ''extra'' brace here %%%%
    \item[{infile = }]\mytarget{fac:anafast:infile}%
 Defines the input map file. 
	(default= map.fits)
	If not blank, the filename should {\em never} be put between quotes even if it contains
symbols such as {\tt \&, [, ], ?, =} which should be typed literally (ie, unprotected). For instance
 {\tt infile~=~http://site/action?file.fits[2][col FLUX]} is just fine.
    \item[{infile2 = }]\mytarget{fac:anafast:infile2}%
 Defines the 2nd input map file, to be cross-correlated with
	the first one. The 2 maps should match in resolution ($\nside$) and coordinate.
	(default= `', only the auto-correlation of the first map will be computed)
    \item[{outfile = }]\mytarget{fac:anafast:outfile}%
 Defines the output file with the power spectrum. If only
      one input map is provided, {\tt outfile} will contains its auto-spectra,
      if 2 maps are provided, {\tt outfile} will contain their
      cross-spectra. Note in particular that in the latter case, the $C^{T\times E}_l$ power
      spectrum will be build from the $T$ field of the 1st (possibly polarized) map, and the $E$
      field of the second polarized map.
(default= cl\_out.fits)
     \item[{simul\_type = }]\mytarget{fac:anafast:simul_type}%
 Defines which map(s) to analyse, 1=temperature only, 2=temperature AND polarisation.
(default= 1)
     \item[{nlmax = }]\mytarget{fac:anafast:nlmax}%
 Defines the maximum $\ell$ value 
to be used. See the Recommendations for Users. 
(default= 64)
 \item[{maskfile = }]\mytarget{fac:anafast:maskfile}%
 Defines the FITS file containing the pixel mask(s) or
 weighting scheme(s) by which the map(s) read from {\tt infile} will be
 multiplied before analysis. If the file contains several fields, the first
one in which at least one pixel is non-zero will be used. This option can be
 used to, for instance, apply
the WMAP Kp intensity mask to the data (see
\htmladdnormallink{https://lambda.gsfc.nasa.gov}{https://lambda.gsfc.nasa.gov}), 
but it will {\em not} handle the WMAP composite mask correctly.
Can be used in conjonction with {\tt theta\_cut\_deg}. Masked or weighted pixels
 will be correctly accounted when performing the monopole and dipole regression.\\
{\em Note:} The mask's resolution ($\nside$) and ordering can be different from the input map(s)
one's, and the mask will be pro/down-graded and reordered to match the map. On the
other hand, the mask and maps coordinates will always be presumed to match (ie, no
attempt of rotation of the mask will be made).
(default= `': no mask, all valid pixels are used)
 \item[{theta\_cut\_deg = }]\mytarget{fac:anafast:theta_cut_deg}%
 Defines the latitude (in degrees) of 
an optional, straight symmetric cut around the equator.  Pixels located within
that cut ($|b|<${\tt theta\_cut\_deg}) are ignored.
(default= $0^\circ$: no cut)
 \item[{iter\_order = }]\mytarget{fac:anafast:iter_order}%
 Defines the maximum order of quadrature 
iteration to be used. (default=0, no iteration).
For details, see the \htmlref{{\tt map2alm\_iterative}}{sub:map2alm_iterative} routine
described in the \linklatexhtml{''Fortran Subroutines''}{subroutines.pdf}{subroutines.htm} document.
\mylink{fac:anafast:iter_order}{iter\_order}$>0$ can {\em not} be used together with \mylink{fac:anafast:won}{won}=2.
 \item[{outfile\_alms = }]\mytarget{fac:anafast:outfile_alms}%
 Defines the name of the file 
containing the $a_{\ell m}$  coefficients of the first map
which can be written optionally.   (default= no entry ---
$a_{\ell m}$s are not written to a file)
 \item[{outfile\_alms2 = }]\mytarget{fac:anafast:outfile_alms2}%
 Defines the name of the file 
containing the $a_{\ell m}$  coefficients of the second map, if any,
which can be written optionally.   (default= no entry ---
$a_{\ell m}$s are not written to a file)
 \item[{plmfile = }]\mytarget{fac:anafast:plmfile}%
 Defines the name for an input file
    containing  precomputed Legendre polynomials $P_{\ell m}$.
(default= no entry --- \thedocid\ executes the recursive evaluation 
of $P_{\ell m}$s)
\item[{w8file = }]\mytarget{fac:anafast:w8file}%
 Defines name for an input file containing ring-based or pixel-based
  weights (depending on the value of \mylink{fac:anafast:won}{won}) in the improved quadrature mode (default= no entry ---
see \htmlref{''Default file names and directories'' in introduction}{fac:subsec:defdir})
\item[{w8filedir = }]\mytarget{fac:anafast:w8filedir}%
 Gives the directory where the ring weight files are
to be found (default= no entry --- \thedocid\ searches the default
directories, see \htmlref{''Default file names and directories'' in introduction}{fac:subsec:defdir})
\item[{won = }]\mytarget{fac:anafast:won}%
 Set this to 1 if ring-based quadrature weight files are to be used,
 or to 2 to use pixel-based weight files instead;
otherwise set it to 0. (default= 0).
\mylink{fac:anafast:won}{won}=2 can {\em not} be used together with \mylink{fac:anafast:iter_order}{iter\_order}$>0$.
\item[{regression = }]\mytarget{fac:anafast:regression}%
 {{Sets the degree of the regression made on the
input map before doing the power spectrum analysis. 
The regression is a minimal variance fit (assuming a uniform noise) 
made on valid (unflagged and unmasked) pixels, out of the symmetric cut (if
any). In case of cut sky analysis, such a regression reduces the monopole
and dipole leakage to higher $\ell$'s.\\
0 : no regression, does the $a\_{\ell m}$ analysis on the raw map\\
1 : removes the best fit monopole first\\
2 : removes the best fit monopole and dipole first\\
default = 0.}}
 
  \end{qulist}
\end{qualifiers}
\vfill

\begin{codedescription}
{Anafast reads one or two binary FITS-files containing a \healpix map. These
files can each contain
a temperature map or both temperature and polarisation (Q,U) maps. Anafast analyses
the map(s) and makes an output ascii-FITS file containing the angular auto or cross
power spectra $C^{TT}_\ell$s
(and $C^{EE}_\ell$, $C^{BB}_\ell$, $C^{TE}_\ell$, $C^{TB}_\ell$ and $C^{EB}_\ell$ if specified, as well
as $C^{ET}_\ell$, $C^{BT}_\ell$ and $C^{BE}_\ell$ if two maps are provided). 
Here $C^{TE}_\ell$ is meant as the power
spectrum built from the $T$ field of the first (polarized) map, and the $E$
field of the second polarized map, while it is the other way around for $C^{ET}_\ell$.
Anafast produces $C_\ell$s up to a specified maximum $\ell$-value
(see Recommendations for Users). 
If requested, the computed $a_{\ell m}$ coefficients 
can  be written to a FITS file. This file can be used in the 
constrained realisation mode  of  \htmlref{synfast}{fac:synfast}. 

Anafast permits two execution modes that allow to improve 
the quadrature accuracy: 
(1) the  ring weight corrected quadrature, and
(2)  the  iterative scheme. 
Using the ring weights does not increase the execution time.  
The precomputed ring weights to be used for each 
\healpix resolution $\nside$ are provided in
the {\tt \$HEALPIX/data} directory. 
The more sophisticated iterative scheme increases the
accuracy more effectively than the weighted ring scheme,
but its disadvantage is that the time for the analysis
increases, 1 iteration takes 3 times as long, 2 iterations 5 times as
long on so forth, since each order of iteration requires one more forward
and backward transform. 

The spherical harmonics evaluation uses  a
recurrence on associated Legendre polynomials $P_{\ell m}(\theta)$. 
This recurrence consumed most of the CPU time used by \thedocid\ up to version
2.15. We have therefore included an option to load precomputed values for the
$P_{\ell m}(\theta)$ from a file generated by the \healpix facility
\htmlref{plmgen}{fac:plmgen}. Since the introduction of accelerated spherical
harmonic transforms in \healpix v2.20, this feature is obsolete and should no
longer be used. 

When dealing with polarized signal maps, the \thedocid\ behavior will depend on the value of the \texttt{POLCCONV} FITS keyword
(see \htmlref{note on POLCCONV}{intro:polcconv} in \linklatexhtml{The \healpix Primer}{intro.pdf}{intro.htm})

In \htmlref{version 3.50}{sub:new3p50} a bug affecting the result of \thedocid has been fixed.
It occured previously when 
\mylink{fac:anafast:iter_order}{\texttt{iter\_order}}$>0$ was used in conjonction
with a \mylink{fac:anafast:maskfile}{\texttt{maskfile}} and/or a restrictive 
\mylink{fac:anafast:theta_cut_deg}{\texttt{theta\_cut\_deg}}). The result was correct when
the mask (if any) was applied to the map prior to the \thedocid calling, or when no iteration was requested.}
\end{codedescription}

\begin{datasets}
{
\begin{tabular}{p{0.3\hsize} p{0.35\hsize}} \hline  
  \textbf{Dataset} & \textbf{Description} \\ \hline
                   &                      \\ %%% for presentation
  data/weight\_ring\_n0xxxx.fits & Files containing ring weights
                   for the \thedocid\ improved quadrature mode.\\ 
                   &                      \\ \hline %%% for presentation
\end{tabular}
} 
\end{datasets}

\begin{support}
  \begin{sulist}{} %%%% NOTE the ''extra'' brace here %%%%
  \item[\htmlref{synfast}{fac:synfast}] This \healpix facility can generate a map for analysis by \thedocid.
  \item[\htmlref{alteralm}{fac:alteralm}] This \healpix Fortran facility can be
  used to modify the $a_{\ell m}$ extracted by \thedocid\ before they are used as
  constraints on a \htmlref{synfast}{fac:synfast} run.
  \item[\htmlref{plmgen}{fac:plmgen}] This \healpix facility can be used to generate precomputed Legendre polynomials.		
  \end{sulist}
\end{support}

\begin{examples}{1}
{
\begin{tabular}{ll} %%%% use this tabular format %%%%
anafast  \\
\end{tabular}
}
{
Anafast runs in interactive mode --- self-explanatory.
}
\end{examples}

\vfill\newpage

\begin{examples}{2}
{%
\begin{tabular}{ll} %%%% use this tabular format %%%%
anafast  filename \\
\end{tabular}%
}
{When 'filename' is present, \thedocid\ enters the non-interactive mode and parses
its inputs from the file 'filename'. This has the following
structure: the first entry is a qualifier which announces to the parser
which input immediately follows. If this input is omitted in the
input file, the parser assumes the default value.
If the equality sign is omitted, then the parser ignores the entry.
In this way comments may also be included in the file.
In this example, the file contains the following qualifiers:\hfill\newline
\fileparam{\mylink{fac:anafast:simul_type}{simul\_type}%
 = 1}
\fileparam{\mylink{fac:anafast:nlmax}{nlmax}%
 = 64}
\fileparam{\mylink{fac:anafast:theta_cut_deg}{theta\_cut\_deg}%
 = 0}
\fileparam{\mylink{fac:anafast:iter_order}{iter\_order}%
 = 0}
\fileparam{\mylink{fac:anafast:infile}{infile}%
 = map.fits}
\fileparam{\mylink{fac:anafast:outfile}{outfile}%
 = cl\_out.fits}
\fileparam{\mylink{fac:anafast:regression}{regression}%
 = 0}
Anafast reads the map from {\em map.fits}, makes an analysis and produces $C^T_l$s up to l=64.
This powerspectrum is saved in the file {\em cl\_out.fits}. 
No galactic cut is excised and no iterations are performed.
As \mylink{fac:anafast:regression}{regression}{}%
 is set to 0 (its default value) the map is analyzed as
is, without prior best fit removal of the monopole nor the dipole.

Since\hfill\newline
\fileparam{\mylink{fac:anafast:infile2}{infile2}%
}
\fileparam{\mylink{fac:anafast:outfile_alms}{outfile\_alms}%
}
\fileparam{\mylink{fac:anafast:outfile_alms2}{outfile\_alms2}%
}
\fileparam{\mylink{fac:anafast:w8file}{w8file}%
}
\fileparam{\mylink{fac:anafast:w8filedir}{w8filedir}%
}
\fileparam{\mylink{fac:anafast:plmfile}{plmfile}%
}
\fileparam{\mylink{fac:anafast:maskfile}{maskfile}%
}
were omitted, they take their default values (empty strings). 
This means that no file for precomputed
Legendre polynomials is read, no second map is read, no mask is applied, and \thedocid\ does not save the $a_{\ell m}$ values
from the analysis.

Also since\hfill\newline
\fileparam{\mylink{fac:anafast:won}{won}%
}
is not given, it takes its default value 0, which means that quadrature 
weights are not used.

}
\end{examples}

\begin{release}
  \begin{relist}
    \item Initial release (\healpix 0.90)
    \item Optional non-interactive operation. Proper FITS file
    support. Improved reccurence algorithm for $P_{\ell m}(\theta)$ which can compute to higher $\ell$ values. New functionality: arbitrary order of iterations, precomputed
    $P_{\ell m}$, dumping of $a_{\ell m}$. (\healpix 1.00)
    \item New functionality: possibility of removing the best fit monopole
    and dipole. New Parser. Can be linked to FFTW (\healpix 1.20)
    \item New functionality: addition of {\tt{maskfile}} (\healpix 2.0)
    \item Bug correction: correct interaction of iterative scheme with masked pixels (\healpix 2.01)
    \item New functionality: cross-correlation of 2 maps; Correction of this documentation: the code expects {\tt maskfile} and
not {\tt mask\_file}  (\healpix 2.1)
    \item Bug correction: now correctly supports mask pro/down-grading
    \item Support for pixel-based quadrature weights when \mylink{fac:anafast:won}{\texttt{won=2}} (\healpix 3.40)
  \end{relist}
\end{release}

\begin{messages}
{
\begin{tabular}{p{0.25\hsize} p{0.1\hsize} p{0.35\hsize}} \hline  
  \textbf{Message} & \textbf{Severity} & \textbf{Text} \\ \hline
                   &                   &   \\ %%% for presentation
can not allocate memory for array xxx &  Fatal & You do not have
                   sufficient system resources to run this
                   facility at the map resolution you required. 
  Try a lower map resolution.  \\ 
                   &                   &   \\ \hline %%% for presentation
\end{tabular}
} 
\end{messages}
%
\newpage
%
