% -*- LaTeX -*-


\renewcommand{\facname}{{init\_healpix }}
\renewcommand{\FACNAME}{{INIT\_HEALPIX }}

\sloppy



\title{\healpix IDL Facility User Guidelines}
\docid{init\_healpix} \section[init\_healpix and !healpix system variable]{ }
\label{idl:init_healpix}
\docrv{Version 1.0}
\author{Eric Hivon}
\abstract{This document describes the \healpix IDL facility \facname.}




\begin{facility}
{This IDL facility creates an IDL system variable (!HEALPIX) containing various
\healpix related quantities}
{src/idl/misc/init\_healpix.pro}
\end{facility}

\begin{IDLformat}
{\FACNAME [,\mylink{idl:init_healpix:VERBOSE}{VERBOSE=}%
]}
\end{IDLformat}

\begin{keywords}
  \begin{kwlist}{} %%% extra brace
    \item[VERBOSE\mytarget{idl:init_healpix:VERBOSE}%
 =] if set, turn on the verbose mode, giving a short
    description of the variables just created.
  \end{kwlist}
\end{keywords}  

\begin{codedescription}
{\thedocid\  defines the IDL system variable and structure !HEALPIX
containing several quantities and character string necessary to \healpix, eg : allowed
resolution parameters Nside, full path to package directory, package version...}
\end{codedescription}



\begin{related}
  \begin{sulist}{} %%%% NOTE the ``extra'' brace here %%%%
    \item[idl] version \idlversion or more is necessary to run \thedocid.	
    \item[!HEALPIX] IDL system variable defined by \thedocid.
  \end{sulist}
\end{related}

\begin{examples}{1}
{
\begin{tabular}{ll} %%%% use this tabular format %%%%
\thedocid,/verbose
\end{tabular}
}
{\thedocid\  will create the system variable !Healpix, and give a short
description of the tags available, as shown below

\texttt{\footnotesize{Initializing !HEALPIX system variable \newline
This system variable contains some information on Healpix : \newline
!HEALPIX.VERSION   = current version number,\newline
!HEALPIX.DATE      = date of release,\newline
!HEALPIX.DIRECTORY = directory containing Healpix package,\newline
!HEALPIX.PATH      = structure containing:\newline
!HEALPIX.PATH.BIN  = structure containing binary path :\newline
!HEALPIX.PATH.BIN.CXX  =     C++\newline
!HEALPIX.PATH.BIN.F90  =     Fortran90\newline
!HEALPIX.PATH.DATA = path to data subdirectory,\newline
!HEALPIX.PATH.DOC  = path to doc subdirectories (.html, .pdf),\newline
!HEALPIX.PATH.SRC  = path to src subdirectory,\newline
!HEALPIX.PATH.TEST = path to test subdirectory,\newline
!HEALPIX.NSIDE     = list of all valid values of Nside parameter,\newline
!HEALPIX.BAD\_VALUE = value of flag given to missing pixels in FITS files,\newline
!HEALPIX.COMMENT   = this description.
}}
}

\end{examples}%
%----------------
\begin{examples}{2}
{
\begin{tabular}{ll} %%%% use this tabular format %%%%
help, !healpix, /structure
\end{tabular}
}
{will print the content of the !Healpix system structure.
}
\end{examples}
% %----------------
% \begin{examples}{3}
% {
% \begin{tabular}{ll} %%%% use this tabular format %%%%
% print, !healpix.comment, form='(a)'
% \end{tabular}
% }
% {will print the content of the !Healpix system structure.
% }
% \end{examples}

