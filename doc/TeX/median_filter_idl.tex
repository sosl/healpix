% -*- LaTeX -*-

\renewcommand{\facname}{{median\_filter }}
\renewcommand{\FACNAME}{{MEDIAN\_FILTER }}
\sloppy

\title{\healpix IDL Facility User Guidelines}
\docid{\facname} \section[\facname]{ }
\label{idl:median_filter}
\docrv{Version 1.1}
\author{Eric Hivon}
\abstract{This document describes the \healpix facility \facname.}




\begin{facility}
{This IDL facility allows the median filtering of a Healpix map.}
{src/idl/toolkit/median\_filter.pro}
\end{facility}

\begin{IDLformat}
{\FACNAME(InputMap, Radius, MedianMap [,ORDERING=, /RING, /NESTED, /FILL\_HOLES, /DEGREES, /ARCMIN])}
\end{IDLformat}

\begin{qualifiers}
  \begin{qulist}{} %%%% NOTE the ``extra'' brace here %%%%
	\item[{InputMap}] (IN)
	either an IDL array containing a full sky Healpix map to filter ('online' usage), 
        or the name of an external FITS file containing a full sky or cut sky map

 	\item[{Radius}] (IN)
	  radius of the disk on which the median is computed.
            It is in Radians, unless {\tt /DEGREES} or {\tt /ARCMIN} are set

 	\item[{MedianMap}] (OUT)
	   either an IDL variable containing on output the filtered map,
        or the name of an external FITS file to contain the map. Should be of
	   same type of {\tt InputMap}. Flagged pixels (ie, having the value
	   {\tt !healpix.bad\_value}) are left unchanged, unless {\tt /FILL\_HOLES} is set.

  \end{qulist}
\end{qualifiers}

\begin{keywords}
  \begin{kwlist}{} %%% extra brace
    \item[{/ARCMIN}] If set, {\tt Radius} is in arcmin rather than radians

    \item[{/DEG}] If set, {\tt Radius} is in degrees rather than radians

	\item[{/FILL\_HOLES}] If set, flagged pixels are replaced with the
	median of the valid pixels found within a distance {\tt Radius}. If
	there are any.

 	\item[{/NESTED}] Same as ORDERING='NESTED'

 	\item[{ORDERING=}] 
	  Healpix map ordering, should be either 'RING' or 'NESTED'. Only
	  applies to 'online' usage.

 	\item[{/RING}] Same as ORDERING='RING'

   \end{kwlist}
\end{keywords}

\begin{codedescription}
{\facname allows the median filtering of a Healpix map. Each pixel
  of the output map is the median value of the input  map pixels found within a disc of given
  radius centered on that pixel. Flagged
  pixels can be either left unchanged or 'filled in' with that same scheme. \\
If the map is polarized, each of the three Stokes components is filtered
  separately. \\
The input and output can either be arrays or FITS files, but they to be both
  arrays or both FITS files.}
\end{codedescription}



\begin{related}
  \begin{sulist}{} %%%% NOTE the ``extra'' brace here %%%%
  \item[idl] version \idlversion or more is necessary to run \facname
  \end{sulist}
\end{related}


\begin{example}
{
\begin{tabular}{l} %%%% use this tabular format %%%%
median\_filter  ('map.fits', 10., /arcmin, 'med.fits') \\
\end{tabular}
}
{Writes in 'med.fits' the median filtered map of 'map.fits' using a disc radius
  of 10 arcmin}
\end{example}


\begin{example}
{
\begin{tabular}{l} %%%% use this tabular format %%%%
map =  randomn(seed, nside2npix(256)) \\
median\_filter  (map, 0.5, /deg, med) \\
\end{tabular}
}
{Returns in {\tt med} the median filtered map of {\tt map} using a disc radius
  of 0.5 degrees}
\end{example}


