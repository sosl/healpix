% -*- LaTeX -*-

\sloppy

\title{\healpix IDL Facility User Guidelines}
\docid{ismoothing} \section[ismoothing]{ }
\label{idl:ismoothing}
\docrv{Version 1.0}
\author{Eric Hivon}
\abstract{This document describes the \healpix IDL facility \thedocid.}

\begin{facility}
{This IDL facility provides an interface to F90 '\htmlref{smoothing}{fac:smoothing}' facility. It can be
used to smooth a \healpix map by an arbitrary circular 'beam' defined by its
Legendre window function (or its FWHM if it is assumed Gaussian)}
{src/idl/interfaces/ismoothing.pro}
\end{facility}

\begin{IDLformat}
{ISMOOTHING, 
\mylink{idl:ismoothing:map1_in}{map1\_in},  
\mylink{idl:ismoothing:map2_out}{map2\_out},[
\mylink{idl:ismoothing:beam_file}{beam\_file=},
\mylink{idl:ismoothing:binpath}{binpath=},
\mylink{idl:ismoothing:double}{/double},
\mylink{idl:ismoothing:fwhm_arcmin}{fwhm\_arcmin=},
\mylink{idl:ismoothing:help}{/help},
\mylink{idl:ismoothing:iter_order}{iter\_order=},
\mylink{idl:ismoothing:keep_tmp_files}{keep\_tmp\_files=}, 
\mylink{idl:ismoothing:lmax}{lmax=},
\mylink{idl:ismoothing:lmax}{nlmax=},
\mylink{idl:ismoothing:nested}{/nested},
\mylink{idl:ismoothing:ordering}{ordering=}, 
\mylink{idl:ismoothing:plmfile}{plmfile=},
\mylink{idl:ismoothing:regression}{regression=},
\mylink{idl:ismoothing:ring}{/ring},
\mylink{idl:ismoothing:simul_type}{simul\_type=},
\mylink{idl:ismoothing:silent}{/silent},
\mylink{idl:ismoothing:theta_cut_deg}{theta\_cut\_deg=},  
\mylink{idl:ismoothing:tmpdir}{tmpdir=},
\mylink{idl:ismoothing:won}{/won},
\mylink{idl:ismoothing:w8file}{w8file=},
\mylink{idl:ismoothing:w8dir}{w8dir=}%            
]}
\end{IDLformat}

\begin{qualifiers}
  \begin{qulist}{} %%%% NOTE the ``extra'' brace here %%%%
   \item[map1\_in] \mytarget{idl:ismoothing:map1_in} required input: input map, can be a FITS file, or a memory array containing the
        map to smooth
    \item[map2\_out] \mytarget{idl:ismoothing:map2_out} required output: output smoothed map, can be a FITS file, or a memory array
  \end{qulist}
\end{qualifiers}

\begin{keywords}
  \begin{kwlist}{} %%% extra brace
 \item[beam\_file=] \mytarget{idl:ismoothing:beam_file} beam window function, either a FITS file or an array
		(see \htmlref{''Beam window function files''}{fac:subsec:beamfiles} section
		in the \linklatexhtml{\healpix Fortran Facilities document}{facilities.pdf}{facilities.htm}). 

 \item[binpath=] \mytarget{idl:ismoothing:binpath} full path to back-end routine \default {\$HEXE/smoothing, then \$HEALPIX/bin/smoothing}\\
              -- a binpath starting with / (or $\backslash$), $~$ or \$ is interpreted as absolute\\
              -- a binpath starting with ./ is interpreted as relative to current directory\\
              -- all other binpathes are relative to \$HEALPIX

 \item[/double] \mytarget{idl:ismoothing:double} if set, I/O is done in double precision \default {single precision I/O}

 \item[fwhm\_arcmin=] \mytarget{idl:ismoothing:fwhm_arcmin} gaussian beam Full Width Half Maximum in arc-minutes \default {0}

 \item[/help]      \mytarget{idl:ismoothing:help} if set, prints extended help

\item[iter\_order=] \mytarget{idl:ismoothing:iter_order} order of iteration in the analysis \default {0}

\item[/keep\_tmp\_files] \mytarget{idl:ismoothing:keep_tmp_files} if set, temporary files are not discarded at the end of the
                  run

 \item[lmax=, nlmax=]   \mytarget{idl:ismoothing:lmax} maximum multipole of smoothing \default {determined by back-end routine (ie, smoothing)}

\item[/nested] \mytarget{idl:ismoothing:nested} if set, signals that *all* maps and mask read online are in
   NESTED scheme (does not apply to FITS file), 
\mylink{idl:ismoothing:ring}{{\tt /ring}} and 
\mylink{idl:ismoothing:ordering}{{\tt Ordering}}

\item[ordering=] \mytarget{idl:ismoothing:ordering} either 'RING' or 'NESTED', ordering of online maps and masks,
see 
\mylink{idl:ismoothing:ring}{{\tt /ring}} and 
\mylink{idl:ismoothing:nested}{{\tt /nested}}

\item[plmfile=] \mytarget{idl:ismoothing:plmfile} FITS file containing precomputed Spherical Harmonics (deprecated) \default {no file}

\item[regression=] \mytarget{idl:ismoothing:regression} 0, 1 or 2, regress out best fit monopole and/or dipole before
    alm analysis
  \default {0, analyze raw map} 

\item[/ring] \mytarget{idl:ismoothing:ring} see 
\mylink{idl:ismoothing:nested}{{\tt /nested}} and 
\mylink{idl:ismoothing:ordering}{{\tt Ordering}} above

\item[simul\_type=] \mytarget{idl:ismoothing:simul_type} 1 or 2, analyze temperature only or temperature + polarization

\item[/silent]    \mytarget{idl:ismoothing:silent} if set, works silently

\item[theta\_cut\_deg=] \mytarget{idl:ismoothing:theta_cut_deg} cut around the equatorial plane 

\item[tmpdir=]      \mytarget{idl:ismoothing:tmpdir} directory in which are written temporary files 
\default {IDL\_TMPDIR (see IDL documentation)}

\item[/won]     \mytarget{idl:ismoothing:won} if set, a weighting scheme is used to improve the quadrature
    \default {apply weighting}

\item[w8file=]    \mytarget{idl:ismoothing:w8file} FITS file containing weights 
     \default {determined automatically by back-end routine}.
   Do not set this keyword unless you really know what you are doing

\item[w8dir=]     \mytarget{idl:ismoothing:w8dir} directory where the weights are to be found 
        \default {determined automatically by back-end routine}

  \end{kwlist}
\end{keywords}  

\begin{codedescription}
{\thedocid\ is an interface to '\htmlref{smoothing}{fac:smoothing}' F90 facility. It
requires some disk space on which to write the parameter file and the other
temporary files. Most data can be provided/generated as an external FITS
file, or as a memory array.}
\end{codedescription}



\begin{related}
  \begin{sulist}{} %%%% NOTE the ``extra'' brace here %%%%
    \item[idl] version \idlversion or more is necessary to run \thedocid.
    \item[smoothing] F90 facility called by \thedocid.
    \item[\htmlref{beam2bl}{idl:beam2bl}] This IDL facility computes a transfer
(or window) function $b(l)$ (such as the ones required by \thedocid) for a given
circular beam profile $b(\theta)$
    \item[\htmlref{ialteralm}{idl:ialteralm}] IDL Interface to F90 \htmlref{alteralm}{fac:alteralm}
    \item[\htmlref{ianafast}{idl:ianafast}] IDL Interface to F90 \htmlref{anafast}{fac:anafast} and C++ anafast\_cxx
    \item[\htmlref{iprocess\_mask}{idl:iprocess_mask}] IDL Interface to F90 \htmlref{process\_mask}{fac:process_mask}
%    \item[\htmlref{ismoothing}{idl:ismoothing}] IDL Interface to F90 \htmlref{smoothing}{fac:smoothing}
    \item[\htmlref{isynfast}{idl:isynfast}] IDL Interface to F90 \htmlref{synfast}{fac:synfast}
  \end{sulist}
\end{related}

\begin{example}
{
\begin{tabular}{l} %%%% use this tabular format %%%%
 whitenoise = randomn(seed, \htmlref{nside2npix}{idl:nside2npix}(256))  \\
 \thedocid, whitenoise, rednoise, fwhm=120, /ring, simul=1,/silent  \\
 \htmlref{mollview}{idl:mollview}, whitenoise,title='White noise'  \\
 \htmlref{mollview}{idl:mollview}, rednoise,  title='Smoothed white Noise'  
\end{tabular}
}
{
will generate and plot a white noise map and its smoothed version
}
\end{example}


