% -*- LaTeX -*-

\sloppy

\title{\healpix IDL Facility User Guidelines}
\docid{ismoothing} \section[ismoothing]{ }
\label{idl:ismoothing}
\docrv{Version 1.0}
\author{Eric Hivon}
\abstract{This document describes the \healpix IDL facility \thedocid.}

\begin{facility}
{This IDL facility provides an interface to F90 'smoothing' facility}
{src/idl/interfaces/ismoothing.pro}
\end{facility}

\begin{IDLformat}
{ISMOOTHING,  map1\_in, map\_out [, beam\_file=, binpath=, double=,
fwhm\_arcmin=, 
help=, iter\_order=, keep\_tmp\_files=, 
       lmax=, nlmax=, nested=, ordering=, plmfile=, regression=, ring=, 
       simul\_type=, silent=, theta\_cut\_deg=, tmpdir=, won=, w8file=, w8dir=]}
\end{IDLformat}

\begin{qualifiers}
  \begin{qulist}{} %%%% NOTE the ``extra'' brace here %%%%
   \item[map1\_in] required input: input map, can be a FITS file, or a memory array containing the
        map to smooth
    \item[map2\_out] required output: output smoothed map, can be a FITS file, or a memory array
  \end{qulist}
\end{qualifiers}

\begin{keywords}
  \begin{kwlist}{} %%% extra brace
 \item[beam\_file=] beam window function, either a FITS file or an array

 \item[binpath=] full path to back-end routine \default {\$HEXE/smoothing, then \$HEALPIX/bin/smoothing}\\
              -- a binpath starting with / (or $\backslash$), $~$ or \$ is interpreted as absolute\\
              -- a binpath starting with ./ is interpreted as relative to current directory\\
              -- all other binpathes are relative to \$HEALPIX

 \item[/double]    if set, I/O is done in double precision \default {single precision I/O}

 \item[fwhm\_arcmin=] gaussian beam FWHM in arcmin \default {0}

 \item[/help]      if set, prints extended help

\item[iter\_order=] order of iteration in the analysis \default {0}

\item[/keep\_tmp\_files] if set, temporary files are not discarded at the end of the
                  run

 \item[lmax=, nlmax=]   maximum multipole of smoothing \default {determined by back-end routine (ie, smoothing)}

\item[/nested] if set, signals that *all* maps and mask read online are in
   NESTED scheme (does not apply to FITS file), see also /ring and Ordering

\item[ordering=] either 'RING' or 'NESTED', ordering of online maps and masks,
 see /ring and Ordering

\item[plmfile=] FITS file containing precomputed Spherical Harmonics \default {no file}

\item[regression=] 0, 1 or 2, regress out best fit monopole and/or dipole before
    alm analysis
  \default {0, analyze raw map} 

\item[/ring] see /nested and Ordering above

\item[simul\_type=] 1 or 2, analyze temperature only or temperature + polarization

\item[/silent]    if set, works silently

\item[theta\_cut\_deg=] cut around the equatorial plane 

\item[tmpdir=]      directory in which are written temporary files 
\default {IDL\_TMPDIR (see IDL documentation)}

\item[/won]     if set, a weighting scheme is used to improve the quadrature
    \default {apply weighting}

\item[w8file=]    FITS file containing weights 
     \default {determined automatically by back-end routine}.
   Do not set this keyword unless you really know what you are doing

\item[w8dir=]     directory where the weights are to be found 
        \default {determined automatically by back-end routine}

  \end{kwlist}
\end{keywords}  

\begin{codedescription}
{\thedocid\ is an interface to 'smoothing' F90 facility. It
requires some disk space on which to write the parameter file and the other
temporary files. Most data can be provided/generated as an external FITS
file, or as a memory array.}
\end{codedescription}



\begin{related}
  \begin{sulist}{} %%%% NOTE the ``extra'' brace here %%%%
    \item[idl] version \idlversion or more is necessary to run \thedocid.
    \item[smoothing] F90 facility called by \thedocid.
    \item[\htmlref{ianafast}{idl:ianafast}] IDL Interface to F90 anafast and C++ anafast\_cxx
    \item[\htmlref{isynfast}{idl:isynfast}] IDL Interface to F90 synfast
%    \item[\htmlref{ismoothing}{idl:ismoothing}] IDL Interface to F90 smoothing
  \end{sulist}
\end{related}

\begin{example}
{
\begin{tabular}{l} %%%% use this tabular format %%%%
 whitenoise = randomn(seed, \htmlref{nside2npix}{idl:nside2npix}(256))  \\
 \thedocid, whitenoise, rednoise, fwhm=120, /ring, simul=1,/silent  \\
 \htmlref{mollview}{idl:mollview}, whitenoise,title='White noise'  \\
 \htmlref{mollview}{idl:mollview}, rednoise,  title='Smoothed white Noise'  
\end{tabular}
}
{
will generate and plot a white noise map and its smoothed version
}
\end{example}


