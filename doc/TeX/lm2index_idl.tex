% -*- LaTeX -*-

\sloppy

\title{\healpix IDL Facility User Guidelines}
\docid{lm2index} \section[lm2index]{ }
\label{idl:lm2index}
\docrv{Version 2.0}
\author{Eric Hivon}
\abstract{This document describes the \healpix facility lm2index.}

\begin{facility}
{This IDL routine provides a means to convert the $a_{\ell m}$ degree and order
$(\ell, m)$ into the index $i=\ell^2 + \ell + m + 1$ (in order to be fed to
alm2fits routine for instance)}
{src/idl/misc/lm2index.pro}

\end{facility}

\begin{IDLformat}
{LM2INDEX, 
\mylink{idl:lm2index:l}{l}, 
\mylink{idl:lm2index:m}{m}, 
\mylink{idl:lm2index:index}{index}
}
\end{IDLformat}

\begin{qualifiers}
  \begin{qulist}{} %%%% NOTE the ``extra'' brace here %%%%
    \item[l\mytarget{idl:lm2index:l}%
] Long array containing on INPUT the order $\ell$. 
    \item[m\mytarget{idl:lm2index:m}%
] Long array containing on INPUT the degree $m$. 
    \item[index\mytarget{idl:lm2index:index}%
] Long array containing on OUTPUT the index \hfill\newline
                 $i=\ell^2 + \ell + m + 1$.
  \end{qulist}
\end{qualifiers}

\begin{codedescription}
{\thedocid\ converts $(\ell, m)$ into $i=\ell^2 + \ell + m + 1$. Note that by
definition $0 \le |m|\le \ell$ (the routine does not check for this).
}
\end{codedescription}



\begin{related}
  \begin{sulist}{} %%%% NOTE the ``extra'' brace here %%%%
    \item[idl] version \idlversion or more is necessary to run \thedocid.
    \item[\htmlref{fits2alm}{idl:fits2alm}] reads a FITS file containing
    $a_{\ell m}$ values.
    \item[\htmlref{alm2fits}{idl:alm2fits}] writes $a_{\ell m}$ values into a FITS file.
    \item[\htmlref{index2lm}{idl:index2lm}] routine complementary to \thedocid:
    converts $i=\ell^2 + \ell + m + 1$ into $(\ell, m)$.
  \end{sulist}
\end{related}

\begin{example}
{
\begin{tabular}{l} %%%% use this tabular format %%%%
\thedocid, l, m, index \\
\end{tabular}
}
{
will return in {\tt index} the value $\ell^2 + \ell + m + 1$
}
\end{example}

