

\sloppy

\title{\healpix Fortran Subroutines Overview}
\docid{parse\_init,~parse\_int,~...,~parse\_finish} \section[parse\_init, parse\_int, $\ldots$, parse\_finish]{ }
\label{sub:parse_xxx}
\docrv{Version 1.1}
\author{Martin Reinecke, Eric Hivon}
\abstract{This document describes the \healpix Fortran90 subroutines in module paramfile\_io.}

\begin{facility}
{The Fortran90 module paramfile\_io contains functions to obtain
parameters from parameter files or interactively}
{\modParamfileIo}
\end{facility}

\begin{arguments}
{
\begin{tabular}{p{0.3\hsize} p{0.05\hsize} p{0.1\hsize} p{0.45\hsize}} \hline  
\textbf{name\&dimensionality} & \textbf{kind} & \textbf{in/out} & \textbf{description} \\ \hline
                   &   &   &                           \\ %%% for presentation
fname & CHR & IN & file containing the simulation parameters.
                   If empty, parameters are obtained interactively.\\
handle & PMF & INOUT & Object of type (paramfile\_handle) used to store parameter information \\
keyname & CHR & IN & name of the required parameter \\
default & XXX & IN & optional argument containing the default value for
                     a given simulation parameter; must be of
                     appropriate type. \\
vmin & XXX & IN & optional argument containing the minimum value for
                     a given simulation parameter; must be of
                     appropriate type. \\
vmax & XXX & IN & optional argument containing the maximum value for
                     a given simulation parameter; must be of
                     appropriate type. \\
descr & CHR & IN & optional argument containing a description of the
                   required simulation parameter \\
filestatus & CHR & IN & optional argument. If present, the parameter
                   must be a valid filename. If filestatus=='new',
                   the file must not exist; if filestatus=='old',
                   the file must exist. \\
code & CHR & IN & optional argument. Contains the name of the executable.\\
silent & LGT & IN & optional argument. If set to {\tt .true.} the parsing
routines will run silently in non-interactive mode (except for warning or error
messages, which will always appear). This is mainly intended for MPI usage where
many processors parse the same parameter file: {\tt silent} can be set to
{\tt .true.} on all CPUs except one.
\end{tabular}
}
\end{arguments}
\newpage

%\ mylink: to avoid automatic processing by make_internal_links.sh

\rule{\hsize}{0.7mm}
\textsc{\large{\textbf{ROUTINES: }}}\hfill\newline
{\tt handle = parse\_init (fname [,silent])} 

\quad initializes the parser to work on the file fname, or interactively, if fname is empty

{\tt intval = parse\_int (handle, keyname [, default, vmin, vmax, descr])} 

{\tt longval = parse\_long (handle, keyname [, default, vmin, vmax, descr])} 

{\tt realval = parse\_real (handle, keyname [, default, vmin, vmax, descr])} 

{\tt doubleval = parse\_double (handle, keyname [, default, vmin, vmax, descr])} 

{\tt stringval = parse\_string (handle, keyname [, default, descr, filestatus])} 

{\tt logicval = parse\_lgt (handle, keyname [, default, descr])} 

\quad These routines obtain integer(i4b), integer(i8b), real(sp), real(dp), character(len=*) and logical values,
respectively. \\
Note: {\tt parse\_string} will expand all environment variables of
the form \$\{XXX\} (eg: \$\{HOME\}). It will also replace a {\em leading} 
\texttt{\textasciitilde/}
%\verb+~+$\!${\tt /}
%%%{\tt \tilde{}/} 
by the user's home directory.

{\tt call parse\_summarize (handle [, code])}

\quad if the parameters were set interactively, this routine will print out a 
parameter file performing the same settings.

{\tt call parse\_check\_unused (handle [, code])}

\quad if a parameter file was read, this routine will print out all the parameters
found in the file but not used by the code. Intended at detecting typos in
parameter names.

{\tt call parse\_finish (handle)}

\quad frees the memory

\begin{example}
{
program who\_r\_u \\
use healpix\_types \\
use paramfile\_io \\
use extension \\
\\
implicit none \\
type(paramfile\_handle) :: handle \\
character(len=256) :: parafile, name \\
real(DP) :: age \\
\\
parafile = ''  \\
if (\htmlref{nArguments()}{sub:narguments} == 1) call \htmlref{getArgument}{sub:getargument}(1, parafile)  \\
handle = parse\_init(parafile)  \\
name  = parse\_string(handle, 'name',descr='Enter your last name: ')  \\
age   = parse\_double(handle, 'age', descr='Enter your age in years: ', \&   \\
   \& default=18.d0,vmin=0.d0)  \\
call parse\_summarize(handle, 'who\_r\_u')  \\
end program who\_r\_u 
}
{If a file is provided as command line argument when running the executable {\tt who\_r\_u}, that file
  will be parsed in search of the lines starting with 'name =' and 'age =',
  otherwise the same questions will be asked interactively.
}
\end{example}

\begin{related}
  \begin{sulist}{} %%%% NOTE the ``extra'' brace here %%%%
  \item[\htmlref{concatnl}{sub:concatnl}] generates from a set of strings the
  multi-line description
  \item[\htmlref{nArguments}{sub:narguments}] returns the number of
  command line arguments
  \item[\htmlref{getArgument}{sub:getargument}] returns the list of command line arguments
  \end{sulist}
\end{related}

\rule{\hsize}{2mm}

\newpage
