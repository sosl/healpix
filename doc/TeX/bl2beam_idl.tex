% -*- LaTeX -*-


\sloppy

\title{\healpix IDL Facility User Guidelines}
\docid{bl2beam} \section[bl2beam]{ }
\label{idl:bl2beam}
\docrv{Version 1.0}
\author{Eric Hivon}
\abstract{This document describes the \healpix IDL facility bl2beam.}

\begin{facility}
{This IDL facility computes a circular beam profile $b(\theta)$ from its transfer (or window) function $b(l)$.
}
{src/idl/misc/bl2beam.pro}
\end{facility}

\begin{IDLformat}
{%\mylink{idl:bl2beam:beam}
{beam}=\thedocid(
\mylink{idl:bl2beam:bl}{bl}, 
\mylink{idl:bl2beam:theta}{theta}, %
[%
\mylink{idl:bl2beam:arcmin}{/ARCMIN} ,
\mylink{idl:bl2beam:degrees}{/DEGREES}, 
\mylink{idl:bl2beam:radians}{/RADIANS}%
])}
\end{IDLformat}

\begin{qualifiers}
  \begin{qulist}{} %%%% NOTE the ``extra'' brace here %%%%
%     \item[beam]\mytarget{idl:bl2beam:bl}%
% 	output (circular) beam profile $b(\theta)$
    \item[bl] \mytarget{idl:bl2beam:bl}%
      input $b(l)$ window function of beam
    \item[theta] \mytarget{idl:bl2beam:theta}%
    angle $\theta$ at which the output beam $b(\theta)$ is to be computed.
  \end{qulist}
\end{qualifiers}

\begin{keywords}
  \begin{kwlist}{} %%% extra brace
    \item[/ARCMIN] \mytarget{idl:bl2beam:arcmin}%
	if set, \mylink{idl:bl2beam:theta}{$\theta$} is in arcmin
    \item[/DEGREES] \mytarget{idl:bl2beam:degrees}%
	if set, $\theta$ is in degrees
    \item[/HELP] \mytarget{idl:bl2beam:help}%
	if set, prints out the help header and exits
    \item[/RADIANS] \mytarget{idl:bl2beam:radians}%
	if set, $\theta$ is in radians
  \end{kwlist}
\end{keywords}  

\begin{codedescription}
{Since an arbitrary beam is related to its SH transform via
\begin{equation}
	b({\bf{r}}) = \sum_{lm} b_{lm} Y_{lm}({\bf{r}}),
\end{equation}
a circular beam has a radial profile (as returned by \thedocid)
   \begin{equation}
	b(\theta) = \sum_l  b(l) P_l(\theta) \frac{2l+1}{4 \pi},
	\end{equation}
   where $P_l$ is Legendre Polynomial
   and \begin{equation}
b(l)=b_{l0} \sqrt{\frac{4 \pi}{2l+1}}
       \end{equation}
 is the beam window (or transfer)
function, expected as input to \thedocid%
}
\end{codedescription}



\begin{related}
  \begin{sulist}{} %%%% NOTE the ``extra'' brace here %%%%
    \item[idl] version \idlversion or more is necessary to run \thedocid.
    \item[\htmlref{beam2bl}{idl:beam2bl}] facility to performs the inverse transform \thedocid.
    \item[\htmlref{bl2fits}{idl:bl2fits}] facility to write a $b(l)$ window function into a FITS file.
    \item[\htmlref{fits2cl}{idl:fits2cl}] facility to read a $b(l)$ window
function from a FITS file
  \end{sulist}
\end{related}

\begin{example}
{
\begin{tabular}{l} %%%% use this tabular format %%%%
bl = \htmlref{gaussbeam}{idl:gaussbeam}(15.d0, 4000, 1) \\
theta = dindgen(1000)/100. \\
beam = \thedocid(bl, theta, /arcmin)\\
plot, theta, beam\\
\end{tabular}
}
{
generates a beam window function for a 15arcmin-FWHM gaussian beam, computes the
beam profile for angles in [0,10] arcmin, and plots the beam profile.
}
\end{example}



