
\sloppy


\title{\healpix Fortran Facility User Guidelines}
\docid{plmgen} \section[plmgen]{\nosectionname}
\label{fac:plmgen}
\docrv{Version 1.1}
\author{Frode K.~Hansen}
\abstract{This document describes the \healpix facility PLMGEN.}

\begin{facility}
{This program can be used to create a file containing 
the precomputed values of the associated Legendre polynomials
$P_{lm}(\theta)$ (and, if requested, of the tensor spherical harmonics) 
for faster execution of the \healpix map analysis/synthesis. 
The  map resolution parameter, $nsmax$,
and the maximum value of the spherical harmonic order $\ell_{max}$
must be specified.

{\bf Note:} Since the introduction of optimized spherical harmonic transforms
in HEALPix v2.20, this code has become obsolete and should no longer be
used.}
{src/f90/plmgen/plmgen.f90}
\end{facility}

\begin{f90facility}
{plmgen}
\end{f90facility}

\begin{qualifiers}
  \begin{qulist}{} %%%% NOTE the ``extra'' brace here %%%%
    \item[{nsmax = }]\mytarget{fac:plmgen:nsmax}%
 Defines the resolution parameter for the map to 
be analysed/synthesized with the precomputed harmonics. 
	(default= 32)
    \item[{nlmax = }]\mytarget{fac:plmgen:nlmax}%
 Defines the $\ell_{max}$ value for the execution.
(default= 64)
     \item[{simul\_type = }]\mytarget{fac:plmgen:simul_type}%
 Defines whether only scalar, or scalar and 
tensor harmonics are to be precomputed, 1=scalar only, 2=scalar AND tensor.
(default=1)
\item[{outfile = }]\mytarget{fac:plmgen:outfile}%
 Defines the name for the file that will 
contain the precomputed harmonics.
(default='plm.fits')
  \end{qulist}
\end{qualifiers}

\begin{codedescription}
{
The recursion of Legendre polynomials 
and tensor harmonics during the  analysis and synthesis 
of \healpix maps can be time consuming. 
Especially when repetitive applications are desired
there is no need to compute the recursions every time. 
For such applications the values of $P_{\ell m}(\theta)$  
can be precomputed with \thedocid\ 
and stored in a file. When using \htmlref{synfast}{fac:synfast} or \htmlref{anafast}{fac:anafast} 
this file can be read in to 
shorten  the analysis/synthesis execution time. 

The memory (and disc) consumption of \thedocid\    is $8 N_\lambda N_p$ bytes, with
$N_\lambda = {\tt nsmax} ({\tt nlmax}+1)({\tt nlmax}+2)$ and
$N_p$ is either 1 or 3, depending whether tensor harmonics are computed. 

Currently an extra limitation $N_\lambda < 2^{31} = 2147483648$ also applies,
corresponding to, eg, {\tt lmax} $\le 1446$ for {\tt nsmax} $=1024$.}
\end{codedescription}

\begin{datasets}
{
\begin{tabular}{p{0.3\hsize} p{0.35\hsize}} \hline  
  \textbf{Dataset} & \textbf{Description} \\ \hline
                   &                      \\ %%% for presentation
 None required & \\
                   &                      \\ \hline %%% for presentation
\end{tabular}
} 
\end{datasets}

\begin{support}
  \begin{sulist}{} %%%% NOTE the ``extra'' brace here %%%%
  \item[\htmlref{synfast}{fac:synfast}] This \healpix facility can generate a map using precomputed harmonics made from plmgen.
  \item[\htmlref{anafast}{fac:anafast}] This \healpix facility can analyse a map using precomputed harmonics.
  \item[plm\_gen] Fortran subroutine used to generate the harmonics
  \end{sulist}
\end{support}

\begin{examples}{1}
{
\begin{tabular}{ll} %%%% use this tabular format %%%%
plmgen  \\
\end{tabular}
}
{
plmgen runs in interactive mode, self-explanatory.
}
\end{examples}

\vfill\newpage

\begin{examples}{2}
{
\begin{tabular}{ll} %%%% use this tabular format %%%%
plmgen  filename \\
\end{tabular}
}
{When `filename' is present, plmgen enters the non-interactive mode 
and parses
its inputs from the file `filename'. This has the following
structure: the first entry is a qualifier which announces to the parser
which input immediately follows. If this input is omitted in the
input file, the parser assumes the default value.
If the equality sign is omitted, then the parser ignores the entry.
In this way comments may also be included in the file.
In this example, the file contains the following qualifiers:\hfill\newline
\fileparam{\mylink{fac:plmgen:simul_type}{simul\_type}%
 = 1}
\fileparam{\mylink{fac:plmgen:nsmax}{nsmax}%
 = 32}
\fileparam{\mylink{fac:plmgen:nlmax}{nlmax}%
 = 86}
\fileparam{\mylink{fac:plmgen:outfile}{outfile}%
 = plm.fits}
}

Creates a binary FITS file called 'plm.fits' containing Legendre polynomials 
up to $\ell$ and $m$ values of 86 for a $\mylink{fac:plmgen:nsmax}{nsmax}%
=32$ map. 
Legendre polynomials for all $\ell$ and $m$ 
values for each angle $\theta$ corresponding to all of the \healpix
pixel center rings will be 
created.
\end{examples}

\begin{release}
  \begin{relist}
    \item Initial release \healpix 1.00
  \end{relist}
\end{release}

\begin{messages}
{
\begin{tabular}{p{0.25\hsize} p{0.1\hsize} p{0.35\hsize}} \hline  
  \textbf{Message} & \textbf{Severity} & \textbf{Text} \\ \hline
                   &                   &   \\ %%% for presentation
can not allocate memory for array xxx &  Fatal & You do not have
                   sufficient system resources to run this
                   facility at the map resolution you required. 
  Try a lower map resolution.  \\ 
                   &                   &   \\ %%% for presentation
Error: these values of Nside and l\_max $\ldots$ are too large $\ldots$ &  Fatal & You are exceeding
the limitation on Nside and l\_max. 
  Try a lower l\_max.  \\ 
                   &                   &   \\ \hline %%% for presentation
\end{tabular}
} 
\end{messages}

\rule{\hsize}{2mm}

\newpage
