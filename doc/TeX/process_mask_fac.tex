
\sloppy


\title{\healpix Fortran Facility User Guidelines}
\docid{process\_mask} \section[process\_mask]{\nosectionname}
\label{fac:process_mask}
\docrv{Version 1.1}
\author{Eric Hivon}
\abstract{This document describes the \healpix facility PROCESS\_MASK.}

\begin{facility}
{This code can be used to modify a binary mask by removing small clusters of bad
or invalid pixels (hereafter 'holes') and by computing the distance of each
valid pixel to the closest invalid one, with the purpose of, for instance,
defining a new apodized mask} 
{src/f90/process\_mask/process\_mask.F90}
\end{facility}

\begin{f90facility}
{process\_mask [parameter\_file]}
\end{f90facility}

% \begin{options}
%   \begin{optionlistwide}{} %%%% NOTE the ``extra'' brace here %%%%
%     \item[{\tt -d}]
%     \item[{\tt -}{\tt -}{\tt double}] double precision mode (see Notes on double/single precision modes on page~\pageref{page:ioprec})
%     \item[{\tt -s}]
%     \item[{\tt -}{\tt -}{\tt single}] single precision mode (default)
%   \end{optionlistwide}
% \end{options}

\begin{qualifiers}
  \begin{qulist}{} %%%% NOTE the ``extra'' brace here %%%%
\item[{mask\_file = }] Input binary mask FITS file
\item[{hole\_min\_size = }] Minimal size (in pixels) of invalid regions to be kept
       (can be used together with hole\_min\_surf\_arcmin2 below, the result will
be the largest of the two). \default{0}
\item[{hole\_min\_surf\_arcmin2 = }] Minimal surface area (in arcmin\^{}2) of invalid regions to be kept
       (can be used together with hole\_min\_size above,
        the result will be the largest of the two). \default{0.0}
\item[{filled\_file = }] Optional output FITS file to contain mask with
filled-in small holes (as defined above). \default{'', no output file}
\item[{distance\_file = }] Optional output FITS file to contain angular distance
(in radians) from valid pixel to the closest invalid one. \default{'', no output file}

  \end{qulist}
\end{qualifiers}

\begin{codedescription}
{For a given input binary mask, in which pixels have either value 0 (=invalid) or 1 (=valid),
this code produces a map containing for each valid pixel,
its distance (in Radians, measured between pixel centers) to the closest invalid pixel.

This distance map can then be used to define an apodized mask.

Two pixels are considered adjacent if they have at least {\em one point} in common 
(eg, a pixel corner or a pixel side).

It is possible to treat small holes (=cluster of adjacent invalid pixels) as valid,
by specifying a minimal number of pixels and/or minimal surface area (whichever is the largest),
and the resulting new mask can be output.

The output FITS files have the same ordering as the input mask
(even though the processing is done in NESTED ordering).

{\footnotesize{The algorithmic complexity of the distance calculation is expected to scale like $\propto \npix^p
\propto\nside^{2p}$ with $p$ in $[1.5,2]$ depending on the mask topology, even
though the code has been optimized to reduce the number of calculations by a
factor $10^2$ to $10^3$ compared to a naive implementation, and the most
computationally intensive loops are parallelized with OpenMP.
On a 3.06GHz Intel Core 2 Duo, the distances on a \htmladdnormallink{$\nside=512$ Galactic + Point sources mask}{http://lambda.gsfc.nasa.gov/product/map/dr4/masks_get.cfm} can be computed in a few
seconds, while a similar $\nside=2048$ mask takes a minute or less to process.
For totally arbitrary masks though, the return times can be multiplied by as
much as 10.}}

%
}
\end{codedescription}

% \begin{datasets}
% {
% \begin{tabular}{p{0.3\hsize} p{0.35\hsize}} \hline  
%   \textbf{Dataset} & \textbf{Description} \\ \hline
%                    &                      \\ %%% for presentation
%  None required & \\
%                    &                      \\ \hline %%% for presentation
% \end{tabular}
% } 
% \end{datasets}

\begin{support}
  \begin{sulist}{} %%%% NOTE the ``extra'' brace here %%%%
  \item[mollview] IDL routine to view the input and output masks and the angular
distance map.
%%%  \item[\htmlref{anafast}{fac:anafast}] This \healpix facility can analyse an
%%up/downgraded map.
  \item[mask\_tools] F90 module containing the routines   
	{\tt dist2holes\_nest}, %{sub:dist2holes_nest}
	{\tt fill\_holes\_nest}, %{sub:fill_holes_nest}
	{\tt maskborder\_nest}, %{sub:maskborder_nest}
	{\tt size\_holes\_nest}  %{sub:size_holes_nest}
used in \thedocid\ and
described in the \linklatexhtml{''Fortran Subroutines''}{subroutines.pdf}{subroutines.htm} document
  \end{sulist}
\end{support}

\begin{examples}{1}
{
\begin{tabular}{ll} %%%% use this tabular format %%%%
process\_mask  \\
\end{tabular}
}
{
process\_mask runs in interactive mode, self-explanatory.
}
\end{examples}

%\vfill\newpage

\begin{examples}{2}
{
\begin{tabular}{ll} %%%% use this tabular format %%%%
process\_mask  filename \\
\end{tabular}
}
{When `filename' is present, process\_mask enters the non-interactive mode and parses
its inputs from the file `filename'. This has the following
structure: the first entry is a qualifier which announces to the parser
which input immediately follows. If this input is omitted in the
input file, the parser assumes the default value.
If the equality sign is omitted, then the parser ignores the entry.
In this way comments may also be included in the file.
In this example, the file contains the following qualifiers:\hfill\newline
\fileparam{mask\_file = wmap\_temperature\_analysis\_mask\_r9\_5yr\_v3.fits}
\fileparam{hole\_min\_size =    100}
\fileparam{distance\_file = !/tmp/dist\_wmap.fits}
process\_mask computes the distance in Radians from each valid pixel to the closest invalid
pixel for WMAP-5 mask 'wmap\_temperature\_analysis\_mask\_r9\_5yr\_v3.fits', ignoring
the holes containing fewer than 100 pixels, and outputs the result in '/tmp/dist\_wmap.fits'.}
\end{examples}

\begin{release}
  \begin{relist}
    \item (Initial release \healpix 3.00)
%    \item Extension to multi-dimensional maps (\healpix 1.20)
  \end{relist}
\end{release}

% \begin{messages}
% {
% \begin{tabular}{p{0.25\hsize} p{0.1\hsize} p{0.35\hsize}} \hline  
%   \textbf{Message} & \textbf{Severity} & \textbf{Text} \\ \hline
%                    &                   &   \\ %%% for presentation
% can not allocate memory for array xxx &  Fatal & You do not have
%                    sufficient system resources to run this
%                    facility at the map resolution you required. 
%   Try a lower map resolution.  \\ 
%                    &                   &   \\ \hline %%% for presentation
% \end{tabular}
% } 
% \end{messages}

\rule{\hsize}{2mm}

\newpage
