\sloppy
\docid{plm\_gen}\section[plm\_gen]{ }
\label{sub:plm_gen}
\docrv{Version 2.0}
\author{Eric Hivon}
\abstract{This document describes the \healpix Fortran90 subroutine PLM\_GEN.}

\begin{facility}
{This routine computes the latitude dependent part $\lambda_{\ell m}$ of the
  spherical harmonics ($Y_{\ell m}(\theta,\phi) = \lambda_{\ell m}(\theta) e^{i m \phi}$) of spin 0 and 2
  (see \healpix primer)
  used to synthetize or analyze \healpix maps of temperature and polarisation.}
{\modAlmTools}
\end{facility}

\begin{f90format}
{nsmax, nlmax, nmmax, plm}
\end{f90format}

\begin{arguments}
{
\begin{tabular}{p{0.4\hsize} p{0.05\hsize} p{0.1\hsize} p{0.35\hsize}} \hline  
\textbf{name~\&~dimensionality} & \textbf{kind} & \textbf{in/out} & \textbf{description} \\ \hline
                   &   &   &                           \\ %%% for presentation
nsmax & I4B & IN & The $\nside$ value for which to compute the $\lambda_{\ell m}$. \\
nlmax & I4B & IN & The maximum multipole order $\ell$ of the generated $\lambda_{lm}$. \\
nmmax & I4B & IN & The maximum degree $m$ of the generated $\lambda_{lm}$. \\
plm(0:n\_plm-1, 1:p) & DP & OUT & The $\lambda_{lm}$ values, either for temperature only
                   ($p=1$) or temperature and polarisation ($p=3$). The number
                    of $\lambda_{\ell m}$ is n\_plm =
                    nsmax*(nmmax+1)*(2*nlmax-nmmax+2). They are stored in the
                    order of increasing order $\ell$, increasing degree $m$, for
                    all the \healpix ring colatitudes $\theta$ from North Pole to Equator, ie
 		   $\lambda_{00}(\theta_1)$, $\lambda_{10}(\theta_1)$, $\lambda_{20}(\theta_1)$,
                    \ldots, $\lambda_{11}(\theta_1)$, $\lambda_{21}(\theta_1)$;
                    \ldots,  $\lambda_{00}(\theta_2)$ \ldots \\
\end{tabular}
}
\end{arguments}

\begin{example}
{
use healpix\_types \\
use alm\_tools, only : plm\_gen \\
integer(I4B) :: nside, lmax, mmax, n\_plm\\
real(DP), dimension(:,:), allocatable :: plm \\
\ldots \\
nside=256 ; lmax=512 ; mmax=lmax\\
npix=nside2npix(nside)\\
n\_plm=nside*(mmax+1)*(2*lmax-mmax+2)\\
allocate(plm(0:n\_plm-1,1:3))\\
\ldots \\
call \thedocid(nside, lmax, mmax, plm)  \\
}
{
Computes the spherical harmonics for temperature and polarisation for $N_{side}= 256$, up to 512 in $\ell$ and $m$.
}
\end{example}

\begin{modules}
  \begin{sulist}{} %%%% NOTE the ``extra'' brace here %%%%
  \item[compute\_lam\_mm, get\_pixel\_layout, ]
  \item[gen\_lamfac,gen\_mfac, gen\_normpol, ] 
  \item[gen\_recfac, init\_rescale, l\_min\_ylm] Ancillary routines used
  for $\lambda_{\ell m}$ recursion
  \item[\textbf{misc\_utils}] module, containing:
  \item[assert\_alloc] routine to print error message, when an array can not be
  allocated properly
  \end{sulist}
\end{modules}

\begin{related}
  \begin{sulist}{} %%%% NOTE the ``extra'' brace here %%%%
   \item[\htmlref{alm2map}{sub:alm2map}] routine generating maps of temperature
   and polarisation from their $a_{\ell m}$ that can use precomputed $\lambda_{\ell
   m}$ generated by \thedocid
   \item[\htmlref{map2alm}{sub:map2alm}] routine analysing maps of temperature
   and polarisation that can use precomputed $\lambda_{\ell
   m}$ generated by \thedocid
  \item[plmgen] executable using \thedocid\ to compute the $\lambda_{\ell m}$ and
  writting them on disc 
%%   \item[anafast] executable that may use the $\lambda_{\ell m}$ produced by \thedocid\ to analyse maps
%%   \item[smoothing] executable that may use the $\lambda_{\ell m}$ produced by \thedocid\ to smooth maps
%%   \item[synfast] executable that may use the $\lambda_{\ell m}$ produced by \thedocid\ to synthesize maps.
  \end{sulist}
\end{related}

\rule{\hsize}{2mm}

\newpage
