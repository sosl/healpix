
\sloppy


\title{\healpix Fortran Facility User Guidelines}
\docid{sky\_ng\_sim} \section[sky\_ng\_sim]{\nosectionname}
\label{fac:sky_ng_sim}
\docrv{Version 1.0}
\author{Eric Hivon}
\abstract{This document describes the \healpix facility SKY\_NG\_SIM.}

\begin{facility}
{This program can be used to create temperature \healpix maps computed as realisations 
of random {\em Non-Gaussian} fields on a sphere (either even power of a Gaussian
distribution, or Simple Harmonics Oscillator PDF, see Description section for
details). \\
It is directly adapted from the NGSIMS code described in 
\htmladdnormallink{Rocha et al, MNRAS, 357, 1 (2005)}%
{http://cdsads.u-strasbg.fr/abs/2005MNRAS.357....1R}%
\\
The operation count is dominated by a term scaling as
 ${\cal {O}}(N_{\rm pix}^{1/2} \ell_{\rm max}^2)$. 
The map angular power spectrum, resolution, Gaussian beam FWHM or arbitrary beam window  
and random seed for the simulation can be selected by the user.
}
{src/f90/ngsims\_full\_sky/sky\_ng\_sim.F90}
\end{facility}

\begin{f90facility}
{sky\_ng\_sim [parameter\_file]}
\end{f90facility}


\begin{qualifiers}
  \begin{qulistwide}{} %%%% NOTE the ``extra'' brace here %%%%
     \item[{simul\_type = }] Defines the simulation type, 1=temperature map only,
       3=temperature and its first spatial derivatives,
       4=temperature and its first and second spatial derivatives.
(default= 1).
    \item[{infile = }] Defines the input power spectrum file,
	(default= HEALPIX/test/cl.fits).
%
    \item[{outfile\_alms = }] Defines the FITS file in which to output $a_{lm}$ used
      for the simulation (default= `')
%
    \item[{outfile = }] Defines the output map file, (default= test.fits).
%
    \item[{nsmax = }] Defines the resolution of the map.
(default= 32)
%
     \item[{nlmax = }] Defines the maximum $\ell$ value 
to be used in the simulation. WARNING: $\ell_{max}$ can not exceed
the value $4\cdot$ {\tt nsmax}, because the coefficients of the  average Fourier 
pixel window functions
are precomputed and provided up to this limit.
(default= 2*{\tt nsmax})
%
    \item[{fwhm\_arcmin = }] Defines the FWHM size in arcminutes 
of the simulated Gaussian beam.
(default= 0.0)
%
    \item[{beam\_file = }] Defines the FITS file describing the
    Legendre window
    function of the circular beam to be used for the
    simulation. If set to an existing file name, it will override the
    {\tt fhwm\_arcmin} given above. (default=`')
%
     \item[{windowfile = }] Defines the input filename  for the pixel
    smoothing windows 
(default= pixel\_window\_n????.fits, see Notes on default files and directories
on page~\pageref{page:defdir})
%
     \item[{winfiledir = }] Defines the directory in which windowfile
    is located (default : see Notes on default files and directories on
page~\pageref{page:defdir}).
%
      \item[{iseed = }] Defines the seed of the pseudo-random sequence to be used 
for the generation of the non-gaussian white noise (default= 1)
%
     \item[{plot =}] If \thedocid was linked with the PGPLOT library during compilation, and
if {\tt plot} is set to (case unsensitive) {\tt .true.}, {\tt t}, {\tt yes}, {\tt y}  or
1, then the histogram of the simulated non-gaussian is produced, overplotted
with the theoretical PDF; the histogram of the final map pixel values,
overplotted with a PDF of a gaussian of same mean and variance is subsequently
produced.
(default=.false.)
%
     \item[{pdf\_choice=1}] Choice of non-Gaussian PDF to use: 1= Simple
Harmonics oscillator (see Eq~\ref{eq:nongauss_sho} below)
%
    \begin{itemize}
      \item[{sigma0= }] RMS of oscillator ground state
%
      \item[{na= }] Integer in \{0, 20\}. Number of $\alpha$ coefficients to be
given (default=3).
Note: analytical calculation of the PDF moments
can only be done for {\tt na}$\le 3$. 
%
      \item[{alpha\_1=, alpha\_2=, ... }] Real values of $\alpha_i$ coefficients for
$i$ in $[1, {\tt na}]$
     \end{itemize}
%
     \item[{pdf\_choice=2}] Choice of non-Gaussian PDF to use: 2=Power of a
Gaussian (see Eq~\ref{eq:nongauss_powgauss} below)
      \begin{itemize}
     \item[{npower = }] Positive integer in \{1,4\} (default=1). The gaussian
will be set to the power 2*{\tt npower}.
     \end{itemize}
%
  \end{qulistwide}
\end{qualifiers}
\newpage

\begin{codedescription}
{%
A random non-Gaussian white noise map is generated, using either
%-------
\begin{itemize}
%
\item a simple linear harmonic oscillator, where the PDF of the pixel
temperature $t$ is
	\begin{equation}
	\rho_{\rm SHO}(t) = \left| \psi_n\right|^2 = e^{-t^2/2\sigma_0^2} \left|\sum_{i=0}^{n} \alpha_i
C_i H_i \left(\frac{t}{\sqrt{2}\sigma_0}\right)\right|^2 \label{eq:nongauss_sho}
	\end{equation}
where $H_i$ are the Hermite polynomials, $C_i$ their normalization constants,
$\sigma_0^2$ the variance of the (Gaussian) ground state $\left|\psi_0\right|^2,$
$\alpha_i$ for $i\ge 1$ are free parameters, while $\alpha_0 =
\left(1 - \sum_i^n |\alpha_i|^2\right)^{1/2}$ is constrained;
%
\item or, an even power of a gaussian PDF, where the temperature of pixel $q$ is
	\begin{equation}
	t_q = g_q^{2 P} \label{eq:nongauss_powgauss}
	\end{equation}
where $g$ is a zero mean, unit variance Gaussian variable, and $P$ is
a user-defined positive integer.
\end{itemize}
%--------
The resulting map is analyzed into its $a_{lm}$ coefficients, which are then
multiplied by the beam, pixel and spectrum window
\begin{equation}
	a_{lm} \longrightarrow a_{lm} \left[C(l)\right]^{1/2} B(l) w_{\rm pix}(l) \nonumber
\end{equation}
The resulting $a_{lm}$ coefficients are turned back into a map, which is
therefore non-gaussian, with an effective angular power spectrum $C(l) B(l)^2
w_{\rm pix}(l)^2$ (Rocha et al, 2005).
\\
Notes: the code has been modified from the original NGSIMS package in several
respects:
the seed parameter is named {\tt iseed} instead of {\tt idum}, to be consistent
with other \healpix simulation codes; and the SHO generator has been
dramatically sped up, without loss of accuracy. Moreover, just like in \htmlref{{\tt synfast}}{fac:synfast} facility,
it is now possible to output the $a_{lm}$ coefficients being used ({\tt
outfile\_alms} option), and the spatial derivatives of the final map can also be
output ({\tt simul\_type} option).
}
\end{codedescription}


\begin{datasets}
{
\begin{tabular}{p{0.3\hsize} p{0.35\hsize}} \hline  
  \textbf{Dataset} & \textbf{Description} \\ \hline
                   &                      \\ %%% for presentation
  /data/pixel\_window\_nxxxx.fits & Files containing pixel windows for
                   various nsmax.\\ 
                   &                      \\ \hline %%% for presentation
\end{tabular}
} 
\end{datasets}

\begin{support}
  \begin{sulist}{} %%%% NOTE the ``extra'' brace here %%%%
  \item[\htmlref{generate\_beam}{sub:generate_beam}] This \healpix Fortran
subroutine generates or reads the $B(\ell)$ window function used in \thedocid
  \item[\htmlref{map2gif}{fac:map2gif}] This \healpix Fortran facility can be used to visualise the
  output map.
  \item[mollview] This \healpix IDL facility can be used to visualise the
  output map.
  \item[\htmlref{anafast}{fac:anafast}] This \healpix Fortran facility can analyse a \healpix map and 
     	       save the $a_{lm}$ and $C_l$s to be read by sky\_ng\_sim.
		
  \end{sulist}
\end{support}

\begin{examples}{1}
{
\begin{tabular}{ll} %%%% use this tabular format %%%%
sky\_ng\_sim  \\
\end{tabular}
}
{
sky\_ng\_sim runs in interactive mode, self-explanatory.
}
\end{examples}

%\vfill\newpage

\begin{examples}{2}
{
\begin{tabular}{ll} %%%% use this tabular format %%%%
sky\_ng\_sim  filename \\
\end{tabular}
}
{When 'filename' is present, {\tt sky\_ng\_sim} enters the non-interactive mode and parses
its inputs from the file 'filename'. This has the following
structure: the first entry is a qualifier which announces to the parser
which input immediately follows. If this input is omitted in the
input file, the parser assumes the default value.
If the equality sign is omitted, then the parser ignores the entry.
In this way comments may also be included in the file.
In this example, the file contains the following qualifiers:\hfill\newline
\fileparam{simul\_type = 1}
\fileparam{nsmax = 128}
\fileparam{nlmax = 256}
\fileparam{fwhm\_arcmin = 30.0}
\fileparam{infile = cl.fits}
\fileparam{pdf\_choice = 1}
\fileparam{iseed =  1}
\fileparam{na = 3}
\fileparam{sigma0 = 1.0}
\fileparam{alpha\_1 = 0.0}
\fileparam{alpha\_2 = 0.0}
\fileparam{alpha\_3 = 0.2}
\fileparam{outfile = !test\_ngfs.fits}
{\tt sky\_ng\_sim} reads in the $C_l$ power spectrum in 'cl.fits' up to $l=256$, and produces the map
'map.fits' which has $N_{\rm side}=128$. The non-gaussian white noise was generated
assuming a SHO PDF (see Eq~\ref{eq:nongauss_sho} above) with $\sigma_0=1$ and
$\alpha_i = (0, 0, 0.2)$.
The map is convolved with a beam of FWHM 30.0 arcminutes. The $iseed=1$ sets
the random seed for the realisation. A different $iseed$ would have given a different 
realisation from the same power spectrum. And finally, since {\tt simul\_type =
1} only the map (and not its spatial derivatives) will be output.

Since \hfill\newline
\fileparam{beam\_file }
\fileparam{windowfile }
\fileparam{outfile\_alms }
were omitted, they take their default values (empty strings). 
This means respectively that no beam were read, that {\tt sky\_ng\_sim} attempts to find the pixel
window files in the default directories (see page~\pageref{page:defdir}), and
that the $a_{lm}$ generated and used to produce the map were not output.
}
\end{examples}

\begin{release}
  \begin{relist}
    \item Initial release (\healpix 2.10)
  \end{relist}
\end{release}

\begin{messages}
{
\begin{tabular}{p{0.25\hsize} p{0.1\hsize} p{0.35\hsize}} \hline  
  \textbf{Message} & \textbf{Severity} & \textbf{Text} \\ \hline
                   &                   &   \\ %%% for presentation
can not allocate memory for array xxx &  Fatal & You do not have
                   sufficient system resources to run this
                   facility at the map resolution you required. 
  Try a lower map resolution.  \\ 
                   &                   &   \\ \hline %%% for presentation

this is not a binary table & & the fitsfile you have specified is not 
of the proper format \\
                   &                   &   \\ \hline %%% for presentation
there are undefined values in the table! & & the fitsfile you have specified is not 
of the proper format \\
                  &                   &   \\ \hline %%% for presentation
the header in xxx is too long & & the fitsfile you have specified is not 
of the proper format \\
                  &                   &   \\ \hline %%% for presentation
XXX-keyword not found & & the fitsfile you have specified is not 
of the proper format \\
                  &                   &   \\ \hline %%% for presentation
found xxx in the file, expected:yyyy & & the specified fitsfile does not
contain the proper amount of data. \\
                   &                   &   \\ \hline %%% for presentation



\end{tabular}
} 
\end{messages}

\rule{\hsize}{2mm}

\newpage
