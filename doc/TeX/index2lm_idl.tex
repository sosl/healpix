% -*- LaTeX -*-

\sloppy

\title{\healpix IDL Facility User Guidelines}
\docid{index2lm} \section[index2lm]{ }
\label{idl:index2lm}
\docrv{Version 2.0}
\author{Eric Hivon}
\abstract{This document describes the \healpix facility index2lm.}

\begin{facility}
{This IDL routine provides a means to convert the $a_{\ell m}$ index $i=\ell^2 +
\ell + m + 1$ (as returned by eg the fits2alm routine) into $\ell$ and $m$.}
{src/idl/misc/index2lm.pro}

\end{facility}

\begin{IDLformat}
{INDEX2LM, index, l, m}
\end{IDLformat}

\begin{qualifiers}
  \begin{qulist}{} %%%% NOTE the ``extra'' brace here %%%%
    \item[index] Long array containing on INPUT the index \hfill\newline
                 $i$ = $\ell^2$ + $\ell$ + $m$ + 1.
    \item[l] Long array containing on OUTPUT the order $\ell$. It has the same
    size as {\tt index}.
    \item[m] Long array containing on OUTPUT the degree $m$. It has the same
    size as {\tt index}.
  \end{qulist}
\end{qualifiers}

\begin{codedescription}
{\thedocid\ converts $i=\ell^2 + \ell + m + 1$ into $(\ell, m)$. Note that the index $i$ is only
defined for $0 \le |m|\le \ell$.
}
\end{codedescription}



\begin{related}
  \begin{sulist}{} %%%% NOTE the ``extra'' brace here %%%%
    \item[idl] version \idlversion or more is necessary to run \thedocid.
    \item[\htmlref{fits2alm}{idl:fits2alm}] reads a FITS file containing
    $a_{\ell m}$ values.
    \item[\htmlref{alm2fits}{idl:alm2fits}] writes $a_{\ell m}$ values into a FITS file.
    \item[\htmlref{lm2index}{idl:lm2index}] routine complementary to \thedocid:
    converts $(\ell, m)$ into $i$ = $\ell^2$ +
    $\ell$ + $m$ + 1.
  \end{sulist}
\end{related}

\begin{example}
{
\begin{tabular}{l} %%%% use this tabular format %%%%
\thedocid, index, l, m \\
\end{tabular}
}
{
will return in {\tt l} and {\tt m} the order $\ell$ and degree $m$ such that {\tt index} $=\ell^2 +
ell + m + 1$
}
\end{example}

